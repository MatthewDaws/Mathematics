\documentclass[a4paper,12pt]{article}

\usepackage[margin=2cm]{geometry}
\usepackage{amsmath,amssymb,amsthm}
\usepackage[all]{xy}
\usepackage{url, wrapfig}

\theoremstyle{plain}
\newtheorem{proposition}{Proposition}[section]
\newtheorem{theorem}[proposition]{Theorem}
\newtheorem{corollary}[proposition]{Corollary}
\newtheorem{lemma}[proposition]{Lemma}
\newtheorem{claim}[proposition]{Claim}
\newtheorem{definition}[proposition]{Definition}
\newtheorem{example}[proposition]{Example}
\newtheorem{question}[proposition]{Question}
\theoremstyle{definition}
\newtheorem{remark}[proposition]{Remark}

\newcommand{\ip}[2]{{\langle {#1} , {#2} \rangle}}
\newcommand{\mc}[1]{\mathcal{#1}}
\renewcommand{\mod}{\textsf{Mod-}}
\newcommand{\mend}{\operatorname{End}}
\renewcommand{\hom}{\operatorname{Hom}}
\newcommand{\unkn}{\underline{\ \ }}
\newcommand{\ann}{\operatorname{Ann}}

\begin{document}

\title{Commentary on ``$K_0$ of purely infinite simple regular rings''}
\author{Matt Daws}
\maketitle

\begin{abstract}
We give the details for some of the results from the start of \cite{agp}.
\end{abstract}

We assume throughout that $R$ is a (unital) ring.  We write $\mod R$ for the category of right
$R$-modules, and for $M,N\in\mod R$ we write $\hom_R(M,N)$ for the morphisms: the right $R$-module
homomorphisms.  We write $\mend_R(M)$ for $\hom_R(M,M)$.

For us, a ring is \emph{simple} exactly when it has no proper two-sided ideals.  We warn the reader
that this is not completely standard terminology, compare \cite[Chapter~XVII, Section~5]{lang} for example.

\section{Idempotents}

\begin{definition}
Let $R$ be a ring, and let $e,f\in R$ be idempotents.  Then $e,f$ are (Murray-von Neumann) \emph{equivalent}
if there are $x\in eRf, y\in fRe$ with $e=xy, f=yx$.  In this case, we write $e\sim f$.
\end{definition}

Suppose we merely have $x,y\in R$ with $e=xy, f=yx$.  As $e^2=e$ we have $xy = xyxy = xfy = xf^2y$ and so
$xy = e = e^3 = exye = exf fye$.  Similarly we can show that $yx = fye exf$, and so it is no loss of
generality to suppose that $x\in eRf$ and $y\in fRe$ as in the definition.

When $e$ is an idempotent, $eR$ is a complemented (right) submodule of $R$.

\begin{lemma}\label{lem:1}
For idempotents $e,f\in R$ we have that $e\sim f$ if and only if $eR \cong fR$ in $\mod R$.
\end{lemma}
\begin{proof}
If $e\sim f$ with $e=xy, f=yx$ as in the definition, then define $\theta:eR \rightarrow fR$ by $\theta(a) = ya$.  This is defined, as if
$a\in eR$ then $ea=a$ so $ya = y(xy)a = fya \in fR$.  Clearly $\theta\in\hom_R(eR,fR)$.  Similarly, there
is $\phi\in\hom_R(eR,fR)$ given by $\phi(b)=xb$.  Then $\theta\phi(b) = yxb = fb = b$ for $b\in fR$, and
$\phi\theta(a) = xya = ea = a$ for $a\in eR$.  Hence $eR \cong fR$ in $\mod R$.

Conversely, let $e,f$ be idempotents, and let $\theta\in\hom_R(eR,fR)$ be an isomorphism.  Let $y=\theta(e)$
and $x=\theta^{-1}(f)$.  For $a\in eR$ as $a=ea$ we have $\theta(a) = \theta(ea) = \theta(e)a = ya$; similarly
for $b\in fR$ we have $\theta^{-1}(b) = xb$.  Thus, for $a\in eR$ we have $a = \theta^{-1}(\theta(a))
= xya$ and for $b\in fR$ we have $b = yxb$.  In particular, $e=xye$ and $f=yxf$.  As $y\in fR$ and $x\in eR$ we have $fy=y, ex=x$.  Set $a = fye = ye$ and set $b = exf = xf$.  Then $ab = yexf = yxf = f$ and $ba = xfye = xye = e$ and so $e\sim f$.
\end{proof}

\begin{definition}
Two idempotents $e,f$ are \emph{orthogonal} when $ef=fe=0$, written $e \perp f$.
\end{definition}

If $e \perp f$ then $(e+f)^2=e+f$ so $e+f$ is an idempotent. Conversely, if $2x=0$ implies $x=0$ in $R$,
then if $(e+f)^2 = e+f$ then $ef+fe = 0$.  Then $0 = e0 = ef + efe$ and $0 = 0e = efe + fe$ so $ef = fe$
and so $ef = fe = 0$.  We need this condition, for with $R = \mathbb Z / 2\mathbb Z \oplus \mathbb Z / 2\mathbb Z$
and $e=(1,1), f=(0,1)$ we have that $ef = fe = f\not=0$ but $e+f = (1,0)$ which is an idempotent.

When $e\perp f$, it is easy to see that $(e+f)R \cong eR \oplus fR$.

\begin{lemma}[{\cite[Lemma~1.1]{agp}}]\label{lem:4}
Let $e\in R$ be an idempotent, and suppose there are right ideals $(A_i)_{i=1}^n$ in $R$ with
$eR = A_1 \oplus \cdots \oplus A_n$.  Then there exist pairwise orthogonal idempotents $e_i\in R$ with
$e=\sum_i e_i$ and $A_i=e_iR$ for each $i$.
\end{lemma}
\begin{proof}
Consider $S = \mend_R(eR)$ which acts on the left of $eR$.  That $eR = A_1 \oplus \cdots \oplus A_n$ is
a decomposition of right $R$-modules means that the projection onto $A_i$ gives an idempotent $f_i$ in $S$.
Thus the $f_i$ are pairwise orthogonal, sum to the identity, and $f_i(eR) = A_i$ for each $i$.

Define $\theta:S\rightarrow eRe$ by $\theta(f) = f(e)$.  For $f\in\mend_R(eR)$ we have $f(e) = f(e^2) = f(e)e$
and so $f(e)\in eRe$ and $\theta$ is defined.  Furthermore, for $x\in R$ we have $f(ex) = f(e)x$ while for any
$eye\in eRe$ we have that $x \mapsto eyex$ defines a member of $\mend_R(eR)$.  Thus $\theta$ is an isomorphism,
and setting $e_i = \theta(f_i)$ gives the required conclusion.
\end{proof}

\begin{definition}
An idempotent $e\in R$ is \emph{infinite} if there are orthogonal idempotents $f,g\in R$ with $e=f+g$ and
$e\sim f$ with $g\not=0$.
\end{definition}

By the Lemma, $e$ is infinite if and only if $eR$ is isomorphic to a proper direct summand of itself.
In this case, we say that $eR$ is \emph{directly infinite}.

\begin{lemma}\label{lem:5}
Let $R$ be a ring and let $e$ be an infinite idempotent.  If $e\sim f$ then also $f$ is infinite.
\end{lemma}
\begin{proof}
As $e$ is infinite there are orthogonal idempotents $a,b$ with $e=a+b$ and $e\sim a$.
As a useful aside, we claim that $ea=ae=ea$ and $eb=be=b$.  Indeed, $a+b = e = e^2 = e(a+b) = (a+b)e$.
Multiply by $a$ on either side to get $a = a(a+b)e = ae$ and $a = e(a+b)a = ea$.  The same argument works for $b$.

There are $u\in eRa, v\in aRe$ with $uv = e, vu = a$.  As $e\sim f$ there are $x\in eRf$ and $y\in fRe$
with $xy = e$ and $yx = f$.  Set $c = yax$ and $d = ybx$.  Then $c^2 = yaxyax = yaeax = yax = c$, and similarly
$d^2=d$, and $cd = yaxybx = yaebx = yabx = 0$, and similarly $dc=0$.  Then $c+d = y(a+b)x = yex = yx = f$ so
$f$ is the sum of the orthogonal idempotents $c,d$.

Observe now that $(yux)(yvx) = yuevx = yuaeavx = yuavx = yuvx = yex = yx = f$ and $(yvx)(yux) = yveux = yvux
= yax = c$ so $f\sim c$, showing that $f$ is infinite.
\end{proof}

\begin{lemma}\label{lem:3}
Let $P\in\mod R$ and suppose that $P\cong Q\oplus A$ with $Q$ being directly infinite.  Then also $P$
is directly infinite.
\end{lemma}
\begin{proof}
That $Q$ is directly infinite means that $Q\cong Q\oplus B$ for some $0 \not= B\in\mod R$. Then
\[ P \cong    Q\oplus A \cong    Q\cong B\oplus A \cong    P \oplus B, \]
and so $P$ is directly infinite.
\end{proof}

\begin{definition}
A ring $R$ is \emph{purely infinite} if every nonzero right ideal of $R$ contains an infinite idempotent.
\end{definition}



\section{Some ring theory}

Given a set $I$ write $R^I$ for the direct sum of $I$-many copies of $R$.  One construction of this
is to consider all functions $f:I\rightarrow R$ which have $f(i)\not=0$ for only finitely many $i$, and endow
with point-wise operations.  Similarly define $M^I$ for $M\in\mod R$.

Recall that $F\in\mod R$ is \emph{free} if $P$ is isomorphic to $R^I$ for some $I$.  A module $P\in\mod R$ is
\emph{projective} if given any $N,M\in\mod R$, any $g\in\hom_R(P,M)$ and any surjective $f\in\hom_R(N,M)$,
there is some $h\in\hom_R(P,N)$ with $fh=g$:
\[ \xymatrix{ &  N \ar@{->>}[d]^f \\
P \ar@{..>}[ru]^{\exists\, h} \ar[r]^g &
M } \]

The following is standard:

\begin{lemma}
For $P\in\mod R$ the following are equivalent:
\begin{enumerate}
\item $P$ is projective;
\item $P$ is the direct summand of a free module.  That is, there is a free module $F$ and $Q\in\mod R$ with
$F \cong P\oplus Q$;
\item The functor $\hom_R(P,\unkn)$, from $\mod R$ to the category of Abelian groups, is exact.
\end{enumerate}
\end{lemma}

We shall say that $X\in\mod R$ \emph{generates} $\mod R$, or is a \emph{generator}, if, whenever we have $M,N\in\mod R$ and $f,g\in\hom_R(M,N)$ with $f \alpha = g \alpha$ for all $\alpha\in\hom_R(X,M)$, then $f=g$.  This is the usual Category Theory
notion of a ``generator'' or \emph{separator}, though we warn the reader that this is not completely settled,
compare \cite{qy1}.

As $\mod R$ is Abelian, it suffices to take $g=0$ in the definition.  To be more prosaic, if given $h\in\hom_R(M,N)$ there exists $\alpha\in\hom_R(X,M)$ with $h\alpha\not=0$ then for $f,g\in\hom_R(M,N)$ set $h=f-g$, find a suitable $\alpha$, and then note that $h\alpha\not=0$ exactly when $f\alpha \not= g\alpha$.

For the following, compare the definition of ``generator'' given in \cite[Chapter~XVII, Section~7]{lang}.

\begin{lemma}\label{lem:2}
$X\in\mod R$ is a generator if and only if for each $M\in\mod R$ there is some index set $I$ and a surjective
$f\in\hom_R(X^I,M)$.  If $M$ is finitely generated, we may choose $I$ to be finite.
\end{lemma}
\begin{proof}
If $X$ is a generator, then let $N\subseteq M$ be sum of $\alpha(X)$ as $\alpha$ varies through $\hom_R(X,M)$.
Notice that $N$ can be obtained as the image of some morphism $X^I\rightarrow M$ for some index set $I$.
If $N\not=M$ then $M/N$ is non-zero, and so the quotient map $h:M\rightarrow M/N$ is non-zero.  Thus there is
$\alpha\in\hom_R(X,M)$ with $h\alpha \not = 0$.  However, this means that $\alpha(X)$ is not a subset of $N$,
contradiction.

Suppose now that $M$ is finitely generated, with generators $m_1,\cdots,m_n$.  We have already shown that
for each $i$ there is are $\alpha_1,\cdots,\alpha_j \in \hom_R(X,M)$ and $x_1,\cdots,x_j\in X$ with
$m_i = \sum_k \alpha_k(x_k)$.  Thus for $a\in R$ we have that $m_i a = \sum_k \alpha_k(x_k a)$.  It follows
that we can find a finite $I$ and $\alpha:X^I\rightarrow M$ surjective.

Conversely, if every module is a homomorphic image of $X^I$ for some $I$, then let $h\in\hom_R(M,N)$ be non-zero.
Select $I$ and $\alpha\in \hom_R(X^I,M)$ a surjection.  Such an $\alpha$ is given by a family $(\alpha_i)_{i\in I}$
in $\hom_R(X,M)$ in the obvious way.  If $h\alpha_i=0$ for all $i$, then also $h\alpha=0$ which contradicts $\alpha$ being surjective and $h\not=0$.  Thus for some $i$ with have $h\alpha_i\not=0$, as required to show that $X$ is a generator.
\end{proof}

The following is inspired by \cite{qy1}.

\begin{lemma}
Let $X\in\mod R$ be projective.  Then
$X\in\mod R$ is a generator if and only if for each non-zero $M\in\mod R$ we have that $\hom_R(X,M)\not=0$.
\end{lemma}
\begin{proof}
The ``only if'' case follows immediately from the lemma above (and holds for any $X$).  With $X$ assumed
projective, suppose that $\hom_R(X,M)\not=0$ for all $M\not=0$.  We first give a category-theoretic proof.
Let $h\in\hom_R(M,N)$ be non-zero.  The following sequence is exact
\[ 0 \rightarrow \ker(h) \rightarrow M \rightarrow M/\ker(h) \rightarrow 0, \]
and so as $X$ is projective, the following sequence is also exact
\[ 0 \rightarrow \hom_R(X,\ker(h)) \rightarrow \hom_R(X,M) \rightarrow \hom_R(X,M/\ker(h))
\rightarrow 0. \]
As $h\not=0$ we have that $\ker(h)\not= M$ and so $\hom_R(X,M/\ker(h)) \not =0$ by hypothesis.
By exactness, there is some $\alpha\in \hom_R(X,M)$ which does not map to $0$ in $\hom_R(X,M/\ker(h))$,
that is, with $h\alpha\not=0$.  This is exactly what we needed to prove to show that $X$ is a generator.

Let us now give a more direct proof.  Let $h\in\hom_R(M,N)$ be non-zero, and again consider $M/\ker(h)$ which is non-zero as $h$ is non-zero, so by assumption, there is $\beta:X\rightarrow M/\ker(h)$ non-zero.  As
$X$ is projective, there is $\alpha$ making the diagram commute.
\[ \xymatrix{ &  M \ar@{->>}[d] \\
X \ar@{..>}[ru]^{\exists\, \alpha} \ar[r]_-\beta &
M/\ker(h) } \]
If $h\alpha=0$ then $\alpha(X)\subseteq\ker(h)$, which implies that $\beta=0$ which is not so.  Thus
$h\alpha\not=0$, as required to show that $X$ is a generator.
\end{proof}

For $n\geq 1$ consider $R^n$ as a right $R$-module.  If $R^n = X\oplus Y$ for some submodules $X,Y$ then
there is an idempotent $e\in \hom_R(R^n)$ with $X = e(R^n)$.  Furthermore, $\hom_R(R^n)$ may be identified
with $M_n(R)$ \emph{acting on the left} of $R^n$.

Given $M\in\mod R$ let $\ann(M) = \{ x\in R : mx=0 \ (m\in M) \}$.  For $x\in\ann(M)$ and $y,z\in R$ we
have that $myxz = (my)x(z) = 0z = 0$ for all $m\in M$, so $yxz\in\ann(M)$.  Thus $\ann(M)$ is an ideal in $R$.
If $I \trianglelefteq R$ is an ideal, then let $M = R/I$ as a right module.  Then $x\in\ann(M)$ exactly
when $(y+I)x = 0$ for all $y\in R$, that is, $Rx \subseteq I$, that is, $x\in I$.  So $\ann(R/I) = I$ and
hence all ideals of $R$ arise in this way.

Finally, we need one small part of the Morita equivalence between $R$ and $M_n(R)$, which is the claim
that every $M_n(R)$ module is of the form $M^n$ (thought of as row vectors, with $M_n(R)$ acting on the
right by matrix multiplication) for some $M\in \mod R$.  This claim can be proved directly if one wishes.

\begin{proposition}
Let $e\in M_n(R)$ be an idempotent, and set $P = e R^n$ considered as a right $R$-module.  $P$ is
a generator if and only if $e$ is not contained in any proper two-sided ideal of $M_n(R)$.
\end{proposition}
\begin{proof}
For $M\in\mod R$ non-zero, we wish to show that $\hom_R(P,M)\not=0$.  Let $R^n$ have basis $(e_i)_{i=1}^n$
so every $x\in R^n$ can be written as $x = \sum_{i=1}^n e_i x_i$ for some $x_i\in R$.
We identify
\[ \hom_R(R^n,M) \cong \hom_R(R,M) \oplus \cdots \oplus \hom_R(R,M) \cong M^n, \]
so $f\in \hom_R(R^n,M)$ is identified with $(\xi_i)\in M^n$ when $f(x) = \sum \xi_i x_i$ for $x\in R^n$.

Similarly, given $f\in \hom_R(P,M)$ we can set $\xi_i = f(e e_i)$ as $P=eR^n$.  We now identify which
families $(\xi_i)$ can occur; perhaps there is an easy way to see this, but I will resort to calculation.
Let $e$ have matrix $(x_{ij})$ so $ee_i = \sum_j e_{ji} e_j \in R^n$.  Given some $(\xi_i)\in M^n$ the map
$ex \mapsto \sum \xi_i x_i$ is well-defined exactly when $ex=0 \implies \sum_i \xi_i x_i=0$.  If we consider
$\xi = (\xi_i)$ as a row vector, and $x=(x_i)$ as a column vector, we may write $\xi x = 0$.  That $ex=0$ is
equivalent to $(1-e)x=x$.  So our condition is that for all $y$, with $x = (1-e)y$, we need that
$0 = \xi x = \xi(1-e)y = \xi y - \xi e y$, where now $e\in M_n(R)$ acts on the right of the row vector $\xi$.
That is, $\xi y =\xi e y$ for all $y$, so $\xi = \xi e$.

Thus $P$ is a generator if and only if for every $M$ there is $\xi\in M^n$ with $\xi e = \xi$; equivalently,
$M^n e \not= 0$.  Notice that we have turned $M^n$ into a right $M_n(R)$ module in the obvious way, and so
our condition is equivalent to $e\not\in\ann(M^n) \trianglelefteq M_n(R)$.  As any ideal in $M_n(R)$ arises 
as $ann(M^n)$ for some $M$, the result follows.
\end{proof}

Let $e_{ij}\in M_n(R)$ be the matrix with $1$ in the $(i,j)$th place.  The diagonal embedding $R\rightarrow M_n(R)$ 
is a ring homomorphism which turns $M_n(R)$ into a left $R$-module, with the action of $x\in R$ just left 
multiplication on every matrix element.  Thus a general $x\in M_n(R)$ can be written uniquely as
\[ x = \sum_{i,j=1}^n x_{ij} e_{ij} \]
for some $(x_{ij})$ in $R$.

Let $I\trianglelefteq M_n(R)$ be some non-zero ideal.  As $1 = \sum_i e_{ii}$ there is $x\in I$ and some
$i$ with $e_{ii} x \not = 0$.  Similarly, we can then find $j$ with $e_{ii} x e_{jj} \not = 0$.  So there is
$y\in R$ with $y e_{ij} \in I$.  By multiplying by suitable $e_{kl}$ on the left and right, we see that
$y e_{ij} \in I$ for all $i,j$.  Let $J\subseteq R$ be the collection of such $y$.  Clearly $J$ is an ideal,
and so $M_n(J) \subseteq I$.  However, our argument shows that if $x = (x_{ij}) \in I$ then each $x_{ij} \in J$.
Thus $I = M_n(J)$.  If $J \trianglelefteq R$ is an ideal, then $M_n(J)$ is an ideal, and so we have completely
classified the ideals of $M_n(R)$.

We can finally prove the following; the argument leading here is inspired by \cite{qy2}.

\begin{proposition}\label{prop:1}
Let $R$ be a simple ring.  Any finitely-generated projective module $P\in\mod R$ is a generator for $\mod R$.
\end{proposition}
\begin{proof}
As $P$ is finitely-generated there is a surjective morphism $R^n \rightarrow P$ for some $n$, and as $P$ is
projective, there is $P'\in\mod R$ with $R^n \cong P \oplus P'$.  Let $e\in M_n(R)$ be the resulting idempotent
with $eR^n \cong P$.  By the previous proposition, $eR^n$ is a generator if and only $e$ is not contained in
a proper ideal of $M_n(R)$.  As $R$ is simple, the preceding discussion shows that $M_n(R)$ is simple, and
so the result follows.
\end{proof}


\section{Purely infinite rings}

We continue to follow \cite{agp}.

\begin{proposition}[{\cite[Lemma~1.4]{agp}}]\label{prop:2}
Let $R$ be a simple ring, and let $P$ and $Q$ be finitely generated projective modules in $\mod R$.
If $P$ is directly infinite, then there is a non-zero $A\in\mod R$ with $P \cong Q \oplus A$.
\end{proposition}
\begin{proof}
As $P$ is directly infinite there is $0\not=B\in\mod R$ with $P \cong P \oplus B$.  Thus also $P \cong
P \oplus B \cong P\oplus B\oplus B$; by induction, $P\cong P\oplus B^n$ for any $n\geq 1$.  As $B$ is
the direct summand of a projective, it is itself projective.  There is some $m$ and a surjective morphism
$R^m \rightarrow P \cong P\oplus B$, and so also there is a surjective morphism $R^m\rightarrow B$, so
$B$ is finitely generated.  By Proposition~\ref{prop:1}, $B$ is a generator, and so Lemma~\ref{lem:2}
gives $n\geq 1$ and a surjection $B^n \rightarrow Q$.  As $Q$ is projective there is a splitting and we
can find $C\in\mod R$ with $B^n \cong Q \oplus C$.  Thus
\[ P \cong P \oplus B^n \cong P \oplus Q \oplus C \cong Q \oplus (P\oplus C), \]
and $P\oplus C\not=0$ as $P\not=0$.
\end{proof}

\begin{proposition}[{\cite[Proposition~1.5]{agp}}]\label{prop:3}
Let $R$ be a purely infinite, simple ring.  All non-zero finitely generated projective right $R$-modules
are directly infinite; equivalently, all non-zero idempotents in $M_n(R)$ are infinite.

If $P,Q$ are non-zero finitely generated projective right $R$-modules, there is $A\in\mod R$ with
$P\cong Q\oplus A$.
\end{proposition}
\begin{proof}
As $R$ is purely infinite, there is some infinite idempotent $e\in R$.  Set $P = eR$ so $P$ is a finitely
generated, projective, directly infinite right $R$-module.  Let $Q$ be a non-zero finitely generated projective right $R$-module, so by Proposition~\ref{prop:2}, there is $A\in\mod R$ with $P\cong Q\oplus A$.  As $P=eR$
we may identify $Q$ with a non-zero right ideal $I$ in $R$.  Thus $I$ contains an infinite
idempotent $f$, and so $fR \subseteq I$ is a directly infinite right $R$-module which is a direct summand
of $Q$.  The claim now follows from Lemma~\ref{lem:3}.

The equivalence with all non-zero idempotents in $M_n(R)$ being infinite follows as if $e\in M_n(R)$ is
an idempotent, then $eR^n$ is a finitely generated projective right $R$-module, and so directly infinite,
which is equivalent to $e$ being infinite.  The converse is similar.

The final claim is immediate from Proposition~\ref{prop:2}.
\end{proof}

We now come to our main theorem.

\begin{theorem}\label{thm:1}
Let $R$ be a simple ring.  Then $R$ is purely infinite if and only if:
\begin{enumerate}
\item $R$ is not a division ring; and
\item for every $0\not=a\in R$ there are $b,c\in R$ with $bac=1$.
\end{enumerate}
\end{theorem}
\begin{proof}
Assume $R$ is purely infinite.  Then $R$ contains an infinite idempotent, and so $R$ is not a division ring.
Given $0\not=a\in R$, the non-zero right ideal $aR$ contains an infinite idempotent $e$.  By the final claim of
Proposition~\ref{prop:3} there is $A\in\mod R$ with $eR \cong R \oplus A$.  By Lemma~\ref{lem:4} we can find
orthogonal idempotents $f,g$ with $e=f+g$ and with $fR \cong R$, that is, $f\sim 1$, by Lemma~\ref{lem:1}.
Thus there are $x\in fR, y\in Rf$ with $xy=f, yx=1$.  As $f\in eR$ (compare the proof of Lemma~\ref{lem:5})
and as $e\in aR$, we have $f\in aR$ so $f=ar$ for some $r\in R$.  Hence 
\[ 1 = yxyx = yfx = (y)a(rx), \]
as required.

Conversely, suppose the two conditions hold.  Let $I$ be a non-zero right ideal in $R$.
Either $I\not=R$ in which case set $J=I$, or, as $R$ is not a division ring, there is a proper
right ideal $J$ in $R$.
[\footnote{This seems to be well-known.  Such an ideal exists, for if not, we would have that
$xR=R$ for all $x\in R$, and so for each $x$ there is $y$ with $xy=1$.  If $e\in R$ is an idempotent
then there is $z$ with $ez=1$, so $1 = ez = eez = e1 = e$.  As $xy=1$ we have that $yx$ is an idempotent,
so $yx=1$.  Thus $R$ is a division ring, contradiction.}]
Let $a\in J$ be non-zero, so there are $b,c\in R$ with $bac=1$.  Then $e=acb$ is an idempotent in $aR\subseteq J$,
so $e\not=1$ and $e\in I$.
As $(eac)(be) = e$ and $(be)(eac) = 1$ we have that $e\sim 1$.  Thus $1$ is infinite, but then
as $e$ is equivalent to $1$, also $e$ is infinite, by Lemma~\ref{lem:5}.  Hence $R$ is purely infinite.
\end{proof}


\begin{thebibliography}{99}

\bibitem{agp} P. Ara, K. R. Goodearl\ and\ E. Pardo, $K_0$ of purely infinite simple regular rings, $K$-Theory {\bf 26} (2002), no.~1, 69--100. MR1918211

\bibitem{lang} S. Lang, {\it Algebra}, revised third edition, Graduate Texts in Mathematics, 211, Springer-Verlag, New York, 2002. MR1878556

\bibitem{qy1} Q. Yuan, Generators, blog post available at \url{https://qchu.wordpress.com/2015/05/17/generators/}

\bibitem{qy2} Q. Yuan, When do projective modules give generators?, URL (version: 2020-10-28): \url{https://math.stackexchange.com/q/3884117}

\end{thebibliography}

\end{document}









%%%%%%%%%%% Mortia Equivalence stuff

For $n\geq 1$ consider $R^n$ as a right $R$-module.  If $R^n = X\oplus Y$ for some submodules $X,Y$ then
there is an idempotent $e\in \hom_R(R^n)$ with $X = e(R^n)$.  Now, $\hom_R(R^n)$ may be identified
with $M_n(R)$ \emph{acting on the left} of $R^n$, and this gives us motivation to study $R^n$ as a left
module of $M_n(R)$.
There is a Morita equivalence between $R$ and $M_n(R)$; let us sketch elements of this.  Given $M$ a left $R$-module, naturally $M^n$ is a left $M_n(R)$ module considering $M^n$ as column vectors, with $M_n(R)$ acting on the left by multiplication.

Let $e_{ij}\in M_n(R)$ be the matrix with $1$ in the $(i,j)$th place.  The diagonal embedding $R\rightarrow M_n(R)$ 
is a ring homomorphism which turns $M_n(R)$ into a left $R$-module, with the action of $x\in R$ just left 
multiplication on every matrix element.  Thus a general $x\in M_n(R)$ can be written uniquely as
\[ x = \sum_{i,j=1}^n x_{ij} e_{ij} \]
for some $(x_{ij})$ in $R$.  Notice that $e_{11} \sim e_{ii}$ as $e_{11} = e_{1i} e_{i1}$ and $e_{ii} = e_{i1}
e_{1i}$.  Let $N\in\mod M_n(R)$.  As in Lemma~\ref{lem:1}, we find that $e_{11} N \sim e_{ii} N$ for the mutual
inverses $\theta:e_{11} N \rightarrow e_{ii} N; x \mapsto e_{i1} x$ and 
$\phi:e_{ii}N \rightarrow e_{11}N; x \mapsto e_{1i} x$.  

Let $N\in\mod M_n(R)$, and set $X = e_{11} N$ considered as an $R$-module, by restriction of the action.
As $1 = \sum_{i=1}^n e_{ii}$ in $M_n(R)$, we see that $N\cong \sum_{i=1}^n e_{ii} N$, and so
the above discussion shows that
\[ \Phi:N\rightarrow X^n; \quad \xi \mapsto \big( e_{1i} e_{ii} \xi \big)_{i=1}^n
= \big( e_{1i} \xi \big)_{i=1}^n, \]
is an isomorphism on $R$-modules.  We claim that $\Phi$ is also an $M_n(R)$-homomorphism.  Given
$x = \sum x_{ij} e_{ij} \in M_n(R)$, we have that
\[ \Phi(x\xi) = \Big( e_{1i} \sum_{j,k} x_{jk} e_{jk} \xi \Big)_{i=1}^n
= \Big( \sum_k x_{ik} e_{1k} \xi \Big)_{i=1}^n
= x \Phi(\xi), \]
given what the action of $M_n(R)$ on $R^n$ is.  Hence every $M_n(R)$ module arises as $X^n$ for some $X\in\mod R$.

Similar arguments establish that every $M_n(R)$-module homomorphism $M^n\rightarrow N^n$ arises from a (unique) $R$-module homomorphism $M\rightarrow N$ which acts coordinate-wise on $M^n$.  Thus we have an equivalence of categories.  Exactly the same argument works for right modules.\footnote{We actually use the right module
version in the end.  But had typed this out for right modules, then re-worked it for left modules, and now
cannot be bothered to type out again ``on the right''.}
