\documentclass[a4paper,11pt]{article}
\usepackage[utf8]{inputenc}
\usepackage[margin=2cm]{geometry}
\usepackage{latexsym, amsmath, amsthm, amssymb}
\usepackage[all]{xy}

% Drafting
%\usepackage{showkeys, backref}

\newcommand{\mc}[1]{{\mathcal{#1}}}
\newcommand{\mf}{\mathfrak}
\newcommand{\Rep}{\operatorname{Rep}}
\newcommand{\Cl}{\operatorname{Cl}}
\newcommand{\Eq}{\operatorname{Eq}}
\newcommand{\Sp}{\operatorname{Sp}}
\newcommand{\op}{{\operatorname{op}}}
\newcommand{\bin}{{\operatorname{bin}}}
\newcommand{\asy}{{\operatorname{asy}}}
\newcommand{\indlim}{\varinjlim}
\newcommand{\ba}{\textsf{BA}}

\newtheorem{lemma}{Lemma}[section]
\newtheorem{proposition}[lemma]{Proposition}
\newtheorem{theorem}[lemma]{Theorem}
\newtheorem{corollary}[lemma]{Corollary}
\theoremstyle{definition}
\newtheorem{definition}[lemma]{Definition}
\newtheorem{example}[lemma]{Example}


\title{Inductive limits of Banach algebras}
\author{Matt Daws}
\date{Last knowingly modified 7 June 2020}

% 234567890123456789012345678901234567890123456789012345678901234567890123456789012345678901234567890

\begin{document}

\maketitle

\begin{abstract}
We provide an overview of inductive limits of Banach algebras, correcting a few misleading
statements in the literature.
\end{abstract}

\tableofcontents

\section{Introduction}

The inductive (or direct) limit of Banach algebras have not been widely studied, perhaps reflecting
a lack of applications in the literature.  Inductive limits are widely studied for locally convex
spaces -- often when the space appearing in the inductive system are Banach spaces, but the limit is
not in general a Banach space, for example, \cite[Chapter~IV, Section~5]{conway}.  They have also
been a useful tool in the theory of $C^*$-algebras, see references below.  For Banach algebras,
there are a number of misleading statements in standard textbooks, which we wish to correct.  We do
not pretend to make a literature review, instead giving such references as we are aware of.

Let $C$ be a general category, let $I$ be a directed set (often $\mathbb N$ with the usual
ordering), for each $i\in I$ let $A_i$ be an object in $C$, and for $i\leq j$ let $\varphi_{j,i}:
A_i\rightarrow A_j$ be morphism.  We require that for $i\leq j\leq k$ that $\varphi_{k,j} \circ
\varphi_{j,i} = \varphi_{k,i}$.  We warn the reader that some authors write $\varphi_{i,j}$
instead.  This is an \emph{inductive system}, written as $((A_i), (\varphi_{j,i}))$ or sometimes
just $(A_i)$, the latter being an abuse of notation, as the ``connecting morphisms'' $\varphi_{j,i}$
are vitally important.

The \emph{inductive limit} of the system, if it exists, is an object $A$ and morphisms $\phi_i:
A_i\rightarrow A$ such that:
\begin{itemize}
\item For each $i\leq j$ the following diagram commutes:
\[ \xymatrix{ A_i \ar[rr]^-{\varphi_{j,i}} \ar[rd]_{\phi_i} && A_j \ar[ld]^{\phi_j} \\
& A
} \]
\item Let $B$ is an object and suppose we have morphisms $\psi_i:A_i\rightarrow B$ with, as above,
$\psi_i = \psi_j \circ \varphi_{j,i}$ for all $i\leq j$.  We call $(B,(\psi_i))$ a \emph{coherent
system}.  Then there is a unique morphism
$\phi:A\rightarrow B$ making the following diagram commutes:
\[ \xymatrix{ & A_i \ar[ld]_{\phi_i} \ar[rd]^{\psi_i} \\
A \ar[rr]_{\phi} && B
} \]
\end{itemize}
A standard argument, using the ``unique'' clause, shows that if the limit exists, it is unique,
up to isomorphism in the category.  We write $\indlim A_i$ for the inductive limit (again, with an
abuse of notation, as we have suppressed the connecting morphisms).

This presentation follows \cite[Section~6.2]{rll}.  In category theory, the construction can be
much extended to the notion of a \emph{colimit}, see \cite[Chapter~5]{leinster} for example.  For
a combined category theory and functional analytic take, see \cite{cas}.
A very readable introduction, which ultimately focuses on $C^*$-algebras, is \cite[Appendix~L]{wo}.
For $C^*$-algebras, see also \cite[Section~II.8.2]{blenc}.



\section{Banach algebras}

Having now set the scene using this language of categories, we will now perform a u-turn and present
the standard construction, which unfortunately does not have a simple interpretation in terms of
category theory.  Below we return to the more general setting.  We follow
\cite[Definition~1.3.4]{palmer} and \cite[Section~3.3]{blackadar}, both of which make slightly
misleading claims which we note below.  We are not aware of any other standard textbooks covering
this material.\footnote{But we have not even tried to make a literature review.}

For the remainder of this section, $((A_i), (\varphi_{j,i}))$ shall be an inductive system of Banach
algebras.  Thus each $A_i$ is a Banach algebra (to be sure, this means that $A_i$ has a contractive
product) and each $\varphi_{j,i}$ is a bounded algebra homomorphism.  Essentially identical
arguments will give analogous results for Banach space, Banach modules over Banach algebras, and so
forth.

\begin{definition}\label{defn:1}
Suppose that for each $i\in I$
\[ \limsup_{j\geq i} \|\varphi_{j,i}\| = \lim_{j\geq i} \sup_{k\geq j} \|\varphi_{k,i}\|
=
\inf_{j\geq i} \sup_{k\geq j} \|\varphi_{k,i}\| \]
is finite.  Then we say that our system is a \emph{normed inductive system} of Banach algebras.
\end{definition}

Given a normed inductive system, the classical construction of $\indlim A_i$ is as follows.
Consider the disjoint union of the $A_i$ with the equivalence relation that, for $a_i\in A_i,
a_j\in A_j$, we have $a_i \sim a_j$ if and only if there is $k\geq i, k\geq j$ with
$\varphi_{k,i}(a_i) = \varphi_{k,j}(a_j)$.  A slightly tedious check shows that the quotient becomes
a vector space.  Given an equivalence class $[a]$ with representative $a\in A_i$, it is easy to see
that $\limsup_{j\geq i} \|\varphi_{j,i}(a)\|$ defines a seminorm which is submultiplicative.
The quotient by the null ideal of this seminorm gives a normed algebra, and the completion is
$\indlim A_i$.  We define $\phi_i:A_i \rightarrow \indlim A_i$ by setting $\phi_i(a)$ to
be the equivalence class of $a$.

An alternative construction is the following.  Let $\ell^\infty(A_i)$ be the Banach algebra of all
bounded families $(a_i)$, where $a_i\in A_i$ for each $i$, with pointwise operations.
Let $c_0(A_i)$ be the collection of families $(a_i)\in \ell^\infty(A)$ such that, for each
$\epsilon>0$, there is $i_0\in I$ so that $\|a_i\|<\epsilon$ for $i\geq i_0$.  Then $c_0(A_i)$
is a closed ideal in $\ell^\infty(A_i)$.  Denote by $\mf A$ the quotient algebra
$\ell^\infty(A_i) / c_0(A_i)$.  We shall abuse notation, and write $(a_i)\in\ell^\infty(A_i)$
for the equivalence class of $\mf A$ which it defines.

Given $i\in I$, by assumption, there is $j\geq i$ so that
$K = \sup_{k\geq j} \|\varphi_{k,i}\| < \infty$.  Then, for $a\in A_i$, we define
\[ a_k = \begin{cases} \varphi_{k,i}(a) & : k\geq j, \\ 0 & : \text{otherwise}. \end{cases} \]
Thus $\|a_k\| \leq K\|a\|$ for each $k$, and so $(a_k) \in \mf A$.  Suppose we chose $j'\in I$
instead, also with $\sup_{k\geq j} \|\varphi_{k,i}\| < \infty$, and form $(a_k')$ using $j'$.
There is $k_0\in I$ with $k_0 \geq j$ and $k_0 \geq j'$, so for $k\geq k_0$ we have that
$a_k = a_k'$, and so $(a_k) = (a_k')$ in $\mf A$.  It follows that we have a well-defined map
$\phi_i:A_i\rightarrow\mf A$ which is a homomorphism, and is seen to satisfy that $\|\phi_i\| \leq
\limsup_{j\geq i} \|\varphi_{j,i}\|$.  We introduce the notation that $\phi_i(a) =
( \varphi_{k,i}(a) )_{k\geq j}$ where $j\geq i$ is any suitable choice.

Given $j\geq i$ there is $k\geq j$ so that both $\sup_{l\geq k} \|\varphi_{l,i}\|<\infty$ and
$\sup_{l\geq k} \|\varphi_{l,j}\|<\infty$.  For $a\in A_i$, it follows that $\phi_i(a) =
( \varphi_{l,i}(a) )_{l\geq k}$ and that $\phi_j(\varphi_{j,i}(a)) =
( \varphi_{l,j}(\varphi_{j,i}(a)) )_{l\geq k} = 
( \varphi_{l,i}(a) )_{l\geq k}$.  Hence $\phi_i = \phi_j \circ \varphi_{j,i}$.
Set $\indlim A_i$ to be the closure of $\{\phi_i(a) : i\in I, a\in A_i \}$.  For $i\leq j$, as
$\phi_i = \phi_j \circ \varphi_{j,i}$, it follows that $\phi_i(A_i) \subseteq \phi_j(A_j)$.
Thus $\indlim A_i$ is the closure of an increasing union of algebras, and hence is a Banach algebra.

Here is an alternative description of a dense subalgebra of $\indlim A_i$.
Define $\mf A_0$ to be the equivalence classes in $\mf A$ with representatives $(a_i)$ such that,
for some $i_0$, we have that $\varphi_{k,j}(a_j) = a_k$ for $i_0 \leq j \leq k$.  We claim that
$\mf A_0$ is equal to the union of the images of the $\phi_i$, and hence is dense in
$\indlim A_i$.  If $a\in A_i$ and $j\geq i$ is suitable, then $\phi_i(a) = ( \varphi_{k,i}(a)
)_{k\geq j}$, and so with $i_0=j$, if $i_0 \leq k \leq l$ then $\varphi_{l,k}( \varphi_{k,i}(a) )
= \varphi_{l,i}(a)$ and so $\phi_i(a)\in \mf A_0$.  Conversely, given $(a_i)$ with $i_0$ such that
$\varphi_{k,j}(a_j) = a_k$ for $i_0 \leq j \leq k$, let $a = a_{i_0} \in A_{i_0}$, so that
$\varphi_{k,i_0}(a) = a_k$ for $k\geq i_0$, and hence $(a_i) = \phi_{i_0}(a)$ in $\mf A$.

Let us compare this construction with the ``classical construction'' sketched above.  Clearly we
define the same equivalence relation on the disjoint union of the $A_i$.  Given $a\in A_i$ we have
that $\|\phi_i(a)\| = \|(\varphi_{k,i}(a))_{k\geq j}\|_{\mf A}$ for a suitable $j\geq i$, and by
the definition of $\mf A$, this is equal to $\limsup_{k\geq j} \|\varphi_{k,i}(a)\|$ which is
equal to $\limsup_{k\geq i} \|\varphi_{k,i}(a)\|$.  Thus our construction agrees with the
classical one.

We now turn to verifying the universal property.  In fact, this is not considered in
\cite{blackadar}, while we believe that \cite{palmer} is wrong to say that the universal property
always holds.    Suppose that $B$ is a Banach algebra with bounded homomorphisms $\psi_i:A_i
\rightarrow B$ such that $\psi_i = \psi_j \circ \varphi_{j,i}$ for each $i\leq j$.  For $a\in A_i$
we must define $\phi(\phi_i(a)) = \psi_i(a)$, which motivates the following definition.  Given
$(a_i) \in \mf A_0$ with $\varphi_{k,j}(a_j) = a_k$ for $i_0 \leq j\leq k$, define
$\phi((a_i)) = \psi_j(a_j)$ for any $j\geq i_0$.  This is well-defined, for $\psi_j(a_j) =
\psi_j(\varphi_{j,i_0}(a_{i_0})) = \psi_{i_0}(a_{i_0})$, and so the definition does not depend on
$j$, and hence also does not depend on the choice of $i_0$.  To ensure that $\phi$ is bounded,
there must exist a constant $K$ with
\[ \|\psi_i(a)\| \leq K \|\phi_i(a)\| = K \limsup_{j\geq i} \|\varphi_{j,i}(a)\|
\qquad (i\in I, a\in A_i). \]
When such a $K$ exists, $\phi$ extends by continuity to $\indlim A_i$, with $\|\phi\|\leq K$.

We summarise the construction as follows.

\begin{theorem}\label{thm:1}
Let $(A_i)_{i\in I}$ be a family of Banach algebras indexed by a direct set $I$.  For each
$i\leq j$ suppose there is a bounded homomorphism $\varphi_{j,i}:A_i\rightarrow A_j$ with
$\varphi_{k,j} \circ \varphi_{j,i} = \varphi_{k,i}$ for $i\leq j\leq k$, and such that
$\limsup_{j\geq i} \|\varphi_{j,i}\| < \infty$ for each $i\in I$.  

There exists a Banach algebra $A=\indlim A_i$ and bounded homomorphisms $\phi_i:A_i\rightarrow
A$ with $\|\phi_i(a)\| = \limsup_{j\geq i} \|\varphi_{j,i}(a)\|$ for $a\in A_i$, and such that
$\phi_j \circ \varphi_{j,i} = \phi_i$ for $i\leq j$,
and with the property that if $B$ is a Banach algebra and
$\psi_i:A_i\rightarrow B$ are homomorphisms with $\psi_j \circ \varphi_{j,i} = \psi_i$ for
$i\leq j$, and
$\|\psi_i(a)\| \leq \limsup_{j\geq i} \|\varphi_{j,i}(a)\|$ for $i\in I, a\in A_i$, then there is
a unique contractive homomorphism $\phi:A\rightarrow B$ with $\phi\circ\phi_i = \psi_i$.
\end{theorem}

In the statement of the theorem, we could weaken the norm condition on $\phi_i$ to
$\|\phi_i(a)\| \leq \limsup_{j\geq i} \|\varphi_{j,i}(a)\|$ for $a\in A_i$.  Then the universal
property would imply that actually we must have equality.  There is a certain appeal to stating the
condition in this weaker way, as it would make the conditions on $A$ and $B$ become similar: see
Theorem~\ref{thm:2} below for a variant where the conditions become fully symmetric.

A standard diagram chase shows that all such $A$ satisfying the conclusions of the theorem are
isometrically isomorphic.  Furthermore, the ``uniqueness' clause in the existence of $\phi$ can be
used to show that the union of the images of the $A_i$ must be dense in $A$.

For completeness, we now give the details of the claims in the previous two paragraphs.

\begin{proof}
Firstly, suppose we have a Banach algebra $A_0$ and a coherent family of homomorphisms $\phi_i^0:A_i
\rightarrow A_0$ such that $\|\phi_i^0(a)\| \leq \limsup_{j\geq i} \|\varphi_{j,i}(a)\|$ for each
$i$ and $a\in A_i$, and suppose further that $A_0$ satisfies the universal property.  In particular,
we can apply the universal property to $B=\indlim A_i$ to find that there is a (unique) contractive
homomorphism $\phi:A_0 \rightarrow \indlim A_i$ with $\phi\circ\phi_i^0 = \phi_i$ for each $i$.
Then
\[ \limsup_{j\geq i} \|\varphi_{j,i}(a)\| = \|\phi_i(a)\| = \|\phi(\phi_i^0(a))\|
\leq \|\phi_i^0(a)\| \leq \limsup_{j\geq i} \|\varphi_{j,i}(a)\| \]
and so we have equality throughout, as claimed.

Let $A$ satisfying the stated universal property, and now let $B$ be the closure of the union of
the images of the $\phi_i$.  We aim to show that $B=A$.  We can apply the universal property with
$\psi_i:A_i\rightarrow B$ being the corestriction of $\phi_i$.  Thus there is a unique contractive
homomorphism $\phi:A\rightarrow B$ with $\phi\circ\phi_i = \psi_i$, that is, $\phi$ is the identity
on $B$, and is hence a projection.  We now apply the universal property to $A$ itself, so there
is a unique contractive homomorphism $A\rightarrow A$ intertwining the maps $\phi_i$.  However,
both the identity homomorphism and $\phi$ satisfy this, so $\phi$ is the identity, and $A=B$ as
claimed.

Finally, if $A$ satisfies the stated properties, then applying the universal property to
$B = \indlim A_i$ shows that there is a contractive homomorphism $A\rightarrow\indlim A_i$
intertwining the coherent homomorphisms.  Reversing the roles of $A$ and $\indlim A_i$ gives
a homomorphism $\indlim A_i \rightarrow A$, and by the density of the images of the $\phi_i$,
these two homomorphisms must be mutual inverses.  As they are contractive, they must actually
be isometric, as claimed.
\end{proof}

It is too much to expect that the ``universal property'' holds without norm control, as the next
example shows.

\begin{example}\label{ex:one}
Let $A$ be a Banach algebra, and let our directed set be $\mathbb N$ with the usual ordering.
Let $A_n = A$ for each $n\in\mathbb N$, and define $\varphi_{n+1,n}$ to be $1/2$ of the identity,
which we denote by $\varphi_{n+1,n} = \frac12$.  Then $\varphi_{n+k,n} = 2^{-k}$ for $k\geq 1$.
This defines a normed inductive system of Banach algebras, as for each $n$ we have that
$\limsup_{m\geq n} \|\varphi_{m,n}\| = \limsup_{k\geq 1} 2^{-k} = 0$.  Indeed, it is now easy to
check that $\indlim A_n = \{0\}$.

Now set $B=A$ and define $\psi_n = 2^{n-1}$ as a map $A_n\rightarrow B$.  For $n\leq m$, we see that
$\psi_m \circ \varphi_{m,n} = 2^{m-1} 2^{n-m} = \psi_n$.  So while $(B, (\psi_n))$ is a system,
there can be no map $\phi:\indlim A_n\rightarrow B$ with $\phi\circ\phi_n = \psi_n$.  Thus, without
some norm condition on the family $(\psi_n)$, we cannot hope to factor through $\indlim A_n$.
\end{example}



\subsection{Weaker conditions}

Rather than requiring that $\limsup_{j\geq i} \|\varphi_{j,i}\| < \infty$ for each $i$, we could
instead ask that $\limsup_{j\geq i} \|\varphi_{j,i}(a)\| < \infty$ for each $i$ and $a\in A_i$.
Let us follow through the above construction.  Given $i\in I$ and
$a\in A_i$, there is $j\geq i$ so that $K=\sup_{k\geq j} \|\varphi_{j,i}(a)\| < \infty$.  Then
again define
\[ a_k = \begin{cases} \varphi_{k,i}(a) &: k\geq j, \\ 0 &:\text{otherwise}. \end{cases} \]
Thus $\|a_k\| \leq K$ for each all $k$ and so $(a_k)\in\mf A$.  Again, the equivalence class which
$(a_k)$ defines is independent of the choice of $j$.  Hence we obtain a well-defined homomorphism
$\phi_i:A_i\rightarrow\mf A$, with
\[ \|\phi_i(a)\| = \limsup_{j\geq i} \|\varphi_{j,i}(a)\| \qquad (a\in A_i). \]
However, there seems to be no reason why $\phi_i$ need be bounded, in general, and
Example~\ref{ex:2} below shows that $\phi_i$ can indeed fail to be bounded.

To obtain a bounded $\phi_i$ we need to impose the condition that for each $i$ there is $K_i$ with
$\limsup_{j\geq i} \|\varphi_{j,i}(a)\| \leq K_i\|a\|$ for each $a\in A_i$.  Then $\|\phi_i\| \leq
K_i$.  Given this, the remainder of the previous construction still works, and we may again define
$\indlim A_i$ as a closed subalgebra of $\ell^\infty(A_i) / c_0(A_i)$.  The same universal property
will hold.  We state this as a theorem.

\begin{theorem}\label{thm:2}
Let $(A_i)_{i\in I}$ be a family of Banach algebras indexed by a direct set $I$.  For each
$i\leq j$ suppose there is a bounded homomorphism $\varphi_{j,i}:A_i\rightarrow A_j$ with
$\varphi_{k,j} \circ \varphi_{j,i} = \varphi_{k,i}$ for $i\leq j\leq k$.  Suppose further that for
each $i\in I$ there is $K_i$ with $\limsup_{j\geq i} \|\varphi_{j,i}(a)\| \leq K_i \|a\|$ for each
$a\in A_i$.

There exists a Banach algebra $A=\indlim A_i$ and bounded homomorphisms $\phi_i:A_i\rightarrow
A$ with $\|\phi_i(a)\| \leq \limsup_{j\geq i} \|\varphi_{j,i}(a)\|$ for $a\in A_i$, and such that
$\phi_j \circ \varphi_{j,i} = \phi_i$ for $i\leq j$,
and with the property that if $B$ is a Banach algebra and
$\psi_i:A_i\rightarrow B$ are bounded homomorphisms with $\psi_j \circ \varphi_{j,i} = \psi_i$ for
$i\leq j$, and
$\|\psi_i(a)\| \leq \limsup_{j\geq i} \|\varphi_{j,i}(a)\|$ for $i\in I, a\in A_i$, then there is
a unique contractive homomorphism $\phi:A\rightarrow B$ with $\phi\circ\phi_i = \psi_i$.
\end{theorem}

Again, the universal property implies that we actually have equality: $\|\phi_i(a)\| =
\limsup_{j\geq i} \|\varphi_{j,i}(a)\|$ for $a\in A_i$ and $i\in I$.

It is stated in \cite{blackadar, palmer} that this condition implies that  $\limsup_{j\geq i}
\|\varphi_{j,i}\| < \infty$ for each $i$, by the Uniform Boundedness Principle.  When $I=\mathbb N$
with the usual ordering, this is correct, as here $\limsup_{j\geq i} \|\varphi_{j,i}\| < \infty$
is equivalent to $\sup_{j\geq i} \|\varphi_{j,i}\| < \infty$, and similarly for the case of
$\varphi_{j,i}(a)$.  For more general directed sets $I$ this is no longer true, as the following
counter-example shows.

\begin{example}\label{ex:1}
The following studies an inductive system of Banach spaces, but these can be turned into Banach
algebras by equipping each space with the zero product.  Let $I$ be the collection of pairs
$(F,n)$ where $F\subseteq\ell^1$ is finite, and $n\in\mathbb N$, turned into a directed set for the
partial order $(F,n) \leq (F',n')$ where $F\subseteq F'$ and $n\leq n'$.  Let $0\in I$ be an extra
element with $0\leq i$ for all $i\in I$.  For each $i\in I$ set $A_i = \ell^1$.

Let $\ell^1$ have standard unit vector basis $(e_t)$.  For $a,n\in\mathbb N$ define $T_{a,n}:
\ell^1\rightarrow\ell^1$ by $T_{a,n}(e_a) = n e_a$ and $T_{a,n}(e_t) = e_t$ for $t\not=a$, extending
$T_{a,n}$ to $\ell^1$ by linearity and continuity.  Then $\|T_{a,n}\| = n$.

For $i=(F,n)\in I$ define $\varphi_{i,0} = T_{a,n}$ where $a$ satisfies that
\[ \sum_{t\in\mathbb N} |x_t| + (n-1)|x_a| < \|x\|+n^{-1} \qquad (x=(x_n) \in F). \]
As $F\subseteq \ell^1$ is finite, such a $a$ does exist.  Indeed, the inequality is equivalent to
$(n-1)|x_a| < n^{-1}$, that is, $|x_a| < 1/(n(n-1))$, for each $x\in F$.
Then $\|\varphi_{i,0}(x)\| = \|T_{a,n}(x)\| < \|x\|+n^{-1}$ for any $x\in F$.   Due to the
definition of the partial order, if $j\geq i$ then also $\|\varphi_{j,0}(x)\| < \|x\|+n^{-1}$ for
any $x\in F$.

For each $i\in I$ define $\varphi_{i,i}$ to be the identity.  To obtain a system, we require that
for $k\geq j\not=0$ that
\[ \varphi_{k,j} = \varphi_{k,0} \circ \varphi_{j,0}^{-1}. \]
Notice that $\varphi_{j,0}^{-1}$ does exist, and is a contraction.  With this definition, we do
have an inductive system.

Finally, for $x\in\ell^1$ and $i\in I$, set $y=\varphi_{i,0}^{-1}(x)$.  There is some $j=(F,n)\in I$
with $y\in F$ and $j\geq i$.  Then, for $k\geq j$,
\[ \|\varphi_{k,i}(x)\| = \|\varphi_{k,0}(y)\| < \|y\|+n^{-1} \leq \|x\| + n^{-1}. \]
Thus $\limsup_{k\geq i} \|\varphi_{k,i}(x)\| \leq \limsup_{k\geq j} \|\varphi_{k,i}(x)\|
< \|x\| + n^{-1}$, and so the ``weaker'' condition holds.

However, given $i\in I$, choose $i\leq j$ with $j=(F,n)\in I$.  Let $k=(F',m)$ for some $m\geq n$
and $F'\supseteq F$, so that $k\geq j$.  Then $\varphi_{i,0} = T_{a,p}$ for some $a,p\in\mathbb N$.
By choosing $F'$ finite but sufficiently large, we can ensure that $\varphi_{k,0} = T_{b,m}$ where
$b$ cannot equal $a$.  Then
\[ \varphi_{k,i}(e_b) = \varphi_{k,0}( \varphi_{i,0}^{-1}(e_b) )
= T_{b,m}( T_{a,p}^{-1}(e_b) )
= T_{b,m}(e_b)
= m e_b, \]
and so $\|\varphi_{k,i}\|\geq m$.  Thus $\limsup_{j\geq i} \|\varphi_{j,i}\| =\infty$, and
so we do not have a normed inductive system.
\end{example}

We now present a further counter-example (again, we work just with Banach spaces, algebras been
obtained for the zero product) to show that it is possible to have that
$\limsup_{j\geq i} \|\varphi_{j,i}(a)\| < \infty$ for each $i$ and $a\in A_i$, while 
there is no constant $K_i$ with $\limsup_{j\geq i} \|\varphi_{j,i}(a)\| \leq K_i\|a\|$ for
$a\in A_i$.  Again, we would think that such an example would be impossible because of the
Principle of Uniform Boundedness, but using $\limsup$ and not $\sup$ changes things.

\begin{example}\label{ex:2}
Let $E$ be some infinite-dimensional Banach space, and pick an everywhere defined but unbounded
functional $\phi:E\rightarrow\mathbb{C}$.  We can find $\phi$ using the axiom of choice to pick a
Hamel basis for $E$, for example.  Let $I$ be the collection of finite-dimensional subspaces of $E$
partially ordered by inclusion.  Let $0\in I$ be another element with $0\leq F$ for each $F\in I$.

Given $F\in I$, so that $F$ is a finite-dimensional subspace of $E$, the functional $\phi$
restricted to $F$ is bounded, and so admits a Hahn-Banach extension to $E$, say $\mu_F\in E^*$.
Pick $x_F\in F$ with $\mu_F(x_F)=0$ and $\|x_F\|=1$ (here we need to suppose that $F$ has dimension
at least 2).  Let $T_F:E\rightarrow E$ be the operator defined by $T_F(x) = x - \mu_F(x) x_F$ for
$x\in E$.  Then $T_F$ is bounded, with inverse $T_F^{-1}(x) = x + \mu_F(x)x_F$.  Indeed,
\begin{align*}
T_F( x + \mu_F(x)x_F ) &= x + \mu_F(x)x_F - \mu_F(x)x_F - \mu_F(x)\mu_F(x_F) x_F = x, \\
T_F^{-1}( x - \mu_F(x) x_F ) &= x - \mu_F(x) x_F + \mu_F(x)x_F - \mu_F(x) \mu_F(x_F)x_F = x
\end{align*}

For $F\in I$ let $E_F=E$.
Define $\varphi_{F,0} = T_F$ and as before, given $0\not=F \leq G$ define $\varphi_{G,F}
= T_G T_F^{-1}$, and finally define $\varphi_{F,F}$ to be the identity.  Then
$((E_F),(\varphi_{G,F}))$ is an inductive system.

Fix $F\in I$ and $x\in E$.  For $G\geq F$,
\[ \varphi_{G,F}(x) = T_G( T_F^{-1}(x) ) = T_G( x + \mu_F(x)x_F )
= x + \mu_F(x)x_F - \mu_G(x)x_G - \mu_F(x)\mu_G(x_F)x_G. \]
As $F\leq G$ we see that $x_F\in G$ and so $\mu_G(x_F) = \phi(x_F) = \mu_F(x_F) = 0$.
If $G$ is sufficiently large so that $x\in G$ then $\mu_G(x) = \phi(x)$, and so
\[ \varphi_{G,F}(x) = x + \mu_F(x)x_F - \phi(x)x_G. \]
Thus
\[ \|\varphi_{G,F}(x)\| \leq \|x + \mu_F(x)x_F\| + |\phi(x)|, \qquad
\|\varphi_{G,F}(x)\| \geq |\phi(x)| - \|x + \mu_F(x)x_F\|. \]
Hence
\[ |\phi(x)| - \|x + \mu_F(x)x_F\| \leq
\limsup_{G\geq F} \|\varphi_{G,F}(x)\| \leq
\|x + \mu_F(x)x_F\| + |\phi(x)|. \]
In particular, $\limsup_{G\geq F} \|\varphi_{G,F}(x)\| < \infty$ for each $x\in E_F$.
However, we also see that there is no constant $K$ so that $\limsup_{G\geq F} \|\varphi_{G,F}(x)\|
\leq K \|x\|$ for every $x\in E_F$, because $\phi$ is unbounded.
\end{example}




\section{From a categorical perspective}

If the reader has encountered Banach spaces in a category theory setting before, they will know that
there are two main choices for which morphisms to consider:
\begin{itemize}
\item all bounded linear maps.  From an analytic perspective, this seems natural, but the resulting
category is somewhat difficult to manage;
\item just contractive linear maps.  This might seem slightly artificial, but it leads to a somewhat
better behaved category.
\end{itemize}

Let $\ba_1$ be the category of Banach algebras with contractive homomorphisms.  Given an inductive
system $((A_i), (\varphi_{j,i}))$ in $\ba_1$, it is immediate that this is a normed inductive
system, in the sense of Definition~\ref{defn:1}.  By Theorem~\ref{thm:1} we can form
$\indlim A_i$.  Now let $(B,(\psi_i))$ be any coherent system.
As each $\psi_j$ is a contraction, for $i\in I$ and $a\in A_i$, we have that
\[ \|\psi_i(a)\| = \|\psi_j(\varphi_{j,i}(a))\|
\leq \|\varphi_{j,i}(a)\| \]
for any $j\geq i$, and so $\|\psi_i(a)\| \leq \limsup_{j\geq i}\|\varphi_{j,i}(a)\|$.
Thus $(B,(\psi_i))$ satisfies the conditions in Theorem~\ref{thm:1}, and so we conclude that
$\indlim A_i$ is the inductive limit in $\ba_1$.  Recall that isomorphisms in $\ba_1$ are
\emph{isometric} isomorphisms.

Indeed, abstract category theoretic ideas work for $\ba_1$.  We recall that the inductive limit is
really an example of a \emph{colimit}, \cite[Chapter~5]{leinster}.  If we have all coproducts, and
coequalisers, then all colimits exists; although we remark that in this particular case, this does
not lead to a very efficient construction of $\indlim A_i$.  Quickly, we recall that coproducts in
$\ba_1$ are simply direct sums normed with the $\ell^1$ norm (which also works for Banach algebras).
Coequalisers are given as certain quotient maps.  Indeed, given Banach spaces $X,Y$ and contractions
$f,g:X\rightarrow Y$, let $Y_0$ be the closed linear span of $\{ f(x)-g(x) : x\in X \}$ in $Y$.
Then the quotient map $Y\rightarrow Y/Y_0$ is the coequaliser of $f,g$.  For Banach algebras, take
instead the closed ideal generated by $\{ f(x)-g(x) : x\in X \}$.

\subsection{For bounded linear maps}

We consider now $\ba$, the category of Banach algebra and bounded homomorphisms.  The situation here
seems much more complicated.

\begin{example}\label{ex:3}
We return to
Example~\ref{ex:one}, where $A_n=A$ for each $n\in\mathbb N$ and $\varphi_{n+1,n} = \frac12$.
Then $\indlim A_n = \{0\}$, but we noticed that we could define $B=A$ and $\psi_n =
2^{n-1}:A_n\rightarrow B$, and then $(B,(\psi_n))$ is a coherent system, but of course there is
no suitable homomorphism $\indlim A_n \rightarrow B$.  Hence $\indlim A_n$ is not an inductive
limit in $\ba$.

However, $B$ actually satisfies the universal property.  Indeed, let $(C,(\phi_n))$ be a coherent
system.  Then $\phi_1 = \phi_n \circ \varphi_{n,1} = 2^{1-n} \phi_n$ and so
$\phi_n = 2^{n-1} \phi_1$ for each $n$.  We can then define $\phi: B\rightarrow C$ by
$\phi = \phi_1$ (recalling that $B=A=A_n$).  Then $\phi\circ\psi_n = 2^{n-1}\phi_1
= \phi_n$, and so we have factored through the coherent system $(B,(\psi_n))$.  We conclude that
$B$ is the inductive limit in $\ba$.
\end{example}

\begin{example}
Consider the construction in Example~\ref{ex:1}.  We shall show that $\indlim A_i$ is the inductive
limit in $\ba$.

We first compute $\indlim A_i$, in the sense of Theorem~\ref{thm:2}.  We can take $K_i=1$ for each
$i$.  Furthermore, for $i\in I$, let $x\in\ell^1$ and set $y = \varphi_{i,0}^{-1}(x)$.  We showed
that there is $j\geq i$ so that for $k\geq j$ we have $\varphi_{k,i}(x) = \varphi_{k,0}(y)$ has
norm close to $\|y\| \leq \|x\|$.  Then $\phi_i(a) = (\varphi_{k,i}(x)) = (\varphi_{k,0}(y)) \in
\mf A$.  The construction of $\varphi_{k,0}$ shows that in fact $\| \varphi_{k,0}(y)-y\|$ is
arbitrarily small for large $k$.  Thus $\phi_i(a) = (y) = (\varphi_{i,0}^{-1}(a)) \in\mf A$.
We conclude that $\indlim A_i$ is isomorphic to $\ell^1$, and $\phi_i$ is identified with
$\varphi_{i,0}^{-1}$.

Let $(B,(\psi_i))$ be a coherent system.  Then $\psi_0 = \psi_i\circ\varphi_{i,0}$ and so
$\psi_i = \psi_0\circ\varphi_{i,0}^{-1}$.  We may hence define $\phi:\indlim A_i = \ell^1
\rightarrow B$ by $\phi = \psi_0$.  Then $\phi \circ \phi_i = \psi_0 \circ \varphi_{i,0}^{-1} =
\psi_i$.  Thus we have factored our system, and we conclude that $\indlim A_i$ is the inductive
limit in $\ba$.
\end{example}

\begin{example}
Now consider the construction in Example~\ref{ex:2}.  Let $(B,(\psi_F))$ be a coherent system.
Then $\psi_0 = \psi_F \circ \varphi_{F,0} = \psi_F \circ T_F$ so that $\psi_F = \psi_0 \circ
T_F^{-1}$.  Given $x\in F$ we have that $T_F^{-1}(x) = x + \phi(x)x_F$ and so $\psi_F(x)
= \psi_0(x) + \phi(x) \psi_0(x_F)$.  Thus
\[ \|\psi_F\| \|x\| \geq \|\psi_F(x)\| \geq |\phi(x)|\|\psi_0(x_F)\| - \|\psi_0(x)\|
\geq |\phi(x)|\|\psi_0(x_F)\| - \|\psi_0\| \|x\|. \]
As $\phi$ is unbounded, this inequality can only hold when $\psi_0(x_F)=0$.
We hence conclude that $\psi_0(x_F)=0$ for all $F\not=0$, and that hence $\psi_F = \psi_0$ for
each $F$.

Let $E_0$ be the closed linear span of $\{ x_F : F\in I, F\not=0 \}$, and let $q:E\rightarrow E/E_0$
be the quotient map.  Let $\phi_F = q: E_F=E \rightarrow E/E_0$.  For $F\leq G$ and $x\in E$
\begin{align*}
\phi_G \circ \varphi_{G,F}(x) &= q \circ T_G \circ T_F^{-1}(x)
= q(T_G(x + \mu_F(x) x_F)) \\
&= q( x - \mu_G(x)x_G + \mu_F(x) x_F - \mu_F(x)\mu_G(x_F) x_G)
= q(x) = \phi_F(x).
\end{align*}
Thus $(E/E_0,(\phi_F))$ is a coherent system.

Furthermore, $\psi_0 = \psi_F$ must factor through $q$ as $\psi_0$ sends $E_0$ to $\{0\}$.
Hence any coherent system factors, and we conclude that $(E/E_0,(\phi_F))$ is the inductive limit
in $\ba$.
\end{example}

\begin{example}
Let's explore a very simple example.  Let $I$ be some directed set, and set $A_i=\mathbb C$ for each
$i\in I$.  Suppose we have an inductive system $(\varphi_{j,i})$, so each $\varphi_{j,i}$ can be
identified with a complex number, and $\varphi_{k,j}\varphi_{j,i}=\varphi_{k,i}$ for all $i\leq j
\leq k$.

For simplicity set $I=\mathbb N$ so we need only specify $\varphi_{n+1,n} = \alpha_n$ say.  Then
$\varphi_{j,i} = \alpha_i \alpha_{i+1}\cdots\alpha_{j-1}$.  There are then exactly two cases.
Firstly, if $\alpha_n=0$ for arbitrarily large $n$, then for each $i$ there is some $j>i$ with
$\varphi_{j,i}=0$.  It follows that if $(B,(\psi_i))$ is any coherent system, then $\psi_i=0$
for all $i$.  Thus $0$ is the category theoretic inductive limit.

Now assume that $\alpha_n$ is eventually non-zero, in which case we can identify the (category
theoretic) inductive limit as $A=\mathbb C$.  Choose $n_0$ so that $\alpha_n\not=0$ for $n\geq n_0$,
set $\phi_{n_0} = 1$, and set
\[ \phi_n = \alpha_{n_0}^{-1} \alpha_{n_0+1}^{-1} \cdots \alpha_{n-1}^{-1} \qquad (n>n_0). \]
Finally set $\phi_n  = \alpha_n \alpha_{n+1} \cdots \alpha_{n_0-1}$ for $n<n_0$.  For $i\leq j$
we have that
\begin{align*}
\phi_j \varphi_{j,i} &= \phi_j \alpha_i \alpha_{i+1} \cdots \alpha_{j-1} \\
&= \begin{cases} 
\alpha_{n_0}^{-1} \cdots \alpha_{j-1}^{-1} \times \alpha_i \alpha_{i+1} \cdots \alpha_{j-1}
&: j\geq n_0 \\
\alpha_j \cdots \alpha_{n_0-1} \times \alpha_i \alpha_{i+1} \cdots \alpha_{j-1}
&: j\leq n_0
\end{cases} \\
&= \begin{cases} 
\alpha_{n_0}^{-1} \cdots \alpha_{i-1}^{-1} &: j\geq n_0, i\geq n_0 \\
\alpha_i \alpha_{i+1} \cdots \alpha_{n_0-1} &: j\geq n_0, i\leq n_0 \\
\alpha_i \alpha_{i+1} \cdots \alpha_{n_0-1} &: j\leq n_0
\end{cases} \\
&= \phi_i,
\end{align*}
and so we have a coherent system.

Let $(B,(\psi_i))$ be a coherent system, and set $p_i = \psi_1(1)$ an idempotent in $B$.  Then
for $n>n_0$ we have $p_{n_0} = \alpha_{n_0} \alpha_{n_0+1} \cdots \alpha_{n-1} p_n = \phi_n p_n$
so $p_n = \phi_n p_{n_0}$, while for $n<n_0$ we have $p_n = \alpha_n \alpha_{n+1} \cdots
\alpha_{n_0-1} p_{n_0} = \phi_n p_{n_0}$.  Thus, if we define $\phi:A=\mathbb C\rightarrow B$ by
$\phi(1) = p_{n_0}$, then $\psi_n(1) = p_n = \phi_n \phi(1)$ and so $\psi_n = \phi \phi_n$ for each
$n$.  Thus $A$ is universal.
\end{example}

\begin{example}\label{ex:4}
We again work with Banach spaces for simplicity.
Let $I=\mathbb N$, set $E_1 = \mathbb C$ and $E_{n+1} = E_n \oplus_1 \mathbb C = \ell^1_{n+1}$.
Define $\varphi_{n+1,n} : E_n \rightarrow E_n \oplus_1 \mathbb C$ to be inclusion on the first
component.  Thus we obtain a normed inductive system, and so $\indlim A_i$ exists.

Let $(B,(\psi_n))$ be some coherent system, and let $y_1 = \psi_1(1) \in B$.  Then a typical element
of $E_2$ is $(a,b)$ for some $a,b\in\mathbb C$, and coherence ensures that $\psi_2((a,b)) = ay_1 +
by_2$ for some $y_2\in B$.  Similarly we find a sequence $(y_n)$ in $B$ so that for
$a = (a_i)_{i=1}^n \in E_n$ we have that $\psi_n(x) = \sum_{i=1}^n a_i y_i$.  Notice that every
$\psi_n$ is bounded, regardless of the norms of the $(y_n)$.  We can reverse this argument, and so
we conclude that coherent systems biject with sequences $(y_n)$ in $B$.

From Theorem~\ref{thm:1} or Theorem~\ref{thm:2}, $\indlim A_i$ is universal for coherent systems
$(B,(\psi_n))$ where each $\psi_n:\ell^1_n\rightarrow B$ is a contraction.  This corresponds to the
sequence $(y_n)$ in $B$ satisfying $\|y_n\|\leq 1$ for each $n$.  It follows that $\indlim A_n$ is
actually (isometrically isomorphic to) $\ell^1$.

We claim that there is no inductive limit in the category theory sense.  Indeed, if $(E,\phi_n)$
were a universal coherent system, then as in the previous paragraph, we find a sequence $(x_n)$ in
$E$.  By universality, for any Banach space $B$ and any sequence $(y_n)$ in $B$, there is a
(unique) bounded linear map $\phi:E\rightarrow B$ with $\phi(x_n) = y_n$ for each $n$.  To obtain a
contradiction, let $(\alpha_n)$ be some sequence of positive numbers, let $B=\ell^1$ with unit
vector basis $(\delta_n)$, and let $y_n = \alpha_n \delta_n$.  If $\phi$ exists then
\[ |\alpha_n| = \|y_n\| = \|\phi(x_n)\| \leq \|\phi\| \|x_n\| \qquad (n\in\mathbb N), \]
and so $\sup_n |\alpha_n| \|x_n\|^{-1} < \infty$.  As $(x_n)$ is fixed, we can easily find a
sequence $(\alpha_n)$ to give a contradiction.
\end{example}

\begin{example}
We now adapt the previous example.  Have $(E_n)$ as before, but now let $\varphi_{n+1,n}$ be twice
the inclusion on the first component.  Then $\indlim A_i$ is not defined.  A coherent system
$(B,(\psi_n))$ still bijects with a sequence $(y_n)$ in $B$, but now for $x = (x_i) \in E_n$ we
have that $\psi_n(x) = \sum_{i=1}^n 2^{i-n} x_i y_i$.  However, the same argument will show that
there is no inductive limit in the category theory sense.
\end{example}

We hence see that in $\ba$ the category theoretical inductive limit may or may not exist, and
that the existence of $\indlim A_i$ (in either sense) seems unrelated.  This raises two questions:
\begin{itemize}
\item Can we characterise those inductive systems in $\ba$ which have an inductive limit in $\ba$?
\item What relation does $\indlim A_i$ have to category theoretical constructions?
\end{itemize}
We do not know the answer to either of these questions.



\subsection{Algebraic inductive limits}

Let us think a little about the purely algebraic inductive limit.  Let $\textsf{Alg}$ be the
category of complex algebras with algebra homomorphisms.  It is well-known that there are always
inductive limits: we give a slightly unusual construction.

Let $((A_i), (\varphi_{j,i}))$ be an inductive system in $\textsf{Alg}$.  Let $\prod_i A_i$ be the
algebraic direct sum of the $(A_i)$, that is, families $(a_i)$ where $a_i\in A_i$ for each $i$,
with coordinate wise operations.  Let $A_0$ be the collection of those $(a_i)\in \prod_i A_i$ such
that there is $i_0$ so that $\varphi_{j,i}(a_i) = a_j$ for any $i_0\leq i\leq j$.  Then $A_0$ is
a subalgebra.

Let $I$ be the collection of those $(a_i) \in \prod_i A_i$ such that there is $i_0$ with $a_i=0$
for $i_0 \leq i$.  Then $I$ is an ideal, so we may consider the quotient algebra $\prod_i A_i / I$.
Notice that if $a=(a_i)\in A_0$ and $b=(b_i)\in I$ then there are $i_0,i_1$ with $b_i=0$ for
$i\leq i_1$ and $\varphi_{j,i}(a_i) = a_j$ for $i_0 \leq i \leq j$.  Then there is $i_2$ greater
than both $i_0, i_1$, and for $i_2 \leq i \leq j$ we have both $b_i=0$ and $\varphi_{j,i}(a_i) =
a_j$.  Thus $a+b = (a_i+b_i)\in A_0$.  We conclude that $A_0 + I = A_0$, in particular,
$I\subseteq A_0$.

Let $A$ be the algebra $A_0 / I$.  Notice that alternatively, $A$ is (canonically isomorphic to)
the image of $A_0$ in $\prod_i A_i / I$.  For each $k$ and $a\in A_k$ define $(a_i)$ by $a_i
= \varphi_{i,k}(a)$ if $i\geq k$, and $a_i=0$ otherwise.  Set $\phi_k(a) = (a_i)$.  Then $\phi_k$
is a homomorphism, and it is easy to see that the resulting family $(\phi_i)$ is coherent.

Let $B$ be an algebra with a coherent system $\psi_i:A_i\rightarrow B$.  Given $a=(a_i)+I\in A$
there is $i_0$ with $\varphi_{j,i}(a_i) = a_j$ for $i_0\leq i\leq j$.  Define $\phi(a) =
\psi_i(a_i) \in B$ for any $i\geq i_0$.  If also $j\geq i_0$ there is $k$ greater than both $i,j$
and so $\psi_k(a_k) = \psi_k(\varphi_{k,i}(a_i)) = \psi_i(a_i)$ and similarly $\psi_k(a_k)
= \psi_j(a_j)$.  It follows that $\phi$ is well-defined.  Clearly $\phi\circ\phi_i = \psi_i$ for
each $i$.  It follows that $A$ is the (algebraic) inductive limit.

Supposing each $A_i$ is a Banach algebra and each $\varphi_{j,i}$ is bounded, we might seek an
algebra norm on $A$ making the maps $\phi_i$ bounded.  We could then take the norm completion of
$A$ to obtain a Banach algebra.  Let us use this idea to make a few, somewhat inconclusive,
observations.

Suppose $\overline{A}$ is an inductive limit of the Banach algebras $(A_i)$; here we are being
deliberately vague as to what this might mean.  Let $\overline\phi_i$ be the coherent maps
$A_i\rightarrow\overline A$.  In particular, this means that $(\overline A, (\overline\phi_i))$
is a coherent system of algebras, and so there is a unique $\overline\phi : A \rightarrow
\overline A$.  If $(B,(\psi_i))$ is a coherent system of Banach algebras, then we have a unique
$\psi:A\rightarrow B$, and we seek a bounded homomorphism $\overline\psi:\overline A\rightarrow B$,
making the following diagram commute:
\[ \xymatrix{ A \ar[rr]^{!\,\psi} \ar[dd]_{!\,\overline\phi} && B \\
& A_i \ar[lu]^{\phi_i} \ar[ru]^{\psi_i} \ar[ld]_{\overline\phi_i} \\
\overline A \ar@/_20pt/[rruu]_{\exists ?\,\overline\psi}
 } \]
If $\overline\psi$ exists then $\ker\overline\phi \subseteq \ker\psi$, and conversely, if this
holds, then $\overline\psi$ will exist, at least as a linear map.  Let's explore this in a couple
of examples.

\begin{example}
Let's consider Example~\ref{ex:one} again.  As each connecting map is an isomorphism of algebras, we
see that $A$, the algebraic inductive limit, is just the algebra we started with.  With $\overline A
= \indlim A_n$ we have $\overline A = \{0\}$, and so we see that we need $\psi=0$.  However, this is
actually not too restrictive.  Let $(B,(\psi_n))$ be a coherent system, so that $\psi_1 =
\psi_n \circ \varphi_{n,1} = 2^{1-n} \psi_n$ and hence $\psi_n = 2^{n-1} \psi_1$, for each $n$.
It is not hard to see that when $A$ is identified with $A_1$, then $\psi$ is identified with
$\psi_1$.  Thus we recover the previous result that only when $\psi_1=0$, that is, the coherent
system is trivial, does $\overline\psi$ exist.

If, however, we take $\overline A$ as the inductive limit in $\ba_1$, that is, $\overline A
= A_1$, as in Example~\ref{ex:3}, then $\overline\phi$ is just the identity.  From the computation
in the previous paragraph, if $(B,(\psi_n))$ is a coherent system in $\ba_1$, then in particular
$\psi_1$ is bounded, and so $\psi$ is bounded, and we may take $\overline\psi = \psi$.
\end{example}

\begin{example}
Now consider the example from Example~\ref{ex:4}, where we work with vector spaces, or Banach
spaces, and not algebras, for simplicity.  Notice that the argument shows that coherent systems of
vector spaces are just sequences $(y_n)$ in $B$.  By universality, it follows that $A$ is the
algebraic direct sum of $\mathbb N$ many copies of $\mathbb C$, that is, the vector space of
all functions $f:\mathbb N\rightarrow\mathbb C$ with $f(n)=0$ for all but finitely many $n$.
Let $(\delta_n) \subseteq A$ be the obvious basis.  Then $\psi:A\rightarrow B$ is the unique linear
map with $\psi(\delta_n) = y_n$ for each $n$.

Any norm on $A$ will lead to some choice of $\overline A$.  However $\overline\phi$ is always
injective, and so $\overline\psi$ will always exists as a linear map.  However, we learn nothing
about whether $\overline\psi$ is bounded.
\end{example}



\subsection{Uses}

There are perhaps two possible uses of inductive limits of Banach spaces:
\begin{itemize}
\item One naturally comes across a (normed) inductive system when working on some problem,
and identifying the inductive limit will be useful;
\item We are interested in constructing examples of Banach algebras, using inductive limits.
\end{itemize}
We are not actually aware of any example in the literature, but surely the 2nd must occur somewhere.

We now study the second case, that of constructing examples using inductive limits.

\begin{proposition}
Let $A$ be a Banach algebra, and let $\mc S$ be a collection of closed subalgebras of $A$ which
when ordered by inclusion becomes a directed set (that is, given $B,C\in\mc S$ there is $D\in\mc S$
with $B\subseteq D$ and $C\subseteq D$).  For $B,C\in\mc S$ with $B\subseteq C$ let $\varphi_{C,B}:
B\rightarrow C$ be the inclusion map.  Then $((B)_{B\in\mc S}, (\varphi_{C,B}))$ is a normed
inductive system, and $\indlim \mc S$ is equal to the closure of the union of $\mc S$.
\end{proposition}
\begin{proof}
We need only prove that $\indlim\mc S$ has the stated form; we may suppose that the union of $\mc S$
is dense in $A$, and we shall verify the universal property from Theorem~\ref{thm:1}.  The maps
$\phi_B:B\rightarrow A$, for $B\in\mc S$, are simply the inclusion maps.  Let $A_1$
be a Banach algebra with coherent maps $\psi_B:B\rightarrow A_1$ for $B\in\mc S$ such that
$\|\psi_B(a)\| \leq \limsup_{C\supseteq B} \|\varphi_{C,B}(a)\| = \|a\|$ for each $a\in B$, that is,
$\psi_B$ is a contraction.  We shall show there is a unique contractive homomorphism $\phi:A
\rightarrow A_1$ with $\phi \circ \phi_B = \phi|_B = \psi_B$ for $B\in\mc S$.

Given $a\in A$ in the union of the $\mc S$ define $\phi(a) = \psi_B(a)$ for any $B\in\mc S$ with
$a\in B$.  If also $C\in\mc S$ with $a\in C$ then given $D\in\mc S$ containing $B$ and $C$, by
coherence, we have that $\psi_B(a) = \psi_D(\varphi_{D,B}(a)) = \psi_D(a)$ and similarly
$\psi_C(a) = \psi_D(a)$, so $\psi_B(a) = \psi_C(a)$.  Thus $\phi$ is well-defined.  It remains to
show that $\phi$ is a contraction (and so extends by continuity to all of $A$) but this is
immediate as $\|\phi(a)\| = \|\psi_B(a)\| \leq \|a\|$ for each $a\in B$, for any $B\in\mc S$.
\end{proof}

Let $(A_i, (\varphi_{j,i}))$ be a normed inductive system (or, indeed, an inductive system
satisfying the weaker condition of Theorem~\ref{thm:2}), and let $A = \indlim A_i$.  For each $i$
let $B_i$ be the closure of $\phi_i(A_i)$ in $A$, so that $B_i \subseteq B_j$ when $i\leq j$.
Then $\mc S = \{ B_i : i\in I \}$ is a family of closed subalgebras of $A$ which becomes a directed
set under inclusion, and so the above proposition shows that $A = \indlim \mc S$.  Thus, if we are
only interested in the construction of $A$, then it is no loss of generality to work with normed
inductive systems in which each connecting morphism is an isometry.  The same argument applies to
the category theoretic inductive limit in $\ba$.




\begin{thebibliography}{99}

\bibitem{blackadar} B. Blackadar, {\it $K$-theory for operator algebras}, second edition, Mathematical Sciences Research Institute Publications, 5, Cambridge University Press, Cambridge, 1998. MR1656031

\bibitem{blenc} B. Blackadar, {\it Operator algebras}, Encyclopaedia of Mathematical Sciences, 122, Springer-Verlag, Berlin, 2006. MR2188261

\bibitem{cas} J. M. F. Castillo, The hitchhiker guide to categorical Banach space theory. Part I, Extracta Math. {\bf 25} (2010), no.~2, 103--149. MR2814472

\bibitem{conway} J. B. Conway, {\it A course in functional analysis}, second edition, Graduate Texts in Mathematics, 96, Springer-Verlag, New York, 1990. MR1070713

\bibitem{leinster} T. Leinster, {\it Basic category theory}, Cambridge Studies in Advanced Mathematics, 143, Cambridge University Press, Cambridge, 2014. MR3307165

\bibitem{palmer} T. W. Palmer, {\it Banach algebras and the general theory of $^*$-algebras. Vol. I}, Encyclopedia of Mathematics and its Applications, 49, Cambridge University Press, Cambridge, 1994. MR1270014

\bibitem{rll} M. R\o rdam, F. Larsen\ and\ N. Laustsen, {\it An introduction to $K$-theory for $C^*$-algebras}, London Mathematical Society Student Texts, 49, Cambridge University Press, Cambridge, 2000. MR1783408

\bibitem{wo} N. E. Wegge-Olsen, {\it $K$-theory and $C^*$-algebras}, Oxford Science Publications, The Clarendon Press, Oxford University Press, New York, 1993. MR1222415

\end{thebibliography}


\end{document}
