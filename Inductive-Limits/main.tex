\documentclass[a4paper,11pt]{article}
\usepackage[utf8]{inputenc}
\usepackage[margin=2cm]{geometry}
\usepackage{latexsym, amsmath, amsthm, amssymb}
\usepackage[all]{xy}

% Drafing
%\usepackage{showkeys, backref}

\newcommand{\mc}[1]{{\mathcal{#1}}}
\newcommand{\mf}{\mathfrak}
\newcommand{\Rep}{\operatorname{Rep}}
\newcommand{\Cl}{\operatorname{Cl}}
\newcommand{\Eq}{\operatorname{Eq}}
\newcommand{\Sp}{\operatorname{Sp}}
\newcommand{\op}{{\operatorname{op}}}
\newcommand{\bin}{{\operatorname{bin}}}
\newcommand{\asy}{{\operatorname{asy}}}
\newcommand{\indlim}{\varinjlim}

\newtheorem{lemma}{Lemma}[section]
\newtheorem{proposition}[lemma]{Proposition}
\newtheorem{theorem}[lemma]{Theorem}
\newtheorem{corollary}[lemma]{Corollary}
\theoremstyle{definition}
\newtheorem{definition}[lemma]{Definition}
\newtheorem{example}[lemma]{Example}


\title{Inductive limits of Banach algebras}
\author{Matt Daws}
\date{Last knowing modified 12 April 2020}

% 234567890123456789012345678901234567890123456789012345678901234567890123456789012345678901234567890

\begin{document}

\maketitle

\begin{abstract}
We provide an overview of inductive limits of Banach algebras, correcting a few misleading
statements in the literature.

This is a very early version.
\end{abstract}

\section{Introduction}

The inductive (or direct) limit of Banach algebras have not been widely studied, perhaps reflecting
a lack of applications in the literature.  Inductive limits are widely studied for locally convex
spaces -- often when the space appearing in the inductive system are Banach spaces, but the limit is
not in general a Banach space, for example, \cite[Chapter~IV, Section~5]{conway}.  They have also
been a useful tool in the theory of $C^*$-algebras, see references below.  For Banach algebras,
there are a number of misleading statements in standard textbooks, which we wish to correct.  We do
not pretend to make a literature review, instead giving such references as we are aware of.

Let $C$ be a general category, let $I$ be a directed set (often $\mathbb N$ with the usual
ordering), for each $i\in I$ let $A_i$ be an object in $C$, and for $i\leq j$ let $\varphi_{j,i}:
A_i\rightarrow A_j$ be morphism.  We require that for $i\leq j\leq k$ that $\varphi_{k,j} \circ
\varphi_{j,i} = \varphi_{k,i}$.  We warn the reader that some authors write $\varphi_{i,j}$
instead.  This is an \emph{inductive system}, written as $((A_i), (\varphi_{j,i}))$ or sometimes
just $(A_i)$, the latter being an abuse of notation, as the ``connecting morphisms'' $\varphi_{j,i}$
are vitally important.

The \emph{inductive limit} of the system, if it exists, is an object $A$ and morphisms $\phi_i:
A_i\rightarrow A$ such that:
\begin{itemize}
\item For each $i\leq j$ the following diagram commutes:
\[ \xymatrix{ A_i \ar[rr]^-{\varphi_{j,i}} \ar[rd]_{\phi_i} && A_j \ar[ld]^{\phi_j} \\
& A
} \]
\item If $B$ is an object and we have morphisms $\psi_i:A_i\rightarrow B$ with, as above,
$\psi_i = \psi_j \circ \varphi_{j,i}$ for all $i\leq j$, then there is a unique morphism
$\phi:A\rightarrow B$ making the following diagram commutes:
\[ \xymatrix{ & A_i \ar[ld]_{\phi_i} \ar[rd]^{\psi_i} \\
A \ar[rr]_{\phi} && B
} \]
\end{itemize}
A standard argument, using the ``unique'' clause, shows that if the limit exists, it is unique.
We write $\indlim A_i$ for the inductive limit (again, with an abuse of notation, as we have
suppressed the connecting morphisms.

This presentation follows \cite[Section~6.2]{rll}.  In category theory, the construction can be
much extended to the notion of a \emph{colimit}, see \cite[Chapter~5]{leinster} for example.  For
a combined category theory and functional analytic take, see \cite{cas}.
A very readable introduction, which ultimately focuses on $C^*$-algebras, is \cite[Appendix~L]{wo}.
For $C^*$-algebras, see also \cite[Section~II.8.2]{blenc}.


\section{Banach algebras}

If the read has encountered Banach spaces in a category theory setting before, they will know that
there are two main choices for which morphisms to consider:
\begin{itemize}
\item all bounded linear maps.  From an analytic perspective, this seems natural, but the resulting
category is somewhat difficult to manage;
\item just contractive linear maps.  This might seem slightly artificial, but it leads to a somewhat
better behaved category.
\end{itemize}

We shall choose the first, more general choice for morphisms.  In this setting, the inductive limit
does not always exist, but we study when it does.

Having now set the scene using this language of categories, we will now perform a u-turn and present
the standard construction, which unfortunately does not have a simple interpretation in terms of
category theory.  Below we return to the more general setting.  We follow
\cite[Definition~1.3.4]{palmer} and \cite[Section~3.3]{blackadar}, both of which make slightly
misleading claims which we note below.  We are not aware of any other standard textbooks covering
this material.\footnote{But we have not even tried to make a literature review.}

For the remainder of this section, $((A_i), (\varphi_{j,i}))$ shall be an inductive system of Banach
algebras.  Thus each $A_i$ is a Banach algebra (to be sure, this means that $A_i$ has a contractive
product) and each $\varphi_{j,i}$ is a bounded algebra homomorphism.  Essentially identical
arguments will give analogous results for Banach space, Banach modules over Banach algebras, and so
forth.

Suppose further that for each $i\in I$
\[ \limsup_{j\geq i} \|\varphi_{j,i}\| = \lim_{j\geq i} \sup_{k\geq j} \|\varphi_{k,i}\|
=
\inf_{j\geq i} \sup_{k\geq j} \|\varphi_{k,i}\| \]
is finite.  Then we say that our system is a \emph{normed inductive system} of Banach algebras.
[\footnote{Notice the mistake here?}]

Given a normed inductive system, the classical construction of $\indlim A_i$ is as follows.
Consider the disjoint union of the $A_i$ with the equivalence relation that, for $a_i\in A_i,
a_j\in A_j$, we have $a_i \sim a_j$ if and only if there is $k\geq i, k\geq j$ with
$\varphi_{k,i}(a_i) = \varphi_{k,j}(a_j)$.  A slightly tedious check shows that the quotient becomes
a vector space.  Given an equivalence class $[a]$ with representative $a\in A_i$, it is easy to see
that $\limsup_{j\geq i} \|\varphi_{j,i}(a)\|$ defines a seminorm which is submultiplicative.
The quotient by the null ideal of this seminorm gives a normed algebra, and the completion is
$\indlim A_i$.  We define $\phi_i:A_i \rightarrow \indlim A_i$ by setting $\phi_i(a)$ to
be the equivalence class of $a$.

An alternative construction is the following.  Let $\ell^\infty(A_i)$ be the Banach algebra of all
bounded families $(a_i)$, where $a_i\in A_i$ for each $i$, with pointwise operations.
Let $c_0(A_i)$ be the collection of families $(a_i)\in \ell^\infty(A)$ such that, for each
$\epsilon>0$, there is $i_0\in I$ so that $\|a_i\|<\epsilon$ for $i\geq i_0$.  Then $c_0(A_i)$
is a closed ideal in $\ell^\infty(A_i)$.  Denote by $\mf A$ the quotient algebra
$\ell^\infty(A_i) / c_0(A_i)$.  We shall abuse notation, and write $(a_i)\in\ell^\infty(A_i)$
for the equivalence class of $\mf A$ which it defines.

Given $i\in I$, by assumption, there is $j\geq i$ so that
$K = \sup_{k\geq j} \|\varphi_{k,i}\| < \infty$.  Then, for $a\in A_i$, we define
\[ a_k = \begin{cases} \varphi_{k,i}(a) & : k\geq j, \\ 0 & : \text{otherwise}. \end{cases} \]
Thus $\|a_k\| \leq K\|a\|$ for each $k$, and so $(a_k) \in \mf A$.  Suppose we chose $j'\in I$
instead, also with $\sup_{k\geq j} \|\varphi_{k,i}\| < \infty$, and form $(a_k')$ using $j'$.
There is $k_0\in I$ with $k_0 \geq j$ and $k_0 \geq j'$, so for $k\geq k_0$ we have that
$a_k = a_k'$, and so $(a_k) = (a_k')$ in $\mf A$.  It follows that we have a well-defined map
$\phi_i:A_i\rightarrow\mf A$ which is a homomorphism, and is seen to satisfy that $\|\phi_i\| \leq
\limsup_{j\geq i} \|\varphi_{j,i}\|$.  We introduce the notation that $\phi_i(a) =
( \varphi_{k,i}(a) )_{k\geq j}$ where $j\geq i$ is any suitable choice.

Given $j\geq i$ there is $k\geq j$ so that both $\sup_{l\geq k} \|\varphi_{l,i}\|<\infty$ and
$\sup_{l\geq k} \|\varphi_{l,j}\|<\infty$.  For $a\in A_i$, it follows that $\phi_i(a) =
( \varphi_{l,i}(a) )_{l\geq k}$ and that $\phi_j(\varphi_{j,i}(a)) =
( \varphi_{l,j}(\varphi_{j,i}(a)) )_{l\geq k} = 
( \varphi_{l,i}(a) )_{l\geq k}$.  Hence $\phi_i = \phi_j \circ \varphi_{j,i}$.
Set $\indlim A_i$ to be the closure of $\{\phi_i(a) : i\in I, a\in A_i \}$.  For $i\leq j$, as
$\phi_i = \phi_j \circ \varphi_{j,i}$, it follows that $\phi_i(A_i) \subseteq \phi_j(A_j)$.
Thus $\indlim A_i$ is the closure of an increasing union of algebras, and hence is a Banach algebra.

Here is an alternative description of a dense subalgebra of $\indlim A_i$.
Define $\mf A_0$ to be the equivalence classes in $\mf A$ with representatives $(a_i)$ such that,
for some $i_0$, we have that $\varphi_{k,j}(a_j) = a_k$ for $i_0 \leq j \leq k$.  We claim that
$\mf A_0$ is equal to the union of the images of the $\phi_i$, and hence is dense in
$\indlim A_i$.  If $a\in A_i$ and $j\geq i$ is suitable, then $\phi_i(a) = ( \varphi_{k,i}(a)
)_{k\geq j}$, and so with $i_0=j$, if $i_0 \leq k \leq l$ then $\varphi_{l,k}( \varphi_{k,i}(a) )
= \varphi_{l,i}(a)$ and so $\phi_i(a)\in \mf A_0$.  Conversely, given $(a_i)$ with $i_0$ such that
$\varphi_{k,j}(a_j) = a_k$ for $i_0 \leq j \leq k$, let $a = a_{i_0} \in A_{i_0}$, so that
$\varphi_{k,i_0}(a) = a_k$ for $k\geq i_0$, and hence $(a_i) = \phi_{i_0}(a)$ in $\mf A$.

Let us compare this construction with the ``classical construction'' sketched above.  Clearly we
define the same equivalence relation on the disjoint union of the $A_i$.  Given $a\in A_i$ we have
that $\|\phi_i(a)\| = \|(\varphi_{k,i}(a))_{k\geq j}\|_{\mf A}$ for a suitable $j\geq i$, and by
the definition of $\mf A$, this is equal to $\limsup_{k\geq j} \|\varphi_{k,i}(a)\|$ which is
equal to $\limsup_{k\geq i} \|\varphi_{k,i}(a)\|$.  Thus our construction agrees with the
classical one.

We now turn to verifying the universal property.  In fact, this is not considered in
\cite{blackadar}, while we believe that \cite{palmer} is wrong to say that the universal property
always holds.    Suppose that $B$ is a Banach algebra with bounded homomorphisms $\psi_i:A_i
\rightarrow B$ such that $\psi_i = \psi_j \circ \varphi_{j,i}$ for each $i\leq j$.  For $a\in A_i$
we must define $\phi(\phi_i(a)) = \psi_i(a)$, which motivates the following definition.  Given
$(a_i) \in \mf A_0$ with $\varphi_{k,j}(a_j) = a_k$ for $i_0 \leq j\leq k$, define
$\phi((a_i)) = \psi_j(a_j)$ for any $j\geq i_0$.  This is well-defined, for $\psi_j(a_j) =
\psi_j(\varphi_{j,i_0}(a_{i_0})) = \psi_{i_0}(a_{i_0})$, and so the definition does not depend on
$j$, and hence also does not depend on the choice of $i_0$.  To ensure that $\phi$ is bounded,
there must exist a constant $K$ with
\[ \|\psi_i(a)\| \leq K \|\phi_i(a)\| = K \limsup_{j\geq i} \|\varphi_{j,i}(a)\|
\qquad (i\in I, a\in A_i). \]
When such a $K$ exists, $\phi$ extends by continuity to $\indlim A_i$.

We summarise the construction as follows.

\begin{theorem}
Let $(A_i)_{i\in I}$ be a family of Banach algebras indexed by a direct set $I$.  For each
$i\leq j$ suppose there is a bounded homomorphism $\varphi_{j,i}:A_i\rightarrow A_j$ with
$\varphi_{k,j} \circ \varphi_{j,i} = \varphi_{k,i}$ for $i\leq j\leq k$, and such that
$\limsup_{j\geq i} \|\varphi_{j,i}\| < \infty$ for each $i\in I$.  

There exists a Banach algebra $A=\indlim A_i$ and bounded homomorphisms $\phi_i:A_i\rightarrow
A$ with $\|\phi_i(a)\| = \limsup_{j\geq i} \|\varphi_{j,i}(a)\|$ for $a\in A_i$, and such that
$\phi_j \circ \varphi_{j,i} = \phi_i$ for $i\leq j$,
and with the property that if $B$ is a Banach algebra and
$\psi_i:A_i\rightarrow B$ are homomorphisms with $\psi_j \circ \varphi_{j,i} = \psi_i$ for
$i\leq j$, and
$\|\psi_i(a)\| \leq \limsup_{j\geq i} \|\varphi_{j,i}(a)\|$ for $i\in I, a\in A_i$, then there is
a unique contractive homomorphism $\phi:A\rightarrow B$ with $\phi\circ\phi_i = \psi_i$.
\end{theorem}

In the statement of the theorem, we could weaken the norm condition on $\phi_i$ to
$\|\phi_i(a)\| \leq \limsup_{j\geq i} \|\varphi_{j,i}(a)\|$ for $a\in A_i$.  Then the universal
property would imply that actually we must have equality.  A standard diagram chase shows that all
such $A$ satisfying the conclusions of the theorem are isometrically isomorphic.

It is too much to expect that the ``universal property'' holds without norm control, as the next
example shows.

\begin{example}\label{ex:one}
Let $A$ be a Banach algebra, and let our directed set be $\mathbb N$ with the usual ordering.
Let $A_n = A$ for each $n\in\mathbb N$, and define $\varphi_{n+1,n}$ to be $1/2$ of the identity,
which we denote by $\varphi_{n+1,n} = \frac12$.  Then $\varphi_{n+k,n} = 2^{-k}$ for $k\geq 1$.
This defines a normed inductive system of Banach algebras, as for each $n$ we have that
$\limsup_{m\geq n} \|\varphi_{m,n}\| = \limsup_{k\geq 1} 2^{-k} = 0$.  Indeed, it is now easy to
check that $\indlim A_n = \{0\}$.

Now set $B=A$ and define $\psi_n = 2^{n-1}$ as a map $A_n\rightarrow B$.  For $n\leq m$, we see that
$\psi_m \circ \varphi_{m,n} = 2^{m-1} 2^{n-m} = \psi_n$.  So while $(B, (\psi_n))$ is a system,
there can be no map $\phi:\indlim A_n\rightarrow B$ with $\phi\circ\phi_n = \psi_n$.  Thus, without
some norm condition on the family $(\psi_n)$, we cannot hope to factor through $\indlim A_n$.
\end{example}



\subsection{Weaker conditions}

Rather than requiring that $\limsup_{j\geq i} \|\varphi_{j,i}\| < \infty$ for each $i$, we could
instead ask that $\limsup_{j\geq i} \|\varphi_{j,i}(a)\| < \infty$ for each $i$ and $a\in A_i$.
It is immediate that the construction above still works, leading to $\indlim A_i$ as a closed
subalgebra of $\ell^\infty(A_i) / c_0(A_i)$.  The same universal property will hold.  We state a
more general theorem.

\begin{theorem}
Let $(A_i)_{i\in I}$ be a family of Banach algebras indexed by a direct set $I$.  For each
$i\leq j$ suppose there is a bounded homomorphism $\varphi_{j,i}:A_i\rightarrow A_j$ with
$\varphi_{k,j} \circ \varphi_{j,i} = \varphi_{k,i}$ for $i\leq j\leq k$, and such that
$\limsup_{j\geq i} \|\varphi_{j,i}(a)\| < \infty$ for each $i\in I$ and $a\in A_i$.

There exists a Banach algebra $A=\indlim A_i$ and bounded homomorphisms $\phi_i:A_i\rightarrow
A$ with $\|\phi_i(a)\| \leq \limsup_{j\geq i} \|\varphi_{j,i}(a)\|$ for $a\in A_i$, and such that
$\phi_j \circ \varphi_{j,i} = \phi_i$ for $i\leq j$,
and with the property that if $B$ is a Banach algebra and
$\psi_i:A_i\rightarrow B$ are homomorphisms with $\psi_j \circ \varphi_{j,i} = \psi_i$ for
$i\leq j$, and
$\|\psi_i(a)\| \leq \limsup_{j\geq i} \|\varphi_{j,i}(a)\|$ for $i\in I, a\in A_i$, then there is
a unique contractive homomorphism $\phi:A\rightarrow B$ with $\phi\circ\phi_i = \psi_i$.
\end{theorem}

It is stated in \cite{blackadar, palmer} that this condition implies that  $\limsup_{j\geq i}
\|\varphi_{j,i}\| < \infty$ for each $i$, by the Uniform Boundedness Principle.  When $I=\mathbb N$
with the usual ordering, this is correct, as here $\limsup_{j\geq i} \|\varphi_{j,i}\| < \infty$
is equivalent to $\sup_{j\geq i} \|\varphi_{j,i}\| < \infty$, and similarly for the case of
$\varphi_{j,i}(a)$.  For more general directed sets $I$ this is no longer true, as the following
counter-example shows.

\begin{example}
The following studies an inductive system of Banach spaces, but these can be turned into Banach
algebras by equipping each space with the zero product.  Let $I$ be the collection of pairs
$(F,n)$ where $F\subseteq\ell^1$ is finite, and $n\in\mathbb N$, turned into a directed set for the
partial order $(F,n) \leq (F',n')$ where $F\subseteq F'$ and $n\leq n'$.  Let $0\in I$ be an extra
element with $0\leq i$ for all $i\in I$.  For each $i\in I$ set $A_i = \ell^1$.

Let $\ell^1$ have standard unit vector basis $(e_t)$.  For $a,n\in\mathbb N$ define $T_{a,n}:
\ell^1\rightarrow\ell^1$ by $T_{a,n}(e_a) = n e_a$ and $T_{a,n}(e_t) = e_t$ for $t\not=a$, extending
$T_{a,n}$ to $\ell^1$ by linearity and continuity.  Then $\|T_{a,n}\| = n$.

For $i=(F,n)\in I$ define $\varphi_{i,0} = T_{a,n}$ where $a$ satisfies that
\[ \sum_{t\in\mathbb N} |x_t| + (n-1)|x_a| < \|x\|+n^{-1} \qquad (x=(x_n) \in F). \]
As $F\subseteq \ell^1$ is finite, such a $a$ does exist.  Indeed, the inequality is equivalent to
$(n-1)|x_a| < n^{-1}$, that is, $|x_a| < 1/(n(n-1))$, for each $x\in F$.
Then $\|\varphi_{i,0}(x)\| = \|T_{a,n}(x)\| < \|x\|+n^{-1}$ for any $x\in F$. 

For each $i\in I$ define $\varphi_{i,i}$ to be the identity.  To obtain a system, we require that
for $k\geq j\not=0$ that
\[ \varphi_{k,j} = \varphi_{k,0} \circ \varphi_{j,0}^{-1}. \]
Notice that $\varphi_{j,0}^{-1}$ does exist, and is a contraction.  With this definition, we do
have an inductive system.

Finally, for $x\in\ell^1$ and $i\in I$, set $y=\varphi_{i,0}^{-1}(x)$.  There is some $j=(F,n)\in I$
with $y\in F$ and $j\geq i$.  Then, for $k\geq j$,
\[ \|\varphi_{k,i}(x)\| = \|\varphi_{k,0}(y)\| < \|y\|+n^{-1} \leq \|x\| + n^{-1}. \]
Thus $\limsup_{k\geq i} \|\varphi_{k,i}(x)\| \leq \limsup_{k\geq j} \|\varphi_{k,i}(x)\|
< \|x\| + n^{-1}$, and so the ``weaker'' condition holds.

However, given $i\in I$, choose $i\leq j$ with $j=(F,n)\in I$.  Let $k=(F',m)$ for some $m\geq n$
and $F'\supseteq F$, so that $k\geq j$.  Then $\varphi_{i,0} = T_{a,p}$ for some $a,p\in\mathbb N$.
By choosing $F'$ finite but sufficiently large, we can ensure that $\varphi_{k,0} = T_{b,m}$ where
$b$ cannot equal $a$.  Then
\[ \varphi_{k,i}(e_b) = \varphi_{k,0}( \varphi_{i,0}^{-1}(e_b) )
= T_{b,m}( T_{a,p}^{-1}(e_b) )
= T_{b,m}(e_b)
= m e_b, \]
and so $\|\varphi_{k,i}\|\geq m$.  Thus $\limsup_{j\geq i} \|\varphi_{j,i}\| =\infty$, and
so we do not have a normed inductive system.
\end{example}




\section{From a categorical perspective}

Let us start by returning to Example~\ref{ex:one}, where $A_n=A$ for each $n\in\mathbb N$ and
$\varphi_{n+1,n} = \frac12$.  Then $\indlim A_n = \{0\}$, but we noticed that we could define $B=A$
and $\psi_n = 2^{n-1}:A_n\rightarrow B$, and then $(B,(\psi_n))$ is a ???.  Indeed, $B$ also
satisfies the universal property.  For, if $C$ is a Banach algebra, and $(\phi_n)$ a sequence of
bounded homomorphisms $\phi_n:A_n\rightarrow C$ with $\phi_{n+1} \circ \varphi_{n+1,n} = \phi_n$ for
each $n$, then we must have that $\phi_n = 2^{n-1} \phi_1$ for each $n$.  We can then define $\phi:
B\rightarrow C$ by $\phi = \phi_1$ (recalling that $B=A=A_n$).  Then $\phi\circ\psi_n = 2^{n-1}\phi_1
= \phi_n$, and so we have factored the ??? $(B,(\psi_n))$.






\begin{thebibliography}{99}

\bibitem{blackadar} B. Blackadar, {\it $K$-theory for operator algebras}, second edition, Mathematical Sciences Research Institute Publications, 5, Cambridge University Press, Cambridge, 1998. MR1656031

\bibitem{blenc} B. Blackadar, {\it Operator algebras}, Encyclopaedia of Mathematical Sciences, 122, Springer-Verlag, Berlin, 2006. MR2188261

\bibitem{cas} J. M. F. Castillo, The hitchhiker guide to categorical Banach space theory. Part I, Extracta Math. {\bf 25} (2010), no.~2, 103--149. MR2814472

\bibitem{conway} J. B. Conway, {\it A course in functional analysis}, second edition, Graduate Texts in Mathematics, 96, Springer-Verlag, New York, 1990. MR1070713

\bibitem{leinster} T. Leinster, {\it Basic category theory}, Cambridge Studies in Advanced Mathematics, 143, Cambridge University Press, Cambridge, 2014. MR3307165

\bibitem{palmer} T. W. Palmer, {\it Banach algebras and the general theory of $^*$-algebras. Vol. I}, Encyclopedia of Mathematics and its Applications, 49, Cambridge University Press, Cambridge, 1994. MR1270014

\bibitem{rll} M. R\o rdam, F. Larsen\ and\ N. Laustsen, {\it An introduction to $K$-theory for $C^*$-algebras}, London Mathematical Society Student Texts, 49, Cambridge University Press, Cambridge, 2000. MR1783408

\bibitem{wo} N. E. Wegge-Olsen, {\it $K$-theory and $C^*$-algebras}, Oxford Science Publications, The Clarendon Press, Oxford University Press, New York, 1993. MR1222415

\end{thebibliography}


\end{document}
