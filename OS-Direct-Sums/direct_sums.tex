\documentclass[a4paper,11pt]{article}

\usepackage[margin=2cm]{geometry}
\usepackage{amsmath,amsthm,amssymb,latexsym}
\usepackage[all]{xy}

\theoremstyle{plain}
\newtheorem{proposition}{Proposition}[section]
\newtheorem{theorem}[proposition]{Theorem}
\newtheorem{corollary}[proposition]{Corollary}
\newtheorem{lemma}[proposition]{Lemma}
\theoremstyle{definition}
\newtheorem{definition}[proposition]{Definition}
\newtheorem{example}[proposition]{Example}

\newcommand{\mc}[1]{\mathcal{#1}}
\newcommand{\ip}[2]{\langle{#1},{#2}\rangle}
\newcommand{\osip}[2]{\langle\langle{#1},{#2}\rangle\rangle}
\newcommand{\proten}{\widehat\otimes}
\newcommand{\id}{\operatorname{id}}

\begin{document}

\title{Direct sums of Operator spaces}
\author{Matthew Daws}
\date{Last knowing updated October 2019}

\maketitle

\begin{abstract}
We give a short, self-contained account of the $\oplus_1$ and $\oplus_\infty$ norms of
operator spaces, and the duality between them.
\end{abstract}


\section{Introduction}

An \emph{operator space} is a Banach space $E$ together with a norms on each matrix
space $M_n(E)$, for $n=1,2,\cdots$, which satisfy Ruan's axioms.  Alternatively, $E$ comes
with an embedding into $\mc B(H)$, the bounded operators on a Hilbert space, where
$M_n(E)$ inherits the norm from $M_n(\mc B(H)) = \mc B(H^n)$.  We refer the reader to
\cite{er, pis, pau} for further details.

Given a family of operators spaces $(E_i)_{i\in I}$ there is an obvious way to form an
analogue of the Banach space $\ell^\infty$-direct-sum, namely $E = \oplus_{i\in I}^\infty
E_i$ which consists of all families $(x_i)_{i\in I}$ where $x_i\in E_i$ for each $i$,
and $\sup_i \|x_i\| < \infty$.  The matrix norms are defined as follows.  We identify
$M_n(E)$ with $\oplus_{i\in I}^\infty M_n(E_i)$ as vector spaces in the obvious way, and
then define
\[ \|x\|_n = \sup_{i\in I} \| x_i \|_n \qquad (x = (x_i) \in 
\oplus_{i\in I}^\infty M_n(E_i)). \]
Verifying Ruan's axioms is routine, and so we have an operator space structure on
$\oplus_{i\in I}^\infty E_i$.

We can similarly define the Banach space $\oplus_{i\in I}^1 E_i$ consisting of all
sequences $x = (x_i)$ with $\|x\| = \sum_i \|x_i\|<\infty$.  It is standard (copy
the proof that the dual of $\ell^1$ is $\ell^\infty$) that the dual space of
$\oplus_{i\in I}^1 E_i$ is $\oplus_{i\in I}^\infty E_i^*$ with the dual-pairing element
wise.  Furthermore, if $I$ is finite, then the dual space of
$\oplus_{i\in I}^\infty E_i$ is $\oplus_{i\in I}^1 E_i^*$.  Finally, for arbitrary $I$,
the natural map from $\oplus_{i\in I}^1 E_i$ into the dual of
$\oplus_{i\in I}^\infty E_i^*$ is isometric.

Motivated by this, we \emph{define} the operator space structure on
$\oplus_{i\in I}^1 E_i$ to be that given by the embedding into the dual of
$\oplus_{i\in I}^\infty E_i^*$.  This note will be concerned with other characterisations
of this operator space structure.


\subsection{In the literature}

The book by Effros and Ruan only considers the $\infty$-sum, see \cite[page~38]{er}.
Pisier's book treats both sums in \cite[Section~2.6]{pis}, and defines the $1$-sum
via the ``universal property'' discussed below.  That the duality results
hold is stated without proof.  To the best of our knowledge, \cite{pau} does not consider
the direct sum (which is not surprising, as it is not, as such, a book about operator
spaces).  The $\infty$-sum is considered in \cite[Section~1.2.17]{blm} and the $1$-sum in
\cite[Section~1.4.13]{blm} using our ``duality'' definition.

To the best of our knowledge, the $1$-sum was first introduced in \cite[Section~3]{b}.
Here $\oplus_{i\in I}^1 E_i$ is again defined by the injection into the dual of
$\oplus_{i\in I}^\infty E_i^*$.  Again, that the duality results hold are stated without
proof.  



\section{The $\infty$-sum}

As above, we identify $M_n(\oplus_{i\in I}^\infty E_i)$ with $\oplus_{i\in I}^\infty M_n(E_i)$,
and so induce an operator space structure on $\oplus_{i\in I}^\infty E_i$.
For each $i\in I$ define $\iota_i:E_i\rightarrow 
\oplus_{i\in I}^\infty E_i$ and $P_i:\oplus_{i\in I}^\infty E_i \rightarrow E_i$ to be
the natural injection and quotient maps.  It is easy to see that $\iota_i$ is a
complete isometry, and $P_i$ is a complete quotient map.

We can alternatively characterise $\oplus^\infty$ by the following universal property.
Let $(E_i)$ be a family of operator spaces, and let $F$ be an operator space.
Given a family $(T_i)$ of complete contractions $T_i:F\rightarrow E_i$, there is a
complete contraction $T:F\rightarrow \oplus_{i\in I}^\infty E_i$ given by $T(x) = (T_i(x))$.
Indeed, $\oplus^\infty$ does satisfy this property, as for $x\in M_n(F)$ we have
\[ \|(T)_n(x)\| = \sup_i \|(T_i)_n(x)\| \leq \sup_i \|T_i\|_{cb} \|x\| \leq \|x\|, \]
and so $T$ is a complete contraction.

Let us be more precise about this: we mean that there exists an operator space $E$ and
complete contractions $\pi_i:E\rightarrow E_i$ such that if $F$ is any operator space,
and $T_i:F\rightarrow E_i$ a family of complete contractions, then there is a unique
complete contraction $T:F\rightarrow E$ with $\pi_i\circ T = T_i$ for each $i$.  That is,
$(E,\pi_i)$ is a \emph{product} in the category of operator spaces, see Section~\ref{sec:cat}
below.  As shown below, $E$ is necessarily unique if it exists, and furthermore, we have
maps $\kappa_i:E_i\rightarrow E$ with $\pi_i\circ\kappa_j = \delta_{i,j}$.

Consider $F = \oplus_{i\in I}^\infty E_i$ and the complete contractions $T_i:F\rightarrow E_i$
being $T_i = P_i$.  Then there is a complete contraction $T:F\rightarrow E$ with $\pi_i\circ T
= P_i$.  Given $x = (x_i) \in M_n(F) = \oplus_{i\in I}^\infty M_n(E_i)$ and $\epsilon>0$
there is $j\in I$ with $\|x_j\|_{M_n(E_j)} > \|x\| - \epsilon$.  Thus
\[ x_j = \pi_j(T(x)) \implies \|x\| - \epsilon < \|x_j\| \leq \|T(x)\| \leq \|x\|. \]
As $\epsilon>0$ was arbitrary, we conclude that $T$ is a complete isometry.  By uniqueness
of $E$, we must have that $T$ is a completely isometric isomorphism.




\section{Equivalence of definitions}

Consider the following two operator space structures on $\oplus_{i\in I}^1 E_i$:
\begin{itemize}
\item The ``dual structure'' given by the embedding into $\oplus_{i\in I}^\infty E_i^*$;
\item The ``universal property structure'' is that for any operator space $F$ and any
family $(T_i)$ of complete contractions $T_i:E_i\rightarrow F$, then the operator
$\oplus T_i : \oplus_{i\in I}^1 E_i\rightarrow F; (x_i) \mapsto \sum_i T_i(x_i)$ exists,
and is a complete contraction.
\end{itemize}

Notice that as a \emph{Banach space}, it is immediately clear that the dual structure gives
the usual Banach space structure on $\oplus_{i\in I}^1 E_i$.  Further, working just with
Banach spaces, it is clear that the usual norm on $\oplus_{i\in I}^1 E_i$ does satisfy the
given universal property.  Consider now applying the universal property to the family of
maps $(\iota_i)$ where $\iota_i: E_i \rightarrow \oplus_{j\in I}^1 E_j$.
This shows that the universal property, for Banach spaces, determines the
usual norm on $\oplus_{i\in I}^1 E_i$.

Thus, if the universal property structure exists, then as a Banach space it is just
the Banach space direct sum $\oplus_{i\in I}^1 E_i$.  For example, $M_n(\oplus_{i\in I}^1 E_i)$
can be identified, as a vector space, with $\oplus_{i\in I}^1 M_n(E_i)$.

\begin{proposition}
There exists an operator space structure on $\oplus_{i\in I}^1 E_i$ which satisfies the
universal property.  Indeed, given a sufficiently large Hilbert space $H$, the collection
$\mc C$ of families of complete contractions $T_i:E_i\rightarrow\mc B(H)$ is a set, and
given $u = (u_i) \in M_n(\oplus_{i\in I}^1 E_i) \cong \oplus_{i\in I}^i M_n(E_i)$
we define $\|u\|_n$ to be the supremum over $\mc C$ of $\|\sum_i (T_i)_n(u_i)\|$,
this norm computed in $M_n(\mc B(H))$.
\end{proposition}
\begin{proof}
Let $J$ be a set so that each $E_i$ contains a dense subset of cardinality at most $|J|$,
say $E_i'\subseteq E_i$.
Let $F$ be an operator space, and let $(T_i)$ be a family of complete contractions
$T_i:E_i\rightarrow F$.  Let $F_0$ be the collection
\[ \Big\{ \sum_{i\in I'} T_i(x_i) : x_i\in E_i', I'\subseteq I\text{ is finite} \Big\}. \]
Let $F_1$ be the rational-linear span of $F_0$, so $F_1$ is dense in the closed linear
span of $F_0$ which equals the closure of $\{ \sum_i T_i(x_i) : (x_i)\in \oplus_{i\in I}^1 E_i\}$.  Thus, we may replace $F$ by the closure of $F_1$, and we lose no information.
Thus, in the definition of the universal property
structure, we can limit $F$ to having a dense subset of cardinality of at most $|I\times J|$. 
If we set $H = \ell^2(I\times J)$ then we can completely isometrically embed such $F$ into
$\mc B(H)$.  In conclusion, it suffices to consider families of complete contractions mapping
into $\mc B(H)$ for this fixed $H$.

The family $\mc C$ is a set.  Given $(T_i)\in\mc C$ and $u=(u_i)\in M_n(\oplus_{i\in I}^1 E_i)$
we have that $\sum_i \|(T_i)_n(u_i)\| \leq \sum_i \|u_i\| < \infty$ and so $\|u\|_n$ is
well-defined.  Given the argument above, it is now clear that given any operator space $F$
and any family of complete contractions $(S_i)$ with $S_i:E_i\rightarrow F$, then for any
$u=(u_i)\in M_n(\oplus_{i\in I}^1 E_i)$ we have that $\| \sum_i S_i(u_i) \|$ computed in
$M_n(F)$, is $\leq \|u\|_n$.  Thus $\oplus S_i$ is a complete contraction, as required.
\end{proof}

Having established that the universal property structure exists, we now want to show that
this equals to dual structure.  Given $\mu = (\mu_i) \in M_n(\oplus_{i\in I}^\infty E_i^*)$
each $\mu_i \in M_n(E_i^*) = \mc{CB}(E_i, M_n)$; let $T_i:E_i\rightarrow M_n$ be the
induced map.  If $\|\mu\|\leq 1$ then $\|\mu_i\|_n\leq 1$ and so $T_i$ is a complete
contraction, for each $i$.  Given $x = (x_i) \in M_m(\oplus_{i\in I}^\infty E_i)$ the dual
pairing is
\[ \osip{\mu}{x} = \sum_i \osip{\mu_i}{x_i} \in M_{mn}
= \sum_i (T_i)_n(x_i) = (\oplus_i T_i)_n(x). \]
From this calculation, it follows that the identity on $\oplus_{i\in I}^i E_i$ gives a
complete contraction from the universal property structure to the dual structure.

Conversely, given an operator space $F$ and $x\in M_n(F)$, we know that
\[ \|x\|_n = \sup\big\{ \|\osip{\mu}{x}\| : \mu\in M_m(F^*), \|\mu\|_m\leq 1,
m\in\mathbb N\big\}. \]
(This is equivalent to $F\rightarrow F^{**}$ being a complete contraction.  Further,
by Smith's lemma, we can just fix $m=n$, see the discussion in \cite[Section~3.2]{er}.)
For $\mu\in M_m(F^*) = \mc{CB}(F,M_m)$ a complete contraction, and given $T_i:E_i\rightarrow
F$ a complete contraction, also $S_i = \mu\circ T_i:E_i\rightarrow M_m$ is a complete
contraction.  Given $x=(x_i)\in M_n(\oplus_{i\in I}^1 E_i)$ we hence see that
\begin{align*}
& \sup\big\{ \|\sum_i (T_i)_n(x)\| : (T_i) \text{ complete contractions to }F\big\}\\
&= \sup\big\{ \|\osip{\mu}{\sum_i (T_i)_n(x)}\| : (T_i) \text{ complete contractions},
\|\mu\|\leq 1 \big\} \\
&\leq \sup\big\{ \|(S_i)_n(x)\| : (T_i) \text{ complete contractions to }M_m \big\}.
\end{align*}
However, this final supremum is just the dual structure norm.  In conclusion, we have
hence shown that the universal property structure and the dual structure agree.

The universal property structure is useful for proving existence of certain maps, but as
we have seen, the dual structure is easier to work with if one wishes to compute the
norm of an element.



\section{Duality results}

We now turn our attention of the duals of the $1$-sum and $\infty$-sum.  The first proof
is routine, but we find that the second claim is non-trivial to show.

\begin{proposition}
The dual of $\oplus_{i\in I}^1 E_i$ is $\oplus_{i\in I}^\infty E_i^*$ for the natural duality.
\end{proposition}
\begin{proof}
For Banach spaces, this is true, and so we need only show that the operator space structure
on $(\oplus_{i\in I}^1 E_i)^*$ is $\oplus_{i\in I}^\infty E_i^*$.  By the definition of
the (dual) operator space structure on $\oplus_{i\in I}^1 E_i$ we have that
\[ \Big\| \sum_i \osip{\mu_i}{x_i} \Big\|_{M_{nm}}
\leq \|(\mu_i)\|_n \|(x_i)\|_m \qquad \big((\mu_i)\in M_n(\oplus_{i\in I}^\infty E_i^*),
(x_i)\in M_m(\oplus_{i\in I}^1 E_i\big)). \]
Denote by $\|\cdot\|_n^*$ the norm on $M_n(\oplus_{i\in I}^\infty E_i^*)$ induced by
identifying this with $(\oplus_{i\in I}^1 E_i)^*$.  Thus $\|\mu\|_n^* \leq \|\mu\|_n$.

Conversely, given $\mu=(\mu_i) \in M_n(\oplus_{i\in I}^\infty E_i^*)$ fix $i$ and $\epsilon>0$,
so as $M_n(E_i^*) = \mc{CB}(E_i,M_n)$, there is $x\in M_m(E_i)$ with $\|x\|_m\leq 1$ and
$\|\osip{\mu_i}{x}\|\geq \|\mu_i\|-\epsilon$.  Then
\[ \|\osip{\mu}{(\iota_i)_m(x)}\| = \|\osip{\mu_i}{x}\|
\geq \|\mu_i\|-\epsilon, \]
so as $\|(\iota_i)_m(x)\|\leq 1$, it follows that $\|\mu\|_n^* \geq \|\mu_i\|-\epsilon$.
As $\epsilon,i$ were arbitrary, it follows that $\|\mu\|_n^* \geq \sup_i \|\mu_i\|
= \|\mu\|$.

Hence the norms on $M_n(\oplus_{i\in I}^\infty E_i^*)$ agree, as required.
\end{proof}

\begin{proposition}
The natural embedding of $\oplus_{i\in I}^1 E_i^*$ into the dual of $\oplus_{i\in I}^\infty
E_i$ is a complete isometry.
\end{proposition}
\begin{proof}
By definition, the natural pairing gives rise to a complete isometry from
$\oplus_{i\in I}^1 E_i^*$ to the dual of $\oplus_{i\in I}^\infty E_i^{**}$.  That is,
given $\mu=(\mu_i)\in M_n(\oplus_{i\in I}^1 E_i^*)$ and $\epsilon>0$ there is
$x=(x_i)\in M_m(\oplus_{i\in I}^\infty E_i^{**})$ with $\|\osip{x}{\mu}\|\geq
\|\mu\|-\epsilon$ and $\|x\|\leq 1$.
Now, $\osip{x}{\mu} = \sum_i \osip{x_i}{\mu_i}$.

For an operator space $E$ we have that $M_m(E^{**}) = M_m(E)^{**}$ as a Banach space (see
\cite[Proposition~7.1.6]{er}, compare \cite[Lemma~4.1.1]{er}).  Following \cite[Section~4.1]{er}
let $T_m(E^*)$ be $M_m(E^*)$ together with the norm induced by the pairing between $M_m(E^*)$
and $M_m(E)$ given by
\[ \ip{\mu}{x} = \sum_{j,k=1}^n \ip{\mu_{jk}}{x_{jk}} \in\mathbb C
\qquad \big( x\in M_m(E), \mu\in T_m(E^*) \big). \]
Then $M_m(E)^* = T_m(E^*)$ and $T_m(E^*)^* = M_m(E^{**})$.

Given $\lambda\in E^*$ and $j_0,k_0$ define $\mu_{jk} = \lambda$ if $j=j_0, k=k_0$,
and $0$ otherwise.  Then $\mu\in T_m(E^*)$ and $\ip{\mu}{x} = \ip{\lambda}{x_{j_0, k_0}}$.
In this way, given $\phi\in M_n(E^*)$, there is a finite-dimensional subspace
$X\subseteq T_m(E^*)$ depending on $\phi$, such that if $\phi\in M_m(E^{**})$ and
$x\in M_m(E)$ satisfy
\[ \ip{\Phi}{\mu} = \ip{\mu}{x} \qquad (\mu\in X), \]
then $\osip{\Phi}{\phi} = \osip{\phi}{x}$.

For each $i\in I$ choose $X_i$ using $\mu_i\in M_n(e_i)$.
By Goldstein's Theorem, we can find $y_i \in M_m(E_i)$ with $\|y_i\| < \|x_i\|+\epsilon
\leq 1+\epsilon$ and with $\ip{x_i}{\mu} = \ip{\mu}{y_i}$ for $\mu\in X_i$.
It follows that $\osip{x_i}{\mu_i} = \osip{\mu_i}{y_i}$.  Then $y=(y_i) \in
M_m(\oplus_{i\in I}^\infty E_i)$ with $\|y\|\leq 1+\epsilon$ and
\[ \osip{\mu}{y} = \sum_i \osip{\mu_i}{y_i} = \sum_i \osip{x_i}{\mu_i}
= \osip{x}{\mu}. \]
As $\epsilon>0$ was arbitrary, it follows that $\oplus_{i\in I}^\infty E_i$ norms
$\oplus_{i\in I}^1 E_i^*$, as required.
\end{proof}

\begin{corollary}
If $I$ is finite, then the dual of $\oplus_{i\in I}^\infty E_i$ is $\oplus_{i\in I}^1
E_i^*$.
\end{corollary}
\begin{proof}
The underlying Banach spaces agree, and so do the operator space structures by the
above result.
\end{proof}


\section{For operators}

\begin{proposition}\label{prop:one}
For any family $(E_i)$ of operator spaces, and an operator space $F$, we have that:
\begin{enumerate}
\item\label{prop:one:1}
$\mc{CB}(\oplus_{i\in I}^1 E_i, F) = \oplus_{i\in I}^\infty\mc{CB}(E_i,F)$;
\item\label{prop:one:2}
$\mc{CB}(F, \oplus_{i\in I}^\infty E_i) = \oplus_{i\in I}^\infty \mc{CB}(F,E_i)$.
\end{enumerate}
\end{proposition}
\begin{proof}
By the universal property, if $(T_i)$ is a bounded family in $\oplus_{i\in I}^\infty
\mc{CB}(E_i,F)$ then $T = \oplus_i T_i \in \mc{CB}(\oplus_{i\in I}^1 E_i, F)$.
Conversely, given $T\in \mc{CB}(\oplus_{i\in I}^1 E_i, F)$ let $T_i = T \circ \iota_i
\in\mc{CB}(E_i,F)$ so $\|T_i\|_{cb} \leq \|T\|_{cb}$ and $T = \oplus_i T_i$.  Thus,
at the level of Banach spaces, we have the required equality.

By definition, we have that
\begin{gather*}
M_n\big( \mc{CB}(\oplus_{i\in I}^1 E_i, F) \big)
= \mc{CB}(\oplus_{i\in I}^1 E_i, M_n(F)), \\
M_n\Big( \oplus_{i\in I}^\infty\mc{CB}(E_i,F) \Big)
= \oplus_{i\in I}^\infty M_n\big(\mc{CB}(E_i,F)\big)
= \oplus_{i\in I}^\infty\mc{CB}(E_i,M_n(F)).
\end{gather*}
Thus also the $n$th matrix levels agree, and we have shown (\ref{prop:one:1}).

For (\ref{prop:one:2}) given $(T_i) \in \oplus_{i\in I}^\infty \mc{CB}(F,E_i)$ we
can define $T:F\rightarrow \oplus_{i\in I}^\infty E_i$ by $T(x) = (T_i(x))$.  Then,
for $x\in M_n(F)$, we have from $M_n(\oplus_{i\in I}^\infty E_i)
= \oplus_{i\in I}^\infty M_n(E_i)$ that
\[ \|(T)_n(x)\| = \sup_{i\in I} \| (T_i)_n(x) \|
\leq \|x\| \sup_{i\in I} \|T_i\|_{cb}. \]
Thus $T\in \mc{CB}(F, \oplus_{i\in I}^\infty E_i)$.  Conversely, given
$T\in \mc{CB}(F, \oplus_{i\in I}^\infty E_i)$ define $T_i = P_i\circ T \in
\mc{CB}(F,E_i)$ so that $\|T_i\|_{cb}\leq \|T\|_{cb}$ for each $i$.  We have hence
established that $\mc{CB}(F, \oplus_{i\in I}^\infty E_i) =
\oplus_{i\in I}^\infty \mc{CB}(F,E_i)$ as Banach spaces.  Again, that the operator spaces
structures are also equal follows from
\begin{gather*}
M_n\big( \mc{CB}(F, \oplus_{i\in I}^\infty E_i) \big)
= \mc{CB}(F, \oplus_{i\in I}^\infty M_n(E_i)), \\
M_n\big( \oplus_{i\in I}^\infty \mc{CB}(F,E_i) \big)
= \oplus_{i\in I}^\infty M_n(\mc{CB}(F,E_i))
= \oplus_{i\in I}^\infty \mc{CB}(F,M_n(E_i)).
\end{gather*}
\end{proof}

By the usual duality between $\mc{CB}$ and the operator space projective tensor product
$\proten$, \cite[Chapter~7]{er}, we have the following.

\begin{corollary}
For any family $(E_i)$ of operator spaces, and an operator space $F$, we have that:
\begin{enumerate}
\item $(\oplus_{i\in I}^1 E_i) \proten F \cong \oplus_{i\in I}^1 (E_i\proten F)$;
\item $F \proten (\oplus_{i\in I}^1 E_i) \cong \oplus_{i\in I}^1 (F\proten E_i)$.
\end{enumerate}
\end{corollary}
\begin{proof}
We have that $((\oplus_{i\in I}^1 E_i) \proten F)^* \cong \mc{CB}(\oplus_{i\in I}^1 E_i, F^*)$
and $(\oplus_{i\in I}^1 E_i\proten F)^* \cong \oplus_{i\in I}^\infty \mc{CB}(E_i,F^*)$,
the isomorphisms respecting our other identifications.

The second claim follows by commutativity of the projective tensor product, or from
observing that $(F \proten (\oplus_{i\in I}^1 E_i))^* = \mc{CB}(F, \oplus_{i\in I}^\infty
E_i^*)$ while $(\oplus_{i\in I}^1 (F\proten E_i))^* \cong \oplus_{i\in I}^\infty
\mc{CB}(F,E_i^*)$.
\end{proof}


\section{Category theory aspects}\label{sec:cat}

We shall consider the category of operator spaces and complete contractions.
Given $(E_i)$ a family of operator spaces, the category theoretic \emph{product} is an
operator space $E$ and morphisms $\pi_i:E\rightarrow E_i$ such that, for any $F$ and morphisms
$T_i:F\rightarrow E_i$ there is a unique $T:F\rightarrow E$ with $\pi_i\circ T = T_i$ for
each $i$.  As a diagram, we have
\[ \xymatrix{ F \ar[r]^-{T_i} \ar@{-->}[rd]_-{\exists!\, T} & E_i \\
& E \ar[u]_-{\pi_i} } \]

If a product exists, it is unique.  Let us remind ourselves why this is true.  Let
$(E',\pi'_i)$ be another product.  Consider the universal property of $E$ applied to the
maps $\pi'_i:E'\rightarrow E_i$ which yields a map $\phi$; and similarly consider the universal
property of $E'$ applied to the maps $\pi_i:E\rightarrow E_i$ which yields a map $\phi'$.
Finally, consider the universal property of $E$ applied to the map $\pi_i:E\rightarrow E_i$,
which of course has the diagram as shown on the right.
\[ \xymatrix{ E' \ar[r]^-{\pi_i'} \ar@{-->}[rd]_-{\exists!\, \phi} & E_i \\
& E \ar[u]_-{\pi_i} }
\qquad\qquad
\xymatrix{ E \ar[r]^-{\pi_i} \ar@{-->}[rd]_-{\exists!\, \phi'} & E_i \\
& E' \ar[u]_-{\pi'_i} }
\qquad\qquad
\xymatrix{ E \ar[r]^-{\pi_i} \ar@{-->}[rd]_-{\id} & E_i \\
& E \ar[u]_-{\pi_i} } \]
Consider now $\phi \circ \phi'$ which has the diagram
\[ \xymatrix{ E \ar[r]^-{\phi'} \ar[rd]_-{\pi_i} &
E' \ar[r]^-{\phi} \ar[d]^-{\pi_i'} & E \ar[ld]^-{\pi_i} \\
& E_i } \]
By uniqueness, we must hence have that $\phi\circ\phi' = \id$.  Similarly
$\phi'\circ\phi = \id$.  Hence $\phi$ (and $\phi'$) is a completely isometric isomorphism.

We showed above that $\oplus_{i\in I}^\infty E_i$ is a concrete realisation of the product.

The \emph{coproduct} is the same concept, with the ``arrows reversed''.  It follows from
discussion above that $\oplus_{i\in I}^1 E_i$ is a concrete realisation of the coproduct.

With this in mind, Proposition~\ref{prop:one} would follows from abstract category
theoretic considerations.



\begin{thebibliography}{99}

\bibitem{b} D. P. Blecher, The standard dual of an operator space, Pacific J. Math. {\bf 153} (1992), no.~1, 15--30. MR1145913

\bibitem{blm} D. P. Blecher\ and\ C. Le Merdy, {\it Operator algebras and their modules---an operator space approach}, London Mathematical Society Monographs. New Series, 30, The Clarendon Press, Oxford University Press, Oxford, 2004. MR2111973

\bibitem{er} E. G. Effros\ and\ Z.-J. Ruan, {\it Operator spaces}, London Mathematical Society Monographs. New Series, 23, The Clarendon Press, Oxford University Press, New York, 2000. MR1793753

\bibitem{pis} G. Pisier, {\it Introduction to operator space theory}, London Mathematical Society Lecture Note Series, 294, Cambridge University Press, Cambridge, 2003. MR2006539

\bibitem{pau} V. I. Paulsen, {\it Completely bounded maps and dilations}, Pitman Research Notes in Mathematics Series, 146, Longman Scientific \& Technical, Harlow, 1986. MR0868472

\end{thebibliography}

\end{document}