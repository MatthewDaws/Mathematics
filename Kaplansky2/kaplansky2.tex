\documentclass[a4paper,12pt]{article}

\usepackage[margin=2cm]{geometry}
\usepackage[all]{xy}
%\usepackage[left,inline]{showlabels}
\usepackage{amsmath,amsthm,amssymb}
\usepackage{url}
%\usepackage{latexsym}

\theoremstyle{plain}
\newtheorem{proposition}{Proposition}[section]
\newtheorem{theorem}[proposition]{Theorem}
\newtheorem{corollary}[proposition]{Corollary}
\newtheorem{lemma}[proposition]{Lemma}
\theoremstyle{definition}
\newtheorem{definition}[proposition]{Definition}
\newtheorem{example}[proposition]{Example}
\newtheorem{remark}[proposition]{Remark}

\newcommand{\mc}{\mathcal}
\newcommand{\mf}{\mathfrak}
\newcommand{\ip}[2]{\langle{#1},{#2}\rangle}
\newcommand{\G}{\mathbb G}
\newcommand{\vnten}{\overline\otimes}
\newcommand{\proten}{\widehat\otimes}
\newcommand{\aone}{\Box}
\newcommand{\atwo}{\Diamond}
\newcommand{\lin}{\operatorname{lin}}

\begin{document}

\title{Failure of Kaplansky Density}
\author{Matthew Daws}
\maketitle

\section{Introduction}

We give some examples to show how (analogues of) the Kaplansky Density theorem fails,
and make some remarks about the Krein--Smulian Theorem.

The Kaplansky Density Theorem is the following.

\begin{theorem}
Let $H$ be a Hilbert space, and endow $\mc B(H)$ with the weak$^*$-topology coming
from regarding $\mc B(H)$ as the dual space of the trace-class operators $\mc B_*(H)$.
Let $M\subseteq\mc B(H)$ be a von Neumann algebra, by which we mean a self-adjoint,
unital, weak$^*$-closed subalgebra of $\mc B(H)$, and let $A\subseteq M$ be a
C$^*$-algebra which generates $M$, by which we mean that the weak$^*$-closure of $A$
is all of $M$.

Then the unit ball of $A$ is weak$^*$-dense in the unit ball of $M$.
\end{theorem}

We seek counter examples in the setting of dual Banach algebras.

\begin{definition}
Let $A$ be a Banach algebra, and suppose there is a Banach space $E$ and a bounded linear
isomorphism $A\rightarrow E^*$, which we use the induce a weak$^*$-topology on $A$.  We say
that $A$ is a \emph{dual Banach algebra} when the algebra product on $A$ is separately
weak$^*$-continuous.
\end{definition}

If $\theta:A\rightarrow E^*$ is the isomorphism, then the composition
\[ \xymatrix{ E \ar@{^{(}->}[r]^{\kappa_E} & E^{**} \ar[r]^{\theta^*} & A^* } \]
is bounded below.  Letting $F\subseteq A^*$ be the image, the product on $A^*$ is separately
weak$^*$-continuous if and only if $F$ is an $A$-submodule of $A^*$.

Thus, we are interested in cases when $A$ is a dual Banach algebra with weak$^*$-dense subalgebra
$B\subseteq A$, but such that the unit ball of $B$ is not weak$^*$-dense in the unit ball of $A$.
As we are not necessarily interested only in the ``isometric'' setting (for example, we do not
assume $\theta$ is an isometric isomorphism) it is also interesting to seek cases when no ball
(of finite radius) of $B$ has weak$^*$-closure containing the unit ball of $A$.

A final remark is that we can always give a Banach space the zero product, to turn it into a Banach
algebra, we may first seek examples working with Banach spaces.  Later we make remarks on more
``natural'' algebra examples.


\section{Counter-examples}

The following construction is from \cite{oz}, but we work with $\ell^1$ instead of $\ell^\infty$,
for variety, and to give a separable example.  This construction must be folklore, but we do not
know, for example, a textbook reference.

Consider $\ell^1$, a generic element of which we denote as $a = (a_n)_{n\geq 1}$, so $\|a\| =
\sum_{n=1}^\infty |a_n| < \infty$.  We give $\ell^1$ the predual $c_0$ in the usual way.  Pick
$r>0$ and let
\[ X = X_r = \big\{ a=(a_n)\in\ell^1 : r a_1 = \sum_{n=2}^\infty a_n \big\}. \]
Then $X$ is a closed subspace.
We claim that $X$ is weak$^*$-dense in $\ell^1$, which (by Hahn-Banach) is equivalent to showing
that if $x=(x_n)\in c_0$ with $\ip{a}{x}=0$ for each $a\in X$, then $x=0$.  Fix such an $x$, and
firstly choose $N>1$ and set $a=(a_n)$ with $a_1=1, a_N=r$ and $a_n=0$ otherwise.  Then $a\in X$
and
\[ 0 = \ip{a}{x} = x_1 + rx_N. \]
Letting $N\rightarrow\infty$ shows that $x_1=0$, because $x\in c_0$.  However, we then see that
$rX_N=0$ for any $N>1$, and so $x=0$, as required.

Now let $(a_\alpha)$ be a net in $X$ converging weak$^*$ to $e_1\in\ell^1$.  Thus $a^{(\alpha)}_1
\rightarrow 1$ and so
\[ \liminf_\alpha \|a_\alpha\| = \liminf_\alpha |a^{(\alpha)}_1| + \sum_{n\geq 2} |a^{(\alpha)}_n|
\geq \liminf_\alpha |a^{(\alpha)}_1| + r |a^{(\alpha)}_1| = 1+r. \]
In particular, the unit ball of $X$ is not weak$^*$-dense in the unit ball of $\ell^1$.

Now let $E$ be the $\ell^1$-direct sum of infinitely many copies of $\ell^1$ (so $E$ is itself
isometrically isomorphic to $\ell^1$), which has predual the $c_0$-direct sum of $c_0$.  Now let
$X$ be the direct sum of the subspaces $X_n$ for $n\in\mathbb N$.  $X$ is weak$^*$-dense in $E$,
as if $x=(x_n)\in\bigoplus_n c_0$ is annihilated by $X$, then each $x_n=0$ in $c_0$, as $X_n$ is
weak$^*$-dense in $\ell^1$.

Let $e^{(n)}_1$ be the first unit vector basis in the $n$th copy of $\ell^1$.  Let $(k_n)$ be a
rapidly increasing sequence of integers.  Set
\[ a = \sum_n 2^{-n} e^{(k_n)}_1, \]
so $\|a\|=1$.  Let $(a_\alpha)$ be a net in $X$ converging weak$^*$ to $a$.  For each $\alpha$,
let $a_\alpha = a^{(\alpha)}_n \in E = \bigoplus_n \ell^1$, and let $a^{(\alpha)}_n
= (a^{(\alpha,n)}_m) \in \ell^1$.  Thus
\[ \lim_\alpha a^{(\alpha,k_n)}_1 \rightarrow 2^{-n}. \]
For any $M$ there exists $\alpha_0$ so that, if $\alpha\geq \alpha_0$, then
\[ |a^{(\alpha,k_n)}_1| > 2^{-n}(1-M^{-1}) \qquad (1 \leq n \leq M). \]
Thus
\[ \|a^{(\alpha)}_{k_n}\| \geq |a^{(\alpha,k_n)}_1|(1+k_n) > 2^{-n}(1-M^{-1})(1+k_n)
\qquad (1 \leq n \leq M), \]
and so
\[ \|a_\alpha\| > \sum_{n=1}^M 2^{-n}(1-M^{-1})(1+k_n). \]
It follows that
\[ \liminf_\alpha \|a_\alpha\| \geq \sum_{n=1}^M 2^{-n}k_n, \]
for any $M$.  By choosing $k_n$ suitably, this shows that $(a_\alpha)$ cannot be bounded.

We have hence constructed a (separable) Banach space $E$ which is isometrically a dual space,
and found a subspace $X$ of $E$ which is weak$^*$-dense, but such that there is a unit vector
$x_0\in E$ such that no bounded net in $X$ converges weak$^*$ to $x_0$.


\subsection{A more abstract construction}

Let $F$ be any Banach space which is not reflexive, and let $E=F^*$.  Fix $x_0\in F$ a unit vector,
regard $F$ as a closed subspace of $F^{**}$ as usual, and pick $M\in F^{***}=E^{**}$ a unit vector
which annihilates $F$.  For $r>0$ choose $F\in F^{**} = E^*$ a unit vector with $|\ip{M}{F}|
> 1/2$, and define
\[ X = X_r = \{ \mu\in E : r \ip{\mu}{x_0} = \ip{F}{\mu} \}. \]

We claim that $X$ is weak$^*$-dense in $E$.  A quick way to see this is the following.  Notice that
if we define
\[ Y = \lin\{ F-r\kappa_F(x_0) \} = \mathbb C (F-r\kappa_F(x_0)) \subseteq F^{**}, \]
then ${}^\perp Y = X$ and so $X^\perp = Y$ as $Y$ is weak$^*$-closed in $F^{**}$.
In particular,
\[ \{0\} = Y \cap \kappa_F(F) = X^\perp \cap \kappa_F(F) = \kappa_F\big( {}^\perp X \big). \]
It follows that ${}^\perp X = \{0\}$ and so $X$ is weak$^*$-dense in $E=F^*$.

By Hahn-Banach, pick a unit vector $\mu_0\in E$ with $\ip{\mu_0}{x_0}=1$.
Now let $(\mu_\alpha)$ be a net in $X$ converging weak$^*$ to $\mu_0$, so
\[ r = \lim_\alpha r \ip{\mu_\alpha}{x_0} = \lim_\alpha \ip{F}{\mu_\alpha}. \]
We conclude that $\liminf_\alpha \|\mu_\alpha\| \geq r$, because $\|F\|=1$.

Thus, arguing as before, $\ell^1(E)$ admits a weak$^*$-dense subspace $X$ such that there is
$\mu\in \ell^1(E)$ which is not the weak$^*$-limit of any bounded net in $X$.  I have
been unable to decide if this construction is possible in $E$ itself.



\section{For algebras}

Taking the Banach space example from the previous section and giving the zero product yields a
Banach algebra example.

A much more sophisticated argument given by Dowson in \cite{dowson} constructs a normal operator
$T$ on a separable Hilbert space $H$ such that if $A$ denotes the algebra generated by $T$ inside
$\mc B(H)$, then $T^*$ is in the weak$^*$-closure of $A$, but there is no bounded net in $A$
converging weak$^*$ to $T^*$.  Thus, the weak$^*$-closure of $A$ is a von Neumann algebra $M$,
but there is no Kaplansky Density type result for $A$.





\section{The Krein--Smulian Theorem}

The following can be found in, for example, \cite[Chapter~5, Section~12]{conway}.

\begin{theorem}[The Krein--Smulian Theorem]
Let $E$ be a Banach space and let $X\subseteq E^*$ be convex.  If $X\cap\{\mu\in E^*:\|\mu\|\leq r\}$
is weak$^*$-closed for each $r> 0$, then $X$ is weak$^*$-closed.
\end{theorem}

After the proof of the theorem, Conway warns the reader thus:

\begin{quote}
There is a misinterpretation of the Krein--Smulian Theorem that the reader should be warned about.
If $X$ is a weak-star closed convex balanced subset of the unit ball of $E^*$, let $M = \bigcup
\{ rX:r>0\}$.  It is easy to see that $M$ is a linear manifold, but it does not follow that $M$
is weak-star closed.
\end{quote}

We believe that a possible way to reach this erroneous conclusion is as follows.  Denote by
$E^*_{[r]}$ the closed ball $\{\mu\in E^* : \|\mu\|\leq r\}$.  To apply Krein--Smulian, we would
need to show that $M \cap E^*_{[r]}$ were weak$^*$-closed.  It is tempting to think that
$M \cap E^*_{[r]} = rX$, but of course this need not hold.

Let $E$ be the example from before, so there is $\mu_0\in E^*$ a unit vector,
and $Y\subseteq E$ a weak$^*$-dense subspace, such that no bounded net in $Y$ converges weak$^*$ to
$\mu_0$.  Let $X$ be the weak$^*$-closure of $Y \cap E^*_{[1]}$, and form $M$ as above.  Towards
a contradiction, suppose that $M$ is weak$^*$-closed.  Clearly $X\subseteq M$ and so $M = E^*$.
Thus $\mu_0\in M$ and so there is some $r>0$ with $\mu_0 \in rX$, but as $rX$ is the weak$^*$-closure
of $Y \cap E^*_{[r]}$ we obtain a bounded net in $Y$ converging weak$^*$ to $\mu_0$, contradiction.



\begin{thebibliography}{aa}

\bibitem{conway} J. B. Conway, {\it A course in functional analysis}, second edition, Graduate Texts in Mathematics, 96, Springer-Verlag, New York, 1990. MR1070713

\bibitem{dowson} H. R. Dowson, On an unstarred operator algebra, J. London Math. Soc. (2) {\bf 5} (1972), 489--492. MR0315467

\bibitem{oz} N. Ozawa (\url{https://mathoverflow.net/users/7591/narutaka-ozawa}), Ultraweak closure inside a closed ball, URL (version: 2012-07-17): \url{https://mathoverflow.net/q/102411}


\end{thebibliography}


\end{document}
