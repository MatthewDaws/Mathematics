\documentclass[a4paper,11pt]{article}
\usepackage[utf8]{inputenc}
\usepackage[margin=2cm]{geometry}

\usepackage{xcolor}
\definecolor{myblue}{rgb}{0.1 0.1 0.6}
% ``backref'' for drafting; see manual for further options
%\usepackage[backref]{hyperref}
%\usepackage{showkeys}
\usepackage{hyperref}
\hypersetup{
   colorlinks=true,
   linkcolor=myblue,
   citecolor=myblue,
   urlcolor=myblue
}

\usepackage[T1]{fontenc}
\usepackage{beton}
\usepackage[euler-digits, euler-hat-accent]{eulervm}
\DeclareFontSeriesDefault[rm]{bf}{sbc} 

\usepackage{amsmath,amssymb,amsthm}
%\usepackage[all]{xy}
\usepackage{url, enumitem}

\theoremstyle{plain}
\newtheorem{proposition}{Proposition}[section]
\newtheorem{theorem}[proposition]{Theorem}
\newtheorem{corollary}[proposition]{Corollary}
\newtheorem{lemma}[proposition]{Lemma}
\newtheorem{claim}[proposition]{Claim}
\newtheorem{definition}[proposition]{Definition}
\newtheorem{example}[proposition]{Example}
\newtheorem{question}[proposition]{Question}

\theoremstyle{remark}
\newtheorem{remarkx}[proposition]{Remark}
\newtheorem{remarksx}[proposition]{Remarks}
\newtheorem{workingx}[proposition]{Working}
% Some hacks to get a symbol printed at the end of a remark, as it was very unclear (in my
% writing style) where a remark ended and the general flow of the paper (re)started.
\newenvironment{remark}
  {\pushQED{\qed}\renewcommand{\qedsymbol}{$\triangle$}\remarkx}
  {\popQED\endremarkx}
\newenvironment{remarks}
  {\pushQED{\qed}\renewcommand{\qedsymbol}{$\triangle$}\remarksx}
  {\popQED\endremarksx}
\newenvironment{working}
  {\pushQED{\qed}\renewcommand{\qedsymbol}{$\triangle$}\workingx}
  {\popQED\endworkingx}


\newcommand{\mc}[1]{\mathcal{#1}}
\newcommand{\mf}[1]{\mathfrak{#1}}
\newcommand{\msf}[1]{\mathsf{#1}}

\newcommand{\ip}[2]{{\langle {#1} , {#2} \rangle}}
\newcommand{\lin}{\operatorname{lin}}
\newcommand{\id}{\operatorname{id}}

\newcommand{\vnten}{\bar\otimes}

\newcommand{\hh}{\widehat}
\newcommand{\G}{\mathbb{G}}
\renewcommand{\H}{\mathbb{H}}
\newcommand{\op}{{\operatorname{op}}}

\begin{document}

\title{Some notes on weights}
\author{Matt Daws}
\date{September 2024}
\maketitle

\section{Introduction}

I collect some notes about weights on von Neumann algebras.  These vary from the trivial to slightly complicated.  I constantly get confused about the basics, so these are just a personal aide de memoir.  I try to give some references to where I found ideas.

\section{The basics}

We follow the notation of \cite[Chapter~VII]{TakesakiII}.  Much the same material can be found in \cite[Section~10.14]{StratilaZsido}, with a brief summary in \cite[Chapter~I]{Stratila_ModTheoryBook}.  A very brief summary can be found in \cite[Section~7.5]{KadisonRingroseII}, but this final source is, as usual, notable for its very careful proofs.

Given a weight $\varphi \colon M^+ \to [0,\infty]$ we set
\[ \mf p_\varphi = \{ x\in M^+ : \varphi(x)<\infty \}, \quad
\mf n_\varphi = \{ x\in M : \varphi(x^*x)<\infty \}, \quad
\mf m_\varphi = \lin \{ x^*y : x,y\in\mf n_\varphi \}. \]
We say that $\varphi$ is \emph{semi-finite} when $\mf m_\varphi$ is $\sigma$-weakly dense in $M$.  We shall mostly suppose that $\varphi$ is semi-finite, normal and faithful.

As $\mf n_\varphi$ is a left ideal, we see that $\mf m_\varphi \subseteq \mf n_\varphi$; similarly $\mf m_\varphi \subseteq \mf n_\varphi^*$ and so in fact $\mf m_\varphi \subseteq \mf n_\varphi \cap \mf n_\varphi^*$.  So:
\begin{itemize}
\item $\varphi$ semi-finite implies that each of $\mf n_\varphi, \mf n_\varphi^*$ and $\mf n_\varphi\cap\mf n_\varphi^*$ are $\sigma$-weakly dense in $M$.
\end{itemize}
Let $x\in\mf n_\varphi$ and $x=u|x|$ be the polar decomposition.  Then $|x| = u^*x\in\mf n_\varphi$.
% If $x\in\mf m_\varphi^+$ then $x^{1/2}\in\mf n_\varphi$ so $x = x^{1/2}x^{1/2} \in\mf n_\varphi$; whence:
% \begin{itemize}
%   \item $\mf m_\varphi^+ \subseteq \mf n_\varphi$.
% \end{itemize}

Form the GNS construction $(H_\varphi, \pi_\varphi, \Lambda_\varphi)$.  I think \cite{TakesakiII} is too hasty here, and so we follow \cite{KadisonRingroseII}.  As $\varphi$ is faithful, the map $\mf n_\varphi \to H_\varphi; a\mapsto \Lambda_\varphi(a)$ is injective.  Suppose that $\pi_\varphi(x)=0$, which is equivalent to $\Lambda_\varphi(xa)=0$ for each $a\in\mf n_\varphi$, equivalently, $xa=0$ for each $a\in\mf n_\varphi$.  As $\mf n_\varphi$ is $\sigma$-weakly dense in $M$, by continuity of the multiplication, $xa=0$ for each $a\in M$, so $x=0$.  A similarly careful treatment of why $\pi_\varphi$ is normal can be found in \cite{KadisonRingroseII}.

By Kaplansky Density (the cleanest statement is \cite[Theorem~5.3.5, Corollary~5.3.6]{KadisonRingroseI}) the unit ball of $\mf m_\varphi$ is $\sigma$-strongly dense in the unit ball of $M$.  Further, for any $y\in M_+$ there is a net $(y_i)$ in $\mf m_\varphi^+$ with $y_i\to y$ $\sigma$-weakly, and with $\|y_i\| \leq \|y\|$ for each $i$.  However, I do not think it is true in general that we can choose $(y_i)$ increasing with $y_i\leq y$.

As $\mf n_\varphi$ is a left ideal, it has an increasing approximate identity.  Indeed, \cite[Theorem~I.7.4]{TakesakiI} shows that the set $\{ x\in \mf n_\varphi : \|x\|<1 \}$ is upward directed and forms a right approximate identity for $\mf n$.

If we work a bit harder, we can ensure that our approximate identity has further properties, compare \cite[Proposition~3.21]{StratilaZsido}, or indeed the proof of \cite[Theorem~I.7.4]{TakesakiI}.  Here is one way to do this.  Using functional calculus, for $a\geq 0$ we can form the element $(a+\epsilon)^{-1}a$, and then for $0\leq a\leq 1$ we have the estimate (which holds for real-values, so by functional calculus, for operators) $\| a - a^2(a+\epsilon)^{-1}\| < \epsilon$.  Thus $\|xa^2(a+\epsilon)^{-1}-x\| \leq \|x\|\epsilon + \|xa-x\|$.  Thus if $(a_i)$ is our original approximate identity, then $b_i = a_i^2(a_i+\epsilon)^{-1}$ will also be an approximate identity, still have $b_i \in \mf n_\varphi \cap M_+$, but we now also have $b_i \in \mf m_\varphi^+$.
\begin{itemize}
  \item There is a (right) approximate identity $(b_i)$ in $\mf m_\varphi^+ \subseteq \mf n_\varphi$ for $\mf n_\varphi$, that is, $\|ab_i-a\|\to 0$ for $a\in\mf n_\varphi$.
\end{itemize}


\subsection{Semi-finiteness}

We explore some equivalent conditions for what it means for a normal weight to be semi-finite.

\begin{proposition}
Let $\mf n \subseteq M$ be a left ideal, and let $\mf m = \lin \{ x^*y : x,y\in\mf n\}$, which is a $*$-subalgebra of $M$ contained in $\mf n$.  There is a projection $e\in M$ with $Me = \overline{\mf n}^\sigma$ and $eMe = \overline{\mf m}^\sigma$.  As such, $\mf n$ is $\sigma$-weakly dense in $M$ if and only if $\mf m$ is.
\end{proposition}
\begin{proof}
We follow the sketch in \cite[III.1.1.15]{Blackadar_OperatorAlgebrasBook}.  As $\mf n$ is a left ideal, it follows that $\mf m \subseteq \mf n$, and that $\mf m$ is a $*$-algebra.  From \cite[Theorem~I.7.4]{TakesakiI} the set $\{ x\in \mf n_\varphi : x\geq 0, \|x\|<1 \}$ is upward directed and forms a right approximate identity for $\mf n$, say $(a_i)$.

We show that $a_i$ converges strongly to a projection $e$ with $\overline{\mf n}^\sigma = Me$.  There are many ways to show this, but here is a simple-minded approach.  Let $M$ act on $H$, and again, we know that $a_i \rightarrow e$ $\sigma$-strongly,\footnote{We have strong convergence, see \cite[Lemma~5.1.4]{KadisonRingroseI} for example.  By replacing $M$ with $M\otimes 1$ acting on $H\otimes\ell^2$, we convert strong convergence to $\sigma$-strong convergence.} where $e \in M^+$ is the upper bound.  For $x\in\mf n$ we have $xe = \lim_i xa_i = x$, the first limit holding strongly, say (the second limit is in norm, of course).  Set $H_0 = \overline\lin\{ x^*\xi : x\in\mf n, \xi\in H \}$.  As $a_i=a_i^*\rightarrow e$ strongly, it follows that $e(H) \subseteq H_0$.  For $x\in\mf n,\xi\in H$ we have $ex^*\xi = (xe)^*\xi = x^*\xi$ and so $e\eta=\eta$ for each $\eta\in H_0$.  Hence $e=e^2$ is an idempotent with image $H_0$.  Now let $\eta\in H_0^\perp, \xi\in H$, so $(e\eta|\xi) = \lim_i (a_i\eta|\xi) = \lim_i (\eta|a_i^*\xi) = 0$ as $a_i^*\xi\in H_0$ for each $i$.  So $e(H_0^\perp) = \{0\}$ and $e$ is the orthogonal projection onto $H_0$.  As $xe=x$ for each $x\in\mf n$, certainly $\mf n \subseteq Me$, and hence $\overline{\mf n}^\sigma \subseteq Me$.  For $x\in M$, we have $xe = \lim_i xa_i \in \overline{\mf n}^\sigma$, as the limit certainly holds in the $\sigma$-weak topology.  So $Me = \overline{\mf n}^\sigma$ as claimed.

Now let $x,y\in\mf n$, so $x^*y\in\mf m$ and hence $x^*y = ex^*ye \in eMe$.  Hence $\overline{\mf m}^\sigma \subseteq eMe$.  Given $x\in eMe$, we have $x = exe = \lim_i a_i^* x a_i$ as $a_i=a_i^*$, the limit holding in the strong topology say, which follows from the estimate
\begin{align*}
\| (a_ixa_i - exe)\xi \| \leq \| a_ixa_i\xi - a_ixe\xi \| + \| a_ixe\xi -exe\xi\|
\leq \|x\| \|(a_i-e)\xi\| + \| (a_i-e) xe\xi \|,
\end{align*}
remembering that $\|a_i\|\leq 1$.  As $a_i\in\mf n$ also $xa_i\in\mf n$ and so $a_i^* (xa_i) \in \mf m$.  We conclude that $x \in \overline{\mf m}^\sigma$, showing the other inclusion, so $\overline{\mf m}^\sigma = eMe$.

To finish, as $\mf m \subseteq \mf n$, if $\mf m$ is $\sigma$-weakly dense in $M$, then obviously $\mf n$ is as well.  For the converse, we observe that $\overline{\mf n}^\sigma=M$ if and only if $Me=M$, if and only if $e=1$, which implies $\overline{\mf m}^\sigma = eMe = M$.  
\end{proof}

\begin{corollary}
$\varphi$ is semi-finite if and only if $\mf n_\varphi$ is $\sigma$-weakly dense in $M$.
\end{corollary}

In particular, this shows that the meaning of ``semi-finite'' for operator-valued weights, \cite[Definition~IX.4.14]{TakesakiII}, is in accordance with the meaning for weights.

An alternative characterisation of \emph{semifinite} comes from \cite[Proposition~III.2.2.20]{Blackadar_OperatorAlgebrasBook}.  As $\mf m_\varphi$ is a $*$-algebra, its norm closure is a $C^*$-algebra, which has an approximate unit, so by approximation, so does $\mf m_\varphi$.\footnote{To maintain positivity, let $(e_i)$ be an (increasing) positive approximate identity for $\overline{\mf m_\varphi}$.  For each $i$, let $b_i\in\mf m_\varphi$ norm approximate $e_i^{1/2}$, and set $a_i = b_i^*b_i$, so $a_i\geq 0$ is norm close to $e_i$.}

\begin{proposition}\label{prop:semifinitequiv}
Let $\varphi$ be a normal weight on $M$.  The following are equivalent:
\begin{enumerate}
  \item\label{prop:semifinitequiv:one} $\varphi$ is semifinite: $\mf m_\varphi$ is $\sigma$-weakly dense in $M$;
  \item\label{prop:semifinitequiv:two} for each (positive) approximate unit $(a_i)$ for $\mf m_\varphi$, for $x\in M_+$ we have $\varphi(x) \leq \liminf_i \varphi(a_i x a_i)$;
  \item\label{prop:semifinitequiv:three} for each (positive) approximate unit $(a_i)$ for $\mf m_\varphi$, when $x\in M_+$ has $\varphi(x)=\infty$, we have $\lim_i \varphi(a_i x a_i) = \infty$;
  \item\label{prop:semifinitequiv:four} for each (positive) approximate unit $(a_i)$ for $\mf m_\varphi$, when $x\in M_+$ has $\varphi(x)=\infty$, we have $\sup_i \varphi(a_i x a_i) = \infty$;
\end{enumerate}
\end{proposition}
\begin{proof}
When $\mf m_\varphi$ is $\sigma$-weakly dense in $M$, we have that $a_i \to 1$ $\sigma$-strongly, and so for each $x\in M_+$ we have that $a_i x a_i \to x$ $\sigma$-weakly.  As $\varphi$ is $\sigma$-weakly lower semicontinuous, $\varphi(x) \leq \liminf_i \varphi(a_ixa_i)$, as required to show (\ref{prop:semifinitequiv:one})$\implies$(\ref{prop:semifinitequiv:two}).
Conversely, as above, $\overline{\mf m_\varphi}^\sigma = pMp$ for some projection $p$, and $a_i \to p$ $\sigma$-strongly.  Let $x\in M_+$ with $px=0$.  As $a_i p = a_i$ because $a_i\in\mf m_\varphi$, we have $\varphi(x) \leq \liminf_i \varphi(a_ixa_i) = \liminf_i \varphi(a_ipxa_i) = 0$ so $\varphi(x)=0$ so $x\in\mf m_\varphi^+$, so $x = px = 0$.  Set $x=1-p$ to conclude that $p=1$ as required.

As the $\liminf$ is infinite implies the limit is, clearly (\ref{prop:semifinitequiv:two})$\implies$(\ref{prop:semifinitequiv:three}).  When $x\in\mf m_\varphi^+$, the condition in (\ref{prop:semifinitequiv:two}) holds by lower semicontinuity, and so (\ref{prop:semifinitequiv:three})$\implies$(\ref{prop:semifinitequiv:two}).  Clearly (\ref{prop:semifinitequiv:three})$\implies$(\ref{prop:semifinitequiv:four}).  When (\ref{prop:semifinitequiv:four}) holds, towards a contradiction, suppose that (\ref{prop:semifinitequiv:three}) doesn't, so there is $K>0$ with $J = \{ i : \varphi(a_ixa_i) \leq K \}$ is cofinal (that is, for each $i_0$ there is $i\geq i_0$ with $\varphi(a_ixa_i) \leq K$, this being the opposite of the limit being infinite).  Then $(a_j)_{j\in J}$ is a subnet of $(a_i)$, and so is still an approximate identity, but as $\sup_{j\in J} \varphi(a_jxa_j) \leq K$, we contradict (\ref{prop:semifinitequiv:four}).  Hence (\ref{prop:semifinitequiv:four})$\implies$(\ref{prop:semifinitequiv:three}) as we want.
\end{proof}



\section{Hilbert algebras to von Neumann algebras}\label{sec:HilbAlg}

I find the links between the Hilbert algebra and weights to be a bit obscure in \cite{TakesakiII}: it would perhaps benefit from a nice summary somewhere.  Similarly in \cite{StratilaZsido}.

Let $\mf A$ be a full left Hilbert algebra associated to the weight $\varphi$.  The passage between these objects is detailed in \cite[Section~VII.2]{TakesakiII}.  We have
\begin{gather*}
\mf n_l = \pi_l(\mf B) = \mf n_{\varphi}, \quad \mf n_r = \pi_r(\mf B') = \mf n_{\varphi'}, \quad \pi_l(\mf A) = \mf n_l \cap \mf n_l^*, \quad
\pi_r(\mf A') = \mf n_r \cap \mf n_r^*, \\
\varphi(\pi_l(\eta)^*\pi_l(\xi)) = (\eta|\xi) \qquad (\xi,\eta\in\mf B), \qquad
\varphi'(\pi_r(\eta)^*\pi_r(\xi)) = (\eta|\xi) \qquad (\xi,\eta\in\mf B').
\end{gather*}
For $\eta\in\mf B'$ define $\omega^l_\eta\in M_*^+; x \mapsto (\eta|x\eta)$, and set
\[ \Phi_{l,0} = \{ \omega^l_\eta : \eta\in\mf B', \|\pi_r(\eta)\|<1 \}. \]
Then
\[ \varphi(x) = \sup\{\omega(x) : \omega\in \Phi_{l,0} \} \qquad (x\in M_+). \]

We state the following for the ``left'' algebra, but analogous results hold on the right.  By \cite[Theorem~VI.1.19, Theorem~VIII.1.2]{TakesakiII} we have the modular automorphism group $(\sigma_t)$ on $M$ given by
\[ \pi_l(\sigma_t(x)) = \nabla^{it} x \nabla^{-it} \qquad (x\in M, t\in\mathbb R). \]
The map $J$ is a bijection between $\mf A$ and $\mf A'$ and an anti-homomorphism, \cite[Theorem~VI.1.19]{TakesakiII}.
I find it buried in \cite{TakesakiII} (but see \cite[Theorem~10.12]{TakesakiII}) that
\[ \pi_l(\nabla^{it}\xi) = \nabla^{it} \pi_l(\xi) \nabla^{-it},
\qquad J \pi_l(\xi) J = \pi_r(J\xi), \qquad (\xi\in\mf A). \]
We extend this to left bounded vectors.

\begin{lemma}\label{lem:toB}
For $\xi\in\mf B$ we have that $\pi_l(\nabla^{it}\xi) = \nabla^{it} \pi_l(\xi) \nabla^{-it}$.
\end{lemma}
\begin{proof}
Let $\xi\in\mf B$ and $\eta\in\mf A'$ so also $\nabla^{-it}\eta\in \mf A'$, and
\[ \pi_r(\eta) \nabla^{it}\xi
= \nabla^{it} \nabla^{-it}\pi_r(\eta) \nabla^{it}\xi
= \nabla^{it} \pi_r(\nabla^{-it}\eta) \xi
= \nabla^{it} \pi_l(\xi) \nabla^{-it}\eta. \]
As $\eta$ was arbitrary, we conclude that $\nabla^{it}\xi\in\mf B$ with $\pi_l(\nabla^{it}\xi) = \nabla^{it} \pi_l(\xi) \nabla^{-it}$ as required.
\end{proof}

This lemma is equivalently stated as $\pi_l(\nabla^{it}\Lambda(a)) = \sigma_{t}(a)$ for $a\in\mf n_\varphi$, or further, equivalently as $\nabla^{it}\Lambda(a) = \Lambda(\sigma_t(a))$, as $\Lambda\pi_l = \id$ on $\mf n_\varphi$.

The relation between left and right bounded vectors allows us to show other results.  For example, given $b\in\mf n_\varphi \cap \mf n_\varphi^*$ we have $\Lambda(b)\in \mf A$ so $J\Lambda(b) \in\mf A'$ with $\pi_r(J\Lambda(b)) = JbJ$, and hence
\[ JbJ\Lambda(a) = \pi_r(J\Lambda(b))\Lambda(a) = \pi_l(\Lambda(a))J\Lambda(b) = aJ\Lambda(b) \qquad (a\in\mf n_\varphi, b\in\mf n_\varphi \cap \mf n_\varphi^*). \]

By \cite[Proposition~VI.1.24]{TakesakiII}, the map $\Lambda \colon \mf n_\varphi \to \mf B \subseteq H$ is closed for $\sigma$-strong topology on $M$ and the norm topology on $H$.  Simiarly, when restricted to $\mf n_\varphi\cap\mf n_\varphi^* \to \mf A$, the map is closed for the $\sigma$-strong$^*$ topology and the $\mf D^\sharp$ norm, that is, $\|\xi\|_{\sharp} = (\|\xi\|^2 + \|\xi^\sharp\|^2)^{1/2}$.

We can approximate vectors, \cite[Theorem~VI.1.26]{TakesakiII}.  Given $\xi\in\mf B$ there is a sequence $(\xi_n)\subseteq\mf A$ converging to $\xi$ with $\|\pi_l(\xi_n)\| \leq \|\pi_l(\xi)\|$ for each $n$, whence $\pi_l(\xi_n) \to \pi_l(\xi)$ strongly, because $\pi_l(\xi_n)\eta \to \pi_l(xi)\eta$ for each $\eta\in\mf A'$.

Similarly, if $\mf A_0\subseteq\mf A$ is any left Hilbert algebra with $\mf A_0''=\mf A$, then for $\xi\in\mf A$ there is a sequence $(\xi_n)$ in $\mf A_0$ with $\|\xi_n-\xi\|_\sharp\to 0$ and $\|\pi_l(\xi_n)\| \leq \|\pi_l(\xi)\|$ for each $n$, whence $\pi_l(\xi_n) \to \pi_l(\xi)$ $\sigma$-strong$^*$, using now that $\pi_l(\xi_n)^*\eta = \pi_l(\xi_n^\sharp)\eta = \pi_r(\eta)\xi_n^\sharp$ for $\eta\in\mf A'$.


\subsection{Tomita algebras and smearing}

Given $\xi\in H$ we define
\[ \xi_n = \frac{n}{\sqrt\pi} \int_{\mathbb R} e^{-n^2t^2} \nabla^{it}\xi \ dt \qquad (n>0). \]
(Notice we use a slightly different convention to \cite{TakesakiII,StratilaZsido}.)  As $\mathbb R \to H, t\mapsto \nabla^{it}\xi$ is continuous, dominated convergence shows that $\xi_n \to \xi$ in norm, as $n\to\infty$.  We have that $\xi_n \in D(\nabla^{iz})$ for any $z\in\mathbb C$, with
\[ \nabla^{iz}(\xi_n) = \frac{n}{\sqrt\pi} \int_{\mathbb R} e^{-n^2(t-z)^2} \nabla^{it}\xi \ dt, \]
a fact proved by contour deformation.

In both, for example, \cite[Proof of Theorem~VI.2.2]{TakesakiII} or \cite[Page~303]{StratilaZsido}, it is claimed that if $\xi\in\mf A$ then $\pi_l(\xi_n) \to \pi_l(x)$ strongly.  I do not see why this is immediate.  Firstly, why is $\xi_n\in\mf B$?
\begin{itemize}
\item For $\eta\in\mf A'$ we certainly have that
\[ \pi_r(\eta) \xi_n = \frac{n}{\sqrt\pi} \int_{\mathbb R} e^{-n^2t^2} \pi_r(\eta) \nabla^{it}\xi \ dt 
= \frac{n}{\sqrt\pi} \int_{\mathbb R} e^{-n^2t^2} \nabla^{it}\pi_l(\xi) \nabla^{-it}\eta \ dt, \]
compare the proof of Lemma~\ref{lem:toB}.  For any $x\in\mc B(H)$ the map $t\mapsto \nabla^{it} x \nabla^{-it}$ is strongly continuous, and so we can form the integral in the strong operator topology, to give meaning to
\[ \pi_r(\eta) \xi_n
= \frac{n}{\sqrt\pi} \Big( \int_{\mathbb R} e^{-n^2t^2} \nabla^{it}\pi_l(\xi) \nabla^{-it} \ dt \Big) \eta
= \frac{n}{\sqrt\pi} \Big( \int_{\mathbb R} e^{-n^2t^2} \sigma_t(\pi_l(\xi)) \ dt \Big) \eta. \]
Call this $\pi_l(\xi)_n \eta$.  Thus $\xi_n\in\mf B$ with $\pi_l(\xi_n) = \pi_l(\xi)_n$.

\item An alternative argument is the following.  We can approximate the intergal by a suitable Riemann sum, say given a partition $I_m = \{ t_0 < t_1 < \cdots < t_{N(m)} \}$ we have
\[ \xi_n \approx \frac{n}{\sqrt\pi} \sum_{k=1}^{N(m)} \frac{1}{t_k-t_{k-1}} e^
{-n^2t_k^2} \nabla^{it_k} \xi = \xi_{n,m}, \]
say.  Similarly $\pi_l(\xi)_n$ has the same approximation, but with $\sigma_{t_k}(\pi_l(\xi))$ replacing $\nabla^{it_k}\xi$.  As the sum is finite, we clearly have $\pi_l(\xi_{n,m}) = \pi_l(\xi)_{n,m}$.  As $\mf B \to \mf B(H), \xi\mapsto\pi_l(\xi)$ is closed for the strong topology, as $\lim_m \xi_{n,m} = \xi_n$ and $\lim_m \pi_l(\xi)_{n,m} = \pi_l(\xi)_n$ we have that $\xi_n\in\mf B$ with $\pi_l(\xi_n) = \pi_l(\xi)_n$.
\end{itemize}
We now wish to show that for any $\eta\in H$ we have
\[ \pi_l(\xi_n)\eta = \frac{n}{\sqrt\pi} \int_{\mathbb R} e^{-n^2(t-z)^2} \sigma_t(\pi_l(\xi))\eta \ dt \to \pi_l(\xi)\eta. \]
This is clear, as $t\mapsto \sigma_t(\pi_l(\xi))\eta$ is continuous, so in the limit the integral converges to $\sigma_0(\pi_l(\xi))\eta$.  The ingredients of this argument are in \cite[Proof of Theorem~VI.2.2]{TakesakiII}, but I think it's useful to be explicit.


\section{Weights on opposite algebras}

Let $M$ be a von Neumann algebra and $M^\op$ its opposite algebra, say $M^\op = \{ x^\op : x\in M \}$ with product $x^\op y^\op = (yx)^\op$.  For a subset $X\subseteq M$ write $X^\op = \{ x^\op : x\in X \} \subseteq M^\op$.

Given an nsf weight $\varphi$ on $M$ we define $\varphi^\op$ on $M^\op$ by $\varphi(x^\op) = \varphi(x)$ for $x\in M^{\op +} = (M^+)^\op$.  Then $x^\op\in\mf n_{\varphi^\op}$ if and only if $\varphi^\op(x^{\op *} x^\op) = \varphi(xx^*) < \infty$ if and only if $x^*\in\mf n_\varphi$.  Thus $\mf p_{\varphi^\op} = \mf p_\varphi^\op$ and
\[ \mf m_{\varphi^\op} = \lin \{ x^{\op *} y^\op : x^\op, y^\op \in \mf n_{\varphi^\op} \}
= \lin\{ (yx^*)^\op : x^*, y^* \in \mf n_\varphi \} = \mf m_\varphi^\op. \]
Alternatively, $m_{\varphi^\op} = \lin \mf p_{\varphi^\op} = \lin \mf p_\varphi^\op = \mf m_\varphi^\op$.  By linearity, $\varphi^\op(x^\op) = \varphi(x)$ for each $x\in\mf m_\varphi$.

We have the GNS construction $(L^2(\varphi^\op), \pi^\op, \Lambda^\op)$.  For $x,y\in \mf n_\varphi^*$ we have 
\[ ( \Lambda^\op(x^\op) | \Lambda^\op(y^\op) )
= \varphi^\op(x^{\op *} y^\op)
= \varphi(yx^*)
= (\Lambda(y^*)|\Lambda(x^*))
= (J\Lambda(x^*) | J\Lambda(y^*)). \]
Hence by density there is a unitary $U\colon L^2(\varphi^\op) \to L^2(\varphi)$ with $U\Lambda^\op(x^\op) = J\Lambda(x^*)$ for each $x\in\mf n_\varphi^*$.  For $x\in\mf n_\varphi^*$ and $a\in M$ we have
\[ U \pi^\op(a^\op) U^* J\Lambda(x^*) = U \Lambda^\op(a^\op x^\op)
= J\Lambda((xa)^*) = J \pi(a^*) JJ \Lambda(x^*), \]
and so $U \pi^\op(a^\op) U^* = J \pi(a^*) J$.

We next investigate how $U$ intertwines the maps $S, F, \nabla, J$; see \cite[Lemma~VI.1.5]{TakesakiII} for example.  Recall that $S$ is the closure of the map $\Lambda(x) \mapsto \Lambda(x^*)$ for $x\in\mf n_\varphi\cap\mf n_\varphi^*$, and that
\[ D(F) = \{ \eta\in L^2(\varphi) : \Lambda(x) \mapsto (S\Lambda(x)|\eta) \text{ is bounded} \}; \quad (\eta|S\xi) = (\xi|F\eta) \qquad (\xi\in D(S), \eta\in D(F)). \]
As $F = F^{-1}$ we have that $\eta\in D(F) \Leftrightarrow F(\eta)\in D(F)$ and $F^2(\eta) = \eta$ for $\eta\in D(F)$.  Similarly for $S$.  In particular, for $\xi\in D(S), \eta\in D(F)$ also $\eta'=F(\eta)\in D(F)$ with $F\eta'=\eta$, and so $(S\xi|F\eta) = (S\xi|\eta') = (F\eta'|\xi) = (\eta|\xi)$.

\begin{lemma}
We have that $US^\op = FU$ (so $U$ restricts to bijection $D(S^\op) \to D(F)$) and that $UF^\op = SU$.
\end{lemma}
\begin{proof}
That $US^\op = FU$ means that $D(US^\op) = D(FU)$, equivalently, $D(S^\op) = U^*D(F)$, equivalently, $U D(S^\op) = D(F)$, using that $U$ is unitary, with $US^\op(\xi^\op) = FU(\xi^\op)$ for $\xi^\op \in D(S^\op)$.

Let $\xi^\op\in D(S^\op)$ so there is a sequence $(x_n^\op) \subseteq \mf n_{\varphi^\op}$ with $\Lambda(x_n^\op) \to \xi^\op, \Lambda(x_n^{\op *}) \to S^\op(\xi^\op)$.  For $x\in\mf n_\varphi \cap\mf n_\varphi^*$ we have
\begin{align*}
(S\Lambda(x)|U(\xi^\op))
&= \lim_n (S\Lambda(x)|J\Lambda(x_n^*))
= \lim_n (\Lambda(x_n^*)|JS\Lambda(x))
= \lim_n (S\Lambda(x_n)|FJ\Lambda(x)) \\
&= \lim_n (J\Lambda(x)|\Lambda(x_n))
= \lim_n (J\Lambda(x_n)|\Lambda(x))
= \lim_n (U\Lambda^\op(x_n^{\op *}) | \Lambda(x) ) \\
&= (US^\op(\xi^\op)|\Lambda(x)).
\end{align*}
So $U(\xi^\op) \in D(F)$ with $FU(\xi^\op) = US^\op(\xi^\op)$.

Conversely, and for variety using a different proof strategy, let $\xi\in D(F) = D(\nabla^{-1/2})$ so $J\xi \in D(\nabla^{1/2}) = D(S)$ and hence there is a sequence $(x_n)$ in $\mf n_\varphi \cap\mf n_\varphi^*$ with $\Lambda(x_n) \to J\xi, \Lambda(x_n^*) \to SJ(\xi) = JF(\xi)$.  So $U\Lambda(x_n^\op) = J\Lambda(x_n^*) \to F(\xi)$ and $U\Lambda(x_n^{\op *}) = J\Lambda(x_n) \to \xi$ showing that $U^*F(\xi) \in D(S^\op)$ with $S^\op U^*F(\xi) = U^*(\xi)$.  Set $\xi^\op = U^*F(\xi)$ so $\xi^\op \in S^\op$ with $US^\op(\xi^\op) = \xi$ hence showing that $U$ maps $D(S^\op)$ onto $D(F)$.

That $UF^\op = SU$ follows similarly.
\end{proof}

So $\nabla^\op = F^\op S^\op = U^*S U U^*FU = U^* SF U = U^* \nabla^{-1} U$.  Hence $S^\op = U^* F U = U^* J \nabla^{-1/2} U = U^* J U \nabla^{\op 1/2}$ and $S^\op = U^*FU = U^* \nabla^{1/2} J U = \nabla^{\op\, -1/2} U^*JU$, so by uniqueness, $J^\op = U^*JU$.

Consider now the modular automorphism group $(\sigma_t^\op)$; we use \cite[Section~VIII.1]{TakesakiII} for a reference.  We claim that $\sigma_t^\op(x^\op) = \sigma_{-t}(x)^\op$ for each $x\in M$.    This would leave $\varphi^\op$ invariant, and given $x^\op,y^\op\in\mf n_{\varphi^\op} \cap \mf n_{\varphi^\op}^* = (\mf n_\varphi \cap \mf n_\varphi^*)^\op$ there is a bounded continuous $F$ on $\overline{\mathbb D}$, holomorphic on $\mathbb D$, with
\[ F(t) = \varphi(\sigma_t(x)y), \qquad F(t+i) = \varphi(y\sigma_t(x)) \qquad (t\in\mathbb R). \]
Define $F'(z) = F(i-z)$, so if $z = x+iy$ for $y\in [0,1]$ then $i-z = -x + i(1-y)$ and so $F'$ is also bounded and continuous on $\overline{\mathbb D}$ and holomorphic on $\mathbb D$.  We have
\begin{align*}
F'(t) &= F(i-t) = \varphi(y\sigma_{-t}(x)) = \varphi^\op(\sigma_{-t}(x)^\op y^\op),  \\
F'(t+i) &= F(i-(t+i)) = F(-t) = \varphi(\sigma_{-t}(x) y) = \varphi^\op(y^\op \sigma_{-t}(x)^\op).
\end{align*}
So $F'$ satisfies the requirements for the pair $x^\op, y^\op$.  By uniqueness, our claim follows.  We show this is consistent with $\nabla^\op$.  We have that $J\nabla = \nabla^{-1}J$, but $J$ is also anti-linear, and so $J\nabla^{it} = \nabla^{it}J$ for all $t$.  Hence
\begin{align*}
\pi^\op\big( \sigma^\op_t(x^\op) \big)
&= \nabla^{\op\, it} \pi^\op(x^\op) \nabla^{\op\, -it}
= U^* \nabla^{-it} U U^* J\pi(x^*)J U U^* \nabla^{it} U
= U^* J \nabla^{-it} \pi(x^*) \nabla^{it} J U \\
&= U^* J \pi\big(\sigma_{-t}(x^*)\big) J U
= \pi^\op(\sigma_{-t}(x)^\op)
\end{align*}
showing again that $\sigma_t^\op(x^\op) = \sigma_{-t}(x)^\op$.



\subsection{Commutant acting on standard form}

We suppress $\pi\colon M \to\mc B(L^2(\varphi))$ and regard $M$ as a subalgebra of $\mc B(L^2(\varphi))$.  Then $M' = JMJ$ is isomorphic to $M^\op$ via the map $\theta \colon M^\op \to M'; x^\op \mapsto Jx^*J$.  Let $\varphi^\op$ induce $\varphi'$.  Then $n_{\varphi'} = \theta(n_{\varphi^\op}) = \{ Jx^*J : x^*\in\mf n_\varphi\} = J\mf n_\varphi J$, and similarly $\mf p_{\varphi'} = J\mf p_\varphi J$ and $\mf m_{\varphi'} = J\mf m_\varphi J$.  So $\varphi'(JxJ) = \varphi(x^*)$ for $x\in\mf m_\varphi$.  Then $\theta$ induces a unitary $u \colon L^2(\varphi') \to L^2(\varphi); \Lambda(Jx^*J) \mapsto \Lambda^\op(x^\op)$.

Then $Uu \colon L^2(\varphi') \to L^2(\varphi)$ is $\Lambda'(JxJ) \mapsto J\Lambda(x)$ and $Uu\pi'(Ja^*J)u^*U^* = U \pi^\op(a^\op) U^* = J\pi(a^*)J = Ja^*J$.  By definition, $u \nabla' u^* = \nabla^\op = U^*\nabla^{-1}U$ and so $Uu \nabla' u^*U^* = \nabla^{-1}$, and similarly $Uu J' u^*U^* = J$.  Also $Uu S' u^*U^* = US^\op U^* = F$ and $Uu F' u^*U^* = S$.

In summary, if we identify $L^2(\varphi')$ with $L^2(\varphi)$ via $\Lambda'(JxJ) = J\Lambda(x)$ for $x\in\mf n_\varphi$, then $\pi'$ is the identity representation, and the modular operators are $\nabla' = \nabla^{-1}, J' = J, S' = F, F' = S$.

We can check various things for consistency.  For example, with $x' = Jx^*J  = \theta(x^\op) \in M'$, then we would expect $\sigma'_t(x') = \theta(\sigma^\op_t(x^\op)) = \theta(\sigma_{-t}(x)^\op) = J \sigma_{-t}(x)^* J$.  We also have
\[ \sigma'_t(x')
= \nabla'^{it} x' \nabla'^{-it}
= \nabla^{-it} Jx^*J \nabla^{it}
= J \nabla^{-it} x^* \nabla^{it} J
= J \sigma_{-t}(x^*) J
= J \sigma_{-t}(x)^* J, \]
as expected.


\subsection{Alternative construction}

An alternative construction of $\varphi'$ is given in \cite[Theorem~VII.1.17]{TakesakiII}.  We show here that these definitions do agree.  We recall the definition from \cite{TakesakiII}.  Let $\Phi_{\varphi} = \{ \omega \in M_*^+ : \omega\leq\varphi \}$ and $E_\varphi = \bigcup_{\lambda\geq 0} \lambda\Phi_{\varphi}$.  For each $\omega\in E_\varphi$ there is $h_\omega \in M'_+$ with $(\Lambda(x)|h_\omega\Lambda(y)) = \omega(x^*y)$ for $x,y\in\mf n_\varphi$.  Then set $\mf p_{\varphi'_0} = \{ h_\omega : \omega\in E_\varphi \}$ and define
\[ \varphi'_0(x) = \begin{cases} \|\omega\| &: x = h_\omega \in \mf p_{\varphi'_0}, \\
  \infty &: \text{otherwise.} \end{cases} \qquad (x\in M'_+). \]
Then $\varphi'_0$ is a nsf weight on $M'$.

We use the bijection between left Hilbert algebras and nsf weights, in particular Theorems~2.5 and~2.6 from \cite[Section~VII]{TakesakiII}; see also Section~\ref{sec:HilbAlg} above.
Starting with $\varphi$ on $M$ we form the full left Hilbert algebra $\mf A_\varphi = \Lambda(\mf n_\varphi \cap \mf n_\varphi^*)$.  This gives rise to a weight $\varphi_l$ on $\mc R_l(\mf A_\varphi) \cong M$ and $\varphi_l = \varphi$.  We also obtain $\varphi_r$ on $\mc R_r(\mf A_\varphi) = \mc R_l(\mf A_\varphi)'$ which agrees with $\varphi'_0$.  %By \cite[Theorem~VI.1.19]{TakesakiII} we have that $J \mc R_r(\mf A_\varphi) J = \mc R_l(\mf A_\varphi)$, that is, $JMJ = M'$.

Let $\mf B$ be the algebra of left bounded vectors, and similarly $\mf B'$.  As in Section~\ref{sec:HilbAlg} above, $J \pi_l(\xi) J = \pi_r(J\xi)$ for each $\xi\in\mf A \Leftrightarrow J\xi\in\mf A'$.  Then $\xi\in\mf B$ exactly when there is a bounded operator $\pi_l(\xi)$ satisfying $\pi_l(\xi)\eta = \pi_r(\eta)\xi$ for each $\eta\in\mf A'$, equivalently,
\[ \pi_l(\xi)J\eta = J\pi_l(\eta)J\xi \qquad (\eta\in\mf A). \]
Then $\pi_l(\eta) J\xi = J\pi_l(\xi)J\eta$ for each $\eta\in\mf A$ shows that $J\xi\in\mf B'$ with $\pi_r(J\xi) = J\pi_l(\xi)J$.  We can reverse this argument, and so we conclude that $J\mf B = \mf B'$.

We recall the construction of $\varphi_r$: we have
\[ \varphi_r(x) = \sup\{ \omega(x): \omega\in \Phi_{r,0} \}
\quad\text{where}\quad
\Phi_{r,0} = \{ \omega_\xi|_{M'} : \xi\in\mf B, \|\pi_l(\xi)\|\leq 1 \}. \]
However, we now see that
\[ \Phi_{r,0} = \{ \omega_{J\xi}|_{M'} : \xi\in\mf B', \|\pi_r(\xi)\|\leq 1 \}, \]
which should be compared with $\Phi_{l,0}$ is Section~\ref{sec:HilbAlg}.
Given $x\in M'$ we have $JxJ\in M$ and $\omega_{J\xi}(x) = (J\xi|xJ\xi) = (\xi|JxJ\xi) = \omega_\xi(JxJ)$, so
\[ \varphi_l(JxJ) = \sup\big\{ \omega_{\xi}(JxJ) : \xi\in \mf B', \|\pi_r(\xi)\|< 1 \big\}
= \varphi_r(x). \]
Hence $\varphi_l$ and $\varphi_r$ are related in the way we expect, and as $\varphi_r = \varphi'_0$, we conclude that $\varphi' = \varphi'_0$ as claimed.


% Let $a\in D(\sigma_{-i/2})$ which by \cite[Lemma~VIII.3.18]{TakesakiII} is equivalent to $\varphi(axa^*) \leq \varphi(x)$ for each $x\in M_+$, and implies that $\Lambda(xa^*) = J\sigma_{-i/2}(a)J\Lambda(x)$ for each $x\in\mf n_\varphi$.  Suppose also $a^*\in\mf n_\varphi$, so $\omega = \omega_{\Lambda(a^*)} \in \Phi_\varphi$ and so $\varphi'_0(h_\omega) = \|\Lambda(a^*)\|^2 = \varphi(aa^*)$.  Given $x,y\in\mf n_\varphi$ we have
% \begin{align*}
% \omega(x^*y)
% &= (x\Lambda(a^*)|y\Lambda(a^*))
% = ( J\sigma_{-i/2}(a)J\Lambda(x) | J\sigma_{-i/2}(a)J\Lambda(y) )
% = ( \Lambda(x) | J\sigma_{-i/2}(a)^*\sigma_{-i/2}(a)J\Lambda(y) )
% \end{align*}
% and so $Jh_\omega J = \sigma_{-i/2}(a)^*\sigma_{-i/2}(a)$.  Suppose finally that $\sigma_{-i/2}(a)\in\mf n_\varphi$.  By \cite[Proposition~2.17]{Stratila_ModTheoryBook}, as $a\in \mf n_\varphi^* \cap D(\sigma_{-i/2})$ and $\sigma_{-i/2}(a) \in \mf n_\varphi$, we have $\varphi(aa^*) = \varphi(\sigma_{i/2}(a^*)\sigma_{-i/2}(a))$.  Thus $\varphi'(h_\omega) = \varphi'_0(h_\omega)$.

% In particular, this is all true for $a \in \mf t_\varphi$ the Tomita algebra (that is, $\Lambda(a)$ is in the Tomita algebra associated to the Hilbert algebra $\Lambda(\mf n_\varphi \cap \mf n_\varphi^*)$).  



\bibliographystyle{plain}
\bibliography{thebib.bib}

\end{document}
