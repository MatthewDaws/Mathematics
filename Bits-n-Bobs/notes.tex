\documentclass[twoside,a4paper,12pt]{article}

\usepackage[margin=2cm]{geometry}

\usepackage{amsmath,amssymb,latexsym,amsthm,url}
%\usepackage{showkeys}
%\usepackage[active]{srcltx}
\usepackage[all]{xy}

\theoremstyle{plain}
\newtheorem{proposition}{Proposition}[section]
\newtheorem{theorem}[proposition]{Theorem}
\newtheorem{corollary}[proposition]{Corollary}
\newtheorem{lemma}[proposition]{Lemma}
\theoremstyle{definition}
\newtheorem{definition}[proposition]{Definition}
\newtheorem{example}[proposition]{Example}
\newtheorem{remark}[proposition]{Remark}
\newtheorem{remarks}[proposition]{Remarks}
%\newtheorem{question}[proposition]{Question}
%\newtheorem{problem}[proposition]{Problem}

\newcommand{\ip}[2]{\langle #1,#2 \rangle}
\newcommand{\mc}{\mathcal}
\newcommand{\mf}{\mathfrak}
\newcommand{\G}{{\mathbb G}}
\newcommand{\vnten}{\overline\otimes}
\newcommand{\mor}{\operatorname{Mor}}
\newcommand{\tr}{\operatorname{Tr}}
\newcommand{\re}{\operatorname{Re}}
\newcommand{\lin}{\operatorname{lin}}
\newcommand{\op}{{\operatorname{op}}}
\newcommand{\aff}{\operatorname{Aff}}
\newcommand{\id}{\operatorname{id}}
\newcommand{\aut}{\operatorname{Aut}}

\begin{document}

\title{Smaller notes}
\author{Matthew Daws}
\maketitle

\tableofcontents



\section{Unions of cosets}

This is a careful write-up of my question \cite{q1} at math\emph{overflow} together with the
answer \cite{a1}.  The question is:

\begin{quote}
Let $G$ be a locally compact group, and let $K,L$ be cosets of $G$ (not assumed open or closed)
which each have empty interior.  Does also $K\cup L$ have empty interior?
\end{quote}

The answer is ``no''.  The counter-example comes from considering $G = (\mathbb Z/2\mathbb Z
)^{\mathbb N}$ the infinite product of the cyclic group of order 2.  We shall write $0,1$ for the
elements of $\mathbb Z/2\mathbb Z$, so $1+1=0$.  We give $G$ the product topology, so $G$ is compact
and Hausdorff.  We shall write elements of $G$ as infinite sequences $x = (x_n)$ in
$\mathbb Z/2\mathbb Z$.  Notice that $G$ is compact, abelian, and every $x\in G$ satisfies
that $x+x=0$.

The topology has a base of ``cylinder'' sets, given as follows.  Let $n\in\mathbb N$ and $y=(y_1,
y_2,\cdots,y_n)$ a finite sequence in $\mathbb Z/2\mathbb Z$.  Define
\[ \mathcal O_{n,y} = \big\{ x=(x_k)\in G : x_i = y_i \ (i\leq n) \big\}. \]
These sets form a base for the topology on $G$.  Notice that the intersection of two such sets
is of the same form (or is empty).

Furthermore, notice that for $s,t\in\mathbb Z/2\mathbb Z$ either $s=t$ or $s=t+1$.  Then
\begin{align*} G \setminus \mathcal O_{n,y}
&= \big\{ x : \exists\, 1\leq i\leq n, x_i\not=y_i \big\}
= \bigcup_{i=1}^n \big\{ x : x_i\not=y_i \big\} \\
&= \bigcup_{i=1}^n \big\{ x : x_i=y_i+1 \big\}
\end{align*}
which is the union of (many) basic open sets.  Thus $\mathcal O_{n,y}$ is also closed.

Finally, notice that $\mathcal O_{n,y}$ is a subgroup exactly when $y_i=0$ for $i\leq n$, and so
every $\mathcal O_{n,y}$ is a coset in $G$.

\begin{lemma}\label{lem:1}
$G$ has countably many open subgroups.
\end{lemma}
\begin{proof}
Consider a basic open set $\mc O_{n,y}$.  Given $x,z\in \mc O_{n,y}$, as $x$ and $z$ agree in the
first $n$ coordinates, we see that $(x+z)_k = 0$ for $k\leq n$.  It follows that $\mc O_{n,y}
+ \mc O_{n,y} = \mc O_{n,0}$ an open subgroup.

Now let $H$ be an arbitrary open subgroup, so we can write $H$ as some union of basic open sets.
Let $n$ be minimal with $\mc O_{n,y}\subseteq H$ for some $y$.  Thus $\mc O_{n,0}\subseteq H$
as $H$ is a subgroup.  Any open basic open set $\mc O_{m,z}\subseteq H$ must have $n\leq m$,
and so we see that $\mc O_{m,z} + \mc O_{n,0} = \mc O_{n,z} \subseteq H$.

As $H$ is the union of basic open sets, we conclude that there are finitely many $x_1,\cdots, x_k$
with $H = \bigcup_{i=1}^k \mc O_{n,x_i}$.  Let $y_i \in (\mathbb Z / 2\mathbb Z)^n$ be the
projection of $x_i$ onto the first $n$ coordinates.  As $H$ is a subgroup, it follows that
$\{y_i : 1\leq i \leq k\}$ is a subgroup of $(\mathbb Z / 2\mathbb Z)^n$, say $K$.  Furthermore,
$H$ is exactly the collection of all $x\in G$ such that the projection of $x$ onto the first
$n$ coordinates is in $K$.

It follows that open subgroups of $H$ can be described by $n\in\mathbb N$ and a subgroup $K$ of
$(\mathbb Z / 2\mathbb Z)^n$.  There are only countably many such choices.
\end{proof}

\begin{corollary}
There are countably many closed subgroups of $G$ of index $2$.
\end{corollary}
\begin{proof}
Let $H\leq G$ be a closed subgroup of index $2$.  Then $G\setminus H$ is a coset of $H$ and so
is closed, and so $H$ is open.  The result follows.
\end{proof}

We now consider arbitrary subgroups of $G$.  It is instructive to consider the bijection between
$G$ and $\mc P(\mathbb N)$ the power set of $\mathbb N$, given by $x=(x_n)$ mapping to the set
$A\subseteq \mathbb N$ where $n\in A$ if and only if $x_n=0$.  If $x,y\in G$ biject with $A,B$,
respectively, then $x+y$ bijects with $C$ where $n\in C$ if and only if $x_n+y_n=0$, that is,
$x_n=y_n=0$ or $x_n=y_n=1$, that is, $n\in A\cap B$ or $n\in \mathbb N \setminus (A\cup B)$.
Thus $C = (A\cap B) \cup (\mathbb N \setminus (A\cup B)) = \mathbb N \setminus (A \triangle B)$.

We recall the notion of a \emph{filter} on $\mathbb N$.  This is a subset $\mc F\subseteq
\mc P(\mathbb N)$ with $\emptyset\not\in\mc F$, with, if $A\in\mc F$ and $A\subseteq B$ then also
$B\in\mc F$, and with $A,B\in\mc F\implies A\cap B\in\mc F$.  An \emph{ultrafilter} $\mc U$
is a maximal filter; alternatively, $\mc U$ is a filter with the property that if
$A\in\mc P(\mathbb N)$ then either $A\in\mc U$ or $\mathbb N\setminus A\in\mc U$.

\begin{lemma}
There are $2^{\mf{c}}$ subgroups of index $2$ in $G$.
\end{lemma}
\begin{proof}
Let $\mc U$ be an ultrafilter, and let $H\subseteq G$ be the associated subset.  Given $A,B\in\mc U$
consider $C=\mathbb N \setminus (A\triangle B)$.  Then $A\cap B \subseteq C$ and so $C\in\mc U$.
It follows that $H$ is a subgroup of $G$.  Furthermore, given $A\not\in\mc U$ we know that
$\mathbb N\setminus A\in\mc U$.  Thus if $x\not\in H$ then $1+x\in H$, and as $\emptyset\not\in
\mc U$ also $1\not\in H$.  Thus $H$ is proper, and $G$ is the union of $H$ and $1+H$, so $H$ has
index $2$.

It is well-known (see for example \cite{a2}) that there are $2^c$ ultrafilters on $\mathbb N$,
and so there are (at least) $2^\mf{c}$ subgroups of index $2$ in $G$.  As $G$ bijects with
$\mc P(\mathbb N)$ we have $|G| = 2^{\aleph_0} = \mf{c}$ and so $|\mc P(G)| = 2^{\mf{c}}$.
Thus there are at most $2^{\mf{c}}$ subgroups of any index.
\end{proof}

There hence exists a subgroup $H$ of index $2$ which is not closed.  (In fact this follows more
directly from the existence of non-principle ultrafilters, and the proof of Lemma~\ref{lem:1}.)
Thus $H$ is not open, and so cannot contain any non-empty open set (if $\emptyset\not=
U\subseteq H$ is open then using the group operations we can cover $H$ by translates of $U$ which
shows that $H$ is open, contradiction).  As $G\setminus H$ is a coset of $H$ it follows that
$G\setminus H$ cannot contain any non-empty open set.  Thus $H$ is dense in $G$.  We have also
now answered our original question, as both $H$ and its coset have empty interior, and yet their
union is all of $G$.



\section{Semi-direct products}

This is standard material.  Let $G$ be a group with a subgroup $H$ and a normal subgroup $N$.
The following statements are equivalent:
\begin{enumerate}
\item\label{sdp:1} $G=NH = \{ nh : n\in N, h\in H \}$ and $N \cap H = \{e\}$;
\item\label{sdp:2} for each $g\in G$ there are unique $n\in N, h\in H$ with $g=nh$;
\item\label{sdp:3} for each $g\in G$ there are unique $n\in N, h\in H$ with $g=hn$;
\item\label{sdp:4} for the inclusion $i:H\rightarrow G$ and the quotient $\pi:G\rightarrow G/N$,
the composition $\pi\circ i:H\rightarrow G/N$ is an isomorphism;
\item\label{sdp:5} there is a homomorphism $G\rightarrow H$ that is the identity on $H$ and has
kernel $N$;
\item\label{sdp:6} there is a split short-exact sequence
$1 \rightarrow N \rightarrow G \rightarrow H \rightarrow 1$.
\end{enumerate}

Let us show these equivalences.  If (\ref{sdp:1}) holds then $G=NH$ but if there was a possibly
non-unique way to
write $g = n_1h_1 = n_2h_2$ then $n_2^{-1} n_1 = h_2 h_1^{-1} \in N\cap H=\{e\}$ so $n_1=n_2$ and
$h_1=h_2$, so (\ref{sdp:2}) holds.  Conversely, the uniqueness clause in (\ref{sdp:2}) shows that
if $g\in N\cap H$ then we must have $g=e$.  So (\ref{sdp:1})$\Leftrightarrow$(\ref{sdp:2}).

That (\ref{sdp:1})$\Leftrightarrow$(\ref{sdp:3}) is similar, using the group inverse to show that
$G=NH$ if and only if $G = G^{-1} = H^{-1} N^{-1} = HN$.

Considering (\ref{sdp:4}), $\ker(\pi\circ i) = N\cap H$ and the image of $\pi\circ i$ is $\{Nh :h\in H\}
= \{ Nnh : n\in N, h\in H \}$, so immediately (\ref{sdp:1}) implies (\ref{sdp:4}), while conversely,
if $\pi\circ i$ is onto, then given $g\in G$ there is $h\in H$ with $Nh = Ng$ so $g\in Nh \subseteq NH$
and hence (\ref{sdp:1}) holds.  So (\ref{sdp:1})$\Leftrightarrow$(\ref{sdp:4}).

For (\ref{sdp:5}) consider $p:G\rightarrow H$ a homomorphism with $p(h)=h$ for $h\in H$ and $\ker(p)=N$.
Then $N\cap H=\{e\}$ while for $g\in G$ let $h=p(g)\in H$ so $p(h g^{-1}) = h p(g)^{-1} = e$ and hence
$h g^{-1} = n$ for some $n\in N$, so $g = n^{-1} h \in NH$, so (\ref{sdp:1}) holds.
Conversely, given (\ref{sdp:1}), also (\ref{sdp:4}), let $p:G\rightarrow H$ be the composition of
$(\pi\circ i)^{-1} \circ \pi : G\rightarrow H$.  Then $p$ is a homomorphism, $\ker(p) = \ker(\pi) = N$,
and for $h\in H$ we have $p(h) = h_1$ where $h_1\in H$ is the necessarily unique element with
$h_1N = \pi(h) = hN$, so $h_1=h$ by uniqueness.  Hence (\ref{sdp:5}) holds.

For (\ref{sdp:6}) we have
\begin{equation}
\xymatrix{ 1 \ar[r] & N \ar@{^{(}->}[r]^{\iota} & G
\ar@{->>}@<0.6ex>[r]^{\pi} & H \ar[r] \ar@{..>}@<0.6ex>[l]^{\theta} & 1 }
\label{eq:sdp:1}
\end{equation}
where $\iota$ is an inclusion, $\pi$ a surjection with $\ker(\pi) = \iota(H)$, and we have
$\pi\circ\theta = \id_H$.  We identify $N$ with its image in $G$ which is certainly normal, and then
$\pi$ gives an isomorphism between $G/N$ and $H$.  That $\pi\circ\theta=\id_H$ means $\theta$ is injective,
so we identify $H$ with its image in $G$.  Then $p = \theta\circ\pi$ is an idempotent homomorphism,
which is the identity on $H$, and has $\ker(p) = \ker(\pi) = N$, so (\ref{sdp:5}) holds.
Conversely, given (\ref{sdp:5}), we have inclusions $N\rightarrow G$ and $H\rightarrow G$ giving
$\iota$ and $\theta$.  Consider the map $p:G\rightarrow H$, and set $\pi=p$.  Then $\ker(\pi) = \ker(p)
= N$ while $\pi$ is onto, and $\pi\circ\theta = \id_H$, so (\ref{sdp:6}) holds.

Any of these equivalent conditions define what it means for $G$ to be the \emph{(inner) semidirect
product} denoted $G = N \rtimes H$ (sometimes $G = H \ltimes N$).

\begin{definition}
The \emph{(outer) semidirect product} of groups $N,H$ is defined by specifying a homomorphism
$\varphi:H\rightarrow \aut(N)$ and setting $G = N\times H$ as a set, with product
\[ (n_1,h_1)(n_2,h_2) = (n_1 \varphi(h_1)(n_2), h_1h_2). \]
The identity of $G$ is $(e_N, e_H)$ and the inverse is $(n,h)^{-1} = (\varphi(h^{-1})(n^{-1}), h^{-1})$.
\end{definition}

Given such a construction, we identify $N$ with $\{ (n,e_H) : n\in N \}$ and $H$ with $\{ (e_N,h):h\in H\}$.
These are seen to be subgroups of $G$.  Then $NH = \{ (n,e)(e,h) : n\in N, h\in H \} = G$ and $N\cap H=\{e\}$.
Finally, the product satisfies
\[ h n h^{-1} = (e,h)(n,e)(e,h)^{-1}
= (\varphi(h)(n), h) (e,h^{-1})
= (\varphi(h)(n) \varphi(h)(e), e) = \varphi(h)(n) \in N, \]
for each $n\in N, h\in H$.  Thus $N$ is a normal subgroup and we have verified condition (\ref{sdp:1}).

Conversely, if we have an inner semidirect product $G=NH$ with $N \unlhd G$, then define $\varphi(h)(n)
= hnh^{-1} \in N$, so that $\varphi(h)$ is an automorphism of $N$ for each $h$, and $\varphi:H\rightarrow
\aut(N)$ is an homomorphism.  We define $\theta: G \rightarrow N \times H$ by $\theta(nh) =
(n,h)$.  As a slight aside, notice that this map is a homomorphism if and only if $\theta(h) \theta(n) =
\theta(hn) = \theta( (hnh^{-1}) h ) = \theta( \varphi(h)(n) h )$, which is true by definition.
This ``commutation relation'', showing how elements of $N$ and $H$ commute pass each other, is often
a useful way to think of a semi-direct product.

Thus inner and outer semidirect products are canonically isomorphic.

Finally, we claim that $G$ is isomorphic to $N \rtimes H$ if and only if there is a split short exact
sequence
\[ \xymatrix{ 1 \ar[r] & N \ar@{^{(}->}[r]^{\beta} & G
\ar@{->>}@<0.6ex>[r]^{\alpha} & H \ar[r] \ar@{..>}@<0.6ex>[l]^{\gamma} & 1 } \]
This is just condition (\ref{sdp:6}).  Notice 
\[ \varphi(h)(n) = hnh^{-1} = \beta^{-1}\big( \gamma(h) \beta(n) \gamma(h^{-1}) \big) \]
gives the action of $H$ on $N$.


\subsection{For topological groups}

When $G$ is a topological group, it seems natural to consider the continuous automorphisms of $N$,
and to ask for $\varphi:H\rightarrow N$ to be continuous.  Forming the (outer) semidirect product, we
form the topological product $N \times H$ and the product as above.  We require that $H\times N
\rightarrow N; (h,n) \mapsto \varphi(h)(n)$ is (jointly) continuous, and then we see that the
product on $N \rtimes H$ is (jointly) continuous.  The inverse is continuous, as it is the composition
of $(h,n) \mapsto (h^{-1}, n^{-1})$ with the continuous action map.

Then $N, H$ are closed subgroups of $N\rtimes H$.  Conversely, start with $G=NH$ for some closed
$N,H$ with $N \unlhd G$.  Then in (\ref{sdp:4}) while $i,\pi$ are continuous, so $\pi\circ i$ is continuous,
we require that $(\pi\circ i)^{-1}$ be continuous as well.  This is equivalent to $p$ in (\ref{sdp:5})
being continuous.  For (\ref{sdp:6}), in the split short exact sequence \eqref{eq:sdp:1}, we have that
$\iota,\pi$ are continuous, and we also require that $\theta$ is continuous.  Then $\theta$ is injective,
and furthermore is a homeomorphism onto its range, because $\pi\circ\theta=\id_H$.

It does not seem obvious what conditions we would like to add to (\ref{sdp:1}), (\ref{sdp:2}) or (\ref{sdp:3})
to ensure the correct continuity conditions.  However, notice that given $G=NH$ with $N\unlhd H$, with
$N, H$ closed, we can still define $\varphi(h)(n) = hnh^{-1}$, and continuity of the product in $G$ will
ensure that $\varphi(h)$ is continuous for $h$, and that $H\times N \rightarrow N;
(h,n) \mapsto \varphi(h)(n)$ is continuous.  Thus we can form the outer semidirect product.  The canonical
isomorphism is $G \rightarrow N \times H, nh \mapsto (n,h)$.  The inverse of this is always continuous,
but this map itself might not be.  It is continuous exactly when $(n_i h_i)$ is a net converging to
$e$, we must have $n_i\rightarrow e$ and $h_i\rightarrow e$.  If $G$ is locally compact, this fails to
happen exactly when we can find $(n_i) \subseteq N, (h_i)\subseteq H$ with $n_i\rightarrow \infty,
h_i\rightarrow\infty$ and yet $n_ih_i\rightarrow e$.

\begin{remark}
It is implicitly claimed on page 10 of \cite{kt} that in this setting, the map from the outer
semidirect product $N \rtimes H$ to $G$ is an isomorphism of locally compact groups.  The following
counter-example, from \cite{a3}, shows that this need not be the case.  Let $K$ be an infinite compact
group and let $K_d$ be $K$ with the discrete topology, so $K\not=K_d$ as topological groups.  Let
$G = K_d \times K$ the direct product of groups, so $G$ is locally compact.  Let $N = K_d \times \{e\}$
a closed subgroup, and let $H = \{ (g,g) : g\in K \} \subseteq G$ the diagonal, which is also a closed,
normal subgroup.  Then it's easy to see that $N\cap H=\{e\}$ and $NH = G$.  However, both $N$ and $H$ have
the discrete topology (as the subspace topologies from $G$) and so the outer semidirect product
$N \rtimes H$ will also be discrete.  Hence $N\rtimes H\rightarrow G$ is continuous, but the inverse is
not.

In this example, the map $p$ from (\ref{sdp:5}) is given by $p(g,h) = (h,h)$ which is obviously a
projection onto the diagonal $H$ with kernel $N$.  This is not continuous, as there is a net $(h_i)$ in
$K$ which converges to $e$ without eventually being equal to $e$.  Then for any fixed $g$, we have
$(g,h_i) \rightarrow (g,e)$ in $G$ but $(h_i,h_i) \not\rightarrow (e,e)$.  A similar remark applies
to (\ref{sdp:6}), while in (\ref{sdp:4}) we see that $\pi\circ i$ does not have continuous inverse.
\end{remark}

In \cite{kt}, see page~9, in the locally compact case the Haar measure on $N \rtimes H$ is computed.
For each $h\in H$ the measure on $N$ given by $\lambda_N^h(E) = \lambda_N(\varphi(h)(E))$, for each Borel $E$,
is left-invariant, and so there is $\delta(h)>0$ with $\lambda_N^h(E) = \delta(h) \lambda_N(E)$ for
all $E$.  One can check that $\delta:H\rightarrow (\mathbb R^+, \times)$ is a continuous homomorphism,
and then a left Haar measure on $N \rtimes H$ is given by
\[ \int_{N \rtimes H} f(n,h) \ d(n,h) = \int_H \int_N f(n,h) \delta(h)^{-1} \ dn \ dh. 
\qquad (f\in C_{00}(N \rtimes H)). \]
The modular function is
\[ \Delta_{N \rtimes H}(n,h) = \Delta_N(n) \Delta_H(h) \delta(h)^{-1}. \]


\begin{thebibliography}{aa}

\bibitem{q1} Matthew Daws, Empty interior of union of cosets?, 
   \url{https://mathoverflow.net/q/351764}

\bibitem{kt} E. Kaniuth\ and\ K. F. Taylor, {\it Induced representations of locally compact groups}, Cambridge Tracts in Mathematics, 197, Cambridge University Press, Cambridge, 2013. MR3012851

\bibitem{a2} Brian M. Scott (\url{https://math.stackexchange.com/users/12042/brian-m-scott}), The set of ultrafilters on an infinite set is uncountable, \url{https://math.stackexchange.com/q/83540}

\bibitem{a1} YCor (\url{https://mathoverflow.net/users/14094/ycor}), Empty interior of union of cosets?,
   \url{https://mathoverflow.net/q/351846}

\bibitem{a3} YCor (\url{https://mathoverflow.net/users/14094/ycor}), Topological semi-direct products of groups,
   \url{https://mathoverflow.net/q/425559}

\end{thebibliography}


\end{document}
