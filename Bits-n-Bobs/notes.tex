\documentclass[twoside,a4paper,12pt]{article}
\usepackage[margin=2cm]{geometry}

\usepackage{xcolor}
\definecolor{myblue}{rgb}{0.1 0.1 0.6}
% ``backref'' for drafting; see manual for further options
%\usepackage[backref]{hyperref}
%\usepackage{showkeys}
\usepackage{hyperref}
\hypersetup{
   colorlinks=true,
   linkcolor=myblue,
   citecolor=myblue,
   urlcolor=myblue
}

\usepackage{amsmath,amssymb,latexsym,amsthm}
%\usepackage{showkeys}
%\usepackage[active]{srcltx}
\usepackage{xurl}
%\usepackage[all]{xy}
\usepackage{tikz-cd}
\usepackage[shortlabels]{enumitem}

\theoremstyle{plain}
\newtheorem{proposition}{Proposition}[section]
\newtheorem{theorem}[proposition]{Theorem}
\newtheorem{corollary}[proposition]{Corollary}
\newtheorem{lemma}[proposition]{Lemma}
\theoremstyle{definition}
\newtheorem{definition}[proposition]{Definition}
\newtheorem{example}[proposition]{Example}
\newtheorem{remark}[proposition]{Remark}
\newtheorem{remarks}[proposition]{Remarks}
%\newtheorem{question}[proposition]{Question}
%\newtheorem{problem}[proposition]{Problem}

\newcommand{\ip}[2]{\langle #1,#2 \rangle}
\newcommand{\mc}{\mathcal}
\newcommand{\mf}{\mathfrak}
\newcommand{\G}{{\mathbb G}}
\newcommand{\vnten}{\overline\otimes}
\newcommand{\mor}{\operatorname{Mor}}
\newcommand{\tr}{\operatorname{Tr}}
\newcommand{\re}{\operatorname{Re}}
\newcommand{\lin}{\operatorname{lin}}
\newcommand{\op}{{\operatorname{op}}}
\newcommand{\aff}{\operatorname{Aff}}
\newcommand{\id}{\operatorname{id}}
\newcommand{\aut}{\operatorname{Aut}}
\newcommand{\im}{\operatorname{Im}}

\begin{document}

\title{Smaller notes}
\author{Matthew Daws}
\maketitle

\tableofcontents



\section{Unions of cosets}

This is a careful write-up of my question \cite{q1} at math\emph{overflow} together with the
answer \cite{a1}.  The question is:

\begin{quote}
Let $G$ be a locally compact group, and let $K,L$ be cosets of $G$ (not assumed open or closed)
which each have empty interior.  Does also $K\cup L$ have empty interior?
\end{quote}

The answer is ``no''.  The counter-example comes from considering $G = (\mathbb Z/2\mathbb Z
)^{\mathbb N}$ the infinite product of the cyclic group of order 2.  We shall write $0,1$ for the
elements of $\mathbb Z/2\mathbb Z$, so $1+1=0$.  We give $G$ the product topology, so $G$ is compact
and Hausdorff.  We shall write elements of $G$ as infinite sequences $x = (x_n)$ in
$\mathbb Z/2\mathbb Z$.  Notice that $G$ is compact, abelian, and every $x\in G$ satisfies
that $x+x=0$.

The topology has a base of ``cylinder'' sets, given as follows.  Let $n\in\mathbb N$ and $y=(y_1,
y_2,\cdots,y_n)$ a finite sequence in $\mathbb Z/2\mathbb Z$.  Define
\[ \mathcal O_{n,y} = \big\{ x=(x_k)\in G : x_i = y_i \ (i\leq n) \big\}. \]
These sets form a base for the topology on $G$.  Notice that the intersection of two such sets
is of the same form (or is empty).

Furthermore, notice that for $s,t\in\mathbb Z/2\mathbb Z$ either $s=t$ or $s=t+1$.  Then
\begin{align*} G \setminus \mathcal O_{n,y}
&= \big\{ x : \exists\, 1\leq i\leq n, x_i\not=y_i \big\}
= \bigcup_{i=1}^n \big\{ x : x_i\not=y_i \big\} \\
&= \bigcup_{i=1}^n \big\{ x : x_i=y_i+1 \big\}
\end{align*}
which is the union of (many) basic open sets.  Thus $\mathcal O_{n,y}$ is also closed.

Finally, notice that $\mathcal O_{n,y}$ is a subgroup exactly when $y_i=0$ for $i\leq n$, and so
every $\mathcal O_{n,y}$ is a coset in $G$.

\begin{lemma}\label{lem:1}
$G$ has countably many open subgroups.
\end{lemma}
\begin{proof}
Consider a basic open set $\mc O_{n,y}$.  Given $x,z\in \mc O_{n,y}$, as $x$ and $z$ agree in the
first $n$ coordinates, we see that $(x+z)_k = 0$ for $k\leq n$.  It follows that $\mc O_{n,y}
+ \mc O_{n,y} = \mc O_{n,0}$ an open subgroup.

Now let $H$ be an arbitrary open subgroup, so we can write $H$ as some union of basic open sets.
Let $n$ be minimal with $\mc O_{n,y}\subseteq H$ for some $y$.  Thus $\mc O_{n,0}\subseteq H$
as $H$ is a subgroup.  Any open basic open set $\mc O_{m,z}\subseteq H$ must have $n\leq m$,
and so we see that $\mc O_{m,z} + \mc O_{n,0} = \mc O_{n,z} \subseteq H$.

As $H$ is the union of basic open sets, we conclude that there are finitely many $x_1,\cdots, x_k$
with $H = \bigcup_{i=1}^k \mc O_{n,x_i}$.  Let $y_i \in (\mathbb Z / 2\mathbb Z)^n$ be the
projection of $x_i$ onto the first $n$ coordinates.  As $H$ is a subgroup, it follows that
$\{y_i : 1\leq i \leq k\}$ is a subgroup of $(\mathbb Z / 2\mathbb Z)^n$, say $K$.  Furthermore,
$H$ is exactly the collection of all $x\in G$ such that the projection of $x$ onto the first
$n$ coordinates is in $K$.

It follows that open subgroups of $H$ can be described by $n\in\mathbb N$ and a subgroup $K$ of
$(\mathbb Z / 2\mathbb Z)^n$.  There are only countably many such choices.
\end{proof}

\begin{corollary}
There are countably many closed subgroups of $G$ of index $2$.
\end{corollary}
\begin{proof}
Let $H\leq G$ be a closed subgroup of index $2$.  Then $G\setminus H$ is a coset of $H$ and so
is closed, and so $H$ is open.  The result follows.
\end{proof}

We now consider arbitrary subgroups of $G$.  It is instructive to consider the bijection between
$G$ and $\mc P(\mathbb N)$ the power set of $\mathbb N$, given by $x=(x_n)$ mapping to the set
$A\subseteq \mathbb N$ where $n\in A$ if and only if $x_n=0$.  If $x,y\in G$ biject with $A,B$,
respectively, then $x+y$ bijects with $C$ where $n\in C$ if and only if $x_n+y_n=0$, that is,
$x_n=y_n=0$ or $x_n=y_n=1$, that is, $n\in A\cap B$ or $n\in \mathbb N \setminus (A\cup B)$.
Thus $C = (A\cap B) \cup (\mathbb N \setminus (A\cup B)) = \mathbb N \setminus (A \triangle B)$.

We recall the notion of a \emph{filter} on $\mathbb N$.  This is a subset $\mc F\subseteq
\mc P(\mathbb N)$ with $\emptyset\not\in\mc F$, with, if $A\in\mc F$ and $A\subseteq B$ then also
$B\in\mc F$, and with $A,B\in\mc F\implies A\cap B\in\mc F$.  An \emph{ultrafilter} $\mc U$
is a maximal filter; alternatively, $\mc U$ is a filter with the property that if
$A\in\mc P(\mathbb N)$ then either $A\in\mc U$ or $\mathbb N\setminus A\in\mc U$.

\begin{lemma}
There are $2^{\mf{c}}$ subgroups of index $2$ in $G$.
\end{lemma}
\begin{proof}
Let $\mc U$ be an ultrafilter, and let $H\subseteq G$ be the associated subset.  Given $A,B\in\mc U$
consider $C=\mathbb N \setminus (A\triangle B)$.  Then $A\cap B \subseteq C$ and so $C\in\mc U$.
It follows that $H$ is a subgroup of $G$.  Furthermore, given $A\not\in\mc U$ we know that
$\mathbb N\setminus A\in\mc U$.  Thus if $x\not\in H$ then $1+x\in H$, and as $\emptyset\not\in
\mc U$ also $1\not\in H$.  Thus $H$ is proper, and $G$ is the union of $H$ and $1+H$, so $H$ has
index $2$.

It is well-known (see for example \cite{a2}) that there are $2^c$ ultrafilters on $\mathbb N$,
and so there are (at least) $2^\mf{c}$ subgroups of index $2$ in $G$.  As $G$ bijects with
$\mc P(\mathbb N)$ we have $|G| = 2^{\aleph_0} = \mf{c}$ and so $|\mc P(G)| = 2^{\mf{c}}$.
Thus there are at most $2^{\mf{c}}$ subgroups of any index.
\end{proof}

There hence exists a subgroup $H$ of index $2$ which is not closed.  (In fact this follows more
directly from the existence of non-principle ultrafilters, and the proof of Lemma~\ref{lem:1}.)
Thus $H$ is not open, and so cannot contain any non-empty open set (if $\emptyset\not=
U\subseteq H$ is open then using the group operations we can cover $H$ by translates of $U$ which
shows that $H$ is open, contradiction).  As $G\setminus H$ is a coset of $H$ it follows that
$G\setminus H$ cannot contain any non-empty open set.  Thus $H$ is dense in $G$.  We have also
now answered our original question, as both $H$ and its coset have empty interior, and yet their
union is all of $G$.





\clearpage

\section{Semi-direct products}

This is standard material.  Let $G$ be a group with a subgroup $H$ and a normal subgroup $N$.
The following statements are equivalent:
\begin{enumerate}
\item\label{sdp:1} $G=NH = \{ nh : n\in N, h\in H \}$ and $N \cap H = \{e\}$;
\item\label{sdp:2} for each $g\in G$ there are unique $n\in N, h\in H$ with $g=nh$;
\item\label{sdp:3} for each $g\in G$ there are unique $n\in N, h\in H$ with $g=hn$;
\item\label{sdp:4} for the inclusion $i:H\rightarrow G$ and the quotient $\pi:G\rightarrow G/N$,
the composition $\pi\circ i:H\rightarrow G/N$ is an isomorphism;
\item\label{sdp:5} there is a homomorphism $G\rightarrow H$ that is the identity on $H$ and has
kernel $N$;
\item\label{sdp:6} there is a split short-exact sequence
$1 \rightarrow N \rightarrow G \rightarrow H \rightarrow 1$.
\end{enumerate}

Let us show these equivalences.  If (\ref{sdp:1}) holds then $G=NH$ but if there was a possibly
non-unique way to
write $g = n_1h_1 = n_2h_2$ then $n_2^{-1} n_1 = h_2 h_1^{-1} \in N\cap H=\{e\}$ so $n_1=n_2$ and
$h_1=h_2$, so (\ref{sdp:2}) holds.  Conversely, the uniqueness clause in (\ref{sdp:2}) shows that
if $g\in N\cap H$ then we must have $g=e$.  So (\ref{sdp:1})$\Leftrightarrow$(\ref{sdp:2}).

That (\ref{sdp:1})$\Leftrightarrow$(\ref{sdp:3}) is similar, using the group inverse to show that
$G=NH$ if and only if $G = G^{-1} = H^{-1} N^{-1} = HN$.

Considering (\ref{sdp:4}), $\ker(\pi\circ i) = N\cap H$ and the image of $\pi\circ i$ is $\{Nh :h\in H\}
= \{ Nnh : n\in N, h\in H \}$, so immediately (\ref{sdp:1}) implies (\ref{sdp:4}), while conversely,
if $\pi\circ i$ is onto, then given $g\in G$ there is $h\in H$ with $Nh = Ng$ so $g\in Nh \subseteq NH$
and hence (\ref{sdp:1}) holds.  So (\ref{sdp:1})$\Leftrightarrow$(\ref{sdp:4}).

For (\ref{sdp:5}) consider $p:G\rightarrow H$ a homomorphism with $p(h)=h$ for $h\in H$ and $\ker(p)=N$.
Then $N\cap H=\{e\}$ while for $g\in G$ let $h=p(g)\in H$ so $p(h g^{-1}) = h p(g)^{-1} = e$ and hence
$h g^{-1} = n$ for some $n\in N$, so $g = n^{-1} h \in NH$, so (\ref{sdp:1}) holds.
Conversely, given (\ref{sdp:1}), also (\ref{sdp:4}), let $p:G\rightarrow H$ be the composition of
$(\pi\circ i)^{-1} \circ \pi : G\rightarrow H$.  Then $p$ is a homomorphism, $\ker(p) = \ker(\pi) = N$,
and for $h\in H$ we have $p(h) = h_1$ where $h_1\in H$ is the necessarily unique element with
$h_1N = \pi(h) = hN$, so $h_1=h$ by uniqueness.  Hence (\ref{sdp:5}) holds.

For (\ref{sdp:6}) we have
\begin{equation}
\begin{tikzcd}
1 \arrow[r] &
N \arrow[r, hook, "\iota"] &
G \arrow[r, twoheadrightarrow, shift left=0.6ex, "\pi"] &
H \arrow[r] \arrow[l, dotted, shift left=0.6ex, "\theta"] & 1
\end{tikzcd}
\label{eq:sdp:1}
\end{equation}

where $\iota$ is an inclusion, $\pi$ a surjection with $\ker(\pi) = \iota(H)$, and we have
$\pi\circ\theta = \id_H$.  We identify $N$ with its image in $G$ which is certainly normal, and then
$\pi$ gives an isomorphism between $G/N$ and $H$.  That $\pi\circ\theta=\id_H$ means $\theta$ is injective,
so we identify $H$ with its image in $G$.  Then $p = \theta\circ\pi$ is an idempotent homomorphism,
which is the identity on $H$, and has $\ker(p) = \ker(\pi) = N$, so (\ref{sdp:5}) holds.
Conversely, given (\ref{sdp:5}), we have inclusions $N\rightarrow G$ and $H\rightarrow G$ giving
$\iota$ and $\theta$.  Consider the map $p:G\rightarrow H$, and set $\pi=p$.  Then $\ker(\pi) = \ker(p)
= N$ while $\pi$ is onto, and $\pi\circ\theta = \id_H$, so (\ref{sdp:6}) holds.

Any of these equivalent conditions define what it means for $G$ to be the \emph{(inner) semidirect
product} denoted $G = N \rtimes H$ (sometimes $G = H \ltimes N$).

\begin{definition}
The \emph{(outer) semidirect product} of groups $N,H$ is defined by specifying a homomorphism
$\varphi:H\rightarrow \aut(N)$ and setting $G = N\times H$ as a set, with product
\[ (n_1,h_1)(n_2,h_2) = (n_1 \varphi(h_1)(n_2), h_1h_2). \]
The identity of $G$ is $(e_N, e_H)$ and the inverse is $(n,h)^{-1} = (\varphi(h^{-1})(n^{-1}), h^{-1})$.
\end{definition}

Given such a construction, we identify $N$ with $\{ (n,e_H) : n\in N \}$ and $H$ with $\{ (e_N,h):h\in H\}$.
These are seen to be subgroups of $G$.  Then $NH = \{ (n,e)(e,h) : n\in N, h\in H \} = G$ and $N\cap H=\{e\}$.
Finally, the product satisfies
\[ h n h^{-1} = (e,h)(n,e)(e,h)^{-1}
= (\varphi(h)(n), h) (e,h^{-1})
= (\varphi(h)(n) \varphi(h)(e), e) = \varphi(h)(n) \in N, \]
for each $n\in N, h\in H$.  Thus $N$ is a normal subgroup and we have verified condition (\ref{sdp:1}).

Conversely, if we have an inner semidirect product $G=NH$ with $N \unlhd G$, then define $\varphi(h)(n)
= hnh^{-1} \in N$, so that $\varphi(h)$ is an automorphism of $N$ for each $h$, and $\varphi:H\rightarrow
\aut(N)$ is an homomorphism.  We define $\theta: G \rightarrow N \times H$ by $\theta(nh) =
(n,h)$.  As a slight aside, notice that this map is a homomorphism if and only if $\theta(h) \theta(n) =
\theta(hn) = \theta( (hnh^{-1}) h ) = \theta( \varphi(h)(n) h )$, which is true by definition.
This ``commutation relation'', showing how elements of $N$ and $H$ commute pass each other, is often
a useful way to think of a semi-direct product.

Thus inner and outer semidirect products are canonically isomorphic.

Finally, we claim that $G$ is isomorphic to $N \rtimes H$ if and only if there is a split short exact
sequence
\[ \begin{tikzcd}
1 \arrow[r] &
N \arrow[r, hook, "\beta"] &
G \arrow[r, twoheadrightarrow, shift left=0.6ex, "\alpha"] &
H \arrow[r] \arrow[l, dotted, shift left=0.6ex, "\gamma"] & 1
\end{tikzcd} \]
This is just condition (\ref{sdp:6}).  Notice 
\[ \varphi(h)(n) = hnh^{-1} = \beta^{-1}\big( \gamma(h) \beta(n) \gamma(h^{-1}) \big) \]
gives the action of $H$ on $N$.


\subsection{For topological groups}

When $G$ is a topological group, it seems natural to consider the continuous automorphisms of $N$,
and to ask for $\varphi:H\rightarrow N$ to be continuous.  Forming the (outer) semidirect product, we
form the topological product $N \times H$ and the product as above.  We require that $H\times N
\rightarrow N; (h,n) \mapsto \varphi(h)(n)$ is (jointly) continuous, and then we see that the
product on $N \rtimes H$ is (jointly) continuous.  The inverse is continuous, as it is the composition
of $(h,n) \mapsto (h^{-1}, n^{-1})$ with the continuous action map.

Then $N, H$ are closed subgroups of $N\rtimes H$.  Conversely, start with $G=NH$ for some closed
$N,H$ with $N \unlhd G$.  Then in (\ref{sdp:4}) while $i,\pi$ are continuous, so $\pi\circ i$ is continuous,
we require that $(\pi\circ i)^{-1}$ be continuous as well.  This is equivalent to $p$ in (\ref{sdp:5})
being continuous.  For (\ref{sdp:6}), in the split short exact sequence \eqref{eq:sdp:1}, we have that
$\iota,\pi$ are continuous, and we also require that $\theta$ is continuous.  Then $\theta$ is injective,
and furthermore is a homeomorphism onto its range, because $\pi\circ\theta=\id_H$.

It does not seem obvious what conditions we would like to add to (\ref{sdp:1}), (\ref{sdp:2}) or (\ref{sdp:3})
to ensure the correct continuity conditions.  However, notice that given $G=NH$ with $N\unlhd H$, with
$N, H$ closed, we can still define $\varphi(h)(n) = hnh^{-1}$, and continuity of the product in $G$ will
ensure that $\varphi(h)$ is continuous for $h$, and that $H\times N \rightarrow N;
(h,n) \mapsto \varphi(h)(n)$ is continuous.  Thus we can form the outer semidirect product.  The canonical
isomorphism is $G \rightarrow N \times H, nh \mapsto (n,h)$.  The inverse of this is always continuous,
but this map itself might not be.  It is continuous exactly when $(n_i h_i)$ is a net converging to
$e$, we must have $n_i\rightarrow e$ and $h_i\rightarrow e$.  If $G$ is locally compact, this fails to
happen exactly when we can find $(n_i) \subseteq N, (h_i)\subseteq H$ with $n_i\rightarrow \infty,
h_i\rightarrow\infty$ and yet $n_ih_i\rightarrow e$.

\begin{remark}
It is implicitly claimed on page 10 of \cite{kt} that in this setting, the map from the outer
semidirect product $N \rtimes H$ to $G$ is an isomorphism of locally compact groups.  The following
counter-example, from \cite{a3}, shows that this need not be the case.  Let $K$ be an infinite compact
group and let $K_d$ be $K$ with the discrete topology, so $K\not=K_d$ as topological groups.  Let
$G = K_d \times K$ the direct product of groups, so $G$ is locally compact.  Let $N = K_d \times \{e\}$
a closed subgroup, and let $H = \{ (g,g) : g\in K \} \subseteq G$ the diagonal, which is also a closed,
normal subgroup.  Then it's easy to see that $N\cap H=\{e\}$ and $NH = G$.  However, both $N$ and $H$ have
the discrete topology (as the subspace topologies from $G$) and so the outer semidirect product
$N \rtimes H$ will also be discrete.  Hence $N\rtimes H\rightarrow G$ is continuous, but the inverse is
not.

In this example, the map $p$ from (\ref{sdp:5}) is given by $p(g,h) = (h,h)$ which is obviously a
projection onto the diagonal $H$ with kernel $N$.  This is not continuous, as there is a net $(h_i)$ in
$K$ which converges to $e$ without eventually being equal to $e$.  Then for any fixed $g$, we have
$(g,h_i) \rightarrow (g,e)$ in $G$ but $(h_i,h_i) \not\rightarrow (e,e)$.  A similar remark applies
to (\ref{sdp:6}), while in (\ref{sdp:4}) we see that $\pi\circ i$ does not have continuous inverse.
\end{remark}

In \cite{kt}, see page~9, in the locally compact case the Haar measure on $N \rtimes H$ is computed.
For each $h\in H$ the measure on $N$ given by $\lambda_N^h(E) = \lambda_N(\varphi(h)(E))$, for each Borel $E$,
is left-invariant, and so there is $\delta(h)>0$ with $\lambda_N^h(E) = \delta(h) \lambda_N(E)$ for
all $E$.  One can check that $\delta:H\rightarrow (\mathbb R^+, \times)$ is a continuous homomorphism,
and then a left Haar measure on $N \rtimes H$ is given by
\[ \int_{N \rtimes H} f(n,h) \ d(n,h) = \int_H \int_N f(n,h) \delta(h)^{-1} \ dn \ dh. 
\qquad (f\in C_{00}(N \rtimes H)). \]
The modular function is
\[ \Delta_{N \rtimes H}(n,h) = \Delta_N(n) \Delta_H(h) \delta(h)^{-1}. \]





\clearpage

\section{Modules over algebras of operators}

The following is surely folklore: people will know the construction, but I'm not aware of a reference.

We consider homomorphisms $\pi \colon \mc B(E) \to \mc B(X)$ for Banach spaces $E$ and $X$.  Any such $\pi$ restricts to a homomorphism $\pi \colon \mc F(E) \to \mc B(X)$, where $\mc F(E)$ is the finite-rank operators.  For $e\in E, e^*\in E^*$ we write $e\otimes e^*$ for the rank-one operator $E \ni f \mapsto \ip{e^*}{f} e$.

Choose $e_0\in E, e_0^*\in E^*$ with $\ip{e_0^*}{e_0} = 1$, so that $e_0 \otimes e_0^*\in\mc F(E)$ is an idempotent, and hence so too is $\pi(e_0\otimes e_0^*)$.  Set
\[ Y_0 = \{ x\in X : \pi(e_0\otimes e_0^*)x = x \}, \]
a closed subspace (as the range of an idempotent) in $X$.  Define $\phi \colon E\otimes Y_0 \to X$ by $e\otimes y \mapsto \pi(e\otimes e_0^*)y$.

\begin{lemma}
$\phi$ is an injective map, the image of which is a $\mc B(E)$-submodule of $X$.  Infact, the image of $\phi$ is $\lin \{ \pi(T)x : x\in X, T\in\mc F(E) \}$.
\end{lemma}
\begin{proof}
Given $u\in E\otimes Y_0$ we can find linearly independent $e_1,\cdots,e_n$ in $E$ with $u = \sum_{i=1}^n e_i \otimes y_i$ for some $y_i$ in $Y$.  Then there are $e_j^*\in E^*$ with $\ip_{e_j^*}{e_i} = \delta_{i,j}$.  Suppose $\phi(u)=0$ we see that
\[ 0 = \pi(e_0 \otimes e_j^*) \sum_{i=1}^n \phi(e_i \otimes y_i)
= \sum_{i=1}^n \pi(e_0 \otimes e_j^*) \pi(e_i \otimes e_0^*) y_i
= \pi(e_0 \otimes e_0^*) y_j = y_j, \]
as $y_j \in Y_0$, and using that $\pi$ is a homomorphism.  Hence $y_j=0$ for all $j$ and so $u=0$, showing that $\phi$ is injective.

Given $T\in\mc B(E)$ and $\phi(e\otimes y)$ we see that $\pi(T) \phi(e\otimes y) = \pi(T) \pi(e \otimes e_0^*) y = \pi(T(e)\otimes e_0^*) y = \phi(T(e)\otimes y)$, and so the image of $\phi$ is a $\mc B(E)$-submodule.

The space which we claim equals the image of $\phi$ is $\lin\{ \pi(e\otimes e^*)x : x\in X, e\in E, e^*\in E^* \}$.  Clearly the image of $\phi$ is contained in this.  Conversely, given $x' = \pi(e\otimes e^*)x$, set $y = \pi(e_0 \otimes f^*)x'$ where $f^*\in E^*$ is chosen so that $\ip{f^*}{e}=1$ (such a functional exists via Hahn--Banach).  Then $y = \pi(e_0 \otimes e^*)x$ and so $\pi(e_0\otimes e_0^*)y = y$, hence $y\in Y_0$.  Then $\phi(e\otimes y)= \pi(e\otimes e_0^*)y = \pi(e\otimes e_0^*) \pi(e_0 \otimes e^*)x = \pi(e\otimes e^*)x = x'$ and so $x'$ is in the image of $\phi$, as required.
\end{proof}

In particular, the image of $\phi$ does not depend upon the choice of $e_0, e_0^*$.  Let's explore this a little more: suppose we also have $\ip{e_1^*}{e_1} = 1$ and analogously define $Y_1$ and $\phi_1$.  Define $\alpha_{1,0} \colon Y_0 \to Y_1$ by $y \mapsto \pi(e_1\otimes e_0^*)y$.  It's easy to see that $\alpha_{1,0}$ does map into $Y_1$.  Analogously define $\alpha_{0,1}$.  For $y\in Y_0$ we see that
\[ \alpha_{0,1} \alpha_{1,0} y = \pi(e_0\otimes e_1^*)\pi(e_1\otimes e_0^*)y
= \pi(e_0\otimes e_0^*)y = y, \]
and similarly, $\alpha_{1,0} \alpha_{0,1}$ is the identity on $Y_1$.  Furthermore, for $y\in Y_0$,
\[ \phi_1(e \otimes \alpha_{1,0}y)
= \pi(e\otimes e_1^*) \alpha_{1,0}y
= \pi(e\otimes e_1^*) \pi(e_1\otimes e_0^*) y
= \pi(e\otimes e_0^*) y = \phi(e\otimes y). \]
So we have explicitly found intertwiners between the different maps.

Of course, this construction only tells us about homomorphisms $\mc F(E) \to \mc B(X)$.  If we have a homomorphism of the generalised Calkin algebra $\mc B(E) / \mc K(E)$ then we learn nothing.

When $\mc B(X)$ is an essential $\mc A(E)$-module, the map $\phi$ realises $X$ as the completion of $E \otimes Y_0$.  The action of $\mc A(E)$ (or $\mc B(E)$) is just the standard action on the $E$ tensor factor.

If $S\in\mc B(X)$ commutes with each $\pi(T)$, for $T\in\mc A(E)$, then $y\in Y_0$ implies $Sy\in Y_0$ and then $\phi(e\otimes Sy) = S\phi(e\otimes y)$.  Letting $S_0$ be the restriction of $S$ to $Y_0$, if we view $X$ as the completion of $E\otimes Y_0$, then $1\otimes S_0$ is the operator $S$.  However, in general there seems no reason to suspect that every member of $\mc B(Y_0)$ occurs as some $S_0$.





\clearpage

\section{Polar decompositions}

This material can be found in a variety of sources, but I wanted something in my own presentation to reference.  Fix Hilbert spaces $H,K$.

Let $T\in\mc B(H,K)$ and use the continuous functional calculus to define $|T| = (T^*T)^{1/2} \in \mc B(H)$.  This is the unique positive operator $S$ with $S^2 = T^*T$.
We start with some elementary facts about operators on Hilbert spaces.

\begin{lemma}\label{lem:hilb_basic}
Let $S\in\mc B(H,K)$.  Then:
\begin{enumerate}[(1)]
   \item\label{lem:hilb_basic:1}
   $\im(S)^\perp = \ker S^*$ and so $\overline{\im}(S) = (\ker S^*)^\perp$.
   \item\label{lem:hilb_basic:2}
   $\ker S = \ker S^*S = \ker |S|$.
   \item\label{lem:hilb_basic:3}
   $S$ is an isometry if and only if $S^*S=1$.
\end{enumerate}
\end{lemma}
\begin{proof}
We see that $\xi\in\im(S)^\perp$ if and only if $(\xi|S\eta)=0$ for all $\eta$, if and only if $S^*\xi=0$, so \ref{lem:hilb_basic:1} follows.  For \ref{lem:hilb_basic:2} we note that
\[ S\xi=0 \implies
S^*S\xi=0 \implies
(\xi|S^*S\xi)=0 \implies
\|S\xi\|^2=0 \implies
S\xi=0, \]
and so we have equivalence throughout.  As $|S|^*|S| = S^*S$ also $\ker |S| = \ker S^*S$.

For \ref{lem:hilb_basic:3}, if $S^*S=1$ then $\|S\xi\|^2 = (\xi|S^*S\xi) = \|\xi\|^2$ for each $\xi\in H$ and so $S$ is an isometry.  For the converse, we use the polarisation identity, so for $\xi, \eta\in H$,
\[ (S\xi|S\eta) = \frac14 \sum_{k=0}^3 i^k \|S\xi + (-i)^k S\eta\|^2 
= \frac14 \sum_{k=0}^3 i^k \|\xi + (-i)^k \eta\|^2 
= (\xi|\eta), \]
in the middle step using that $S$ is an isometry.  It follows that $S^*S=1$.
\end{proof}

Next we look at partial isomeries in the abstract.

\begin{lemma}\label{lem:pi}
Let $A$ be a $C^*$-algebra and let $u\in A$.  The following are equivalent:
\begin{enumerate}[(1)]
   \item\label{lem:pi:1} $u^*u$ is a projection;
   \item\label{lem:pi:2} $uu^*u = u$;
   \item\label{lem:pi:3} $u^*uu^* = u^*$;
   \item\label{lem:pi:4} $uu^*$ is a projection.
\end{enumerate}
\end{lemma}
\begin{proof}
If \ref{lem:pi:1} holds, then $u^*u u^*u = u^*u$ and so $(uu^*u-u)^*(uu^*u-u) = u^*uu^*uu^*u - u^*uu^*u - u^*uu^*u + u^*u = 0$, so \ref{lem:pi:2} holds, by the $C^*$-condition (as $a^*a=0 \implies a=0$ for $a\in A$).  If \ref{lem:pi:2} holds then multiply on the left by $u^*$ to see that $u^*u$ is idempotent; clearly $u^*u$ is self-adjoint, and so \ref{lem:pi:1} holds.  Replacing $u$ by $u^*$ shows that \ref{lem:pi:3} and \ref{lem:pi:4} are equivalent.  Finally, \ref{lem:pi:2} and \ref{lem:pi:3} are equivalent by taking adjoints.
\end{proof}

\begin{definition}
A \emph{partial isometry} is $u\in A$ satisfying any of the above equivalent conditions.  The projection $u^*u$ is the \emph{initial projection} and $uu^*$ is the \emph{final projection}.
\end{definition}

\begin{lemma}\label{lem:pi_hilbsp}
For $U\in\mc B(H,K)$ we have that $U$ is a partial isometry if and only if there is a closed subspace $H_0\subseteq H$ such that $U$ restricted to $H_0$ is an isometry, and $U$ restricted to $H_0^\perp$ is $0$.  In this case, $H_0$ is the image of the initial projection, and $U(H_0)$ is the image of the final projection.
\end{lemma}
\begin{proof}
Let $U$ be a partial isometry, and let $H_0$ be the image of the initial projection $U^*U$.  For $\xi = U^*U\xi \in H_0$, we have $\|U\xi\|^2 = (\xi|U^*U\xi) = (\xi|\xi) = \|\xi\|^2$, so conclude that $U$ is an isometry on $H_0$.  By Lemma~\ref{lem:hilb_basic}\ref{lem:hilb_basic:2}, we have $\ker U = \ker U^*U = H_0^\perp$.

Conversely, the restriction of $U$ to $H_0$ is an isometry, and so $(U\xi|U\eta) = (\xi|\eta)$ for all $\xi,\eta\in H_0$, see Lemma~\ref{lem:hilb_basic}\ref{lem:hilb_basic:3}.  Let $\xi\in H$ and write $\xi = \xi_0 + \xi_1 \in H_0 \oplus H_0^\perp$, and similarly for $\eta$, so that $(\eta|U^*U\xi) = (U\eta|U\xi) = (U\eta_0|U\xi_0) = (\eta_0|\xi_0) = (\eta|\xi_0)$.  Thus $U^*U\xi = \xi_0$, and so $U^*U$ is the projection onto $H_0$.  Hence $U$ is a partial isometry with initial space $H_0$.  Obviously $\im(UU^*) \subseteq \im(U)$, but $UU^*U = U$ so $UU^*\xi=\xi$ for any $\xi\in U(H_0)$, and we conclude that $\im(UU^*) = U(H_0)$.
\end{proof}

For a partial isometry $U$ on a Hilbert space, the \emph{initial space} is the range of the projection $U^*U$ and the \emph{final space} is the range of $UU^*$.

We can now construct the polar decomposition.  Given $T\in\mc B(H,K)$ form $|T|=(T^*T)^{1/2}$, and define
\[ U \colon |T|(H) = \{ |T|\xi : \xi\in H\} \to K; \quad
U |T|\xi = T\xi. \]
For $\xi\in H$ we have that $\|T\xi\|^2 = (\xi|T^*T\xi) = \| |T|\xi\|^2$ and so $U$ is an isometry, and hence extends by continuity to the closure of $|T|(H)$.  Set $U$ to be $0$ on the orthogonal complement of $|T|(H)$, so $U$ is now defined on all of $H$.  By construction, $U|T|= T$, and by Lemma~\ref{lem:pi_hilbsp}, $U$ is a partial isometry with initial space $\overline{\im}|T|$ and final space $\overline{\im}T$.  By Lemma~\ref{lem:hilb_basic}, $\overline{\im}|T| = (\ker |T|)^\perp = (\ker T)^\perp = \overline{\im}(T^*)$ and $\overline{\im}T = (\ker T^*)^\perp$.  Also notice that $\ker U = (\im |T|)^\perp = \ker T$.

\begin{proposition}\label{prop:polar_unique}
The polar decomposition is unique in the sense that if $T = VS$ with $S$ positive and $V$ a partial isometry with initial space $\overline{\im} S$ then $U=V$ and $S = |T|$.  We have that $|T|$ belongs to the $C^*$-algebra generated by $T^*T$, and when $H=K$, $U$ belongs to the von Neumann algebra generated by $T$.
\end{proposition}
\begin{proof}
We have $T^*T = S V^*V S = S^2$ as $V^*V$ is the projection onto $\overline{\im} S$, and so by uniqueness of positive square-roots, $|T| = S$.  Then $U|T| = V|T|$ so $U$ and $V$ agree on $\overline{\im} |T|$ which is the initial space of both partial isomeries, and hence $U=V$.

By the continuous functional calculus, $|T| \in C^*(T^*T)$.  When $H=K$, the von Neumann algebra generated by $T$ is the bicommutant $\{ T, T^* \}''$.  Let $S$ commute with $T$ and $T^*$; we need to show that $SU=US$.  Let $\eta \in \ker U = \ker T$ so $TS\eta = ST\eta = 0$ so $S\eta\in \ker T$ so $US\eta=0$.  As $S$ and $|T|$ commute, for $\xi\in H, \eta\in \ker U$, we have
\[ SU(|T|\xi+\eta) = ST\xi = TS\xi = U|T|S\xi = US(|T|\xi+\eta). \]
As $|T|(H) + \ker U$ is dense in $H$, we conclude that $SU=US$, as required.
\end{proof}

Suppose now that $T\in\mc B(H)$ is self-adjoint, so we can write $T = T_+ - T_-$ for some $T_+,T_-$ positive with $T_+ T_- = 0 = T_- T_+$.  Let $H_\pm = \overline{\im}T_\pm$ so $H_+$ and $H_-$ are mutually orthogonal as $(T_+\xi|T_-\eta) = (\xi|T_+T_-\eta) = 0$ for all $\xi,\eta\in H$.  Let $U$ be the operator which is $1$ on $H_+$, $-1$ on $H_-$ and $0$ on $(H_+ \oplus H_-)^\perp$.  Then $U = U^*$ and $U^*U$ is the projection onto $H_+ \oplus H_-$, so $U$ is a partial isometry.  As $\overline{\im}T = (\ker T)^\perp = (H_+ \oplus H_-)^{\perp\perp}$ we see that $U^*U$ is the projection onto $\overline{\im}T$.  As $|T| = T_+ + T_-$ (by uniqueness, or functional calculus) we see that $U|T| = UT_+ + UT_- = T_+ - T_- = T$ and so by uniqueness, we have constructed the polar decomposition.





\clearpage

\section{von Neumann regular elements}

Again, we present some surely well-known results, with proofs, for applications to follow in the next section.

\begin{definition}
Let $A$ be an algebra (or even just a ring).  An element $x\in A$ is \emph{von Neumann regular} if there is $y\in A$ with $xyx = x$.
\end{definition}

\begin{proposition}
Let $E$ be a Banach space and let $x\in \mc B(E)$.  The following are equivalent:
\begin{enumerate}
   \item
   $x$ is von Neumann regular in $\mc B(E)$;
   \item 
   $x$ has closed image and both $\ker(x)$ and $\im(x)$ are complemented subspaces of $E$.
\end{enumerate}
In this case, when $xyx=x$, we have that $xy = e$ is a projection onto $\im(x)$ and $1-yx = f$ is a projection onto $\ker(x)$.  Conversely, given $e$ and $f$ projections onto $\im(x)$, which is closed, and $\ker(x)$, respectively, we can choose $y$ with $xyx=x, xy=e$ and with $yx=1-f$.
\end{proposition}
\begin{proof}
Suppose there is $y\in\mc B(E)$ with $xyx=x$.  Then $(xy)^2 = xyxy = xy$ and $(yx)^2 = yxyx = yx$ so $e=xy$ and $f=1-yx$ are projections.  We have $e(E) \subseteq x(E)$, but as $ex = xyx = x$ also $e(H) \supseteq x(E)$, so $x(E)=e(E)$ and in particular, $x$ has closed, complemented image.  As $xf = x - xyx = 0$ we have $f(E) \subseteq \ker(x)$, but if $x\xi=0$ then $f\xi = \xi - yx\xi = \xi$ so we conclude that $f$ is a projection onto $\ker(x)$.

Conversely, let $x$ have closed image, with $e$ a projection onto $x(E)$ and $f$ a projection onto $\ker(x)$.  Let $x_0$ be the restriction of $x$ to a map $\im(1-f) \to \im(e) = \im(x)$.  If $x_0\xi=0$ then $(1-f)\xi=\xi$ so $f\xi=0$, but also $x\xi=0$, so $f\xi=\xi$, so $\xi=0$ and we conclude that $x_0$ is injective.  Given any $\xi\in E$ we have that $xf\xi=0$ so $x\xi = x(1-f)\xi$ and hence $x_0(1-f)\xi = x\xi$ and we conclude that $x_0$ is surjective.  By the Open Mapping Theorem, $x_0$ is invertible.  Set $y = x_0^{-1} e \colon E \to \im(1-f) \subseteq E$.  Then $xy = x x_0^{-1} e = x_0 x_0^{-1} e = e$.  Given $\xi\in E$ set $\eta = (1-f)\xi$ so as before, $x\xi = x\eta = x_0\eta$ as $\eta\in\im(1-f)$.  Hence $yx\xi = x_0^{-1} ex\xi = x_0^{-1} x\xi = x_0^{-1} x_0 \eta = \eta = (1-f)\xi$.  So $yx = 1-f$.  Finally, we have $xyx = ex = x$, as desired.
\end{proof}

For the following, compare also Proposition~\ref{prop:ws_vnreg} below.

\begin{proposition}\label{prop:vnreg_hilb}
Let $H$ be a Hilbert space, and let $x\in \mc B(H)$.  Then $x$ is von Neumann regular if and only if $x$ has closed image.  When $x$ is von Neumann regular and self-adjoint, we can choose a self-adjoint $y$ with $xyx=x$ and with $xy = yx$ a (self-adjoint) projection.  Further, when $x$ is positive, we can choose $y$ positive.
\end{proposition}
\begin{proof}
In a Hilbert space, all closed subspaces are complemented, so $x$ is von Neumann regular if and only if $x$ has closed image.  Now let $x$ be self-adjoint with closed image.  As $\im(x) = (\ker x)^\perp$, Lemma~\ref{lem:hilb_basic}, we can choose $e$ to be the orthogonal projection onto $\im(x)$, and then $f=1-e$ will be the orthogonal projection onto $\ker(x)$.  Following the previous proof, let $x_0$ be the restriction of $x$ to $\im(1-f) = \im(e) = \im(x)$, so $x_0 \colon \im(x) \to \im(x)$ is invertible, and set $y = x_0^{-1} e$.  Then $xy=e, yx=1-f = e$.  As $x$ is self-adjoint, also $x_0$ is self-adjoint, and so $x_0^{-1}$ is self-adjoint.  As $y = ey = ex_0^{-1}e$, we conclude that $y$ is self-adjoint.  The same argument shows that when $x$ is positive, also $y$ is positive.
\end{proof}





\clearpage

\section{Well-supported elements}

This definition is given in \cite[Definition~II.3.2.8]{blackadar_enc}.  An element $x$ in a $C^*$-algebra $A$ is \emph{well-supported} if $\sigma(x^*x) \setminus\{0\}$ is closed, that is, $\sigma(x^*x) \subseteq \{0\} \cup [\epsilon,\infty)$ for some $\epsilon>0$.  I haven't found an reference for elementary properties of such elements, so here give some elementary proofs, following Blackadar for the results.

As $\sigma(x^*x) \cup\{0\} = \sigma(xx^*)\cup\{0\}$, we see that $x$ is well-supported if and only if $x^*$ is.  Let $x$ be well-supported.  Define $f\colon [0,\infty) \to [0,1]$ by $f(0)=0$ and $f(t)=1$ for $t>0$.  Then $f$ is continuous on $\sigma(x^*x)$ and so by the continuous functional calculus, $p = f(x^*x)$ exists, and $p$ is a projection with $px^*x = x^*x = x^*xp$.

\begin{lemma}\label{lem:ws_projs}
Let $x$ be well-supported, and $f$ be as above, and set $p = f(x^*x)$ and $q = f(xx^*)$.  Then $p$ is a projection, $xp=x$ and $xy=0 \implies py=0$.  Also $q$ is a projection, $qx=x$ and $yx=0 \implies yq=0$.
\end{lemma}
\begin{proof}
As observed above, $px^*x = x^*x = x^*xp$ and so $(px^*-x^*)(xp-x) = px^*xp - x^*xp - px^*x + x^*x = 0$ hence $xp = x$.  If $xy = 0$ then $x^*x yy^* = 0$ and $yy^* x^*x = y (xy)^* x = 0$ so working in the commutative $C^*$-algebra generated by $x^*x, yy^*$ and $1$, we see that $p yy^* = 0 = yy^*p$.  Thus $pyy^*p =0$ so $py=0$.

As $x^*$ is also well-supported, $q$ is defined.  We argue similarly: $(qx-x)(qx-x)^* = qxx^*q - xx^*q - qxx^* + xx^* = 0$ so $qx=x$.  If $yx=0$ then $y^*y$ commutes with $xx^*$ so $y^*yq = 0$ so $yq=0$.
\end{proof}

The next lemma says that for a well-supported element, its polar decomposition exists in the $C^*$-algebra.

\begin{lemma}\label{lem:ws_pd}
Let $x\in A$ be well-supported, let $\pi \colon A \to \mc B(H)$ be a $*$-homomorphism, and let $\pi(x) = u\pi(|x|)$ be the polar decomposition.  Then $u\in\pi(A)$.
\end{lemma}
\begin{proof}
Define $g \colon [0,\infty) \to [0,\infty)$ by $g(0)=0$ and $g(t) = t^{-1/2}$, so again $g$ is continuous on $\sigma(x^*x)$, and so we can set $v = x g(x^*x) \in A$.  Then $g(x^*x) |x| = p$ by the functional calculus, where $p$ is as above.  So $v|x| = xp = x$.  Also $v^*v = g(x^*x) x^*x g(x^*x) = p$ again by the functional calculus.

Let $q$ be the projection onto $\overline{\im}(\pi(|x|)) = \ker(\pi(|x|))^\perp = \ker(\pi(x))^\perp = \overline{\im}(\pi(x^*))$, and as $xp=x$ also $px^*=x^*$ so $p\geq q$.  As $(1-q)x^*=0$, so $x(1-q)=0$, by Lemma~\ref{lem:ws_projs}, we have $p(1-q)=0$.  So $p = pq = q$, and we conclude that $v^*v$ is the projection onto $\overline{\im}(\pi(|x|))$.  By uniqueness, Proposition~\ref{prop:polar_unique}, $\pi(x) = \pi(v) \pi(|x|)$ is the polar decomposition.
\end{proof}

We now make links between well-supported elements and von Neumann regular elements, as considered in the previous section.

\begin{proposition}\label{prop:ws_vnreg}
An element $x\in A$ is well-supported if and only if $x$ is \emph{von Neumann regular} meaning that there is $y\in A$ with $xyx=x$.
\end{proposition}
\begin{proof}
Let $x$ be well-supported.  Again define $g(0)=0, g(t)=t^{-1/2}$ for $t>0$, so as in the proof of Lemma~\ref{lem:ws_pd}, with $u = xg(x^*x)$ we have that $x = u|x|$ is the polar decomposition.  In particular, $uu^*u = u$.  Set $y = g(x^*x) u^*$ so that $xyx = x g(x^*x) u^* u |x| = u u^* u |x| = u |x| = x$.

Conversely, let $x$ be von Neumann regular.  Represent $A$ faithfully on some Hilbert space $H$, and let $x = u|x|$ be the polar decomposition, in $\mc B(H)$.  Then $u^*x = |x|$ and so if we choose $y$ with $xyx = x$ then $|x| yu |x| = |x|$, showing that $|x|$ is also von Neumann regular, in $\mc B(H)$.  We now apply Proposition~\ref{prop:vnreg_hilb}, which tells us that $|x|$ has closed image, and we can choose $y \geq 0$ in $\mc B(H)$ with $|x| y |x| = |x|$.  Furthermore, $e = y|x| =  |x|y$ is a projection onto $\im(|x|) = \ker(|x|)^\perp$.

Let $0 < \lambda \leq \|y\|^{-1}$.  We show that $|x|-\lambda 1$ is bounded below.  Let $\xi\in H$ so $|x|\xi = |x|e \xi = e|x|e\xi$.  Hence, as $e$ and $1-e$ has orthogonal images,
\[ \| |x|\xi - \lambda\xi\|^2
= \| |x|e\xi - \lambda e\xi - \lambda (1-e)\xi \|^2
= \| |x|e\xi - \lambda e\xi\|^2 + \lambda^2 \| (1-e)\xi \|^2. \]
We have $\|e\xi\| = \|y|x|\xi\| \leq \|y\| \| |x|\xi\| = \|y\| \| |x| y|x| \xi\| = \|y\| \| |x| e \xi \|$ and so with $\eta=e\xi$ we see that $\| |x|\eta\| \geq \|y\|^{-1} \|\eta\|$.  So $\| |x| \eta - \lambda\eta\| \geq \| |x|\eta\| - \lambda \|\eta\| \geq (\|y\|^{-1} - \lambda) \|\eta\|$, and thus
\[ \| |x|\xi - \lambda\xi\|^2 \geq (\|y\|^{-1} - \lambda)^2 \|e\xi\|^2 + \lambda^2 \|(1-e)\xi\|^2
\geq (\|y\|^{-1}-\lambda)^2 \|\xi\|^2. \]
So $|x|-\lambda$ is bounded below and hence is an isomorphism onto its range.  As we are working on a Hilbert space, there is $z\in\mc B(H)$ with $z(|x|-\lambda) = 1$, hence also $(|x|-\lambda)z^*=1$.  So $z = z^* = (|x|-\lambda)^{-1}$ and we conclude $\lambda\not\in\sigma(|x|)$.

So $\sigma(|x|) \subseteq \{0\} \cup [c,\infty)$ for some $c>0$ (we can take $c = \|y\|^{-1}$).  Hence $\sigma(x^*x) = \sigma(|x|^2) \subseteq \{0\} \cup [c^2,\infty)$, and $x$ is well-supported.\footnote{This argument could also be made, slightly more directly, by using the notion of the ``approximate point spectrum''.}
\end{proof}

\begin{remark}
The definition of being von Neumann regular might depend upon the algebra.  However, notice that the proof of Proposition~\ref{prop:ws_vnreg} shows that if $A\subseteq\mc B(H)$ and $x\in A$ is von Neumann regular in $\mc B(H)$, then there is $y\in A$ with $xyx=x$.  Hence, for $C^*$-algebras, there is no dependence on the algebra.
\end{remark}

According to Blackadar, sometimes well-supported elements are called \emph{elements with closed range}, which is supported by the following, and the arguments in the previous proposition.  First a lemma.

\begin{lemma}\label{lem:power_factorisation}
Let $A$ be a $C^*$-algebra and let $x\in A$ be positive.  Denote by $C^*(x) \subseteq A$ the $C^*$-algebra generated by $x$.  For any $r>0$ we have that $x^{1/2}$ is in the closure of $x^r C^*(x) \subseteq x^r A$.
\end{lemma}
\begin{proof}
We use a functional calculus argument.  Let $\epsilon>0$ and define $f\colon [0,\infty) \to [0,\infty)$ by $f(t) = t \epsilon^{-1/2-r}$ for $t<\epsilon$, and $f(t) = t^{1/2-r}$ for $t\geq\epsilon$.   Then $f$ is continuous, and for $t\geq\epsilon$ we have $t^r f(t) = t^{1/2}$, while for $t<\epsilon$ we have $t^{1/2} - t^r f(t) = t^{1/2} - t^{1+r} \epsilon^{-1/2-r} = t^{1/2}(1 - (t/\epsilon)^{1/2+r}) \leq t^{1/2} \leq \epsilon^{1/2}$.  Hence $\| x^r f(x) - x^{1/2} \| \leq \epsilon^{1/2}$.  It follows that $x^{1/2}$ is in the closure of $x^r C^*(x)$.
\end{proof}

\begin{proposition}\label{prop:ws_closedrange}
Let $x\in A$.  The following are equivalent:
\begin{enumerate}[(1)]
   \item\label{prop:ws_closedrange:1}
   $Ax$ is closed;
   \item\label{prop:ws_closedrange:2}
   $xA$ is closed;
   \item\label{prop:ws_closedrange:3}
   $x^*Ax$ is closed;
   \item\label{prop:ws_closedrange:4}
   $x$ is well-supported.
\end{enumerate}
\end{proposition}
\begin{proof}
Let $x$ be well-supported, so von Neumann regular, so there is $y$ with $xyx=x$.  Let $(a_n)$ be a sequence in $A$ with $a_n x\to a\in A$.  Then $ayx = \lim_n a_n x yx = \lim_n a_n x =a$ and so $a = ayx \in Ax$.  So \ref{prop:ws_closedrange:4} implies \ref{prop:ws_closedrange:1}, and similarly \ref{prop:ws_closedrange:2}.  Similarly, if $x^*a_nx \to a\in A$ then $x^*y^* a yx = \lim_n x^*y^* x^* a_n x yx = \lim_n x^* a_n x = a$ so $a\in x^*Ax$, and so \ref{prop:ws_closedrange:4} implies \ref{prop:ws_closedrange:3}.

Represent $A$ faithfully on a Hilbert space $H$.  Then $Ax$ is closed in $A$ if and only if $Ax$ is closed in $\mc B(H)$.  Let $x^* = v|x^*|$ be the polar decomposition, so $v^*x^* = |x^*|$.  If $Ax$ is closed in $\mc B(H)$ then for a sequence $(a_n)$ in $A$ with $a_n |x^*| \to a\in\mc B(H)$, we have $a_nx = a_n |x^*|v^* \to av^*$ so $av^*\in Ax$, say $av^*=bx$.  Then $b|x^*| = bxv = av^*v = \lim_n a_n |x^*| v^*v = \lim_n a_n |x^*| = a$ and so $a \in A|x^*|$ and we conclude that $A|x^*|$ is closed.  Conversely, suppose $A|x^*|$ is closed, let $a_nx \to a$, so $a_n|x^*| = a_n xv \to av \in A|x^*|$, say $av = b|x^*|$, so $bx = b|x^*|v^* = avv^* = \lim_n a_n xvv^* = \lim_n a_n |x^*| v^*vv^* = \lim_n a_n |x^*| v^* = \lim_n a_n x = a$ so $a\in Ax$ and we conclude $Ax$ is closed.  So we've shown that $Ax$ is closed if and only if $A|x^*|$ is closed.  Using the polar decomposition of $x$, one similarly shows that $xA$ is closed if and only if $|x|A$ is closed.  Finally, the same argument shows that $x^*Ax$ is closed if and only if $|x^*| A |x^*|$ is closed.

Suppose that $x$ is positive and $xA$ is closed.  Apply the lemma with $r=1$ to conclude that $x^{1/2} \in \overline{xA} = xA$.
% Let $\epsilon>0$ and define $f\colon [0,\infty) \to [0,\infty)$ by $f(t) = \epsilon^{-1/2}$ for $t<\epsilon$, and $f(t) = t^{-1/2}$ for $t\geq\epsilon$.  Then $f(t)t = t^{1/2}$ for $t\geq\epsilon$, while $t^{1/2} - f(t)t = t^{1/2} ( 1- (t/\epsilon)^{1/2}) \leq t^{1/2}$ for $t<\epsilon$.  By the functional calculus, $\| x f(x) - x^{1/2} \| \leq \epsilon^{1/2}$ and so $x^{1/2} \in \overline{xA} = xA$.
Hence there is $a\in A$ with $xa = x^{1/2}$, and so $xaa^*x = x$ showing that $x$ is von Neumann regular, and so well-supported.  The same argument shows that if $Ax$ is closed then $x$ is well-supported.
Thus if \ref{prop:ws_closedrange:1} holds, then $A|x^*|$ is closed, so $|x^*|$ is well-supported, but by definition, this means $x^*$ is well-supported, so $x$ is well-supported.  So \ref{prop:ws_closedrange:4} holds.  Similarly \ref{prop:ws_closedrange:2} implies \ref{prop:ws_closedrange:4}.

Finally, suppose \ref{prop:ws_closedrange:3} holds, so $|x^*|A|x^*|$ is closed.
%Set $y=|x^*|$ and choose $\epsilon>0$.  Define $f(t) = \epsilon^{-3/2}$ for $t<\epsilon$ and $f(t) = t^{-3/2}$ for $t\geq\epsilon$.  Then $f(t)t^2 = t^{1/2}$ for $t\geq\epsilon$, and for $t<\epsilon$ we have $t^{1/2} - f(t)t^2 = t^{1/2} (1 - (t/\epsilon)^{3/2})$
Apply the lemma with $r=2$ to conclude that $|x^*|^{1/2} \in \overline{|x^*|^2 C^*(|x^*|)} = \overline{|x^*| C^*(|x^*|) |x^*|} \subseteq \overline{|x^*|A|x^*|} = |x^*|A|x^*|$.  So there is $a\in A$ with $|x^*|^{1/2} = |x^*| a |x^*|$ and so $|x^*| (a |x^*|^2 a^*) |x^*| = |x^*|$ showing that $|x^*|$ is von Neumann regular, hence well-supported.  Again, this implies \ref{prop:ws_closedrange:4}.
\end{proof}

The above argument was inspired by \cite[Theorem~8]{hm}.  This paper \cite{hm} contains a little more about von Neumann regular elements, and following the citations will find similar papers.

\begin{remark}
We have tacitly worked with unital $C^*$-algebras, so we can smoothly apply the functional calculus.  The continuous functional calculus is constructed by using Gelfand theory applied to a normal element $x\in A$.  Then $C^*(1,x)$ is commutative and we find that it is isomorphic to $C(\sigma(x))$.  Then the map sending $x$ to the ``coordinate function'' $t \in C(\sigma(x))$ extends to polynomials, and by density, then to $C^*(1,x) \cong C(\sigma(x))$.  The inverse map is the continuous functional calculus.

When $A$ is non-unital, we embed $A$ into $A^+$ the unitisation.  By construction, if $f \in C(\sigma(x))$ can be approximated by polynomials with zero constant term, then $f(x) \in A$ not $A^+$.  By the locally compact space version of the Stone--Weierstrass theorem, \cite[Corollary~V.8.3]{conway}, the collection of such $f$ is exactly $C_0(\sigma(x)\setminus\{0\})$.

Hence $|x| = (x^*x)^{1/2} \in A$.  All of the continuous functions we apply to well-supported elements vanish at $0$ by construction, and so work in the non-unital case.  The same applies to the function we used in the proof of Lemma~\ref{lem:power_factorisation}.
\end{remark}


\subsection{Application to positive maps}

Let $A$ be a C$^*$-algebra (often a von Neumann algebra) and let $\varphi \colon A \to A$ be a (completely) positive map.  We say that $\varphi$ is \emph{irreducible} if there is not a non-trivial projection $p$ (so $p\not=0, p\not=1$) with $\varphi(p) \leq M p$ for some $M>0$.  See \cite[Proposition~1]{F} for example, and references therein, for further details and motivation.

\begin{lemma}
Let $v\in A$ be positive and invertible, and let $e\in A$ be a non-trivial projection.  Then $x = vev$ is a well-supported element of $A$.  The projection $p$ associated to $x$, from Lemma~\ref{lem:ws_projs}, is non-trivial.
\end{lemma}
\begin{proof}
There are a number of ways to show this: for example, it follows almost immediately from Proposition~\ref{prop:ws_closedrange}, as $Avev = Aev$ is closed if and only if $Ae$ is closed, because $v$ is invertible, and $Ae$ is always closed as $e$ is idempotent.

Let $p$ be given by Lemma~\ref{lem:ws_projs} applied to $x=vev$.  Then $xp=x$ and $xy=0 \implies py=0$.  As $x\not=0$ clearly $p\not=0$.  If $p=1$ then $xy=0\implies y=0$, but let $y=v^{-1}(1-e)\not=0$, as $e\not=1$, so see that $xy = vev v^{-1}(1-e) = ve(1-e) = 0$, contradiction.  So $p$ is non-trivial.
\end{proof}

\begin{proposition}\label{prop:twisting_irred}
Let $v\in A$ be positive and invertible, let $\varphi\colon A\to A$ be positive, and set $\tilde A \colon A\to A; x \mapsto v^{-1} A(vxv) v^{-1}$, which is positive.  Then $A$ is irreducible if and only if $\tilde A$ is irreducible.
\end{proposition}
\begin{proof}
Let $\tilde A$ be reducible, say $\tilde A(e) \leq M e$ for some non-trivial projection $e$ and $M>0$.  By the lemma, $x=vev$ is positive and well-supported.  Let $p\in A$ be the projection associated to $x$ so $p$ is non-trivial.  As $x$ is well-supported, $\sigma(x^2) \subseteq \{0\} \cup [c^2,C^2]$ for some $0<c<C$, and so $\sigma(x) \subseteq \{0\} \cup [c,C]$.  Let $f(0)=0, f(t)=1$ for $t>0$, so by construction, $p = f(x)$.  As $cf(t) \leq t \leq Cf(t)$ for $t\in\sigma(x)$, we see that $c p \leq x \leq Cp$.  Then $\tilde A(e) \leq Me$ means $A(vev) \leq M vev$, so $A(p) \leq c^{-1} A(x) \leq c^{-1} M x \leq c^{-1}CM p$, in the first step using that $A$ is positive.  Thus $A$ is reducible.  Swapping the roles of $v$ and $v^{-1}$ shows that $A$ reducible implies $\tilde A$ reducible.
\end{proof}

This gives justification to the claim at the start of the proof of \cite[Proposition~4.3]{cgw}.  There we work with a finite-dimensional $C^*$-algebra, for which it is obvious that all elements are well-supported.  I thank Mateusz Wasilewski for correspondance which suggested the approach to proving Proposition~\ref{prop:twisting_irred} in the finite-dimensional case.



\section{Inductive limits of Hilbert spaces}

We use \cite[Chapter~XIV]{tak3} as a reference, though this is standard material (and I suspect the main result of this section is in the literature, if I knew where to look).  Let $(H_n)$ be a seqeuence of Hilbert spaces with $\iota_n\colon H_n \to H_{n+1}$ isometries.  Then $(H_n,\iota_n)$ is an \emph{inductive sequence of Hilbert spaces}.

For $m\geq n$ define $\iota_{m,n} = \iota_{m-1} \circ \iota_{m-2} \circ\cdots\circ \iota_n$ an isometry $H_n \to H_m$, and set $\iota_{n,n} = 1_{H_n}$.  As it is easy to forget the convention, we note this:
\[ \iota_{m,n} \colon H_n \hookrightarrow H_m \qquad (n\leq m). \]
Let $H_\infty$ be the family of sequences $\xi=(\xi_n)$ where $\xi_n\in H_n$ for each $n$, and such that there is some $m$ with $\xi_{n+1} = \iota_n(\xi_n)$ for $n\geq m$.  That is, $\xi_n = \iota_{n,m}(\xi_m)$ for $n\geq m$.  Notice that $H_\infty$ is a vector space for the pointwise operations, and we may define a pre-inner-product by
\[ (\xi|\eta) = \lim_n (\xi_n|\eta_n). \]
This is well-defined, as given $\xi,\eta$ there is some $m$ so that for $n\geq m$ we have $\xi_n = \iota_{n,m}(\xi_m)$ and $\eta_n = \iota_{n,m}(\eta_m)$, and so $(\xi_n|\eta_n) = (\iota_{n,m}(\xi_m)|\iota_{n,m}(\eta_m)) = (\xi_m|\eta_m)$ as $\iota_{n,m}$ is an isometry.  Hence the sequence $(\xi_n|\eta_n)$ is eventually constant, and so the limit certainly exists.  Notice that $(\xi|\xi)=0$ if and only if $(\xi_n|\xi_n)=0$ for sufficiently large $n$, and so we do not have an inner-product, and must quotient by the null space $\{ \xi : (\xi|\xi)=0 \}$.  The completion of the resulting space is $H = \underrightarrow{\lim} (H_n,\iota_n)$ a Hilbert space.

For each $n$ we define
\[ \iota_{\infty, n} \colon H_n \to H_\infty; \quad \xi \mapsto (0,0,\cdots,0,\xi,\iota_{n+1,n}(\xi),\iota_{n+2,n}(\xi),\cdots), \]
where the $\xi$ occurs in the $n$th position.  This is an isometry, and so we can regard $\iota_{\infty,n}$ as a map to $H$.  We have that $\iota_{\infty,n+1} \circ \iota_n (\xi) = (0,\cdots,0,0,\iota_{n+1,n}(\xi), \iota_{n+2,n}(\xi), \cdots)$ which equals $\iota_{\infty,n}(\xi)$ in $H$ (once we have quotiented $H_\infty$ by the null space).  Thus we have the commutative diagram:
\[ \begin{tikzcd}
   H_1 \arrow[r, "\iota_1", hook] \arrow[rrd, "\iota_{\infty,1}"', hook] &
   H_2 \arrow[r, "\iota_2", hook] \arrow[rd, hook] &
   H_3 \arrow[r, "\iota_3", hook] \arrow[d, "\iota_{\infty,3}", hook] &
   \cdots & \arrow[dll, phantom, "\cdots"] \\
   && H
\end{tikzcd} \]

The space $H$ has the following universal property.  Suppose $K$ is a Hilbert space and for each $n$ we have an isometry $u_n \colon H_n \to K$ with $u_{n+1} \circ \iota_n = u_n$, and such that the images of the $u_n$ are dense in $K$.  As $u_{n+1}(H_{n+1}) \supseteq u_n(H_n)$, we see that $K$ is the closure of this increasing family of subspaces.  For $\xi\in H_n$ define $U \colon u_n(\xi) \mapsto \iota_{\infty,n}(\xi) \in H$.  As $u_n(\xi) = u_{n+1}\iota_n(\xi)$ and $\iota_{\infty,n+1}\iota_n(\xi) = \iota_{\infty,n}(\xi)$, we see that $U$ is well-defined on the union of the subspaces $u_n(H_n)$.  As all the maps are isometries, also $U$ is an isometry, and so $U$ extends to an isometry $U \colon K\to H$.  The image of $U$ contains each subspace $\iota_{\infty,n}(H_n)$, and as $H$ is the union of all these subspaces, $U$ is unitary.


\subsection{Relation to the algebraic inductive limit}

We recall the usual construction of the algebraic inductive limit.  We consider the disjoint union of each space $H_n$ and quotient by the equivalence relation that $\xi_i\in H_i$ is related to $\xi_j\in H_j$ if $\iota_{m,i}(\xi_i) = \iota_{m,j}(\xi_j)$ for some $m\geq i,j$.  The relation is transitive as $\iota_{n,m}\circ\iota_{m,i} = \iota_{n,i}$ whenever $n \geq m \geq i$.

The vector space operations on each $H_n$ drop to the equivalence classes.  Given two classes $[\xi_i]$ and $[\xi_j]$ given $m\geq i,j$ we define $([\xi_i]|[\xi_j]) = (\iota_{m,i}(\xi_i)|\iota_{m,j}(\xi_j))_{H_m}$.  As $\iota_{n,m}$ is an isometry for $n\geq m$, we again see that this definition does not depend upon the choice of $m$.  Furthermore, it's seen to be independent of the choice of vector giving the equivalence class.

The maps $\iota_{\infty,n} \colon H_n \to H_\infty$ respect the equivalence relation and give a bijection between the algebraic inductive limit and $H_\infty$, which is unitary for the inner-products we have defined.  Thus we obtain the same construction.


\subsection{As an inverse limit}

The sequence $(H_n, \iota_n^*)$ is an inverse sequence, where a general inverse sequence is $(H_n, q_n)$ where $q_n\colon H_{n+1} \to H_n$ is a coisometry for each $n$.  Of course, every inverse sequence arises from an inductive sequence, and conversely, due to our requirement that each $q_n$ be a coisometry.

We may define
\[ q_{m,n} = q_n \circ q_{n+1} \circ\cdots\circ q_{m-1} \colon H_m \to H_n \qquad (n\leq m). \]
The \emph{algebraic inverse limit} is defined as all sequences $\xi = (\xi_n)$ where $\xi_n\in H_n$ for each $n$, and $q_{m,n}(\xi_m) = \xi_n$ for each $n\leq m$; equivalently, $\xi_n = q_n(\xi_{n+1})$ for each $n$.  This space is a vector space for the pointwise operations.
We define $\underleftarrow{\lim} (H_n, q_n)$ to be the subspace of the algebraic inverse limit consisting of all bounded sequences.
For each $n$ we define $q_{\infty,n} \colon \underleftarrow{\lim} (H_n, q_n) \to H_n$ to be the map $\xi \mapsto \xi_n$, that is, the projection onto the $n$th coordinate, which is a linear map.

Let $\xi=(\xi_n)$ with $\|\xi_n\|\leq K$ for each $n$.  As each $q_n$ is, in particular, a contraction, $\|\xi_n\| = \|q_n(\xi_{n+1})\| \leq \|\xi_{n+1}\|$ and so the sequence of norms $(\|\xi_n\|)$ is increasing, and bounded above, and so converges.
For $n\leq m$, as $\iota_{m,n} = q_{m,n}^* \colon H_n \to H_m$ is an isometry, $q_{m,n}^* \circ q_{m,n}$ is an orthogonal projection of $H_m$ onto the image of $q_{m,n}^*$, which is isometric with $H_n$.  For a bounded sequence $(\xi_n)$ we hence have
\[ \|\xi_m\|^2 = \| q_{m,n}^* \circ q_{m,n}(\xi_m) \|^2 + \| \xi_m - q_{m,n}^* \circ q_{m,n}(\xi_m) \|^2 = \|\xi_n\|^2 + \| \xi_m - \iota_{m,n}(\xi_n) \|^2. \]
Thus $\| \xi_m - \iota_{m,n}(\xi_n) \|$ is small when $n\leq m$ and $n$ is large.  Given another bounded sequence $\eta$ in the inverse limit, we have
\begin{align*}
(\xi_m|\eta_m)
&= ( \iota_{m,n}(\xi_n) + \xi_m - \iota_{m,n}(\xi_n) | \iota_{m,n}(\eta_n) + \eta_m - \iota_{m,n}(\eta_n)  ) \\
&= ( \iota_{m,n}(\xi_n) | \iota_{m,n}(\eta_n) ) 
   + ( \xi_m - \iota_{m,n}(\xi_n) | \iota_{m,n}(\eta_n) )
   + ( \iota_{m,n}(\xi_n) | \eta_m - \iota_{m,n}(\eta_n) ) \\
   &\qquad + ( \xi_m - \iota_{m,n}(\xi_n) | \eta_m - \iota_{m,n}(\eta_n) ) \\
&= ( \xi_n | \eta_n ) 
   + ( q_{m,n}(\xi_m) - \xi_n | \eta_n )
   + ( \xi_n | q_{m,n}(\eta_m) - \eta_n )
   + ( \xi_m - \iota_{m,n}(\xi_n) | \eta_m - \iota_{m,n}(\eta_n) ) \\
&= ( \xi_n | \eta_n ) 
   + ( \xi_m - \iota_{m,n}(\xi_n) | \eta_m - \iota_{m,n}(\eta_n) ),
\end{align*}
using that $\iota_{m,n}$ is an isometry, and that $q_{m,n}(\xi_m) = \xi_n$, and similarly for $\eta$.  As the 2nd term is small, we see that the sequence $((\xi_n|\eta_n))$ is Cauchy and so converges.  We may hence define an inner-product on the subspace of bounded sequences by $(\xi|\eta) = \lim_n (\xi_n|\eta_n)$.  This is an inner-product, as $(\xi|\xi) = \lim_n \|\xi_n\|$ which in an increasing limit, and so equals $0$ only when $\xi_n=0$ for all $n$.  So  $\underleftarrow{\lim} (H_n, q_n)$ is an inner-product space.

Given $\xi = (\xi_n)$ in $\underleftarrow{\lim} (H_n, q_n)$, for large $N$, define $\eta = (\eta_n)$ by setting $\eta_n=\xi_n$ for $n\leq N$ and $\eta_n = \iota_{n,N}(\xi_N)$ for $n>N$.  Then $q_n(\eta_{n+1}) = \eta_n$ for $n<N$, while for $n\geq N$ we have $q_n(\eta_{n+1}) = q_n q_{n+1,N}^*(\xi_N) = q_n q_n^* q_{n-1}^* \cdots q_N^*(\xi_N) = q_{n,N}^*(\xi_N) = \eta_n$.  Thus $\eta$ is in the inverse limit, is bounded, and for $n\geq N$ we have $\|\xi_n - \eta_n\| = \|\xi_n - \iota_{n,N}(\xi_N)\|$ is small.  We conclude that the collection of such $\eta$ is dense in $\underleftarrow{\lim} (H_n, q_n)$; notice also that $\eta\in H_\infty$.

Let $\xi=(\xi_n) \in H_\infty$, say $\xi_{n+1} = \iota_n(\xi_n)$ for $n\geq N$.  Adjust $\xi$ by setting $\xi_n = q_{N,n}(\xi_N)$ for $n\leq N$, and notice that this does not change the equivalence class that $\xi$ defines in $\underrightarrow{\lim} (H_n, \iota_n)$.  For $n<N$ we have $q_n(\xi_{n+1}) = q_n q_{N,n+1}(\xi_N) = q_{N,n}(\xi_N) = \xi_n$, while for $n\geq N$ we have $q_n(\xi_{n+1}) = \iota_n^* \iota_n(\xi_n) = \xi_n$.  Hence $\xi \in \underleftarrow{\lim} (H_n, q_n)$ as of course $(\xi_n)$ is bounded, and notice that the norms of $\xi$ in both spaces is the same.  If $\xi'=(\xi'_n)\in H_\infty$ agrees with $\xi$ in $\underrightarrow{\lim} (H_n, \iota_n)$ then $\xi_n = \xi'_n$ for sufficiently large $n$, and so we obtain the same sequence in $\underleftarrow{\lim} (H_n, q_n)$.  We hence have an isometry $U \colon \underrightarrow{\lim} (H_n, q_n) \to \underleftarrow{\lim} (H_n, q_n)$.  The previous paragraph shows that $U$ has dense range, and so $U$ is a unitary.

Thus the inverse and inductive limits agree; in particular, $\underleftarrow{\lim} (H_n, q_n)$ is a Hilbert space.  Consider $q_{\infty,n}^* \colon H_N \to \underleftarrow{\lim} (H_n, q_n)$ which satisfies, for $\xi\in \underleftarrow{\lim} (H_n, q_n), \eta_n\in H_n$,
\begin{align*}
(\xi|q_{\infty,n}^*(\eta_n))
&= (\xi_n|\eta_n)
= \lim_n (q_{m,n}(\xi_m)|\eta_n)
= \lim_m (\xi_m | \iota_{m,n}(\eta_n)) \\
&= (\xi|(q_{n,1}(\eta_n),\cdots,q_{n-1}(\eta_n),\eta_n, \iota_n(\eta_n), \iota_{n+1,n}(\eta_n), \cdots))
\end{align*}
This shows that $q_{\infty,n}^*(\eta_n) = (q_{n,1}(\eta_n),\cdots,q_{n-1}(\eta_n),\eta_n, \iota_n(\eta_n), \iota_{n+1,n}(\eta_n), \cdots)$, where we can again check that this sequence is in $\underleftarrow{\lim} (H_n, q_n)$.  As adjusting a finite number of elements doesn't change a sequence in $\underrightarrow{\lim} (H_n, q_n)$, we have shown that $U \iota_{\infty, n} = q_{\infty, n} U$.  The inverse limit has the advantage that we obtain a concrete description of the entire Hilbert space, while the inductive limit only constructs a dense subspace.  In particular, we have the following.

\begin{proposition}
Let $(H_n, \iota_n)$ be an inductive sequence of Hilbert spaces.  For each $\xi \in \underrightarrow{\lim} (H_n,\iota_n)$ there is a unique bounded sequence $(\xi_n)$ with $\xi_n\in H_n$ for each $n$, with $\iota_n^*(\xi_{n+1}) = \xi_n$ for each $n$, and such that $\iota_{\infty,n}(\xi_n) \to \xi$.

Conversely, if $(\xi_n)$ is a sequence with $\xi_n\in H_n$ for each $n$, and ``Cauchy'' in the sense that for $\epsilon>0$ there is $N$ such that $\| \iota_{m,n}(\xi_n) - \xi_m \|<\epsilon$ for $m \geq n\geq N$, then there is $\xi \in \underrightarrow{\lim} (H_n,\iota_n)$ with $\iota_{\infty,n}(\xi_n) \to \xi$.
\end{proposition}
\begin{proof}
Given $\xi \in \underrightarrow{\lim} (H_n,\iota_n)$, let $(\xi_n) = U(\xi) \in \underleftarrow{\lim} (H_n,\iota_n)$.  Then
\begin{align*}
U \iota_{\infty,n}(\xi_n)
&= q_{\infty,n}^* U(\xi)
= U(q_{n,1}(\xi_n),\cdots,q_{n,n-1}(\xi_n),\xi_n, \cdots, \iota_{m,n}(\xi_n), \cdots) \\
&= (q_{n,1}(\xi_n),\cdots,q_{n,n-1}(\xi_n),\xi_n, \cdots, \iota_{m,n}(\xi_n), \cdots),
\end{align*}
and so
\[ \| U \iota_{\infty,n}(\xi_n) - U\xi\|
= \lim_m \| \iota_{m,n}(\xi_n) - \xi_m \|, \]
which as observed before is small, if $n$ is large.  So $U \iota_{\infty,n}(\xi_n) \to U\xi$ and hence $\iota_{\infty,n}(\xi_n) \to \xi$.

For the second claim, we again work in $\underleftarrow{\lim} (H_n,\iota_n)$.  Given $\epsilon>0$ select $N$ as in the statement, and for $n\geq N$ define $\eta^{(n)} = q_{\infty,n}^*(\xi_n) \in \underleftarrow{\lim} (H_n,\iota_n)$.
By definition, $\eta^{(n)}_k = \iota_{k,n}(\xi_n)$ for $k\geq n$.  Thus for $m\geq n\geq N$ and $k\leq m$ we have that
\begin{align*}
\| \eta^{(m)}_k - \eta^{(n)}_k \|
&= \| \iota_{k,m}(\xi_m) - \iota_{k,n}(\xi_n) \|
\leq \| \iota_{k,m}(\xi_m) - \xi_k \| + \| \xi_k - \iota_{k,n}(\xi_n) \| < 2\epsilon.
\end{align*}
Hence $\| \eta^{(m)} - \eta^{(n)}\| \leq 2\epsilon$ and so $(\eta^{(n)})$ is Cauchy, so convergent, in $\underleftarrow{\lim} (H_n,\iota_n)$.  Thus $\iota_{\infty,n}(\xi_n)$ converges in $\underrightarrow{\lim} (H_n,\iota_n)$.
\end{proof}


\subsection{Application to infinite tensor products}

Let $H$ be a Hilbert space with unit vector $\xi_0\in H$.  Define $H_n = H^{\otimes n}$ and define connecting maps $\iota_n \colon H_n \to H_{n+1}; u \mapsto u\otimes\xi_0$.  By definition, the infinite tensor product is $(H,\xi_0)^{\otimes\infty} = \underrightarrow{\lim} (H_n,\iota_n)$.  The above proposition shows that we can regard this space as the limit points of sequences $u_n \in H^{\otimes n}$ which are Cauchy in the sense that $u_n \otimes \xi_0^{\otimes (m-n)} - u_m$ is small for $m\geq n$ large.

Suppose that $u_n = \xi_1 \otimes \cdots \otimes \xi_n$ for each $n$.  If $\lim_n \prod_{i\leq n} \|\xi_i\| = 0$ then $\|u_n\|\to 0$ and so $\| u_n \otimes \xi_0^{\otimes(m-n)} - u_m\| \leq \|u_n\| + \|u_m\|$ is small when $n,m$ are large.  Then of course the limit in $(H,\xi_0)^{\otimes\infty}$ is $0$.  Similarly, if $\lim_n \prod_{i\leq n} \|\xi_i\| = \infty$ then $\iota_{\infty,n}(u_n)$ is unbounded, and so cannot converge.

So we assume that $0 < \lim_n \prod_{i\leq n} \|\xi_i\| < \infty$.
%The Cauchy condition becomes that for $\epsilon>0$ there is $N$ such that when $m\geq n\geq N$ we have
%\[ \| \xi_1\otimes\cdots\otimes\xi_n\otimes\xi_0\otimes\cdots\otimes\xi_0 - \xi_1\otimes\cdots\otimes\xi_m \| < \epsilon. \]
For $m\geq n$ we have
\begin{align*}
& \| \xi_1\otimes\cdots\otimes\xi_n\otimes\xi_0\otimes\cdots\otimes\xi_0 - \xi_1\otimes\cdots\otimes\xi_m \|^2 \\
&= \| \xi_1\otimes\cdots\otimes\xi_n\otimes(\xi_0\otimes\cdots\otimes\xi_0 - \xi_{n+1}\otimes\cdots\otimes\xi_m) \|^2 \\
&= \prod_{i=1}^n \|\xi_i\|^2 \Big( 1 + \prod_{i={n+1}}^m \|\xi_i\|^2 - 2\Re \prod_{i={n+1}}^m (\xi_0|\xi_i) \Big).
\end{align*}
The first term is bounded away from $0$ and $\infty$, so we look at the term in brackets, which must be small for large $n$.

We now make some remarks about \cite[Lemma~XIV.1.7]{tak3}, in particular giving a counter-example to one direction of this claim.  We start by looking at some elementary analysis results about infinite products, for which we follow the lecture notes \cite{leonard}, although we allow an infinite product to converge to $0$.

\begin{lemma}\label{lem:cauchy_criterion}
Let $(z_n)$ be a sequence in $\mathbb C$.  The following are equivalent:
\begin{enumerate}
   \item $\lim_n \prod_{i\leq n}z_i$ exists and is non-zero;
   \item no $z_n$ is zero, and for each $\epsilon>0$ there is $N$ such that for each $m\geq n\geq N$ we have $| 1 - \prod_{i=n}^m z_i | < \epsilon$.
\end{enumerate}

\end{lemma}
\begin{proof}
Suppose $\lim_n \prod_{i\leq n}z_i\not=0$, so no $z_i$ is $0$, and setting $p_n = \prod_{i\leq n} z_i$, there is $\delta>0$ so that $|p_n| > \delta$ for all $n$.  There is $N$ so that $|p_n-p_m| < \epsilon\delta$ for each $m \geq n \geq N$.  Then $|1 - \prod_{i=n+1}^m z_i| = |1 - p_m/p_n| =|p_n|^{-1} |p_n-p_m| < \epsilon$.

For the converse, given $\epsilon>0$ select $N$.  Then $|1-z_n|<\epsilon$ for each $n\geq N$ so $1-\epsilon < |z_n| < 1+\epsilon$.  Set $q_m = z_N z_{N+1} \cdots z_m$ for $m\geq N$, so we also have $|1 - q_m| < \epsilon$.  In particular, we may suppose that $\epsilon < 1/2$ so that $1/2 < |q_m| < 3/2$ for all $m\geq N$.  Increasing $N$ if necessary, we may suppose that for $m > n \geq N$,
\[ \Big| \frac{q_m}{q_n} - 1 \Big| = \Big| \prod_{i=n+1}^m z_i - 1 \Big| < \tfrac{2}{3}\epsilon. \]
Then $|q_m - q_n| = |q_n| |q_m/q_n-1| < \frac23 \epsilon|q_n| < \frac32 \frac23 \epsilon = \epsilon$.  Hence $(q_n)_{n\geq N}$ is Cauchy and so converges.  As $\prod_{i=1}^n z_i = q_n \prod_{i=1}^{N-1}z_i$, also the infinite product converges.  As $z_i\not=0$ for each $i$, and as $|q_n| > 1/2$, we also see that the limit is non-zero.
\end{proof}

\begin{proposition}\label{prop:one_plus_inf_prod}
Let $(a_n)$ be a sequence of positive reals.  Then $\lim_n \prod_{i\leq n} (1+a_i)$ converges if and only if $\sum_n a_n$ converges.
\end{proposition}
\begin{proof}
By hypothesis, both series are increasing, and so converge if and only if they are bounded above.  As $x>0$ implies $1+x < e^x$ we see that $\sum_{i\leq n} a_i \leq \prod_{i\leq n} (1+a_i) \leq \exp(\sum_{i\leq n} a_i)$ and the result follows.
\end{proof}

Note of course that if $\lim_n \prod_{i\leq n} (1+a_i)$ converges then its limit is $>1$.  We next come to a notion of ``absolute convergence''.

\begin{corollary}\label{corr:necc_inf_prod_condition}
Let $(a_n)$ be a sequence in $\mathbb C$ such that $\sum_n |a_n|$ converges.  Then $\lim_n \prod_{i\leq n} (1 + a_i)$ converges (and is non-zero if $a_i\not=-1$ for all $i$).
\end{corollary}
\begin{proof}
Note that
\begin{align*}
& \big| (1+a_n)(1+a_{n+1})\cdots(1+a_m) - 1\big|
= \big| a_n+\cdots+a_m + \sum_{n\leq i<j\leq m} a_ia_j + \cdots \big| \\
\leq & |a_n| + \cdots + |a_m| + \sum_{n\leq i<j\leq m} |a_ia_j| + \cdots
= (1+|a_n|)(1+|a_{n+1}|)(1+|a_m|) - 1,
\end{align*}
and so Lemma~\ref{lem:cauchy_criterion} shows that $\lim_n \prod_{i\leq n} (1+|a_n|)$ converges implies that $\lim_n \prod_{i\leq n} (1+a_n)$ converges (and is non-zero if $a_i\not=-1$ for all $i$).  If $\sum_n |a_n|$ converges, then by Proposition~\ref{prop:one_plus_inf_prod}, $\lim_n \prod_{i\leq n} (1+|a_i|)$ converges, and the result follows.
\end{proof}

We now come to one direction of \cite[Lemma~XIV.1.7]{tak3}.

\begin{proposition}
Let $(\xi_n)$ be a sequence in $H$, and set $u_n = \xi_1 \otimes\cdots\otimes \xi_n$ for each $n$.  If $\lim_n \|u_n\| = \lim_n \prod_{i\leq n} \|\xi_i\|$ converges and is non-zero, and $\sum_n |1-(\xi_n|\xi_0)| < \infty$, then $(u_n)$ converges in the infinite tensor product.
\end{proposition}
\begin{proof}
From the discussion above, given $\epsilon>0$ we seek $N$ so that if $m\geq n\geq N$ we have
\[ 1 + \prod_{i={n+1}}^m \|\xi_i\|^2 - 2\Re \prod_{i={n+1}}^m (\xi_0|\xi_i) < \epsilon. \]
Set $a_n = 1-(\xi_0|\xi_n)$ so by hypothesis, $\sum_n |a_n|<\infty$ and so Corollary~\ref{corr:necc_inf_prod_condition} tells us that $\lim_n \prod_{i\leq n} (1+a_n) = \lim_n \prod_{i\leq n} (\xi_0|\xi_i)$ converges and is non-zero.  By Lemma~\ref{lem:cauchy_criterion} there is $N$ so that for $m\geq n\geq N$ we have both
\[ \Big| 1 - \sum_{i=n+1}^m (\xi_0|\xi_i) \Big| < \epsilon, \qquad
\Big| 1 - \sum_{i=n+1}^m \|\xi_i\|^2 \Big| < \epsilon. \]
For $z\in\mathbb C$, as $|2-2\Re z| = |2-z-\overline{z}| \leq |1-z| + |1-\overline{z}|$, we see that $| 2 - 2\Re \sum_{i=n+1}^m (\xi_0|\xi_i)| < 2\epsilon$, and so
\[ \Big|1 + \prod_{i={n+1}}^m \|\xi_i\|^2 - 2\Re \prod_{i={n+1}}^m (\xi_0|\xi_i) \Big|
= \Big| \sum_{i=n+1}^m \|\xi_i\|^2 -1 + 2 - 2\Re \prod_{i={n+1}}^m (\xi_0|\xi_i) \Big|
< 3\epsilon, \]
as we want.
\end{proof}

The converse does not seem to hold, as we now show.  We continue to follow \cite{leonard}.

\begin{proposition}\label{prop:ab_conv}
Let $a_n\geq 0$ for each $n$.  If $\sum_n a_n$ converges then $\lim_n \prod_{i\leq n} (1-a_i)$ converges.  If $\lim_n \prod_{i\leq n} (1-a_i)$ converges then either the limit is $0$ or $\sum_n a_n$ converges.
\end{proposition}
\begin{proof}
If $\sum_n a_n$ converges then Corollary~\ref{corr:necc_inf_prod_condition} tells us that $\lim_n \prod_{i\leq n} (1-a_i)$ converges.

Conversely, suppose $\lim_n \prod_{i\leq n} (1-a_i)$ converges and is non-zero, so by Lemma~\ref{lem:cauchy_criterion}, in particular, $1-(1-a_i) = a_i \to 0$, so there is $N$ such that $a_n<1$ for $n\geq N$.  Towards a contraction, suppose that $\sum_n a_n$ diverges, so also $\lim_n \prod_{i\leq n} (1+a_i)$ diverges.  As $(1-a_i)(1+a_i) = 1-a_i^2 \leq 1$ for $i\geq N$, for $m \geq n \geq N$ we have
\[ \prod_{i=n}^m (1-a_i) \prod_{i=n}^m (1+a_i) = \prod_{i=n}^m (1-a_i^2) \leq 1
\quad\implies\quad
\prod_{i=n}^m (1-a_i) \leq \Big( \prod_{i=n}^m (1+a_i) \Big)^{-1}. \]
Thus $\prod_{i=n}^m (1-a_i)$ is arbitrarily small, but by Lemma~\ref{lem:cauchy_criterion}, is also arbitrarily close to $1$, giving the claimed contraction.
\end{proof}

We now come to our counter-example.  Let
\[ (a_n) = \Big( 1-\tfrac{1}{\sqrt 2}, 1+\tfrac12+\tfrac{1}{\sqrt 2}, 
1-\tfrac{1}{\sqrt 3}, 1+\tfrac13+\tfrac{1}{\sqrt 3}, \cdots \Big). \]
Then $\sum_n |1-a_n| \geq \sum_n n^{-1/2} = \infty$.  However, we claim that $\lim_n \prod_{i\leq n} a_i$ converges, and is non-zero.  We have
\[ \big( 1 - \tfrac{1}{\sqrt{k}} \big) \big( 1 + \tfrac1k + \tfrac{1}{\sqrt{k}} \big)
= \tfrac{1}{k}\big( 1 - \tfrac{1}{\sqrt{k}} \big) + 1 - \tfrac1k
= 1 - \tfrac{1}{k\sqrt k}. \]
By Proposition~\ref{prop:ab_conv}, $\lim_n \prod_{i\leq n} (1-\frac{1}{i\sqrt i})$ converges and is non-zero, and so by Lemma~\ref{lem:cauchy_criterion}, for $\epsilon>0$ there is $N$ so that if $m \geq n \geq N$ we have
\[ \Big| 1 - \prod_{i=2n-1}^{2m} a_i \Big| < \epsilon. \]
Then
\[ \sum_{i=2n}^{2m} a_i = \big(1-\tfrac{1}{\sqrt{n+1}}\big)^{-1}\prod_{i=2n-1}^{2m} a_i, \qquad
\sum_{i=2n-1}^{2m+1} = \big(1+\tfrac{1}{\sqrt{m}}+\tfrac{1}{\sqrt{m+1}}\big)\prod_{i=2n-1}^{2m} a_i, \]
which are then both within $2\epsilon$ of $1$, if $N$ is large enough, and similarly for the final case of $\sum_{i=2n}^{2m+1} a_i$.  By Lemma~\ref{lem:cauchy_criterion} applied in the other direction, we conclude that $\lim_n \prod_{i\leq n} a_i$ converges and is non-zero.

With our Hilbert space example from the previous section, set $\xi_n = a_n \xi_0$ for each $n$.  Then 
\[ \xi_1 \otimes\cdots \otimes \xi_n = a_1a_2\cdots a_n \xi_0\otimes\cdots\otimes\xi_0, \]
and so for $m \geq n$,
\begin{align*}
& \| \xi_1\otimes\cdots\otimes\xi_n\otimes\xi_0\otimes\cdots\otimes\xi_0 - \xi_1\otimes\cdots\otimes\xi_m \| \\
&= \Big| \prod_{i=1}^n a_i - \prod_{i=1}^m a_i \Big|
= \Big| \prod_{i=1}^n a_i \Big| \Big| 1 - \prod_{i=n+1}^m a_i \Big|.
\end{align*}
As $\lim_n \prod_{i\leq n} a_i$ converges, this is arbitrarily small if $n$ is large enough.  So the sequence $(\xi_1\otimes\cdots\otimes\xi_n)$ converges in the infinite tensor product.  However, $\sum_n |1-(\xi_n|\xi_0)| = \sum_n |1-a_n|$ diverges, contrary to \cite[Lemma~XIV.1.7]{tak3}.



\begin{thebibliography}{aa}

\bibitem{blackadar_enc} B.~E. Blackadar, {\it Operator algebras}, Encyclopaedia of Mathematical Sciences Operator Algebras and Non-commutative Geometry, 122 III, Springer, Berlin, 2006; MR2188261

\bibitem{conway} J.~B. Conway, {\it A course in functional analysis}, second edition, 
Graduate Texts in Mathematics, 96, Springer, New York, 1990; MR1070713

\bibitem{cgw} K. Courtney, P. Ganesan, M. Wasilewski, Connectivity for quantum graphs via quantum adjacency operators, arXiv:2505.22519v1 [math.OA]

\bibitem{q1} Matthew Daws, Empty interior of union of cosets?, 
   \url{https://mathoverflow.net/q/351764}

\bibitem{F} D.~R. Farenick, Irreducible positive linear maps on operator algebras, Proc. Amer. Math. Soc. {\bf 124} (1996), no.~11, 3381--3390; MR1340385

\bibitem{hm} R.~E. Harte and M. Mbekhta, On generalized inverses in $C^*$-algebras, Studia Math. {\bf 103} (1992), no.~1, 71--77; MR1184103

\bibitem{kt} E. Kaniuth\ and\ K. F. Taylor, {\it Induced representations of locally compact groups}, Cambridge Tracts in Mathematics, 197, Cambridge University Press, Cambridge, 2013. MR3012851

\bibitem{leonard} I. Leonard, ``MATH 324 Summer 2012'' lecture notes, available at \url{https://www.math.ualberta.ca/~isaac/math324/s12/zeta.pdf}

\bibitem{a2} Brian M. Scott (\url{https://math.stackexchange.com/users/12042/brian-m-scott}), The set of ultrafilters on an infinite set is uncountable, \url{https://math.stackexchange.com/q/83540}

\bibitem{tak3} M. Takesaki, {\it Theory of operator algebras. III}, Encyclopaedia of Mathematical Sciences Operator Algebras and Non-commutative Geometry, 127 8, Springer, Berlin, 2003; MR1943007

\bibitem{a1} YCor (\url{https://mathoverflow.net/users/14094/ycor}), Empty interior of union of cosets?,
   \url{https://mathoverflow.net/q/351846}

\bibitem{a3} YCor (\url{https://mathoverflow.net/users/14094/ycor}), Topological semi-direct products of groups,
   \url{https://mathoverflow.net/q/425559}

\end{thebibliography}


\end{document}
