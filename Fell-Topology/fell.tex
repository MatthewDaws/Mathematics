\documentclass[a4paper,11pt]{article}
\usepackage[utf8]{inputenc}
\usepackage[margin=2cm]{geometry}
\usepackage{latexsym, amsmath, amsthm, amssymb}

\newcommand{\mc}[1]{{\mathcal{#1}}}
\newcommand{\Rep}{\operatorname{Rep}}
\newcommand{\Cl}{\operatorname{Cl}}
\newcommand{\Eq}{\operatorname{Eq}}
\newcommand{\Sp}{\operatorname{Sp}}
\newcommand{\op}{{\operatorname{op}}}
\newcommand{\bin}{{\operatorname{bin}}}
\newcommand{\Corr}{{\operatorname{Corr}}}

\newtheorem{lemma}{Lemma}[section]
\newtheorem{proposition}[lemma]{Proposition}
\newtheorem{theorem}[lemma]{Theorem}
\newtheorem{definition}[lemma]{Definition}


\title{Notes of the Fell topology}
\author{Matt Daws}
\date{Last knowing modified 1 April 2019}

% 2345678901234567890123456789012345678901234567890123456789012345678901234567890

\begin{document}

\maketitle

\begin{abstract}
We provide an overview of the Fell topology on the representation space of a
$C^*$-algebra.  We then, in more detail, study this topology as applied to
correspondences of von Neumann algebras.
\end{abstract}

\section{Introduction}

Fell introduced the eponymously named topology in \cite{fell1}, and further
refined the definitions and properties in \cite{fell2}.  Fix a $C^*$-algebra
$A$.  Given a Hilbert space $H$, let $\Rep(A,H)$ be the collection of all
non-zero $*$-homomorphisms $A\rightarrow\mc B(H)$.  For $\pi\in\Rep(A,H)$ the
\emph{essential space} of $\pi$ is $H^\pi$, the closed linear span of
$\{\pi(a)\xi : a\in A, \xi\in H\}$.  If $H^\pi = H$ then we say that $\pi$
is \emph{non-degenerate}; otherwise $\pi$ is \emph{degenerate}, which we
explicitly allow.  Finally, $\pi$ is \emph{irreducible} if $\pi$ restricted
to $H^\pi$ is irreducible in the usual sense.

We say that $\pi,\pi'\in\Rep(A,H)$ are \emph{equivalent} if there is a unitary
from $H^\pi$ to $H^{\pi'}$ intertwining the representations.  In this case we
write $\pi \sim \pi'$. We say that
$\pi,\pi'$ are \emph{unitarily equivalent} if there is a unitary $U$ on $H$
with $\pi'(a) U = U \pi(a)$ for each $a\in A$.  We write $\pi\cong\pi'$ in
this case.

\begin{lemma}
$\pi\cong\pi'$ if and only if $\pi\sim\pi'$ and $(H^\pi)^\perp$ and 
$(H^{\pi'})^\perp$ have the same dimension.
\end{lemma}
\begin{proof}
Let $U$ intertwine $\pi$ and $\pi'$.
As $U\pi(a)\xi = \pi'(a)U\xi$ it follows that $U(H^\pi) \subseteq H^{\pi'}$.
As $U^*$ intertwines $\pi'$ and $\pi$, also $U^*(H^{\pi'}) \subseteq H^\pi$.
It follows that $U$ restricts to a unitary between $H^\pi$ and $H^{\pi'}$.
Hence $U$ also restricts to a unitary between $(H^\pi)^\perp$ and 
$(H^{\pi'})^\perp$.

Conversely, if $U$ is a unitary between $H^\pi$ and $H^{\pi'}$, and
$(H^\pi)^\perp$ and $(H^{\pi'})^\perp$ have the same dimension, then
we can extend $U$ to a unitary on all of $H$.
\end{proof}

Let $\rho:A\rightarrow\mc B(K)$ be a (possible degenerate) representation.
If $\dim K \leq \dim H$, then by choosing an isometric embedding of $K$ into
$H$, we may regard $\rho$ as a member of $\Rep(A,H)$, say $\pi$.  While $\rho$ depends on the embedding, the equivalence class of $\pi$, with respect to
$\sim$, depends only on $\pi$.
Further, notice that if $\rho:A\rightarrow\mc B(K)$ is irreducible (in the
usual sense) then any $\pi\in\Rep(A,H)$ which we associate with $\rho$ is
irreducible.

If $\rho_i:A\rightarrow\mc B(K_i)$ for $i=1,2$ are representations,
associated to $\pi_i\in\Rep(A,H)$, then notice that $\rho_1$ and $\rho_2$
are unitarily equivalent (in the usual sense) if and only if $\pi_1\sim\pi_2$.
Let us record these observations in a formal way.

\begin{proposition}
Let $R_0$ be the equivalence classes, for unitary equivalence, of
representations of $A$ on Hilbert spaces $K$ of dimension $\leq\dim H$.
There is a bijection between $R_0$ and the $\sim$ equivalence classes of
$\Rep(A,H)$.
\end{proposition}

In particular, this bijection is used without comment in
\cite[Section~2]{fell2}.


\section{The Fell topology}

We define a topology on $\Rep(A,H)$ by taking a sub-basic set about $\pi$
to be sets of the form
\[ M_{a,\xi,\eta,\epsilon}(\pi) =
\big\{ \pi'\in\Rep(A,H) : |(\eta|(\pi(a)-\pi'(a))\xi)|<\epsilon \big\} \]
where $a\in A$, $\xi,\eta\in H^\pi$ and $\epsilon>0$, or of the form
\[ P_{\xi,\epsilon}(T) = \{ \pi'\in\Rep(A,H) : \|P^{\pi'}\xi - \xi\|
< \epsilon \} \]
where $\xi\in H^\pi, \epsilon>0$, where $P^{\pi'}$ is the orthogonal
projection of $H$ onto $H^{\pi'}$.  Note that here
and elsewhere our inner products are linear on the right.
Equivalently, a net $(\pi_i)$ in $\Rep(A,H)$ converges to $\pi$ when
\[ \|\pi_i(a)\xi - \pi(a)\xi\| \rightarrow 0 \qquad (a\in A,\xi\in H^\pi). \]
The equivalence between these definitions is shown in \cite[page~239]{fell2}.
Further, by expanding norms in terms of inner products, it follows that
the $M_{a,\xi,\eta,\epsilon}$ sets alone will be sub-basic.

This topology is $T_0$ (given
$\pi\not=\pi'$ there is an open set containing exactly one of these
representations) but is not $T_1$ (so points need not be closed).  However,
if we consider the subspace of non-degenerate representations, with the
subspace topology, then the subspace is Hausdorff.

For $S \subseteq \Rep(A,H)$ define
\begin{align*}
S^e &= \{ \pi'\in\Rep(A,H) : \pi\sim\pi' \text{ for some } \pi\in S \}, \\
S^u &= \{ \pi'\in\Rep(A,H) : \pi\cong\pi' \text{ for some } \pi\in S \}.
\end{align*}
Then $S \subseteq S^u \subseteq S^e$.  Write $\Cl(S)$ for the closure of
$S$ in $\Rep(A,H)$ for our topology.

\begin{proposition}\label{prop:fell_top_props}
Given $S\subseteq\Rep(A,H)$ we have that:
\begin{enumerate}
\item $(\Cl S)^e \subseteq \Cl S^u$;
\item $(\Cl S)^e \subseteq \Cl S^e$ and $(\Cl S)^u \subseteq \Cl S^u$;
\item $\Cl S^e = \Cl S^u$;
\item Let $S' \subseteq S$ with $S'$ being open in the subspace topology
of $S$.  If $S=S^u$ then ${S'}^u$ is open in the subspace topology of $S$;
if $S=S^e$ then ${S'}^e$ is open in the subspace topology of $S'$;
\item\label{prop:fell_top_props:one}
 If $S$ is open, then $S^e$ and $S^u$ are open;
\item If $S = S^u$ then $(\Cl S)^u = (\Cl S)^e = \Cl S$.
\end{enumerate}
\end{proposition}

We work with the equivalence relation $\sim$.  For $\pi\in\Rep(A,H)$ let
$\Eq\pi$ the equivalence class of $\pi$.  For $S\subseteq\Rep(A,H)$ let
$\Eq S$ be the collection $\{\Eq\pi : \pi \in S\}$.  Finally let $\Eq\Rep(A,H)$
be the space of equivalence classes; we give this the quotient topology, so
that $U\subseteq \Eq\Rep(A,H)$ is open when $\{\pi\in\Rep(A,H):\Eq\pi\in U\}$
is open in $\Rep(A,H)$.

Now consider $S\subseteq \Rep(A,H)$.  There are two ways to give $\Eq S$ a
topology: either the subspace topology from $\Eq\Rep(A,H)$, or give $S$ the
subspace topology from $\Rep(A,H)$, and then use the surjection $S\rightarrow
\Eq S$ to give $\Eq S$ a topology.  In general these are different.

\begin{proposition}\label{prop:uni_equiv_im_unique}
Let $S=S^u$.  Then the two topologies on $\Eq S$ coincide.
\end{proposition}

For example, if $S$ is the collection of $\pi$ with $\dim H^\pi \leq \kappa$
for some cardinal $\kappa$, then $S=S^u$, for if $\pi\cong\rho$ then clearly
$H^\rho$ has the same dimension as $H^\pi$.


\subsection{Spectrum of a $C^*$-algebra}

Let $\hat A$ be the set of irreducible representations of $A$.  We give
$\hat A$ the Hull-Kernel topology.  This can be equivalently described using
the notion of weak containment.  Here we follow \cite[Chapter~3.4]{dix}.
Recall that $f\in A^*$ is a \emph{positive form associated} to a representation
$\pi:A\rightarrow\mc B(K)$ when there is $\xi\in K$ with
$f(a) = (\xi|\pi(a)\xi)$ for $a\in A$.  Similarly we have the notion of
a state associated to $\pi$.

\begin{proposition}\label{prop:wkcon}
Let $\pi$ be a representation of $A$, and let $S$ be a set of representations.
The following are equivalent, and define what is means for $\pi$ to be
\emph{weakly contained} in $S$:
\begin{enumerate}
\item\label{prop:wkcon:one}
$\bigcap_{\rho\in S} \ker\rho \subseteq \ker\pi$;
\item\label{prop:wkcon:two}
Every positive form on $A$ associated to $\pi$ is the weak$^*$-limit
of linear combinations of positive forms associated with $S$;
\item\label{prop:wkcon:three}
Every state on $A$ associated to $\pi$ is the weak$^*$-limit
of states which are sums of positive forms associated with $S$.
\end{enumerate}
\end{proposition}
\begin{proof}
That (\ref{prop:wkcon:three})$\implies$(\ref{prop:wkcon:two}) is clear.
If (\ref{prop:wkcon:two}) holds then if $\rho(a)=0$ for all $\rho\in S$,
then any positive form associated to $S$ will annihilate $a^*a$.
Thus every positive form associated to $\pi$ annihilates $a^*a$, so
$\pi(a)=0$, showing (\ref{prop:wkcon:one}).

The final implication is harder.
\end{proof}

We write $\pi \prec S$ in this case.  If $S=\{\rho\}$ is a singleton, then we
write $\pi\prec\rho$.  By condition (\ref{prop:wkcon:one}) we see that if
$\pi\prec\rho$ and $\rho\prec\sigma$ then $\ker\sigma \subseteq \ker\rho
\subseteq\ker\pi$ and so $\pi\prec\sigma$.

In the following, notice that we need not take sums, unlike in the previous
proposition.

\begin{theorem}\label{thm:wk-con-irrep}
Let $\pi\in\hat A$ and let $S\subseteq\hat A$.  The following are equivalent:
\begin{enumerate}
\item $\pi$ is in the closure of $S$, with respect to the Hull-Kernel
topology;
\item $\pi$ is weakly contained in $S$;
\item at least one non-zero positive form associated with $\pi$ is the
weak$^*$-limit of positive forms associated with $S$;
\item every state associated with $\pi$ is the weak$^*$-limit of states
associated with $S$.
\end{enumerate}
\end{theorem}

Let $S\subseteq\Rep(A,H)$ be the collection of irreducible representations.
Then $\Eq S \subseteq \Eq\Rep(A,H)$ can be identified
with the subset of $\hat A$ consisting of irreducible representations
$\rho:A\rightarrow\mc B(K)$ with $\dim K\leq\dim H$.  Using this, give
$\Eq S$ the subspace topology coming from the Hull-Kernel topology on $\hat A$.
Alternatively, we note that clearly $S = S^u$, and so
Proposition~\ref{prop:uni_equiv_im_unique} tells us that there is a
uniquely defined quotient topology on $\Eq S$ coming from the topology
on $\Rep(A,H)$.

\begin{theorem}
The Hull-Kernel, and quotient, topologies on $S$ coincide.
\end{theorem}

There is a useful link between the quotient topology and weak containment which
works for any representation (not necessarily irreducible).

\begin{theorem}\label{thm:top_wk_con}
Let $\pi\in\Rep(A,H)$ and let $T\subseteq\Rep(A,H)$.  Then $\pi \prec T$ if
and only if $\Eq\pi$ is in the closure of $\Eq T_0$ in the quotient topology,
where $T_0$ is the set of finite direct sums of members of $T$.
\end{theorem}

Another useful reference is \cite[Appendix~F]{bhv}.  While this reference only
considers group representations, it is relatively easy to translate results
back to $C^*$-algebras.  In particular, \cite[Definition~F.2.1]{bhv} defines
\emph{Fell's Topology} as follows.  Let $\mc R$ be the unitary equivalence
classes of representations of $A$, on a Hilbert space
of cardinality less than some fixed cardinal.  A basic open set $W$ about
$\pi\in\mc R$ is of the form $W = W(\pi,(\varphi_i)_{i=1}^n,Q,\epsilon)$,
where $\epsilon>0, Q\subseteq A$ is finite, and the $\varphi_i$ are each
positive forms associated to $\pi$.  There $\rho\in W$ exactly when for each
$\varphi_i$ there is $\psi$, a sum of positive forms associated to $\rho$,
with $|\varphi_i(a) - \psi(a)|<\epsilon$ for each $a\in Q$.

In the following, as usual, we identify $\mc R$ and $\Eq\Rep(A,H)$.

\begin{proposition}\label{prop:fells_is_quotient}
Let $\pi\in\mc R$ and let $(\pi_i)$ be a net in $\mc R$.
The following are equivalent:
\begin{enumerate}
\item\label{prop:fells_is_quotient:one}
$\pi_i \rightarrow \pi$ for Fell's topology on $\mc R$;
\item\label{prop:fells_is_quotient:two}
for any subnet $(\pi_{i(j)})$ of $(\pi_i)$, we have that
$\pi\prec \bigoplus_j \pi_{i(j)}$;
\item\label{prop:fells_is_quotient:three}
for any subnet $(\pi_{i(j)})$ of $(\pi_i)$, and letting $T_0$ be the collection
of finite direct sums of $\{\pi_j\}$, we have that $\pi\prec T_0$;
\item\label{prop:fells_is_quotient:four}
for any subnet $(\pi_{i(j)})$ of $(\pi_i)$, we have that
$\pi$ is in the closure of $\{\pi_j\}$ for Fell's topology on $\mc R$;
\item\label{prop:fells_is_quotient:six}
for any subnet $(\pi_{i(j)})$ of $(\pi_i)$, and letting $T_0$ be as before,
we have that $\Eq\pi$ is in the closure of $\Eq T_0$, for the quotient topology.
\end{enumerate}
\end{proposition}
\begin{proof}
That (\ref{prop:fells_is_quotient:one}) and (\ref{prop:fells_is_quotient:two})
are equivalent follows from the definition of Fell's topology on $\mc R$,
see \cite[Proposition~F.2.2]{bhv}.  That (\ref{prop:fells_is_quotient:one})
and (\ref{prop:fells_is_quotient:four}) are equivalent is a standard fact from
point-set toplogy, see Appendix~\ref{app:one}.
That (\ref{prop:fells_is_quotient:two}) and (\ref{prop:fells_is_quotient:three})
are equivalent follows from the definition of weak containment (we can
approximate a positive form associated with an infinite direct sum of
representations by a form associated with a finite direct sum).
That (\ref{prop:fells_is_quotient:three}) and (\ref{prop:fells_is_quotient:six})
are equivalent follows immediately from Theorem~\ref{thm:top_wk_con}.
\end{proof}

Hence it seems to me that Fell's topology, and the quotient topology on
$\Eq\Rep(A,H)$, do not agree (except on irreducibles).


\subsection{The inner hull kernel topology}

Let $X$ be any topological space, and let $C(X)$ be the family of closed subsets
of $X$.  We define a topology on $C(X)$ by taking as basic open sets
\[ U(A_1,\cdots,A_n) = \{ E\in C(X) : E \cap A_i\not=\emptyset \ (1 \leq i
\leq n) \}, \]
where $n\geq 1$ and each $A_i$ is open and non-empty in $X$.  This is the
\emph{inner topology} on $C(X)$.

For our $C^*$-algebra $A$, as before we consider $\Eq\Rep(A,H)$ to be (bijective
with) the collection of all unitary equivalence classes of non-degenerate
representations of $A$ on a Hilbert space of dimension $\leq\dim H$.  For
$\rho\in\Rep(A,H)$ the \emph{spectrum}, $\Sp(\rho)$, is the collection of those
$\pi\in\hat A$ with $\pi$ weakly contained in $\rho$.  Here we consider $\rho$
as a non-degenerate representation of $A$ on $H^\rho$.  Clearly $\Sp(\rho)$ does 
indeed only depend upon the equivalence class $\Eq\rho$.

\begin{lemma}
$\Sp(\rho)$ is closed in $\hat A$.
\end{lemma}
\begin{proof}
For any $\sigma\in\Sp(\rho)$, by definition, $\sigma\prec\rho$, so by
Proposition~\ref{prop:wkcon}, $\ker\rho \subseteq\ker\sigma$.  It follows
that if $X = \bigcap_{\sigma\in\Sp(\rho)} \ker \sigma$, then $\ker\rho \subseteq
X$.  Now let $\pi\in\hat A$ with $\pi\prec\Sp(\rho)$.  Then $X\subseteq\ker\pi$
so $\ker\rho \subseteq\ker\pi$ so $\pi\prec\rho$ so $\pi\in\Sp(\rho)$.  We have
shown that $\pi\prec\Sp(\rho) \implies \pi\in\Sp(\rho)$.  By 
Theorem~\ref{thm:wk-con-irrep}, we conclude that $\Sp(\rho)$ is closed.
\end{proof}

\begin{lemma}\label{lem:supp_in_dual_A}
For $\pi\in\hat A$, we have that $\Sp(\pi)$ equals the closure of $\{\pi\}$
in $\hat A$.
\end{lemma}
\begin{proof}
Follows directly from Theorem~\ref{thm:wk-con-irrep}.
\end{proof}

We define a closure operation on $\Eq\Rep(A,H)$ as follows.  For $S\subseteq
\Eq\Rep(A,H)$ define $\rho\in\overline{S}$ exactly when $\Sp(\rho)$ is in the
closure of $\{\Sp(\pi) : \pi\in S\}$ in $C(\hat A)$.  This defines the
\emph{inner hull-kernel topology} on $\Eq\Rep(A,H)$.  That the Kuratowski
closure axioms hold follows easily because they must hold for the closure
operation in $C(\hat A)$.

If $\pi,\rho\in\Rep(A,H)$ with $\pi\prec\rho$ and $\rho\prec\pi$ (that is,
$\pi$ and $\rho$ are \emph{weakly equivalent}) then clearly $\Sp(\pi)=\Sp(\rho)$.
It follows that the inner hull-kernel topology will not distinguish $\Eq\pi$
and $\Eq\rho$.  In particular, it does not distinguish $\pi$ from a multiple
of $\pi$.

This definition seems terribly complicated to me, and Fell makes no further
discussion.  There is an alternative presentation in \cite[Chapter~5]{kan},
and in particular a very different definition in the form of
\cite[Definition~5.5]{kan}.  We now show that these definitions actually agree.

\begin{proposition}
The inner hull-kernel topology on $\Eq\Rep(A,H)$ has as basic open
sets collections of the form
\[ \{\Eq\pi\in\Eq\Rep(A,H) : \Sp(\pi) \cap A_i\not=\emptyset \} \]
where $A_1,\cdots,A_n\subseteq \hat A$ are open and non-empty.
\end{proposition}
\begin{proof}
Let us abstract the definition of the closure operation of $\Eq\Rep(A,H)$.
Let $Y$ be a set, $X$ a topological space, and $\theta:Y\rightarrow X$ a map.
Define a closure operation on $Y$ as follows: for $S\subseteq Y$ define
$t\in\overline{S}$ exactly when $\theta(t) \in \overline{\theta(S)}$.
This gives the inner hull-kernel topology when we let $Y=\Eq\Rep(A,H), 
X=C(\hat A)$ and let $\theta$ be the map $\Eq\pi \mapsto \Sp(\pi)$.

Let $\tau=\{\theta^{-1}(U) : U\subseteq X \text{ is open}\}$ which is a
topology on $Y$ (in fact the coarsest topology making $\theta$ continuous).
For $S\subseteq Y$ we have that $t\in\overline{S}$ when, for any open $U\subseteq 
X$ with $\theta(t)\in U$, we have that $\theta(S)\cap U\not=\emptyset$.
Equivalently, for any $V\in\tau$ with $t\in V$, we have that $S\cap V\not=
\emptyset$.  However, this is just the closure operation given by the topology
$\tau$.

Applied to $\Eq\Rep(A,H)$, we just pull back the topology on $C(\hat A)$
using $\Sp$.  But pulling back the basic open sets in $C(\hat A)$ gives exactly
the sets in the statement of the proposition.
\end{proof}

Restrict the inner hull-kernel topology to the irreducibles in $\Eq\Rep(A,H)$.
By Lemma~\ref{lem:supp_in_dual_A}, the basic open sets are of the form
$\{\Eq\pi : \overline{\{\pi\}} \cap A_i\not=\emptyset \}$ for non-empty, open
$A_i$.  But $\overline{\{\pi\}} \cap A_i\not=\emptyset$ exactly when $\pi\in A_i$,
and so we just recover the (subspace) topology of $\hat A$.  Thus, restricted
to the irreducibles, the inner hull-kernel topology agrees with the hull-kernel
topology on $\hat A$.

We can describe the inner hull-kernel topology using positive forms.
Notice that if $f\in A^*$ is a positive form associated with $\pi\in\Rep(A,H)$,
then if $\pi\sim\rho$, also $f$ is associated to $\rho$.  Thus we may speak of
$f$ being associated with an equivalence class $\Eq\pi$.

For $\pi\Eq\Rep(A,H)$, we describe some sets containing $\pi$.  Let $\epsilon>0$,
let $(f_i)_{i=1}^n$ be positive forms associated to $\pi$, and let $(a_i)_{i=1}^m
\subseteq A$.  Define $U$ to be the collection of $\rho\in\Eq\Rep(A,H)$ such that
there are $(g_i)_{i=1}^n$, positive forms associated to $\rho$, so that
\begin{align*}
    | f_i(a_j) - g_i(a_j) | <\epsilon \qquad (1\leq i\leq n, 1\leq j\leq m), \\
    \big| \|f_i\| - \|g_i\| \big| < \epsilon \qquad (1\leq i\leq n).
\end{align*}

\begin{theorem}
Such $U$ forms a basis of open\footnote{Fell does not say ``open'' in \cite{fell2}
but I think such $U$ are open} neighbourhoods of $\pi$ for the inner hull-kernel
topology.
\end{theorem}

Every open inner hull-kernel topology set is also open in the quotient topology
on $\Eq\Rep(A,H)$, but the converse is not true.  Indeed, the inner hull-kernel
topology cannot distinguish between a representation $\pi$ and multiples of
$\pi$, while of course the quotient topology can.

\begin{proposition}
Let $S$ be the set of $\pi\in\Rep(A,H)$ such that $\pi$ is equivalent to a
countably-infinite multiple of $\pi$.  Then the inner hull-kernel topology
and the quotient topology agree on $\Eq S$.
\end{proposition}

We now turn attention to the presentation of the inner hull-kernel topology
given in \cite[Chapter~5]{kan}.  This is in many ways easier to follow
than Fell's original presentation; however, we must be a little careful as
here we fix $H$ of sufficiently large size: the cardinality of $\dim(H)$ should
be greater than or equal to the cardinality of $A$.  It is not clear to me
how much this assumption is really used.

\begin{proposition}
Let $\pi\in\Rep(A,H)$ and $T\subseteq\Rep(A,H)$, and consider the following
conditions:
\begin{enumerate}
\item\label{prop:wk_con_ihk:one} $\pi$ is weakly contained in $T$;
\item\label{prop:wk_con_ihk:two} $\Sp(\pi)$ is contained in the closure of
$\bigcup\{\Sp(\tau) : \tau\in T\}$ in $\hat A$;
\item\label{prop:wk_con_ihk:three} $\Eq\pi$ belongs to the inner hull-kernel
topology closure of $\Eq T$ in $\Eq\Rep(A,H)$.
\end{enumerate}
Then (\ref{prop:wk_con_ihk:three}) $\Rightarrow $(\ref{prop:wk_con_ihk:two})
$\Leftrightarrow$ (\ref{prop:wk_con_ihk:one}), and if $\pi$ is irreducible,
then also (\ref{prop:wk_con_ihk:two}) $\Rightarrow$
(\ref{prop:wk_con_ihk:three}).
\end{proposition}

We can apply this abstract work to the $C^*$-algebra $C^*(G)$; one can
describe the topology in terms of the group alone.  Of note is the result
that the operation of taking the tensor product is jointly continuous for
the inner hull-kernel topology.


\section{Correspondences}

We now apply the above to the study of correspondences of von Neumann algebras.
There is a nice summary of results in \cite[page~316]{ad1}.  There is more
discussion in \cite[Section~1.12]{ad2}, which references the original \cite{cj}.
[\footnote{More brief lit review here?}]

A \emph{correspondence} between von Neumann algebras $M$ and $N$ is a Hilbert
space $H$ with commuting normal representations of $M$ and $N^\op$.  Here we
work with $N^\op$ so as to regard $N$ as a \emph{right} action of $N$ on $H$.
We hence write $x\xi y$ for $x\in M, y\in N, \xi\in H$.
We can linearise this bilinear definition by using the \emph{binormal}
tensor product, as defined in \cite{el}.  This gives a $C^*$-algebra norm on
the algebraic tensor product $M\odot N^\op$, leading to the completion
$M\otimes_{\bin}N^\op$.  Of interest is \cite[Theorem~4.1]{el} which shows
that $M$ is semidiscrete if and only if $M\otimes_\bin N^\op = M\otimes_{\min}
N^\op$ for all $N$, where $M\otimes_{\min} N^\op$ is the $C^*$-algebraic
spatial tensor product.

Write $\Corr(M,N)$ for the collection of all correspondences $H$ from $M$ to
$N$.  Then $\Corr(M,N)$ bijects with the representations of $A = 
M\otimes_\bin N^\op$, say $\pi:A\rightarrow\mc B(H)$ such that the restriction
of $\pi$ to $M$ and $N^\op$ are normal (that is, $\pi$ is separately normal).
Let $S\subseteq\Rep(A,H)$ be the binormal representation.  It is easy to see
that $S = S^e = S^u$, and so the quotient topology on $\Eq S$ is well-defined.

Now restrict $\Corr(M,N)$ to be the correspondences whose Hilbert space
dimension does not exceed some fixed cardinal.
The definition from \cite{cj} and \cite[Section~1.12]{ad2} is that a basic
open neighbourhood of $H_0\in\Corr(M,N)$ is of the form
\[ U=U(H_0; \epsilon,E,F,S) =
\big\{ H : \exists (k_i)_{i=1}^n\subseteq H, \ 
|(k_i|xk_jy) - (h_i|xh_jy)|<\epsilon \ 
(x\in E, y\in F, 1\leq i,j\leq n)
\big\}. \]
Here $E\subseteq M, F\subseteq N$ are finite subsets, $\epsilon>0$, and
$S=(h_i)_{i=1}^n \subseteq H_0$.

It is stated without justification that this is the Fell topology on
$M\otimes_\bin N^\op$.  It seems to me that there is some work here!

Alternatively, we can think of correspondences as self-dual Hilbert 
$C^*$-bimodules, see \cite{ad1, ad2}.  That is, self-dual Hilbert $C^*$-modules
$X$ over $N$ (so $X$ is a right module over $N$, and $X$ has an $N$-valued
inner product) together with a normal $*$-homomorphism from $M$ to the
adjointable operators on $X$, turning $X$ into a left $M$-module.

The associated topology on bimodules is as follows (following
\cite[Section~1.12]{ad2} although we note that the proof presented in \cite{ad2}
assumes that $N$ and $M$ are $\sigma$-finite).  Write $C(M,N)$ for the
$M$-$N$-bimodules.  A basic
open neighbourhood of $X_0\in C(M,N)$ is of the form
\[ V=V(X_0; \mc V, E,S) = \big\{
X : \exists(\eta_i)_{i=1}^n\subseteq X, \ 
(\eta_i|x\eta_j) - (\xi_i|x\xi_j) \in\mc V \ 
(1\leq i,j\leq n, x\in E)
\big\}. \]
Here $\mc V$ is weak$^*$-open neighbourhood of $0$ in $N$, $E\subseteq M$ is
finite, and $(\xi_i)_{i=1}^n \subseteq X_0$.


\subsection{Equivalence with the Fell topology}

We first translate the above topology on correspondences into a statement about
representations.  Let $H$ be a Hilbert space of dimension our fixed cardinal,
let $A = M \otimes_{\bin} N^\op$, and consider $\Rep(A,H)$.  Given a
correspondence $K$ from $M$ to $N$, we can equivalently view $K$ as a
representation $\rho : A\rightarrow\mc B(K)$.  As $\dim K\leq \dim H$ by our
standing assumption, there is an isometry $v:K\rightarrow H$.  Set $\pi(a)
= v\rho(a)v^*$ for $a\in A$, so that $\pi\in\Rep(A,H)$ with $H^\pi = v(K)$.
If $\pi\cong\sigma$ then composing the unitary implementing this equivalence
with $v$, we see that $\sigma$, restricted to $H^\sigma$, is unitarily
equivalent to $\rho$.  A similar remark replies in the case when $\pi\sim\sigma$.
Thus $\Eq\pi$ depends only on $\rho$, and not the choice of $v$.

The topology on $\Corr(M,N)$, viewed as representations of $A$, has as basic
open sets about $\pi_0$ sets of the form
\[ U' = U'(\pi_0; \epsilon, E, S)
= \big\{ \pi : \exists (k_i)_{i=1}^n\subseteq H, \ 
|(k_i|\pi(a)k_j) - (h_i|\pi_0(a)h_j)|<\epsilon \ 
(a\in E, 1\leq i,j\leq n)
\big\}. \]
Here $E\subseteq M\odot N^\op$ is finite, $\epsilon>0$, and
$S=(h_i)_{i=1}^n \subseteq H_0$.  A tedious but routine argument
(compare to Section~\ref{app:app_to_fell}) shows that if
we allow $S\subseteq A$ finite, then we obtain the same topology.
Notice also that these sets satisfy the conditions of 
Theorem~\ref{thm:when_have_basis_nhoods} and so do indeed define a unique
topology.

[\footnote{To be finished.}]



\appendix
\section{Point-set topology}\label{app:one}

\subsection{Nets}

\begin{definition}
A \emph{directed set} is a set $I$ with a relation $\leq$ such that:
$i\leq j$ and $j\leq k$ implies $i\leq k$; if $i\leq j$ and $j\leq i$ then
$i=j$; for any $i,j\in I$ there is $k\in I$ with $i\leq k$ and $j\leq k$.

A \emph{net} in a set $X$ is a function $I\rightarrow X$ from some directed
set $I$.  We typically write $(x_i)_{i\in I}$ for a net.

A \emph{subnet} of a net $(x_i)_{i\in I}$ is a function $J\rightarrow I$,
with $J$ a directed set, written as $j\mapsto i(j)$, and the subnet denoted
by $(x_{i(j)})_{j\in J}$, with: $j_1\leq j_2$ implies that $i(j_1)\leq i(j_2)$,
and for each $i'\in I$ there is $j\in J$ with $i'\leq i(j)$.
\end{definition}

We recall that in a topological space $X$ we say that a net $(x_i)$ converges
to $x$ if for each open set $x\in U$ there is $i_0$ so that $i\leq i_0$ implies
that $x_i\in U$.

\begin{proposition}
If a net $(x_i)$ converges to $x$ then any subnet $(x_{i(j)})_{j\in J}$
also converges to $x$.
\end{proposition}
\begin{proof}
Exercise.
\end{proof}

\begin{proposition}
A net $(x_i)$ converges to $x$ if and only if, for any subnet
$(x_{i(j)})_{j\in J}$, the closure of $\{ x_{i(j)} : j\in J \}$ contains $x$.
\end{proposition}
\begin{proof}
``only if'' follows from the previous proposition.  We show the contrapositive
of ``if'', so suppose that $(x_i)$ does not converge to $x$.  So there is an
open set $x\in U$ so that for each $i_0\in I$ there is $i\geq i_0$ with $x_i
\not\in U$.  Let $J = \{ i\in I : x_i\not\in U \}$, and give $J$ the relative
ordering from $I$.  Given $i,j\in J$ there is $k_0\in I$ with $k_0\geq i,
k_0\geq j$.  By assumption, there is $k\in A$ with $k\geq k_0$, so that $k\geq i,
k\geq j$.  Thus $J$ is a directed set, and furthermore, the inclusion
$J\rightarrow I$ defines a subnet.  As $U \cap \{ x_j : j\in J \}$ is empty,
the closure of $\{ x_j : j\in J \}$ is disjoint from $x$, as required.
\end{proof}

\subsection{Bases}

The notion of a \emph{base} for a topology is standard: a collection $\mc U$ of
subsets of $X$ so that every open set in $X$ is a union of members of $\mc U$.
A \emph{subbase} is a collection $\mc V$ such that if $\mc U$ is the collection
of finite intersections of members of $\mc V$, then $\mc U$ is a base.

A less common notion is to, for each $x\in X$, define a \emph{base of open
neighbourhoods of $x$}.  This is, for each $x\in X$, a collection $\mc U_x$ of
subsets of $X$ such that:
\begin{enumerate}
\item\label{defn:bon:one}
For each $x\in X$ and $A\in\mc U_x$, we have that $x\in A$;
\item\label{defn:bon:two}
Each $\mc U_x$ should be a filter-base for the collection of all open
neighbourhoods of $x$.  That is, each $A\in\mc U_x$ is open, 
and if $U\subseteq X$ is open and contains $x$,
there is $A\in \mc U_x$ with $A\subseteq U$.
\end{enumerate}

\begin{theorem}\label{thm:when_have_basis_nhoods}
Let $X$ be a set and $(\mc U_x)_{x\in X}$ satisfy (\ref{defn:bon:one}).
There is a topology on $X$ with (\ref{defn:bon:two}) holding if and only if:
\begin{enumerate}\setcounter{enumi}{2}
\item\label{defn:bon:three}
For $x\in X$ and $A_1,A_2,\cdots,A_n\in\mc U_x$, there is $A\in \mc U_x$ with
  $A\subseteq A_1\cap\cdots\cap A_n$;
\item\label{defn:bon:four}
For $x,y\in X$ and $B\in\mc U_y$, if $x\in B$, there is $A\in\mc U_x$ with
  $A\subseteq B$.
\end{enumerate}
In this case, the resulting topology on $X$ is unique.
\end{theorem}
\begin{proof}
Suppose a topology $\tau$ exists on $X$ satisfying (\ref{defn:bon:two}).
Then (\ref{defn:bon:three}) holds as each $A_i$ will be open, so we can
apply (\ref{defn:bon:two}) with $U=A_1\cap\cdots\cap A_n$.  Then
(\ref{defn:bon:four}) will hold as $B$ is open and contains $x$.

Conversely, let $\tau_0$ be the topology generated by the $\mc U_x$, that is,
members of $\tau_0$ are arbitrary unions of sets of the form $A_1\cap\cdots\cap
A_n$ where $A_i\in\mc U_{x_i}$ for some family $(x_i)_{i=1}^n$ (we allow
repeats: each $A_i$ could be from the same $\mc U_x$, for example).  Then each
$A\in\mc U_x$ is in $\tau_0$, and if $x\in U\in\tau_0$ then there are
$A_i\in\mc U_{x_i}$ with $x\in A_1\cap\cdots\cap A_n\subseteq U$.  If
$x_i\not=x$ then apply (\ref{defn:bon:four}) to find $A_i'\in\mc U_x$ with
$A_i'\subseteq A_i$.  Thus we may assume that each $A_i$ is actually a member of
$\mc U_x$, and then (\ref{defn:bon:three}) provides $A\in\mc U_x$ with
$A\subseteq A_1\cap\cdots\cap A_n \subseteq U$.  We have hence verified that
(\ref{defn:bon:two}) holds for $\tau_0$.

Now let
\[ \tau_1 = \big\{ B\subseteq X : x\in B \implies \exists\, A\in\mc U_x,
A\subseteq B \ (x\in X) \big\}. \]
Then $\emptyset, X\in \tau_1$ and $\tau_1$ is closed under arbitrary unions.
Condition (\ref{defn:bon:three}) readily implies that $\tau_1$ is closed under
finite intersections, and so $\tau_1$ is a topology.  If $\tau$ is any topology
on $X$ with (\ref{defn:bon:two}) holding, then clearly any $U\in\tau$ is also a
member of $\tau_1$.  Hence $\tau_0 \subseteq \tau \subseteq \tau_1$.

Finally, let $B\in\tau_1$, so if $x\in B$ there is $A_x\in\mc U_x$ with $A_x
\subseteq B$.  Thus $B = \bigcup \{ A_x : x\in B \}$ and so $B\in\tau_0$.  Hence
$\tau_0 = \tau_1$ and so the resulting topology is unique.
\end{proof}

Let $\mc V_x$ be a collection of open sets containing $x$, with the
property that for each $A\in\mc U_x$ there is $B\in\mc V_x$ with $B\subseteq A$.
In the following, we shall refer to this as ``passing to a cofinal family''.
The preceding discussion now easily shows that we can replace $\mc U_x$ by
$\mc V_x$ and still generate the same topology (for example, the $\tau_1$
generated will clearly be the same).


\subsubsection{Application to the Fell topology}\label{app:app_to_fell}

We defined the topology on $\Rep(A,H)$ by specifying as sub-basic sets
$\{ \pi'\in\Rep(A,H) : |(\eta|(\pi(a)-\pi'(a))\xi)|<\epsilon \}$.  That is,
the basic open sets are
\[ \{ \pi'\in\Rep(A,H) : |(\eta_i|(\pi_i(a_i)-\pi'(a_i))\xi_i)|<\epsilon_i 
\ (1\leq i\leq n) \}. \]
However, we might instead define the basic open sets about $\pi$ to be of the
form
\[ \{ \pi'\in\Rep(A,H) : |(\eta_i|(\pi(a_i)-\pi'(a_i))\xi_i)|<\epsilon_i 
\ (1\leq i\leq n) \}. \]
These sets clearly satisfy condition (\ref{defn:bon:three}); we now verify
(\ref{defn:bon:four}).  Suppose that $\pi_0 \in B\in\mc U_{\pi_1}$, say
$|(\eta_i|(\pi_1(a_i)-\pi_0(a_i))\xi_i)|<\epsilon_i$ for $1\leq i\leq n$.  For
each $i$, set $\delta_i = |(\eta_i|(\pi_1(a)-\pi_0(a))\xi_i)|$ and consider the
open neighbourhood $A$ of $\pi_0$ defined by 
\[ A = \{\pi' : |(\eta_i|(\pi_0(a_i)-\pi'(a_i))\xi_i)|<\epsilon_i-\delta_i
(1\leq i\leq n) \}. \]
Then, if $\pi'\in A$, the triangle inequality shows that $\pi'\in B$, so that
$A\subseteq B$ as required.

Thus the Fell topology on $\Rep(A,H)$ is generated by these basic open
neighbourhoods.  As remarked above, we can replace these sets by a ``cofinal
family'' and maintain the topology.  For example, we could define a basic open
neighbourhood of $\pi$ to be a set of the form
\[ \big\{ \pi'\in\Rep(A,H) : |(\eta|(\pi(a)-\pi'(a))\xi)|<\epsilon
\ (\xi,\eta\in E,a\in F) \big\}, \]
where $\epsilon>0$ and $E\subseteq H, F\subseteq A$ are finite.

Alternatively, we could keep the same form, but restrict $E$ to be an
orthonormal set $E=\{e_1,\cdots,e_n\}$.  Indeed, for any $E'\subseteq H$ finite,
let $E$ be an orthonormal basis for the linear span of $E'$.  Choose
$\epsilon>0$.  Suppose that $|(e_i|(\pi(a)-\pi'(a))e_j)|<\delta$ for all
$1\leq i,j\leq n$ and $a\in F$.  Then for $\xi=\sum_i \xi_ie_i$ and
$\eta=\sum_j \eta_je_j$ in $E'$, we have that
\[ |(\eta|(\pi(a)-\pi'(a))\xi)| \leq
\sum_{i,j=1}^n |\eta_j| |\xi_i| \delta
\leq n\delta \Big(\sum_i |\xi_i|^2\Big)^{1/2}
\Big(\sum_j |\xi_j|^2\Big)^{1/2}
= n\delta\|\xi\|\|\eta\|. \]
Set $M=\max\{\|\xi\|:\xi\in E'\}$ so that if $n M^2 \delta < \epsilon$,
we may conclude that $\pi'$ is a member of the basic open neighbourhood defined
by $(\epsilon,E',F)$.

Similarly, we could suppose that each member of $E$ was either a member of
$H^\pi$, or a member of $(H^\pi)^\perp$, as by adjusting $\epsilon$, this again
constitutes passing to a cofinal family.


\subsubsection{Application to quotient topologies}

Let $X$ be a topological space, the topology given by specifying basic open
neighbourhoods $(\mc U_x)_{x\in X}$, and let $f:X\rightarrow Y$ be a surjection,
used to give $Y$ the quotient topology.  This is the same as defining an
equivalence relation on $X$ by $x\sim x'$ if and only if $\phi(x)=\phi(x')$.
In particular, if $A\subseteq X$ then $\{x\in X : x\sim x' \text{ for some }
x\in A\} = \phi^{-1}(\phi(A))$.

We shall assume that if $A\subseteq X$ is open, then $\phi^{-1}(\phi(X))$ is
open; equivalently, $\phi$ is an open map.  By
Proposition~\ref{prop:fell_top_props}(\ref{prop:fell_top_props:one}) the Fell
topology on $\Rep(A,H)$ and the map $Eq$ satisfy this condition.

\begin{proposition}
Let $\psi:Y\rightarrow X$ be some function with $\phi(\psi(y))=y$ for each
$y\in Y$.  Let $\mc V_y = \{ \phi(A) : A\in\mc U_{\psi(y)}\}$.  Then each
$\mc V_y$ is a collection of basic open neighbourhoods of $y$ which together
generate the quotient topology on $Y$.
\end{proposition}
\begin{proof}
For $B=\phi(A)\in\mc V_y$ we have that $x=\psi(y)\in A$ so $\phi(x) = y
\in B$.  As $\phi$ is open, each member of $\mc V_y$ is open.  If $U\subseteq Y$
is open and $y\in U$, then $\phi^{-1}(U)$ is open and $x=\psi(y)\in
\phi^{-1}(U)$.  Thus there is $A\in\mc U_x$ with $A\subseteq\phi^{-1}(U)$, that
is, $\phi(A)\subseteq U$.  So there is $B\in\mc V_y$ with $B\subseteq U$.
Hence the conditions (\ref{defn:bon:one}) and (\ref{defn:bon:two}) hold, and
so the $(\mc V_y)$ generate a unique topology, which must be the quotient
topology.
\end{proof}

Let $\mc R$ be the collection of unitary equivalence classes of representations
of $A$ on Hilbert spaces of dimension no more than $\dim H$.  Then $\mc R$ is
naturally identified with the quotient space $\Eq\Rep(A,H)$.  In this case we
can define $\psi:\mc R\rightarrow\Rep(A,H)$ by, for each representation
$\pi:A\rightarrow\mc B(K)$, picking an isometry $v:K\rightarrow H$, and setting
$\psi(\pi) = v\pi(\cdot)v^* \in\Rep(A,H)$.  The resulting basic open
neighbourhoods of $\pi$ are of the form
\[ \big\{ \pi':A\rightarrow\mc B(K') : \exists v:K'\rightarrow H, \
|(v^*\eta|\pi'(a)v^*\xi) - (u^*\eta|\pi(a)u^*\xi)|<\epsilon \ (\xi,\eta\in E,
a\in F) \big\}, \]
where $v$ is meant to be an isometry depending on $\pi'$, and $\epsilon>0$
and $E\subseteq H,F\subseteq A$ are finite.




\begin{thebibliography}{99}

\bibitem{ad1} C. Anantharaman-Delaroche, Amenable correspondences and approximation properties for von Neumann algebras, Pacific J. Math. {\bf 171} (1995), no.~2, 309--341. MR1372231

\bibitem{ad2} C. Anantharaman-Delaroche, On relative amenability for von Neumann algebras, Compositio Math. {\bf 74} (1990), no.~3, 333--352. MR1055700

\bibitem{bhv} B. Bekka, P. de la Harpe\ and\ A. Valette, {\it Kazhdan's property (T)}, New Mathematical Monographs, 11, Cambridge University Press, Cambridge, 2008. MR2415834

\bibitem{cj} A. Connes\ and\ V. Jones, Property $T$ for von Neumann algebras, Bull. London Math. Soc. {\bf 17} (1985), no.~1, 57--62. MR0766450

\bibitem{dix} J. Dixmier, {\it $C\sp*$-algebras}, translated from the French by Francis Jellett, North-Holland Publishing Co., Amsterdam, 1977. MR0458185

\bibitem{el} E. G. Effros\ and\ E. C. Lance, Tensor products of operator algebras, Adv. Math. {\bf 25} (1977), no.~1, 1--34. MR0448092

\bibitem{fell1} J. M. G. Fell, $C\sp{\ast} $-algebras with smooth dual, Illinois J. Math. {\bf 4} (1960), 221--230. MR0124754

\bibitem{fell2} J. M. G. Fell, Weak containment and induced representations of groups, Canad. J. Math. {\bf 14} (1962), 237--268. MR0150241

\bibitem{kan} E. Kaniuth\ and\ K. F. Taylor, {\it Induced representations of locally compact groups}, Cambridge Tracts in Mathematics, 197, Cambridge University Press, Cambridge, 2013. MR3012851

\end{thebibliography}


\end{document}
