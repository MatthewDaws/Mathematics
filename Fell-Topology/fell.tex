\documentclass[a4paper,11pt]{article}
\usepackage[utf8]{inputenc}
\usepackage[margin=2cm]{geometry}
\usepackage{latexsym, amsmath, amsthm, amssymb}

\newcommand{\mc}[1]{{\mathcal{#1}}}
\newcommand{\Rep}{\operatorname{Rep}}
\newcommand{\Cl}{\operatorname{Cl}}
\newcommand{\Eq}{\operatorname{Eq}}

\newtheorem{lemma}{Lemma}
\newtheorem{proposition}[lemma]{Proposition}
\newtheorem{theorem}[lemma]{Theorem}

\title{Notes of the Fell topology}
\author{Matt Daws}
\date{Last knowing modified 16 March 2019}

\begin{document}

\maketitle

\begin{abstract}
We provide an overview of the Fell topology on the representation space of a
$C^*$-algebra.  We then, in more detail, study this topology as applied to
correspondences of von Neumann algebras.
\end{abstract}

\section{Introduction}

Fell introduced the eponymously named topology in \cite{fell1}, and further
refined the definitions and properties in \cite{fell2}.  Fix a $C^*$-algebra
$A$.  Given a Hilbert space $H$, let $\Rep(A,H)$ be the collection of all
non-zero $*$-homomorphisms $A\rightarrow\mc B(H)$.  For $\pi\in\Rep(A,H)$ the
\emph{essential space} of $\pi$ is $H^\pi$, the closed linear span of
$\{\pi(a)\xi : a\in A, \xi\in H\}$.  If $H^\pi = H$ then we say that $\pi$
is \emph{non-degenerate}; otherwise $\pi$ is \emph{degenerate}, which we
explicitly allow.  Finally, $\pi$ is \emph{irreducible} if $\pi$ restricted
to $H^\pi$ is irreducible in the usual sense.

We say that $\pi,\pi'\in\Rep(A,H)$ are \emph{equivalent} if there is a unitary
from $H^\pi$ to $H^{\pi'}$ intertwining the representations.  In this case we
write $\pi \sim \pi'$. We say that
$\pi,\pi'$ are \emph{unitarily equivalent} if there is a unitary $U$ on $H$
with $\pi'(a) U = U \pi(a)$ for each $a\in A$.  We write $\pi\cong\pi'$ in
this case.

\begin{lemma}
$\pi\cong\pi'$ if and only if $\pi\sim\pi'$ and $(H^\pi)^\perp$ and 
$(H^{\pi'})^\perp$ have the same dimension.
\end{lemma}
\begin{proof}
Let $U$ intertwine $\pi$ and $\pi'$.
As $U\pi(a)\xi = \pi'(a)U\xi$ it follows that $U(H^\pi) \subseteq H^{\pi'}$.
As $U^*$ intertwines $\pi'$ and $\pi$, also $U^*(H^{\pi'}) \subseteq H^\pi$.
It follows that $U$ restricts to a unitary between $H^\pi$ and $H^{\pi'}$.
Hence $U$ also restricts to a unitary between $(H^\pi)^\perp$ and 
$(H^{\pi'})^\perp$.

Conversely, if $U$ is a unitary between $H^\pi$ and $H^{\pi'}$, and
$(H^\pi)^\perp$ and $(H^{\pi'})^\perp$ have the same dimension, then
we can extend $U$ to a unitary on all of $H$.
\end{proof}

Let $\rho:A\rightarrow\mc B(K)$ be a (possible degenerate) representation.
If $\dim K \leq \dim H$, then by choosing an isometric embedding of $K$ into
$H$, we may regard $\rho$ as a member of $\Rep(A,H)$, say $\pi$.  While $\rho$ depends on the embedding, the equivalence class of $\pi$, with respect to
$\sim$, depends only on $\pi$.

\section{The Fell topology}

We define a topology on $\Rep(A,H)$ by taking a sub-basic set about $\pi$
to be sets of the form
\[ M_{a,\xi,\eta,\epsilon}(\pi) =
\big\{ \pi'\in\Rep(A,H) : |(\eta|(\pi(a)-\pi'(a))\xi)|<\epsilon \big\} \]
where $a\in A$, $\xi,\eta\in H$ and $\epsilon>0$.  Note that here and elsewhere
our inner products are linear on the right.
Equivalently, a net $(\pi_i)$ in $\Rep(A,H)$ converges to $\pi$ when
$(\eta|\pi_i(a)\xi) \rightarrow (\eta|\pi(a)\xi)$ for each $a\in A$ and 
$\xi,\eta\in H$.

This is a simpler definition than in \cite{fell1}, but the equivalence
is shown in \cite[?TODO?]{fell2}.  This topology is $T_0$ (given
$\pi\not=\pi'$ there is an open set containing exactly one of these
representations) but is not $T_1$ (so points need not be closed).  However,
if we consider the subspace of non-degenerate representations, with the
subspace topology, then the subspace is Hausdorff.

For $S \subseteq \Rep(A,H)$ define
\begin{align*}
S^e &= \{ \pi'\in\Rep(A,H) : \pi\sim\pi' \text{ for some } \pi\in S \}, \\
S^u &= \{ \pi'\in\Rep(A,H) : \pi\cong\pi' \text{ for some } \pi\in S \}.
\end{align*}
Then $S \subseteq S^u \subseteq S^e$.  Write $\Cl(S)$ for the closure of
$S$ in $\Rep(A,H)$ for our topology.

\begin{proposition}
Given $S\subseteq\Rep(A,H)$ we have that:
\begin{enumerate}
\item $(\Cl S)^e \subseteq \Cl S^u$;
\item $(\Cl S)^e \subseteq \Cl S^e$ and $(\Cl S)^u \subseteq \Cl S^u$;
\item $\Cl S^e = \Cl S^u$;
\item Let $S' \subseteq S$ with $S'$ being open in the subspace topology
of $S$.  If $S=S^u$ then ${S'}^u$ is open in the subspace topology of $S$;
if $S=S^e$ then ${S'}^e$ is open in the subspace topology of $S'$;
\item If $S$ is open, then $S^e$ and $S^u$ are open;
\item If $S = S^u$ then $(\Cl S)^u = (\Cl S)^e = \Cl S$.
\end{enumerate}
\end{proposition}

We work with the equivalence relation $\sim$.  For $\pi\in\Rep(A,H)$ let
$\Eq\pi$ the equivalence class of $\pi$.  For $S\subseteq\Rep(A,H)$ let
$\Eq S$ be the collection $\{\Eq\pi : \pi \in S\}$.  Finally let $\Eq\Rep(A,H)$
be the space of equivalence classes; we give this the quotient topology, so
that $U\subseteq \Eq\Rep(A,H)$ is open when $\{\pi\in\Rep(A,H):\Eq\pi\in U\}$
is open in $\Rep(A,H)$.

Now consider $S\subseteq \Rep(A,H)$.  There are two ways to give $\Eq S$ a
topology: either the subspace topology from $\Eq\Rep(A,H)$, or given $S$ the
subspace topology from $\Rep(A,H)$, and then use the surjection $S\rightarrow
\Eq S$ to give $\Eq S$ a topology.  In general these are different.

\begin{proposition}
Let $S=S^u$.  Then the two topologies on $\Eq S$ coincide.
\end{proposition}


\subsection{Spectrum of a $C^*$-algebra}

Let $\hat A$ be the set of irreducible representations of $A$.  We give
$\hat A$ the Hull-Kernel topology; equivalently, this can be described using
the notion of weak containment.  Here we follow \cite[Chapter~3.4]{dix}.
Recall that $f\in A^*$ is a \emph{positive form associated} to a representation
$\pi:A\rightarrow\mc B(K)$ when there is $\xi\in K$ with
$f(a) = (\xi|\pi(a)\xi)$ for $a\in A$.  Similarly we have the notion of
a state associated to $\pi$.

\begin{proposition}\label{prop:wkcon}
Let $\pi$ be a representation of $A$, and let $S$ be a set of representations.
The following are equivalent, and define what is means for $\pi$ to be
\emph{weakly contained} in $S$:
\begin{enumerate}
\item\label{prop:wkcon:one}
$\bigcap_{\rho\in S} \ker\rho \subseteq \ker\pi$;
\item\label{prop:wkcon:two}
Every positive form on $A$ associated to $\pi$ is the weak$^*$-limit
of linear combinations of positive forms associated with $S$;
\item\label{prop:wkcon:three}
Every state on $A$ associated to $\pi$ is the weak$^*$-limit
of states which are sums of positive forms associated with $S$.
\end{enumerate}
\end{proposition}
\begin{proof}
That (\ref{prop:wkcon:three})$\implies$(\ref{prop:wkcon:two}) is clear.
If (\ref{prop:wkcon:two}) holds then if $\rho(a)=0$ for all $\rho\in S$,
then any positive form associated to $S$ will annihilate $a^*a$.
Thus every positive form associated to $\pi$ annihilates $a^*a$, so
$\pi(a)=0$, showing (\ref{prop:wkcon:one}).

The final implication is harder.
\end{proof}

In the following notice that we need not take sums, unlike in the previous
proposition.

\begin{theorem}
Let $\pi\in\hat A$ and let $S\subseteq\hat A$.  The following are equivalent:
\begin{enumerate}
\item $\pi$ is in the closure of $S$, with respect to the Hull-Kernel
topology;
\item $\pi$ is weakly contained in $S$;
\item at least one non-zero positive form associated with $\pi$ is the
weak$^*$-limit of positive forms associated with $S$;
\item every state associated with $\pi$ is the weak$^*$-limit of states
associated with $S$.
\end{enumerate}
\end{theorem}

Let $S\subseteq\Rep(A,H)$ be the collection of irreducible representations.
Then $\Eq S \subseteq \Eq\Rep(A,H)$ can be identified
with the subset of $\hat A$ consisting of irreducible representations
$\rho:A\rightarrow\mc B(K)$ with $\dim K\leq\dim H$.  Using this, give
$\Eq S$ the subspace topology coming from the Hull-Kernel topology on $\hat A$.
Alternatively, we note that clearly $S = S^u$, and so as above there is a
uniquely defined quotient topology on $\Eq S$ coming from the topology
on $\Rep(A,H)$.

\begin{theorem}
The Hull-Kernel, and quotient, topologies on $S$ coincide.
\end{theorem}


\begin{thebibliography}{99}

\bibitem{dix} J. Dixmier, {\it $C\sp*$-algebras}, translated from the French by Francis Jellett, North-Holland Publishing Co., Amsterdam, 1977. MR0458185

\bibitem{fell1} J. M. G. Fell, $C\sp{\ast} $-algebras with smooth dual, Illinois J. Math. {\bf 4} (1960), 221--230. MR0124754

\end{thebibliography}


\end{document}
