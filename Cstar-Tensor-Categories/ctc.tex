\documentclass[a4paper,11pt]{article}
\usepackage[utf8]{inputenc}
\usepackage[margin=2cm]{geometry}

\usepackage{xcolor}
\definecolor{myblue}{rgb}{0.1 0.1 0.6}
% ``backref'' for drafting; see manual for further options
%\usepackage[backref]{hyperref}
\usepackage{hyperref}
%\usepackage{showkeys}
\hypersetup{
   colorlinks=true,
   linkcolor=myblue,
   citecolor=myblue,
   urlcolor=myblue
}

\usepackage[T1]{fontenc}
% \usepackage[euler-digits, euler-hat-accent]{eulervm}
% \DeclareFontSeriesDefault[rm]{bf}{sbc} 
%\usepackage{beton}
\usepackage{amsmath,amssymb,amsthm}
\usepackage{newpxtext}
\usepackage{eulerpx}

%\usepackage[all]{xy}
\usepackage{url}
\usepackage[shortlabels]{enumitem}

\usepackage{tikz-cd}

\theoremstyle{plain}
\newtheorem{proposition}{Proposition}[section]
\newtheorem{theorem}[proposition]{Theorem}
\newtheorem{corollary}[proposition]{Corollary}
\newtheorem{lemma}[proposition]{Lemma}
\newtheorem{claim}[proposition]{Claim}
\newtheorem{definition}[proposition]{Definition}
\newtheorem{question}[proposition]{Question}

\theoremstyle{remark}
\newtheorem{example}[proposition]{Example}
\newtheorem{remarkx}[proposition]{Remark}
\newtheorem{remarksx}[proposition]{Remarks}
\newtheorem{workingx}[proposition]{Working}
% Some hacks to get a symbol printed at the end of a remark, as it was very unclear (in my
% writing style) where a remark ended and the general flow of the paper (re)started.
\newenvironment{remark}
  {\pushQED{\qed}\renewcommand{\qedsymbol}{$\triangle$}\remarkx}
  {\popQED\endremarkx}
\newenvironment{remarks}
  {\pushQED{\qed}\renewcommand{\qedsymbol}{$\triangle$}\remarksx}
  {\popQED\endremarksx}
\newenvironment{working}
  {\pushQED{\qed}\renewcommand{\qedsymbol}{$\triangle$}\workingx}
  {\popQED\endworkingx}


\newcommand{\mc}[1]{\mathcal{#1}}
\newcommand{\mf}[1]{\mathfrak{#1}}
\newcommand{\msf}[1]{\mathsf{#1}}

\newcommand{\ip}[2]{{\langle {#1} , {#2} \rangle}}
\newcommand{\lin}{\operatorname{lin}}
\newcommand{\id}{\operatorname{id}}

\newcommand{\vnten}{\bar\otimes}

\newcommand{\hh}{\widehat}
\newcommand{\G}{\mathbb{G}}
\renewcommand{\H}{\mathbb{H}}
\newcommand{\op}{{\operatorname{op}}}
\newcommand{\Tr}{{\operatorname{Tr}}}
\newcommand{\im}{{\operatorname{Im}}}
\newcommand{\catVect}{\textsf{Vect}}
\newcommand{\catHilb}{\textsf{Hilb}}



\begin{document}

\title{$C^*$-Tensor Categories}
\author{Matthew Daws}
\date{February 2026}
\maketitle

\begin{abstract}
Some notes about $C^*$-Categories, Tensor Categories, and so forth.

Written somewhat just for myself, as I make a slightly more serious (than in the past) reading around this subject.
\end{abstract}

\section{\texorpdfstring{$C^*$}{CStar}-Categories}

There are lots of nice sources, so I won't make detailed notes here.  The original paper seems to be Ghez, Lima, and Roberts, \cite{GR_Wstar_categories}.  Mitchener's paper \cite{mitchener_cstar_cats} is also very readable.

As categories always have unit morphisms, this means all $C^*$-algebras have units, which is not very natural.  There are no major complications with just looking at ``categories'' where we don't suppose identity morphisms, and functors which don't need to preserve units.  The details are worked out in Ferrier's thesis \cite{ferrier_thesis}.

We follow the axioms in \cite{GR_Wstar_categories}.  A $C^*$-Category is a category $\mc A$ such that:
\begin{enumerate}[(1)]
  \item\label{cstar-cat-axioms:1} each morphism space $\mc A(A,B) = \hom(A,B)$ is a complex vector space where composition of morphisms is bilinear;
  \item\label{cstar-cat-axioms:2} there is an involutive antilinear contravariant endofunctor $*$ of $\mc A$; denote the image of $x\in\mc A(A,B)$ by $x^*\in\mc A(B,A)$;
  \item\label{cstar-cat-axioms:3} for each $x\in\mc A(A,B)$ there is $a\in\mc A(A,A)$ with $x^*x = a^*a$, that is, $x^*x$ is positive; and $x^*x=0\implies x=0$;
  \item\label{cstar-cat-axioms:4} each $\mc A(A,B)$ is a Banach space, and composition of morphisms is contractive: $\|xy\| \leq \|x\| \|y\|$;
  \item\label{cstar-cat-axioms:5} for each $x\in\mc A(A,B)$ we have $\|x^*x\| = \|x\|^2$;
\end{enumerate}

\ref{cstar-cat-axioms:1} means that $\mc A(A,A) = \hom(A)$ is an algebra.
In \ref{cstar-cat-axioms:2}, an ``endofunctor'' just means a functor from $\mc A$ to itself; contravariant means we reverse the directions of arrows, as expected.
So $\hom(A)$ is a $*$-algebra, and by \ref{cstar-cat-axioms:4} and \ref{cstar-cat-axioms:5} is a $C^*$-algebra.  (Note then that the final part of \ref{cstar-cat-axioms:3} is automatic: if $x^*x=0$ then $\|x\|=0$ so $x=0$.)

The axioms corresponding to the $C^*$-condition are more complicated in \cite[Definition 1.2.10]{ferrier_thesis}, but I think this is because Ferrier wishes to consider real-$C^*$-algebras, whereas we work only with the complex numbers; compare \cite[Definition~2.8]{mitchener_cstar_cats}.
Axiom \ref{cstar-cat-axioms:3} appears surprising, but is necessary, see \cite[Example~2.10]{mitchener_cstar_cats}.
The concept of a (possibly non-unital) \emph{algebraoid} is revelant here, \cite{ferrier_thesis,mitchener_cstar_cats}.

A functor between $C^*$-categories is linear and $*$-preserving on morphism spaces.  As for $C^*$-algebras, a functor is automatically contractive, \cite[Proposition~2.14]{mitchener_cstar_cats}.

\begin{itemize}
  \item Hilbert $C^*$-module links
  \item Examples, including $\catHilb$.
  \item Natural transformations: Definition~\ref{defn:natural_transformation}
\end{itemize}




\appendix
\section{Basic category theory}

We follow for example \cite{Leinster_BasicCatTheory}.

\begin{definition}\label{defn:natural_transformation}
A \emph{natural transformation} between two functors $F,G \colon \mc A \to \mc B$, written $\alpha \colon F \Rightarrow G$, is a collection of morphisms $\alpha_A$, for each $A\in\mc A$, such that the following diagrams all commute:
\[ \begin{tikzcd}
  A \arrow[d, "x"] &&  F(A) \arrow[r, "\alpha_A"] \arrow[d, "F(x)"'] & G(A) \arrow[d, "G(x)"] \\
  B &&  F(B) \arrow[r, "\alpha_B"'] & G(B)
\end{tikzcd} \]
\end{definition}

\begin{example}
Let $A$ be an algebra, and consider this as a one-object category.  Then $\hom(A)$ satisfies axiom \ref{cstar-cat-axioms:1}, as $\hom(A)$ is just the algebra $A$.  Let $\catVect$ be the category of vector spaces and linear maps, again this satisfies \ref{cstar-cat-axioms:1}.  A functor $F\colon A \to \catVect$ (assumed linear, of course) is given by a choice of vector space $E = F(A)$ and then an algebra homomorphism $A \to L(E); x \mapsto F(x)$.  This is the same as $E$ begin a left $A$-module.

Consider another functor $G$, say $F = G(A)$.  A natural transformation $\alpha \colon F \Rightarrow G$ is then a morphism $\alpha \colon E \to F$, i.e.\@ a linear map, such that $G(x)\alpha = \alpha F(x)$ for each $x\in A$.  This is the same as saying that $\alpha$ is a module homomorphism.
\end{example}


\bibliographystyle{plain}
\bibliography{ctc.bib}

\end{document}

