\documentclass[a4paper,11pt]{article}
\usepackage[utf8]{inputenc}
\usepackage[margin=2cm]{geometry}

\usepackage{xcolor}
\definecolor{myblue}{rgb}{0.1 0.1 0.6}
% ``backref'' for drafting; see manual for further options
%\usepackage[backref]{hyperref}
%\usepackage{showkeys}
\usepackage{hyperref}
\hypersetup{
   colorlinks=true,
   linkcolor=myblue,
   citecolor=myblue,
   urlcolor=myblue
}

\usepackage[T1]{fontenc}
\usepackage{beton}
\usepackage[euler-digits, euler-hat-accent]{eulervm}
\DeclareFontSeriesDefault[rm]{bf}{sbc} 

\usepackage{amsmath,amssymb,amsthm}
%\usepackage[all]{xy}
\usepackage{url}
\usepackage[shortlabels]{enumitem}
%\usepackage{amscd}
\usepackage{tikz-cd}

\theoremstyle{plain}
\newtheorem{proposition}{Proposition}[section]
\newtheorem{theorem}[proposition]{Theorem}
\newtheorem{corollary}[proposition]{Corollary}
\newtheorem{lemma}[proposition]{Lemma}
\newtheorem{claim}[proposition]{Claim}
\newtheorem{definition}[proposition]{Definition}
\newtheorem{example}[proposition]{Example}
\newtheorem{question}[proposition]{Question}

\theoremstyle{remark}
\newtheorem{remarkx}[proposition]{Remark}
\newtheorem{remarksx}[proposition]{Remarks}
\newtheorem{workingx}[proposition]{Working}
% Some hacks to get a symbol printed at the end of a remark, as it was very unclear (in my
% writing style) where a remark ended and the general flow of the paper (re)started.
\newenvironment{remark}
  {\pushQED{\qed}\renewcommand{\qedsymbol}{$\triangle$}\remarkx}
  {\popQED\endremarkx}
\newenvironment{remarks}
  {\pushQED{\qed}\renewcommand{\qedsymbol}{$\triangle$}\remarksx}
  {\popQED\endremarksx}
\newenvironment{working}
  {\pushQED{\qed}\renewcommand{\qedsymbol}{$\triangle$}\workingx}
  {\popQED\endworkingx}


\newcommand{\mc}[1]{\mathcal{#1}}
\newcommand{\mf}[1]{\mathfrak{#1}}
\newcommand{\msf}[1]{\mathsf{#1}}

\newcommand{\ip}[2]{{\langle {#1} , {#2} \rangle}}
\newcommand{\lin}{\operatorname{lin}}
\newcommand{\id}{\operatorname{id}}

\newcommand{\vnten}{\bar\otimes}

\newcommand{\hh}{\widehat}
\newcommand{\G}{\mathbb{G}}
\renewcommand{\H}{\mathbb{H}}
\newcommand{\op}{{\operatorname{op}}}
\newcommand{\coker}{\operatorname{coker}}

\begin{document}

\title{Interpolation spaces and duality}
\author{Matt Daws}
\date{April 2025}
\maketitle

\begin{abstract}
Some notes on the monograph by Kaijser and Pelletier.
\end{abstract}

We give some notes about the book \cite{KP_InterpolationFunctorsDuality}, written as the author was reading this book.  Part~I of the book makes a quick introduction to doolittle diagrams of Banach spaces, thought of as generalisations of Banach couples, and then shows how the ``real'' ($J$ and $K$ methods) and complex interpolation schemes can be introduced in this generalised setting.

The notes here currently give a detailed account of doolittle diagrams, but little else.  I found the learning curve involved in thinking about doolittle diagrams to be steep, but once overcome, other parts of the book were not so hard to read.



\section{Doolittle Diagrams}

I find the discussion in Section~1 quite hard to follow, so here is a more pedestrian account.

The book works with the category of Banach spaces and bounded linear maps.  Often we might restrict to just contractive maps, giving a subcategory with generally nicer properties: for example, isomorphisms become isometric.  So ``operator'' will mean bounded linear map, unless stated otherwise.

\begin{definition}
A \emph{doolittle diagram} $\overline{X}$ of Banach spaces is a commutative diagram
\[ \begin{tikzcd}
\Delta\overline X \arrow[r, "\delta_0"] \arrow[d, "\delta_1"'] & X_0 \arrow[d, "\sigma_0"] \\
X_1 \arrow[r, "\sigma_1"'] & \Sigma\overline X
\end{tikzcd}
\]
which is both a pullback and a pushout.
\end{definition}

We recall from basic category theory (\cite[Definition~5.1.16]{Leinster_BasicCatTheory} for example) that $\Delta\overline X$ being the \emph{pullback} of $(\sigma_0, \sigma_1)$ means that whenever $Y$ is a Banach space with operators $f_0 \colon Y \to X_0, f_1 \colon Y \to X_1$ with $\sigma_0 f_0 = \sigma_1 f_1$, there is a unique operator $u\colon Y \to \Delta\overline X$ with $\delta_0 u = f_0, \delta_1 u = f_1$.  As a diagram:
\[ \begin{tikzcd}
Y \arrow[rd, dotted, "\exists !\, u"] \arrow[rrd, bend left, "f_0"] \arrow[rdd, bend right, "f_1"']\\
& \Delta\overline X \arrow[r, "\delta_0"] \arrow[d, "\delta_1"'] & X_0 \arrow[d, "\sigma_0"] \\
& X_1 \arrow[r, "\sigma_1"'] & \Sigma\overline X
\end{tikzcd}
\]

Similarly, that $\Sigma\overline X$ is the \emph{pushout} (\cite[Definition~5.2.11]{Leinster_BasicCatTheory} for example) of $(\delta_0, \delta_1)$ means that whenever $Z$ is a Banach space with operators $g_0 \colon X_0 \to Z$ and $f_1 \colon Z_1\to Z$ with $g_0\delta_0 = g_1\delta_1$, there is a unique operator $v\colon \Sigma\overline X \to Z$ with $v\sigma_0 = g_0, v\sigma_1=g_1$.  As a diagram:
\[ \begin{tikzcd}
\Delta\overline X \arrow[r, "\delta_0"] \arrow[d, "\delta_1"'] & X_0 \arrow[d, "\sigma_0"] \arrow[rdd, bend left, "g_0"] \\
X_1 \arrow[r, "\sigma_1"'] \arrow[rrd, bend right, "g_1"'] & \Sigma\overline X \arrow[rd, dotted, "v"] \\
&& Z    
\end{tikzcd} \]

Using the universal properties, it is routine to show that pullbacks and pushouts, if they exist, are unique up to isomorphism.

There are canonical constructions of these objects in the category of Banach spaces.  Firstly, for the pullback, define
\[ D(\sigma_0,\sigma_1) = D\sigma = \{ (\xi_0,\xi_1) : \sigma_0(\xi_0) = \sigma_1(\xi_1) \} \subseteq X_0 \oplus_\infty X_1. \]
As $\sigma_0,\sigma_1$ are continuous, $D\sigma$ is a closed subspace of $X_0 \oplus_\infty X_1$.
Let $\delta_i \colon D\sigma \to X_i$ be the natural projection maps restricted to $D\sigma$, which are contractions satisfying $\sigma_0\delta_0 = \sigma_1\delta_1$.
Given $Y, f_0, f_1$ as above, by definition, $(f_0(\xi), f_1(\xi)) \in D\sigma$ for each $\xi\in Y$ and so we obtain a map $u \colon Y \to D\sigma$.  Notice that $\|u(\xi)\| = \max( \|f_0(\xi)\|, \|f_1(\xi)\| ) \leq \|\xi\| \max(\|f_0\|, \|f_1\|)$ so $u$ is bounded, and contractive if $f_0,f_1$ both are.  Then $\delta_i u = f_i$ and so $u$ satisfies the required property.  Clearly $u$ is unique.  Hence $\Delta\overline X$ is isomorphic (but perhaps not isometric) with $D\sigma$.

For the pushout, set
\[ P(\delta_0,\delta_1) = P\delta = X_0 \oplus_1 X_1 / \overline\lin\{ (\delta_0(\xi), -\delta_1(\xi)) : \xi\in \Delta\overline X \}. \]
Notice that the linear span is superfluous, but the closure is needed, in general.  We shall abuse notation and suppress the quotient when writing elements of $P\delta$.
Let $\sigma_0(\xi_0) = (\xi_0,0)$ for $\xi_0\in X_0$, and simiarly define $\sigma_1(\xi_1) = (0,\xi_1)$ for $\xi_1\in X_1$; both of these operators are contractions.  For $\xi\in \Delta\overline X$ we see that $\sigma_0 \delta_0(\xi) = (\delta_0(\xi), 0) = (0, \delta_1(\xi)) = \sigma_1 \delta_1(\xi)$ by the choice of the subspace to quotient by.
Given $Z, g_0, g_1$ as above, define $v(\xi_0, \xi_1) = g_0(\xi_0) + g_1(\xi_1)$.  This is well-defined as $v(\delta_0(\xi), -\delta_1(\xi)) = g_0\delta_0(\xi) - g_1\delta_1(\xi) = 0$ for each $\xi\in\Delta\overline X$.  As we use the $1$-norm, $\|v\| \leq \max(\|g_0\|, \|g_1\|)$ and so $v$ is contractive if $g_0,g_1$ both are.  Finally, $v \sigma_0 = g_0$ and $v\sigma_1 = g_1$, 
and again $v$ is unique with these properties.  Hence $P\delta$ is the pushout, and so $\Sigma\overline X$ is isomorphic to $P\delta$.

\begin{remark}
We introduce some occasionally useful notation.  Given a subspace $E\subseteq X_0\oplus X_1$ let $E' = \{ (\xi_0, -\xi_1) : (\xi_0,\xi_1) \in E \}$.  (It will be clear from context that this can never be confused with a commutant.)  As the map $(\eta_0,\eta_1) \mapsto (\eta_0, -\eta_1)$ is an isometry, $(E')' = E$, and $E$ is closed if and only if $E'$ is closed.
\end{remark}

Of course, for a doolittle diagram we require compatibility between the constructions, in that we simultaneously have a pushout and a pullback.  The following is useful; we have not found an easily stated analogue starting with $P\delta$.
  
\begin{lemma}\label{lem:when_doolittle_from_pullback}
Let $X_0, X_1, F$ be Banach spaces, and $\sigma_i\colon X_i\to F$ operators, for $i=12$.  The diagram
\[ \begin{tikzcd}
D\sigma \arrow[r, "\delta_0"] \arrow[d, "\delta_1"'] & X_0 \arrow[d, "\sigma_0"] \\
X_1 \arrow[r, "\sigma_1"'] & F
\end{tikzcd} \]
is a doolittle diagram if and only if $\sigma_0(X_0) + \sigma_1(X_1) = F$.
\end{lemma}
\begin{proof}
By construction, $D\sigma = \{ (\xi_0,\xi_1) : \sigma_0(\xi_0) = \sigma_1(\xi_1) \}$ with $\delta_i$ being the projection maps.  As $D\sigma$ is closed, also $(D\sigma)'$ is closed, and so the space we quotient by in forming $P\delta$ is $\{ (\delta_0(\xi), -\delta_1(\xi)) : \xi\in D\sigma \} = \{ (\xi_0, -\xi_1) : (\xi_0,\xi_1) \in D\sigma \} = (D\sigma)'$, no closure required.

Let $F_0 = \sigma_0(X_0) + \sigma_1(X_1) \subseteq F$ a subspace (perhaps not closed) and define $\theta \colon F_0 \to P\delta =  X_0 \oplus_1 X_1 / (D\sigma)'$ by $\theta(\sigma_0(\xi_0) + \sigma_1(\xi_1)) = (\xi_0, \xi_1) + (D\sigma)'$.  This is well-defined, as
\begin{align*}
\sigma_0(\xi_0) + \sigma_1(\xi_1) = \sigma_0(\eta_0) + \sigma_1(\eta_1)
\quad&\Leftrightarrow\quad  \sigma_0(\xi_0-\eta_0) = -\sigma_1(\xi_1-\eta_1)  \\
&\Leftrightarrow\quad  (\xi_0-\eta_0, \xi_1-\eta_1) \in (D\sigma)'   \\
&\Leftrightarrow\quad  (\xi_0-\eta_0, \xi_1-\eta_1) = 0 \text{ in } P\delta   \\
&\Leftrightarrow\quad  (\xi_0, \xi_1) = (\eta_0, \eta_1) \text{ in } P\delta.
\end{align*}
So also $\theta$ is injective, and so by the Open Mapping Theorem, $\theta$ is an isomorphism if and only if $F_0$ is closed (as always $\theta$ is onto).  Let $\sigma'_i \colon X_i \to P\delta$ be the canonical maps, so $\sigma_0'(\xi_0) = (\xi_0,0) = \theta(\sigma_0(\xi_0))$ and similarly $\sigma_1' = \theta\circ\sigma_1$.  The result follows, as the required isomorphism between $F$ and $P\delta$ must be $\theta$.
\end{proof}

\begin{remark}\label{rem:when_equal_in_sigma}
As useful observation of this proof is that in $P\delta$, we have $(\xi_0,0) = (0,\xi_1)$ if and only if $(\xi_0, -\xi_1) \in (D\sigma)'$, if and only if $(\xi_0, \xi_1) \in D\sigma$.
\end{remark}

We now classify doolittle diagrams, up to isomorphism.

\begin{proposition}\label{prop:classification_doolittle}
Let $(X_0, X_1)$ be a pair of Banach spaces.  Given a closed subspace $E$ of $X_0 \oplus_\infty X_1$, setting $\delta_i \colon E \to X_i$ to be the restriction of the natural projection, we have that $\Delta\overline X = E$, and $\Sigma\overline X = P\delta$, gives a doolittle diagram.  Up to isomorphism, all doolittle diagrams arise in this way.
\end{proposition}
\begin{proof}
If $\overline X$ is a doolittle diagram, then $\Delta\overline X$ is isomorphic to $D\sigma$, the pullback of $(\sigma_0, \sigma_1)$.  By construction, $E = D\sigma$ is a closed subspace of $X_0 \oplus_\infty X_1$, and as above, $\Sigma\overline X$ is isomorphic to $P\delta$.

Conversely, given such an $E$, we construct $P\delta$.  To show we have a doolittle diagram, it remains to show that $E, \delta_0, \delta_1$ gives the pullback of $(\sigma_0, \sigma_1)$, these maps coming from the construction of $P\delta$.
%Notice that for any $\xi = (\xi_0, \xi_1) \in E$, we have $(\delta_0(\xi), \delta_1(\xi)) = (\xi_0, \xi_1) = \xi$.  It follows that $E_0 := \{ (\delta_0(\xi), -\delta_1(\xi)) : \xi\in\Delta\overline X\}$ is a closed subspace, as the map $(\eta_0,\eta_1) \mapsto (\eta_0, -\eta_1)$ is an isometry.  By the construction of $P\delta$, we now see that $\sigma_0(\xi_0) = \sigma_1(\xi_1)$ if and only if $(\xi_0, -\xi_1) \in E_0$.  That is, if and only if $(\xi_0,\xi_1) \in E$.  We conclude that $D\sigma = E$ isometrically, and the natural maps giving the pullback are $\delta_0,\delta_1$.  Hence we do obtain a doolittle diagram.
This follows immediately from Lemma~\ref{lem:when_doolittle_from_pullback}.
\end{proof}

Notice that the doolittle diagrams we construct from closed subspaces $E$ have all the map $\delta_i, \sigma_i$ contractive.
If we work in the category of contractive maps, then this proposition classifies doolittle diagrams up to isometric isomorphism.  However, in general, we are free to renorm $D\overline X$ and $\Sigma\overline X$.  I believe that the comment at the bottom of Page~9 of \cite{KP_InterpolationFunctorsDuality} means that we only consider doolittle diagrams coming from this construction, and not renormings.

Henceforth, we shall assume doolittle diagrams arise \emph{isometrically} in this way, so $\Delta\overline X = D\sigma$ isometrically, and $\Sigma\overline X = P\delta$ isometrically.  By Proposition~\ref{prop:classification_doolittle}, we are free to choose $\Delta\overline X$ to be any closed subspace of $X_0\oplus_\infty X_1$, and then $\Sigma\overline X$ is uniquely determined.

Recall from e.g. \cite{BL_Interpolation_Sps_Book} that a \emph{Banach couple} is a pair of Banach spaces $(X_0, X_1)$ which are both realised as subspaces of a Hausdorff topological vector space.  This allows us to form the intersect $X_0 \cap X_1$ and sum $X_0 + X_1$ which then have natural norms
\[ \|\xi\|_{X_0 \cap X_1} = \max\big( \|\xi\|_{X_0}, \|\xi\|_{X_1} \big), \quad
\|\xi\|_{X_0+X_1} = \inf\{ \|\xi_0\|_{X_0} + \|\xi_1\|_{X_1} : \xi = \xi_0+\xi_1 \}. \]
Thus by definition $X_0 \cap X_1 \subseteq X_0 \oplus_\infty X_1$.  Let $\sigma_i \colon X_i \to X_0 + X_1$ be the inclusion maps, and notice that given $(\xi_0,\xi_1) \in X_0 \oplus_\infty X_1$, we have $\sigma_0(\xi_0) = \sigma_1(\xi_1)$ exactly when $\xi_0 = \xi_1$ in $X_0 \cap X_1$.  Hence $X_0 \cap X_1$ is the pullback of $(\sigma_0, \sigma_1)$ and Lemma~\ref{lem:when_doolittle_from_pullback} now shows that
\[ \begin{tikzcd}
X_0 \cap X_1 \arrow[r, "\delta_0"] \arrow[d, "\delta_1"'] & X_0 \arrow[d, "\sigma_0"] \\
X_1 \arrow[r, "\sigma_1"'] & X_0+X_1
\end{tikzcd} \]
is a doolittle diagram.

Notice that here $\sigma_0, \sigma_1$ are injective.  We shall see that this characterises Banach couples, but first we shall investigate the kernel of $\sigma_i$ in the general case.

\begin{lemma}\label{lem:kernels_maps}
For any doolittle diagram, we have that $\delta_0$ is injective on the kernel of $\delta_1$, and $\delta_0(\ker \delta_1) = \ker \sigma_0$.  Similarly $\delta_1(\ker \delta_0) = \ker \sigma_1$.
\end{lemma}
\begin{proof}
As $D\sigma$ is the pullback, we see immediately that
\[ \ker\delta_1 = \{ (\xi_0,\xi_1) \in D\sigma : \xi_1 = 0 \}
= \{ (\xi_0, 0) : \sigma_0(\xi_0) = 0 \}. \]
Obviously, if $(\xi_0,0) \in \ker\delta_1$ then $\delta_0((\xi_0,0)) = \xi_0$ and this is zero only when $(\xi_0,0) = 0$, so $\delta_0$ is injective on $\ker\delta_1$, and $\delta_0(\ker\delta_1) = \{ \xi_0\in X_0 : \sigma_0(\xi_0)=0 \} = \ker\sigma_0$.
\end{proof}

\begin{proposition}\label{prop:when_couple}
A doolittle diagram arises from a Banach couple if and only if $\sigma_0,\sigma_1$ are both injective.
\end{proposition}
\begin{proof}
As observed, Banach couples give rise to doolittle diagrams with $\sigma_0,\sigma_1$ injective.  Conversely, if $\sigma_0,\sigma_1$ are injective, then by the lemma, also $\delta_0,\delta_1$ are injective.  As $\sigma_0,\sigma_1$ are injective, we may use $\Sigma\overline X$ as our ambient space to form $X_0\cap X_1$ and $X_0 + X_1$.  Then $\xi_0 = \xi_1$ exactly when $(\xi_0,0) = (0,\xi_1)$ in $\Sigma\overline X$, that is, $(\xi_0,\xi_1) \in D\sigma$, see Remark~\ref{rem:when_equal_in_sigma}.  Hence $X_0 \cap X_1 = D\sigma$, isometrically.  The norm on $\Sigma\overline X = P\delta$ is then, for $\xi = \xi_0+\xi_1$,
\begin{align*}
\|\xi\| &= \inf\{ \|\eta_0\|_{X_0} + \|\eta_1\|_{X_1} : (\xi_0-\eta_0, \xi_1-\eta_1) \in (D\sigma)' \} \\
&= \inf\{ \|\eta_0\|_{X_0} + \|\eta_1\|_{X_1} : \xi_0-\eta_0 = \eta_1-\xi_1 \} \\
&= \inf\{ \|\eta_0\|_{X_0} + \|\eta_1\|_{X_1} : \xi = \xi_0 + \xi_1 = \eta_0 + \eta_1 \}
= \|\xi\|_{X_0 + X_1}.
\end{align*}
Thus $\Sigma\overline X = X_0+X_1$ isometrically.
\end{proof}

We shall call a doolittle \emph{classical} if it arises from a Banach couple.

\begin{remark}
Notice that $\sigma_0(\xi_0)=0$ if and only if $(\xi_0,0)=0 \in P\delta$ if and only if $(\xi_0,0) \in \Delta\overline X$.  Hence $\sigma_0$ is injective exactly when $(\xi_0,0) \in \Delta\overline X$ implies $\xi_0=0$ (this also follows from Lemma~\ref{lem:kernels_maps}).
As $\Delta\overline X \subseteq X_0\oplus_\infty X_1$ is a subspace, this is equivalent to $(\xi_0,\xi_1), (\xi_0',\xi_1) \in \Delta\overline X$ implies $\xi_0=\xi_0'$.

In this case, we can hence define $\alpha_1 \colon \delta_1(\Delta\overline X) \to X_0; \delta_1((\xi_0,\xi_1)) = \xi_1 \mapsto \xi_0$.  Thus actually $\alpha_1$ maps to $\delta_0(\Delta\overline X)$.  If $\sigma_1$ is injective, we have $\alpha_0 \colon \delta_0(\Delta\overline X) \to \delta_1(\Delta\overline X); \delta_0((\xi_0,\xi_1)) = \xi_0 \mapsto \xi_1$.  When both $\sigma_i$ are injective, the maps $\alpha_0, \alpha_1$ are mutual inverses: indeed, they are just the identity map on $X_0 \cap X_1$.

A standard way to obtain a Banach couple is to start with a contractive injection $\iota \colon X_0 \to X_1$.  Then $X_0 \cap X_1 = X_0$ isometrically, and $X_0 + X_1 = X_1$ isometrically, and the resulting doolittle diagram is
\[ \begin{tikzcd}
\Delta = X_0  \arrow[r, "="] \arrow[d, "\iota"'] & X_0 \arrow[d, "\iota"] \\
X_1 \arrow[r, "="'] & \Sigma = X_1
\end{tikzcd} \]

As a generalisation, consider a closed operator $T \colon X_0 \supseteq D(T) \to X_1$ and let $\Delta$ be the graph of $T$.  Then $\delta_0(\Delta) = D(T)$ and $\delta_0$ is clearly injective, so by Lemma~\ref{lem:kernels_maps}, $\sigma_1$ is injective.  Conversely, if $\sigma_1$ is injective, then we have $\alpha_0 \colon \delta_0(\Delta\overline X) \to \delta_1(\Delta\overline X)$ and its graph is equal to $\Delta\overline X$.
\end{remark}

By definition, in a doolittle diagram, we have that $\sigma_0\circ \delta_0 = \sigma_1 \circ \delta_1$.  Call this map $j \colon \Delta\overline X \to \Sigma\overline X$.  We say that $\overline X$ is \emph{non-trivial} when $j\not=0$.

\begin{proposition}\label{prop:when_j_zero}
We have that $\overline X$ is trivial, that is, $j=0$, if and only if $\Delta = Y_0 \oplus_\infty Y_1$ for closed subspaces $Y_i \subseteq X_i$.
\end{proposition}
\begin{proof}
If $\Delta = Y_0 \oplus_\infty Y_1$ then given any $\xi=(\xi_0,\xi_1)\in \Delta$ we see that also $(\xi_0,0) \in \Delta$ and so $(\xi_0,0) = 0$ in $\Sigma = P\delta$, see Remark~\ref{rem:when_equal_in_sigma}.  So $j(\xi) = \sigma_0 \delta_0(\xi)=0$.

Conversely, if $j=0$ then let $Y_0 = \{ \xi_0\in X_0 : (\xi_0,0)\in \Delta \}$ and similarly define $Y_1$.  Given $\xi=(\xi_0,\xi_1)\in \Delta$, as $0 = j(\xi) = \sigma_0(\xi_0) = (\xi_0,0) \in P\delta$ we have that $(\xi_0,0) \in \Delta$, and similarly $(0,\xi_1)\in\Delta$.  Hence $(\xi_0,\xi_1) \in Y_0 \oplus_\infty Y_1$, as claimed.
\end{proof}

Notice that in this case, $\Sigma = X_0/Y_0 \oplus_1 X_1/Y_1$.

To close this section, we discuss dual spaces.  By Hahn--Banach, we identify $(\Delta\overline X)^*$ with $X_0^* \oplus_1 X_1^* / \Delta^\perp$ and $(\Sigma\overline X)^*$ with $(\Delta')^\perp \subseteq X_0^* \oplus_\infty X_1^*$.  Furthermore, notice that $\sigma_i^* \colon (\Delta')^\perp \to X_i^*$ is the natural projection, and that $\delta_0^*(\xi_0^*) = (\xi_0^*,0) + \Delta^\perp$, and similarly for $\delta_1^*$.

\begin{proposition}\label{prop:dual}
Given a doolittle diagram, also
\[ \begin{tikzcd}
(\Sigma\overline X)^* = (\Delta')^\perp  \arrow[r, "\sigma_0^*"] \arrow[d, "\sigma_1^*"'] & X_0^* \arrow[d, "\delta_0^*"]\\
X_1^* \arrow[r, "\delta_1^*"'] & (\Delta\overline X)^* = X_0^* \oplus_1 X_1^* / \Delta^\perp
\end{tikzcd} \]
\end{proposition}
\begin{proof}
The only thing to check is that $X_0^* \oplus_1 X_1^* / \Delta^\perp = P(\sigma_0^*, \sigma_1^*)$.  This will follow if we show that $\Delta^\perp = ((\Delta')^\perp)'$.  However, $(\xi_0^*, \xi_1^*) \in ((\Delta')^\perp)'$ exactly when $\ip{(\xi_0^*, -\xi_1^*)}{(\xi_0,\xi_1)}=0$ for all $(\xi_0,\xi_1) \in \Delta'$, equivalently, $\ip{(\xi_0^*, -\xi_1^*)}{(\xi_0,-\xi_1)}=0$ for each $(\xi_0,\xi_1)\in \Delta$, that is, $(\xi_0^*, \xi_1^*) \in \Delta^\perp$, as the minus signs cancel.
\end{proof}

We now see the usefulness of doolittle diagrams: these are closed under duality, but by Proposition~\ref{prop:when_couple}, we see that the dual doolittle diagram arises from a Banach couple only when $\delta_0^*$ and $\delta_1^*$ are injective, which is equivalent to $\delta_0,\delta_1$ both having dense range.  Furthermore, even if the dual is a Banach couple, the bidual is only a couple when $\sigma_0^*, \sigma_1^*$ have (norm) dense range, which seems even more restrictive.

With this in mind, we note that the bidual is
\[ \begin{tikzcd}
  ((\Delta')^\perp)')^\perp = \Delta^{\perp\perp} \arrow[r, "\delta_0^{**}"] \arrow[d, "\delta_1^{**}"'] & X_0^{**} \arrow[d, "\sigma_0^{**}"]\\
  X_1^{**} \arrow[r, "\sigma_1^{**}"] & X_0^{**} \oplus_1 X_1^{**} / (\Delta')^{\perp\perp}
\end{tikzcd} \]
The argument in the proof of the lemma shows that $((\Delta')^\perp)' = \Delta^\perp$ from which the identifications arise.  Notice that Hahn--Banach shows that $\Delta^{\perp\perp}$ is isometric to $\Delta^{**}$, and similarly $\Sigma(\overline X^{**}) = X_0^{**} \oplus_1 X_1^{**} / (\Delta')^{\perp\perp}$ is isometric to $(\Sigma\overline X)^{**}$.


\subsection{Morphisms}

\begin{definition}
A morphism of doolittle diagrams $\overline X \to \overline Y$ is a pair $T=(T_0, T_1)$ where $T_i\in\mc B(X_i, Y_i)$ making the following diagram commute:
\[ \begin{tikzcd} & X_0 \arrow[r, "T_0"] & Y_0 \arrow[rd, "\sigma_0"] \\
\Delta\overline X \arrow[ru, "\delta_0"] \arrow[rd, "\delta_1"'] & & &  \Sigma\overline Y \\
& X_1 \arrow[r, "T_1"'] & Y_1 \arrow[ru, "\sigma_1"'] \end{tikzcd} \]
\end{definition}

\begin{lemma}\label{lem:when_morphism}
A pair $T=(T_0,T_1)$ is a morphism if and only if $(\xi_0,\xi_1)\in\Delta\overline X$ implies that $(T_0(\xi_0), T_1(\xi_1)) \in \Delta\overline Y$.
\end{lemma}
\begin{proof}
Let $(\xi_0,\xi_1)\in\Delta\overline X$, so $\sigma_0 T_0 \delta_0(\xi_0,\xi_1) = (T_0(\xi_0), 0)$ and simiarly $\sigma_1 T_1 \delta_1(\xi_0,\xi_1) = (0, T_1(\xi_1))$.  Hence $T$ is a morphism exactly when $(T_0(\xi_0), T_1(\xi_1)) \in \Delta\overline Y$, see Remark~\ref{rem:when_equal_in_sigma}.
\end{proof}

Given a morphism $T$, write $\Delta T \colon \Delta\overline X \to \Delta\overline Y$ for the map given by the lemma.  We also obtain a map $\Sigma T \colon \Sigma\overline X \to \Sigma\overline Y$ as, given $(\xi_0,\xi_1) \in (\Delta\overline X)'$ also $(T_0(\xi_0), T_1(xi_1)) \in (\Delta\overline Y)'$, and hence we obtain a well-defined map $\Sigma T$.  We define the obvious norm,
\[ \|T\| = \max( \|T_0\|, \|T_1\| ). \]
When $\overline X$ arises from a Banach couple, the condition becomes that $T_0(\xi) = T_1(\xi)$ for each $\xi\in X_0 \cap X_1$, and hence we recover the usual notion of a morphism between couples, isometrically.

\begin{proposition}
The morphism space $\overline X \to\overline Y$ is the pullback of the following diagram
\[ \begin{tikzcd}
  & \mc B(X_0, Y_0) \arrow[d, "\alpha_0"] \\
  \mc B(X_1,Y_1) \arrow[r, "\alpha_1"']  & \mc B(\Delta\overline X, \Sigma\overline Y)
\end{tikzcd} \]
where $\alpha_0(T_0)$ is the map $\Delta\overline X \to \Sigma\overline Y; (\xi_0,\xi_1) \mapsto (T_0(\xi_0), 0)$, and similarly for $\alpha_1$.
\end{proposition}
\begin{proof}
As before, the pullback is $\{ (T_0,T_1) : \alpha_0(T_0) = \alpha_1(T_1) \} \subseteq \mc B(X_0,Y_0) \oplus_\infty \mc B(X_1,Y_1)$.  We have that $\alpha_0(T_0) = \alpha_1(T_1)$ if and only if $(T_0(\xi_0),0) = (0,T_1(\xi_1))$ in $\Sigma\overline Y$ for each $(\xi_0,\xi_1)\in\Delta\overline X$, equivalently, $(T_0(\xi_0), T_1(\xi_1))\in\Delta\overline Y$.  So the result follows from Lemma~\ref{lem:when_morphism}.
\end{proof}

From Lemma~\ref{lem:when_morphism} it is clear that if $T\colon \overline X \to \overline Y$ and $S\colon \overline Y\to\overline Z$ then $S\circ T = (S_0\circ T_0, S_1\circ T_1)$ is a morphism $\overline X \to \overline Z$.

By taking adjoints of all the operators involved, compare Proposition~\ref{prop:dual}, it follows immediately that if $(T_0, T_1) \colon \overline X\to\overline Y$ then $(T_0^*, T_1^*) \colon \overline Y^* \to \overline X^*$.

Write $\kappa_X \colon X \to X^{**}$ for the canonical map from a Banach space to its bidual.

\begin{proposition}
The pair $(\kappa_{X_0}, \kappa_{X_1})$ is a morphism $\overline X \to \overline X^{**}$.
\end{proposition}
\begin{proof}
As above, we have $\Delta(\overline X^{**}) = (\Delta\overline X)^{\perp\perp}$.  The result is now immediate.
\end{proof}

Finally, we give another view of morphisms, showing that they can naturally be thought of as morphism of the doolittle diagrams; compare with the discussion in \cite[Section~IV.1]{KP_InterpolationFunctorsDuality}.

\begin{proposition}\label{prop:morphisms_of_diagrams}
A morphism $\overline X \to \overline Y$ may be described by four maps $\Delta T, T_0, T_1, \Sigma T$ making the following diagram commute:
\[ \begin{tikzcd}
\Delta\overline X \arrow[r, "\delta_0"] \arrow[d, "\delta_1"']
\arrow[rrr, bend left, "\Delta T"]
& X_0 \arrow[d, "\sigma_0"] \arrow[rrr, bend left, "T_0"]
& &
\Delta\overline Y \arrow[r, "\delta_0"] \arrow[d, "\delta_1"'] & Y_0 \arrow[d, "\sigma_0"]
\\
X_1 \arrow[r, "\sigma_1"'] \arrow[rrr, bend right, "T_1"']
& \Sigma\overline X \arrow[rrr, bend right, "\Sigma T"']
& &
Y_1 \arrow[r, "\sigma_1"'] & \Sigma\overline Y
\end{tikzcd}
\]
\end{proposition}
\begin{proof}
Given a morphism, we obtain maps $\Delta T$ and $\Sigma T$.  By construction, given $\xi = (\xi_0,\xi_1) \in \Delta\overline X$, we have $T_0 \delta_0(\xi) = T_0(\xi_0) = \delta_0 \Delta T(\xi)$.  Similarly for $\Sigma T$, and by definition of a morphism, the rest of the diagram commutes.

Conversely, given four maps making the diagram commute, we have by definition that $(T_0,T_1)$ forms a morphism.  Then that $T_0 \circ \delta_0 = \Delta_0 \circ \Delta T$ and similarly for $T_1$, we see that $\Delta T$ is uniquely determined by $(T_0,T_1)$.  Similarly for $\Sigma T$.
\end{proof}


\subsection{Constructions}

With reference to Lemma~\ref{lem:kernels_maps}, set $K_i\overline X = \ker(\sigma_i) \subseteq X_i$, so also $K_i\overline X = \delta_i(\ker \delta_{i+1})$ where ``$i+1$'' is interpretted modulo $2$.  The following is immediate.

\begin{lemma}\label{lem:defn_K}
Let $\Delta (K\overline X) = K_0\overline X \oplus_\infty K_1\overline X \subseteq X_0\oplus_\infty X_1$.  The following is a doolittle diagram, say $K\overline X$,
\[ \begin{tikzcd} \Delta(K\overline X) \arrow[r, "\delta_0"] \arrow[d, "\delta_1"'] & K_0\overline X \arrow[d] \\
K_1\overline X \arrow[r] & 0
\end{tikzcd} \]
where $\delta_i$ is the projection map, as usual, and the other maps are zero.
\end{lemma}

For the following, compare Proposition~\ref{prop:when_j_zero}.

\begin{proposition}\label{prop:Ki_and_j}
For any doolittle diagram, we have that
\[ K_0\overline X = \{ \xi_0\in X_0 : (\xi_0,0) \in \Delta\overline X \}, \quad
K_1\overline X = \{ \xi_1\in X_1 : (0,\xi_1) \in \Delta\overline X \} \]
and that $\ker j = \Delta(K\overline X) \subseteq \Delta\overline X$.
\end{proposition}
\begin{proof}
Again using Remark~\ref{rem:when_equal_in_sigma}, we see that $\sigma_0(\xi_0)=0$ if and only if $(\xi_0,0)=0$ in $P\delta$, if and only if $(\xi_0,0) \in \Delta\overline X$.  Similarly for $K_1\overline X$.

As $j(\xi_0,\xi_1) = \sigma_0\delta_0(\xi_0,\xi_1) = (\xi_0,0)$ we see that $j(\xi_0,\xi_1) = 0$ exactly when $(\xi_0,0) \in \Delta$.  Using that also $j=\sigma_1\delta_1$, we see that
\[ \ker j = \{ (\xi_0,\xi_1)\in \Delta : (\xi_0,0), (0,\xi_1) \in \Delta \}
= K_0\overline X \oplus_\infty K_1\overline X, \]
as claimed.
\end{proof}

The following shows how to convert a doolittle diagram into a Banach couple.

\begin{proposition}\label{prop:quotient_by_kernel}
For any doolittle diagram, the following is also a doolittle diagram $\overline X / K\overline X$
\[ \begin{tikzcd}
\Delta\overline X / \Delta(K\overline X) \arrow[r, "\hat\delta_0"] \arrow[d, "\hat\delta_1"'] & X_0 / K_0\overline X\arrow[d, "\hat\sigma_0"] \\
X_1 / K_1\overline X \arrow[r, "\hat\sigma_1"'] & \Sigma\overline X
\end{tikzcd}
\]
Here $\hat\delta_i$ is the quotient operator $\xi+K\overline X \mapsto \delta_i(\xi) + K_i\overline X$, and $\hat\sigma_i$ is the quotient operator $\xi_i + K_i\overline X \mapsto \sigma_i(\xi_i)$.  Further, the maps $\hat\sigma_i$ are injective, and so this diagram comes from a Banach couple.
\end{proposition}
\begin{proof}
As $\Delta(K\overline X) \subseteq \Delta\overline X$, the quotient exists, and as $\delta_i(\Delta(K\overline X)) \subseteq K_i\overline X$, the maps $\hat\delta_i$ are well-defined.  As $K_i\overline X = \ker\sigma_i$, similarly the maps $\hat\sigma_i$ are well-defined.  Clearly $\hat\sigma_i$ is injective.

Set $Y_i = X_i / K_i\overline X$ and $\Delta\overline Y = \{ (\xi_0', \xi_1') \in Y_0 \oplus_\infty Y_1 : \hat\sigma_0(\xi_0') = \hat\sigma_1(\xi_1') \}$.  As $\hat\sigma_0(Y_0) + \hat\sigma_1(Y_1) = \sigma_0(X_0) + \sigma_1(X_1) = \Sigma\overline X$, by Lemma~\ref{lem:when_doolittle_from_pullback}, the following is a doolittle diagram:
\[ \begin{tikzcd}
\Delta\overline Y \arrow[r, "\delta_0"] \arrow[d, "\delta_1"'] & Y_0 \arrow[d, "\hat\sigma_0"] \\
Y_1 \arrow[r, "\hat\sigma_1"'] & \Sigma\overline X
\end{tikzcd} \]
We see that for $(\xi_0,\xi_1)\in \Delta\overline X$ we have that $(\xi_0+K_0\overline X, \xi_1+K_1\overline X) \in \Delta\overline Y$, and the kernel of the resulting map $\Delta\overline X \to \Delta\overline Y$ is $K_0\overline X \oplus K_1\overline X = K\overline X$, while clearly this map is onto.  So $\Delta\overline X / \Delta(K\overline X) = \Delta\overline Y$ as Banach spaces.
Furthermore, in $\Delta\overline X / \Delta(K\overline X)$,
\begin{align*}
\| (\xi_0,\xi_1) + \Delta(K\overline X) \|
&= \inf\{ \max(\|\eta_0\|, \|\eta_1\|) : (\xi_0 - \eta_0, \xi_1-\eta_1) \in \Delta(K\overline X) \} \\
&= \inf\{ \max(\|\eta_0\|, \|\eta_1\|) : \xi_i - \eta_i \in K_i\overline X \} \\
&= \max( \|\xi_0+K_0\overline X\|, \|\xi_1+K_1\overline X\| ),
\end{align*}
as when computing the infimum we can let $\eta_0$ and $\eta_1$ vary independently.  Hence $\Delta\overline X / \Delta(K\overline X) = \Delta\overline Y$ isometrically, which completes the proof.
\end{proof}

By Proposition~\ref{prop:classification_doolittle}, a doolittle diagram only really depends upon $\Delta\overline X$, and there is some choice in $X_0,X_1$.  Indeed, we can always embed $X_i$ into larger spaces, leave $\Delta\overline X$ unchanged, and form a (larger) pullback to complete the diagram.  There is a construction to, in some sense, get around this problem.

Given $\Delta\overline X$ we set $X_i^\circ$ to be the closure of the image of $\delta_i$.  Then by construction, $\Delta\overline X$ is a closed subspace of $X_0^\circ \oplus_\infty X_1^\circ$, let $\Delta\overline X^\circ$ be this subspace, and so we obtain the doolittle diagram $\overline X^\circ$, namely
\[ \begin{tikzcd}
\Delta\overline X^\circ \arrow[r, "\delta_0"] \arrow[d, "\delta_1"'] & X_0^\circ \arrow[d, "\sigma_0"] \\
X_1^\circ \arrow[r, "\sigma_1"'] & \Sigma\overline X^\circ = X_0^\circ \oplus_1 X_1^\circ / (\Delta\overline X)'
\end{tikzcd} \]
The inclusion $X_0^\circ \oplus_1 X_1^\circ \to X_0\oplus_1 X_1$ drops to a map $\Sigma\overline X^\circ \to \Sigma\overline X$ which is isometric (as the space we quotient by is ``the same'').

[\footnote{Some stuff in the book about ``short exact sequences'' and quotients which we'll treat below in the category section.}]

By Lemma~\ref{lem:kernels_maps}, $\sigma_0,\sigma_1$ are injective if and only if $\delta_0,\delta_1$ are injective, so Proposition~\ref{prop:when_couple} shows that $\overline X^\circ$ comes from a Banach couple if and only if $\overline X$ does.  Recall that a Banach couple is \emph{regular} if $X_0\cap X_1$ is dense in both $X_0$ and $X_1$.  Thus the construction of $\overline X^\circ$ can be thought of as ensuring a regularity-like condition.

\begin{proposition}
We have that $K\overline X = K(\overline X^\circ)$, and that $(\overline X / K\overline X)^\circ$ is equal to $\overline X^\circ / (K\overline X)$.
\end{proposition}
\begin{proof}
Proposition~\ref{prop:Ki_and_j} gives that $K_0\overline X = \{ \xi_0\in X_0 : (\xi_0,0) \in \Delta\overline X \}$ and so $K_0\overline X \subseteq X_0^\circ$, and similarly for $K_1$.  Hence $K\overline X = K(\overline X^\circ)$.

Recall the doolittle diagram for $\overline X / K\overline X$ from Proposition~\ref{prop:quotient_by_kernel}.  As $\hat\delta_i$ is the quotient of $\delta_i$, we see that $(X_i / K_i\overline X)^\circ$ is the closure of $X_i^\circ / K_i\overline X$, but this is already closed as $K_i\overline X = K_i\overline X^\circ$.  It follows that $(\overline X / K\overline X)^\circ = \overline X^\circ / (K\overline X)$.
\end{proof}


\subsection{The resulting category}

Write $\msf{Ban}$ for the category of Banach spaces, $\msf{BanCp}$ for the category of Banach couples, and $\msf{BanDL}$ for the category of doolittle diagrams.  As discussed above, $\msf{BanCp}$ is a subcategory of $\msf{BanDL}$, in fact, it is a full subcategory.

We continue to be a little vague about whether we consider all (bounded) morphisms, or just contractive morphisms.  We will make comments about when the difference is important.  The following was written by the author, but much more detail can be found in \cite[Section~IV.3]{KP_InterpolationFunctorsDuality}.

We consider $\msf{BanDL}$ and the subcategory $\msf{BanCp}$.
The zero object in both these categories is the diagram with $X_0 = X_1 = \{0\}$ and the zero maps.  The zero morphisms are the pairs $(0,0)$.  The following is essentially \cite[Proposition~IV.3.2]{KP_InterpolationFunctorsDuality}.

\begin{proposition}
Let $T=(T_0,T_1)$ be a morphism $\overline X \to \overline Y$.  We define $\ker T$ by setting $\Delta(\ker T) = \{ (\xi_0,\xi_1)\in\Delta\overline X : T_0(\xi_0)=0, T_1(\xi_1)=0 \}$.
Treat $\Delta(\ker T)$ as a subspace of $X_0\oplus_\infty X_1$, so that we obtain doolittle diagram $\ker T$.  With $\iota$ given by the formal identity, we obtain a morphism $\ker T \to \overline X$ which is the categorical kernel of $T$.
\end{proposition}
\begin{proof}
Let $\iota_i \colon X_i\to X_i$ be the identity, so $\iota$ is a morphism as $\Delta(\ker T) \subseteq \Delta\overline X$ by definition.
Suppose we have a diagram
\[ \begin{tikzcd}
  \overline X \arrow[rd, "T"] \\
  \overline Z \arrow[u, "S"] \arrow[r, "0"'] & \overline Y
\end{tikzcd} \]
We show the existence of a unique morphism $\phi\circ \overline Z \to \ker T$ with $S = \iota\circ \phi$.  As $T_i\circ S_i = 0$, given $(\eta_0,\eta_1) \in \Delta\overline Z$ we have that $(S_0(\eta_0), S_1(\eta_1)) \in \Delta\overline X$ as $S$ is a morphism, and $(T_0S_0(\eta_0), T_1S_1(\eta_1))= (0,0)$ so $(S_0(\eta_0), S_1(\eta_1)) \in \Delta(\ker T)$.  Thus we can consider $\phi_i$ as $S_i$, giving a morphism $\phi\colon\overline Z \to \ker T$ with $\iota\circ\phi=S$.  As $\iota$ is the formal identity, clearly $\phi$ is uniquely defined.
\end{proof}

As kernels are unique (up to isomorphism) we don't have any choice in this construction: in particular, notice that we only change the $\Delta$ space to form the kernel.  We can similarly construct cokernels; once one unpacks the construction, this agrees with \cite[Proposition~IV.3.2]{KP_InterpolationFunctorsDuality}.

\begin{proposition}
Let $T=(T_0,T_1)$ be a morphism $\overline X \to \overline Y$.  We define $\coker T$ to be the doolittle diagram with pair of spaces $Q_i = Y_i / \overline{T_i(X_i)}$ with natural quotient maps $q_i \colon Y_i \to Q_i$, and $\Delta(\coker T)$ to be the closure of the image of $\Delta Y$, namely the closure of $\{ (q_0(\xi_0), q_1(\xi_1)) : (\xi_0,\xi_1)\in\Delta Y \}$.  Then $q$ gives a morphism $\overline Y \to \coker T$, and this gives the categorical cokernel.
\end{proposition}
\begin{proof}
By construction, $q$ is a morphism, and $q \circ T = 0$.  Let $\overline Q'$ be a doolittle diagram with morphism $q' \colon Y \to \overline Q'$ with $q' \circ T = 0$.  We show the existence of a unique morphism $\phi \colon \coker T \to \overline Q'$ with $\phi\circ q = q'$.

Define $\phi_i \colon q_i(\xi_i) = q_i'(\xi_i)$, which is well-defined, as if $q_i(\xi_i)=0$ then $\xi_i\in \overline {T_i(X_i)}$ and so $q_i'(\xi_i)=0$.  Given $(\xi_0,\xi_1) \in \Delta\overline Y$, we have that $(q_0(\xi_0), q_1(\xi_1)) \in \Delta(\coker T)$ and $(\phi_0q_0(\xi_0), \phi_1q_1(\xi_1)) = (q'_0(\xi_0), q'_1(\xi_1)) \in \Delta\overline Q'$.  By density of such elements in $\Delta(\coker T)$, it follows that $\phi$ is a morphism, with $\phi\circ q = q'$ by construction.  As $q_i$ is onto, $\phi_i$ is uniquely defined, and so $\phi$ is unique.
\end{proof}

This construction suggests a general notion of a ``quotient''.  First we need the notion of a ``subspace''.

\begin{definition}
Let $\overline X$ be a doolittle diagram.  A \emph{subspace} of $\overline X$ is $\overline Y$ where $Y_i\subseteq X_i$ are closed subspaces, and $\Delta Y \subseteq \Delta X$.

The \emph{quotient} is $\overline Q = \overline X / \overline Y$ where $Q_i = X_i / Y_i$ and $\Delta\overline Q$ is the closure of the image of $\Delta\overline X$ in $Q_0 \oplus_\infty Q_1$.  Let $q_i \colon X_i \to Q_i$ be the quotient map.
\end{definition}

\begin{proposition}
The subspace and quotient form doolittle diagrams, with the inclusion $\overline Y\to\overline X$ a morphism, and $q=(q_0,q_1) \colon \overline X \to \overline X/\overline Y$ a morphism.
\end{proposition}
\begin{proof}
The subspace is by definition a doolittle diagram, and obviously the inclusion gives a morphism, as $\Delta Y \subseteq \Delta X$ by definition.  Again, the quotient is doolittle by definition, and $\Delta Q$ is defined so that $q$ forms a morphism.
\end{proof}

The quotient construction has the following universal property.

\begin{proposition}\label{prop:factor_through_quotient}
Let $\overline Y$ be a subspace of $\overline X$.  The quotient $Q = \overline X / \overline Y$ has the property that a morphism $T\colon \overline X \to \overline Z$ factors through $q\colon \overline X \to \overline Q$ if and only if $Y_i \subseteq \ker(T_i)$ for $i=0,1$.
\end{proposition}
\begin{proof}
If we have $T$ and $S\colon\overline Q\to\overline Z$ with $S\circ q = T$, then for $\xi_i\in Y_i$ we have that $T(\xi_i) = S_i q_i(\xi_i) = 0$, so $Y_i \subseteq \ker(T_i)$.  Conversely, if this holds, then define $S_i \colon X_i / Y_i \to Z_i$ by $S_i(\xi_i + Y_i) = T_i(\xi_i)$, which is well-defined by the assumption that $T_i(Y_i) = \{0\}$.  Then $\|S_i\| \leq \|T_i\|$, and for $(\xi_0,\xi_1) \in \Delta\overline X$ we see that $(\xi_0+Y_0, \xi_1+Y_1) \in \Delta\overline Q$ and $(S_0(\xi_0+Y_0), S_1(\xi_1+Y_1)) = (T_0(\xi_0), T_1(\xi_1)) \in \Delta\overline Z$.  By the density of such elements in $\Delta\overline Q$, we conclude that $S$ is a morphism.  By construction, $S\circ q = T$.
\end{proof}

\begin{remark}
It seems strange that the quotient does not depend upon $\Delta\overline Y$.  However, firstly, the factorisation property given by Proposition~\ref{prop:factor_through_quotient} only holds because $\Delta Q$ is minimal, in the sense that the image of $\Delta X$ is dense.  Indeed, let $\overline Q'$ be a doolittle diagram with $Q'_i = X_i / Y_i$ but $\Delta\overline Q'$ arbitrary with $\Delta \overline Q' \supseteq \overline Q$.  The quotient map $q'$ is still a morphism.  However, $q \colon \overline X \to \overline Q$ does not factor through $q'$ unless $\Delta\overline Q' \subseteq \Delta\overline Q$, that is, actually $\overline Q' = \overline Q$.

Of course, perhaps requiring some extra property from $T$ would give an alternative formulation of this factorisation property, using a different notion of quotient.
\end{remark}

\begin{remark}
Continuing, consider how we might somehow use $\Delta\overline Y$ in defining the quotient.  It is perhaps natural to try to use $\Delta\overline X / \Delta\overline Y$ as the pullback.  This would mean a doolittle diagram like
\[ \begin{tikzcd}
\Delta\overline X / \Delta\overline Y \arrow[r, "\delta_0"] \arrow[d, "\delta_1"'] & X_0/Y_0 \arrow[d, "\sigma_0"] \\
X_1/Y_1 \arrow[r, "\sigma_1"'] & P\delta
\end{tikzcd}
\qquad\text{where}\quad
P\delta = (X_0/Y_0 \oplus_1 X_1/Y_1) / \{ (\delta_0\xi,-\delta_1\xi) : \xi\in \Delta\overline X / \Delta\overline Y \}^{\overline{\phantom{\{\ }}^{\|\cdot\|}}. \]
Of course, this means that $\Delta\overline X / \Delta\overline Y \cong D\sigma$ where
\begin{align*}
D\sigma &= \{ (\xi_0+Y_0, \xi_1+Y_1) : \sigma_0(\xi_0+Y_0) = \sigma_1(\xi_1+Y_1) \}  \\
&= \{ (\xi_0+Y_0, -\xi_1+Y_1) : (\xi_0+Y_0, -\xi_1+Y_1) = 0 \in P\delta \} \\
&= \{ (\delta_0\xi,\delta_1\xi) : \xi\in \Delta\overline X / \Delta\overline Y \}^{\overline{\phantom{\{\ }}^{\|\cdot\|}}.
\end{align*}
We need to decide on the maps $\delta_i$, for which the natural choice seems to be $\delta_0 \colon \overline X / \overline Y \to X_0/Y_0; (\xi_0,\xi_1)+\overline Y \mapsto \xi_0+Y_0$, and similarly for $\delta_1$.  These are well-defined, for if $(\xi_0,\xi_1)\in\overline Y$ then $\xi_0\in Y_0$ and $\xi_1\in Y_1$.  For these choices,
\[ D\sigma = \{ (\xi_0+Y_0, \xi_1+Y_1) : (\xi_0,\xi_1)\in\Delta\overline X \}^{\overline{\phantom{\{\ }}^{\|\cdot\|}}, \]
and we're back to $\Delta\overline Q$.

Thus an arbitrary $\Delta\overline Y$ doesn't seem to work.  For which $\Delta\overline Y$ do we have that $\Delta\overline X / \Delta\overline Y \cong \Delta\overline Q$?  The natural map to use is $\Delta\overline X / \Delta\overline Y \ni (\xi_0,\xi_1) + \Delta\overline Y \mapsto (\xi_0+Y_0, \xi_1+Y_0) \in \Delta\overline Q$, which has dense range.  It is injective if and only if $(\xi_0,\xi_1)\in\Delta\overline X \cap Y_0\oplus Y_1 \subseteq \Delta\overline Y$, but as $\Delta\overline Y \subseteq \Delta\overline X$ and $Y_0\oplus Y_1$, we must have equality:
\[ \Delta\overline Y = \Delta\overline X \cap Y_0\oplus Y_1. \]
We also require the map to be bounded below (ideally, to be an isometry).  For $(\xi_0,\xi_1) \in \Delta\overline X$, the two norms are
\begin{gather*}
\|(\xi_0,\xi_1)\|_{\Delta\overline X / \Delta\overline Y}
= \inf\big\{ \max(\|\xi_0-\eta_0\|, \|\xi_1-\eta_1\|) : (\eta_0,\eta_1) \in \Delta\overline X \cap Y_0\oplus Y_1 \big\}, \\
\|(\xi_0,\xi_1)\|_{\Delta\overline Q}
= \inf\big\{ \max(\|\xi_0-\eta_0\|, \|\xi_1-\eta_1\|) : (\eta_0,\eta_1) \in Y_0\oplus Y_1 \big\}.
\end{gather*}
It seems hard to give a characterisation of when this happens.

However, in the special case that $\Delta\overline X \cap Y_0\oplus Y_1 = Y_0 \oplus Y_1$, we obviously do have equality.  This occurs for $K\overline X$, see Lemma~\ref{lem:defn_K}.
\end{remark}

A related issue occurs with our notion of a subspace: do we also have that $\Sigma Y \to \Sigma X$ is an ``inclusion'', which we might take to mean, ``is isometric''?  This is again rare, because while we always have a contractive map (compare Proposition~\ref{prop:morphisms_of_diagrams}), for $(\xi_0,\xi_1) \in Y_0\oplus Y_1$ we have
\begin{gather*}
\|(\xi_0,\xi_1)\|_{\Sigma\overline Y}
= \inf\big\{\max(\|\xi_0-\eta_0\|, \|\xi_1-\eta_1\|) : (\eta_0,\eta_1)\in \Delta\overline Y\big\}, \\
\|(\xi_0,\xi_1)\|_{\Sigma\overline X}
= \inf\big\{\max(\|\xi_0-\eta_0\|, \|\xi_1-\eta_1\|) : (\eta_0,\eta_1)\in \Delta\overline X\big\}.
\end{gather*}
A case when we do have equality is in the construction of $\overline X^\circ$, for the boring reason that there $\Delta\overline X^\circ = \Delta\overline X$.

In \cite[Section~I.2]{KP_InterpolationFunctorsDuality}, the notion of a \emph{short exact sequence} of couples is defined, by requiring that all the four maps (Proposition~\ref{prop:morphisms_of_diagrams}) are exact at each point of the diagram.  While the constructions $\overline X\mapsto K\overline X$ and $\overline X \mapsto \overline X^\circ$ give examples, it seems hard to think of other cases, given the discussion just made.


\subsection{Interpolation}

Classically, given a Banach couple $(X_0, X_1)$, an interpolation space is simply a Banach space $X$ with $X_0\cap X_1 \subseteq X \subseteq X_0+X_1$ (continuous inclusions) such that for every morphism $T$, there is a bounded operator $X\to X$ extending the map on $X_0\cap X_1$.

For a doolittle diagram, the map $j\colon \Delta\overline X \to \Sigma\overline X$ need not be injective, and so the notion of interpolation is more subtle.  For doolittle diagrams $\overline X, \overline Y$ write $\mc B(\overline X,\overline Y)$ for the morphism space, and write $\mc B(\overline X)$ for $\mc B(\overline X,\overline X)$.

The most basis idea is simply that of a $\mc B(\overline X)$-module, equivalently, a Banach space $X$ with a homomorphism $\mc B(\overline X) \to \mc B(X)$.  As always, we have a choice as to whether this homomorphism is assumed contractive, or just bounded.  We can always re-norm $X$ to be contractive, by setting
\[ \| x \|_0 = \sup\{ \|T\cdot x \| : T\in\mc B(\overline X), \|T\|\leq 1 \}. \]
As $1\in\mc B(\overline X)$ acts as the identity on $X$, this norm is equivalent to the given norm.  We see that for $\|T\|\leq $, for $S\in\mc B(\overline X)$ with $\|S\|\leq 1$ also $\|ST\|\leq 1$ and so $\|T\cdot x\|_0 = \sup\{ \|ST\cdot x\| : \|S\|\leq 1 \} \leq \sup\{ \|R\cdot x\| : \|R\|\leq 1 \} = \|x\|_0$, as required to show that the module action for $\|\cdot\|_0$ is contractive.

\begin{definition}\label{defn:int_space}
Let $\overline X$ be a doolittle diagram.  We call an $\mc B(\overline X)$-module a \emph{quasi-interpolation space} for $\overline X$ if additionally there are module maps $\delta \colon \Delta\overline X \to X$ and $\sigma\colon X\to\Sigma\overline X$ with $\sigma\circ\delta = j$.  When $\delta$ has dense range, we call $X$ a \emph{$\Delta$-interpolation space}.  When $\sigma$ is injective, we call $X$ a \emph{$\Sigma$-interpolation space}.
\end{definition}

Notice that $X = \Delta\overline X$ gives a $\Delta$-interpolation space (which is only a $\Sigma$-interpolation space when $j$ is injective).  Similarly $X = \Sigma\Delta$ give a $\Sigma$-interpolation space (which is only a $\Delta$-interpolation space when $j$ is injective).

If $X$ is a $\Delta$-interpolation space, then $\delta$ being a dense-range module maps shows that the $\mc B(\overline X)$-module action on $X$ is uniquely defined.  Similarly for a $\Sigma$-interpolation space.

\cite{KP_InterpolationFunctorsDuality} observes that most $\Delta$-interpolation spaces are also $\Sigma$-interpolation spaces, but \cite[Example~I.3.2]{KP_InterpolationFunctorsDuality} shows that this isn't always so (an example different from $X = \Delta\overline X$).

\bibliographystyle{plain}
\bibliography{thebib.bib}

\end{document}
