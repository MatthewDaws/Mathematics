\documentclass[a4paper,12pt]{article}

\usepackage[margin=2cm]{geometry}
%\usepackage{xypic}
%\usepackage[left,inline]{showlabels}
\usepackage{amsmath,amsthm,amssymb}
\usepackage{url}
%\usepackage{latexsym}

\theoremstyle{plain}
\newtheorem{proposition}{Proposition}[section]
\newtheorem{theorem}[proposition]{Theorem}
\newtheorem{corollary}[proposition]{Corollary}
\newtheorem{lemma}[proposition]{Lemma}
\theoremstyle{definition}
\newtheorem{definition}[proposition]{Definition}
\newtheorem{example}[proposition]{Example}
\newtheorem{remark}[proposition]{Remark}

\newcommand{\mc}{\mathcal}
\newcommand{\mf}{\mathfrak}
\newcommand{\ip}[2]{\langle{#1},{#2}\rangle}
\newcommand{\G}{\mathbb G}
\newcommand{\vnten}{\overline\otimes}
\newcommand{\proten}{\widehat\otimes}
\newcommand{\aone}{\Box}
\newcommand{\atwo}{\Diamond}

\begin{document}

\title{Elementary Kaplansky Density}
\author{Matthew Daws}
\maketitle

\section{Introduction}

We give an ``elementary'' proof of the Kaplansky Density theorem, using Arens
products and standard basic results about C$^*$-algebras.\footnote{Of course,
these could well be considered non-trivial!}  The form of the Kaplansky Density
Theorem we are aiming for is the following:

\begin{theorem}
Let $H$ be a Hilbert space, and endow $\mc B(H)$ with the weak$^*$-topology coming
from regarding $\mc B(H)$ as the dual space of the trace-class operators $\mc B_*(H)$.
Let $M\subseteq\mc B(H)$ be a von Neumann algebra, by which we mean a self-adjoint,
unital, weak$^*$-closed subalgebra of $\mc B(H)$, and let $A\subseteq M$ be a
C$^*$-algebra which generates $M$, by which we mean that the weak$^*$-closure of $A$
is all of $M$.

Then the unit ball of $A$ is weak$^*$-dense in the unit ball of $M$.
\end{theorem}

We state this for the weak$^*$-topology (or ``$\sigma$-weak'' topology), but, for 
example, if $K$ denotes the $\sigma$-strong$^*$-closure of the unit ball of $A$,
then $K$ is weak$^*$-closed, by \cite[Chapter~II, Theorem~2.6]{tak1}, and so also
equals the closed unit ball of $M$.

In the next section we give a careful proof, without much motivation.  We then make
some comments on the proof in the following section, and in the final section make
some closing remarks, and very brief bibliographic comments.



\section{The proof}

We work with the Arens products $\aone$ and $\atwo$.  Let $E$ be any \emph{reflexive}
Banach space, and consider the projective tensor product $E\proten E^*$.  The dual
space can be identified with $\mc B(E,E^{**}) = \mc B(E)$ in the usual way:
\[ \ip{T}{x\otimes\mu} = \ip{\mu}{T(x)} \qquad (T\in\mc B(E), x\in E, \mu\in E^*). \]

Now let $A$ be a Banach algebra, and let $\pi:A\rightarrow\mc B(E)$ be a contractive
homomorphism.  Consider $\kappa = \kappa_{E\proten E^*} : E\proten E^* \rightarrow
(E\proten E^*)^{**} = \mc B(E)^*$ the canonical map from a Banach space to its bidual.
Let $\alpha = \pi^* \circ \kappa : E\proten E^* \rightarrow A^*$, and hence
$\alpha^* = \kappa^* \circ \pi^{**} : A^{**} \rightarrow\mc B(E)$.  To be explicit,
\[ \ip{\mu}{\alpha^*(\Phi)(x)} = \ip{\Phi}{\alpha(x\otimes\mu)}
\qquad (\Phi\in A^{**}, x\in E, \mu\in E^*), \]
and
\[ \ip{\alpha(x\otimes\mu)}{a} = \ip{\mu}{\pi(a)(x)}
\qquad (a\in A, x\in E, \mu\in E^*). \]

\begin{proposition}\label{prop:two}
For either Arens product, $\alpha^*:A^{**} \rightarrow\mc B(E)$ is a homomorphism
which extends $\pi$, in the sense that $\pi = \alpha^* \circ \kappa_A$.
\end{proposition}
\begin{proof}
This is nothing but a long calculation.  Firstly, consider the module action of
$A$ on the image of $\alpha$ in $A^*$.  For $a,b\in A, x\in E, \mu\in E^*$ we have
\begin{align*}
\ip{a\cdot\alpha(x\otimes\mu)}{b} = \ip{\pi(ba)}{x\otimes\mu}
= \ip{\mu}{\pi(b)\pi(a)(x)} = \ip{\alpha(\pi(a)\otimes\mu)}{b}, \\
\ip{\alpha(x\otimes\mu)\cdot a}{b} = \ip{\pi(ab)}{x\otimes\mu}
= \ip{\mu}{\pi(a)\pi(b)(x)} = \ip{\alpha(\otimes\pi(a)^*(\mu))}{b}.
\end{align*}
Now compute the action of $A^{**}$ on the image of $\alpha$ in $A^*$.
For $\Phi\in A^{**}, a\in A,x\in E,\mu\in E^*$, we have
\begin{align*}
\ip{\Phi\cdot\alpha(x\otimes\mu)}{a} &= \ip{\Phi}{\alpha(x\otimes\pi(a)^*(\mu))}
= \ip{\pi(a)^*(\mu)}{\alpha^*(\Phi)(x)} \\
&= \ip{\mu}{\pi(a) \alpha^*(\Phi)(x)}
= \ip{\alpha(\alpha^*(\Phi)(x)\otimes\mu)}{a}, \\
\ip{\alpha(x\otimes\mu)\cdot\Phi}{a} &= \ip{\Phi}{\alpha(\pi(a)(x)\otimes\mu)}
= \ip{\mu}{\alpha^*(\Phi) \pi(a)(x)} \\
&= \ip{\alpha^*(\Phi)^*(\mu)}{\pi(a)(x)}
= \ip{\alpha(x\otimes\alpha^*(\Phi)^*(\mu))}{a}.
\end{align*}
Thus, for $\Phi,\Psi\in A^{**},x\in E,\mu\in E^*$ we have
\begin{align*}
\ip{\mu}{\alpha^*(\Phi\aone\Psi)(x)} &=
  \ip{\Phi\aone\Psi}{\alpha(x\otimes\mu)} =
  \ip{\Phi}{\Psi\cdot\alpha(x\otimes\mu)} \\
  &= \ip{\Phi}{\alpha(\alpha^*(\Psi)(x)\otimes\mu)}
  = \ip{\mu}{\alpha^*(\Phi)\alpha^*(\Psi)(x)}, \\
\ip{\mu}{\alpha^*(\Phi\atwo\Psi)(x)} &=
  \ip{\Phi\atwo\Psi}{\alpha(x\otimes\mu)} =
  \ip{\Psi}{\alpha(x\otimes\mu)\cdot\Phi}
  = \ip{\Psi}{\alpha(x\otimes\alpha^*(\Phi)^*(\mu))} \\
  &= \ip{\alpha^*(\Phi)^*(\mu)}{\alpha^*(\Psi)(x)}
  = \ip{\mu}{\alpha^*(\Phi)\alpha^*(\Psi)(x)}.
\end{align*}
Thus $\alpha^*$ is a homomorphism for both $\aone$ and $\atwo$.  That $\alpha^*\circ
\kappa_A = \pi$ is immediate.
\end{proof}

Now suppose that $A$ has a continuous involution $a\mapsto a^*$.  As $(\cdot)^*$ is
anti-linear, the standard way to form the ``adjoint'' is to define
\[ \ip{\mu^*}{a} = \overline{\ip{\mu}{a^*}} \qquad (\mu\in A^*, a\in A). \]
This ensures that for each $\mu\in A^*$, we have that $\mu^*$ is linear (so that
$\mu^*\in A^*$) and that the map $\mu\mapsto\mu^*$ itself is anti-linear.  Perform
the same operation again to define an involution on $A^{**}$, which is easily seen
to extend the original involution on $A$.

\begin{proposition}\label{prop:three}
The involution thus defined on $A^{**}$ satisfies the relation that for
$\Phi,\Psi\in A^{**}$ we have $(\Phi\aone\Psi)^* = \Psi^* \atwo \Phi^*$
(and so also $(\Phi\atwo\Psi)^* = \Phi^* \aone \Psi^*$).
\end{proposition}
\begin{proof}
We again perform some calculations.  For $a,b\in A, \mu\in A^*$ we have that
\begin{align*}
\ip{\mu^*\cdot a}{b} = \ip{\mu^*}{ab} = \overline{\ip{\mu}{b^* a^*}}
= \overline{\ip{a^*\cdot\mu}{b^*}} = \ip{(a^*\cdot\mu)^*}{b}
\end{align*}
Thus, for $a\in A, \mu\in A^*, \Phi\in A^{**}$,
\begin{align*}
\ip{\Phi\cdot\mu^*}{a} = \ip{\Phi}{\mu^*\cdot a} = \ip{\Phi}{(a^*\cdot\mu)^*}
= \overline{\ip{\Phi^*}{a^*\cdot\mu}} = \overline{\ip{\mu\cdot\Phi^*}{a^*}}
= \ip{(\mu\cdot\Phi^*)^*}{a}
\end{align*}
Thus finally, for $\mu\in A^*$ and $\Phi,\Psi\in A^{**}$,
\begin{align*}
\ip{(\Phi\aone\Psi)^*}{\mu} &= \overline{\ip{\Phi\aone\Psi}{\mu^*}}
= \overline{\ip{\Phi}{\Psi\cdot\mu^*}} = \overline{\ip{\Phi}{(\mu\cdot\Psi^*)^*}}
= \ip{\Phi^*}{\mu\cdot\Psi^*} = \ip{\Psi^* \atwo \Phi^*}{\mu},
\end{align*}
as claimed.
\end{proof}

Now let $H$ be a Hilbert space, and let $A\subseteq\mc B(H)$ be our C$^*$-algebra.
Form the universal representation $\pi:A\rightarrow\mc B(K)$.  That is, $K$ is the
direct sum of the GNS spaces for each state on $A$.  We shall use the Jordan
Decomposition of functionals, see for example \cite[Chapter~III, Proposition~2.1]{tak1}.
We note that the proof is ``elementary'' in the sense of using nothing beyond
\cite[Chapter~I]{tak1}.  The particular claim we need is this:

\begin{proposition}\label{prop:one}
For each $\mu\in A^*$ there is a state on $A$ with GNS space $(H_0,\pi_0)$, and
$\xi,\eta\in H_0$ such that $\ip{\mu}{a} = ( \pi_0(a)\xi | \eta )$ for each $a \in A$.
\end{proposition}

This implies immediately that $\alpha:K\proten K^* \rightarrow A^*$ is a surjection.
Hence $\alpha^*$ is an injection, and so Proposition~\ref{prop:two} implies that
$\aone = \atwo$ on $A^{**}$ (that is, a C$^*$-algebra is \emph{Arens regular}).
We can hence speak of ``the'' product on $A^{**}$, and so Proposition~\ref{prop:three}
shows that $A^{**}$ has an isometric involution defined on it.

\begin{proposition}
The map $\alpha^*:A^{**}\rightarrow\mc B(K)$ is a $*$-homomorphism.
\end{proposition}
\begin{proof}
As usual, we identify $K^*$ with the conjugate space to $K$ using the inner-product
$(\cdot|\cdot)$ on $K$.  For $\xi\in K$ let $\overline{K}\in K^*$ be the functional
thus induced.  For $a\in A, \xi,\eta\in K$ we hence have that
\begin{align*}
\ip{\alpha(\xi\otimes\overline\eta)^*}{a}
= \overline{ (\pi(a^*)\xi |\eta) }
= \overline{ (\pi(a)^*\xi |\eta) }
= \overline{ (\xi |\pi(a)\eta) }
= (\pi(a)\eta|\xi)
= \ip{\alpha(\eta\otimes\overline\xi)}{a}.
\end{align*}
Thus, for $\Phi\in A^{**}, \xi,\eta\in K$, we have that
\begin{align*}
(\alpha^*(\Phi^*)\xi|\eta) &= \ip{\Phi^*}{\alpha(\xi\otimes\overline\eta)}
= \overline{ \ip{\Phi}{\alpha(\xi\otimes\overline\eta)^*} }
= \overline{ \ip{\Phi}{\alpha(\eta\otimes\overline\xi)} }
= \overline{ (\alpha^*(\Phi)\eta|\xi) } \\
&= (\xi|\alpha^*(\Phi)\eta)
= (\alpha^*(\Phi)^*\xi|\eta).
\end{align*}
Hence $\alpha^*$ is a $*$-map, as required.
\end{proof}

Finally, we observe that Proposition~\ref{prop:one} implies a little more.
As $\alpha$ is surjective, $\alpha^*$ is bounded below, and not just injective.
Thus $\alpha^*(A^{**}) \subseteq \mc B(K)$ is a closed self-adjoint subalgebra,
in particular, a C$^*$-algebra.  Hence (see \cite[Chapter~1, Section~5]{tak1})
$\alpha^*$ is an isometry.  In conclusion, we have by elementary means shown that
$A^{**}$ is a C$^*$-algebra.

We now turn to the Kaplansky Density Theorem itself.  Let $M\subseteq\mc B(H)$
be a von Neumann algebra, and let $A\subseteq M$ be a C$^*$-algebra whose
weak$^*$-closure
is all of $M$.  Let $\pi:A\rightarrow\mc B(H)$ be the inclusion, and define $\alpha$
and $\alpha^*$ from $\pi$.  So $\alpha^*:A^{**}\rightarrow\mc B(H)$ is a
weak$^*$-continuous $*$-homomorphism.

Let $M_* = \mc B_*(H) / {}^\perp M$ be the predual of $M$.  That $A$ is weak$^*$-dense
in $M$ is equivalent to ${}^\perp A = {}^\perp M \subseteq \mc B_*(H)$.  It is
easy to see that if $\omega \in {}^\perp A$ then $\alpha(\omega)=0$, and so, the
map $\alpha$ drops to give an injective map $M_* \rightarrow A^*$.  We may hence
regard $\alpha^*$ as a weak$^*$-continuous $*$-homomorphism $A^{**} \rightarrow M$.
Let $I = \ker\alpha^* \subseteq A^{**}$, a weak$^*$-closed self-adjoint ideal in
$A^{**}$.  Then $\alpha^*$ drops to give an injective $*$-homomorphism
$A^{**} / I \rightarrow M$.  Thus this map is an isometry.  However, this implies
that $\alpha:M_* \rightarrow A^*$ is an isometry, and so $\alpha^*:A^{**}
\rightarrow M$ is a metric surjection.

By Hahn-Banach, the unit ball of $A$ is weak$^*$-dense in the unit ball of $A^{**}$,
and hence the unit ball of $A$ is weak$^*$-dense in the unit ball of $M$.


\subsection{Further remarks}

We can in fact give an elementary proof of Proposition~\ref{prop:one}.
Let $A\subseteq\mc B(H)$ be a C$^*$-algebra.  Consider the Hilbert space
$\ell^2(H)$, and let $\pi : \mc B(H) \rightarrow \mc B(\ell^2(H))$ be the
obvious $*$-representation.  Then, for each $\omega\in\mc B_*(H)$ there are
$\xi,\eta\in\ell^2(H)$ with $\|\omega\| = \|\xi\|\|\eta\|$ and
\[ \ip{x}{\omega} = (\pi(x)\xi|\eta) \qquad (x\in\mc B(H)). \]
By the Hahn-Banach theorem, we know that the unit ball of $\mc B_*(H)$ is weak$^*$-dense
in the unit ball of $\mc B_*(H)^{**} = \mc B(H)^*$; this is commonly called Goldstine's
theorem.  Thus there is an ultra-filter $\mc U$ such that $(\mc B_*(H))_{\mc U}
\rightarrow \mc B(H)^*$ is a metric surjection.  Let $K = (\ell^2(H))_{\mc U}$ and
let $\Pi:\mc B(H) \rightarrow\mc B(K)$ be the obvious $*$-representation.  Thus,
for $\mu \in \mc B(H)^*$, there is $(\omega_i) \in (\mc B_*(H))_{\mc U}$ with
$\ip{\mu}{x} = \lim_{i\rightarrow\mc U} \ip{x}{\omega_i}$ for $x\in\mc B(H)$, and
with $\|\tau_i\| = \|\mu\|$ for each $i$.  For each $i$, there are $\xi_i,\eta_i
\in \ell^2(H)$ with $\ip{x}{\tau_i} = (\pi(x)(\xi_i)|\eta_i)$, and with
$\|\xi_i\| \|\eta_i\| = \|\tau_i\|$.  Thus, with $\xi = (\xi_i), \eta = (\eta_i)\in K$,
then $\|\xi\| \|\eta\| = \|\mu\|$ and
\[ \ip{\mu}{x} = (\Pi(x)\xi|\eta) \qquad (x\in\mc B(H)). \]

Now let $\mu\in A^*$, and take a Hahn-Banach extension to $\mc B(H)^*$,
so we can find $\xi,\eta\in K$ with $\|\mu\| = \|\xi\| \|\eta\|$ and
\[ \ip{\mu}{x} = (\Pi(x)\xi|\eta) \qquad (x\in A). \]
This is already enough for our application.  But if we wish to complete the
proof of Proposition~\ref{prop:one}, observe that we may suppose that $\|\xi\|=1$
and $\|\eta\|=\|\mu\|$, and so $(\Pi(\cdot)\xi|\xi)$ defines
a state $\lambda$ on $A$.  If $(K_0,\pi_0,\xi_0)$ is the GNS construction for
$\lambda$ then the map $\pi_0(x)\xi_0 \mapsto \ip{\mu}{x}$ is bounded, as
\[ |\ip{\mu}{x}| = |(\Pi(x)\xi|\eta)| \leq \|\Pi(x)\xi\| \|\eta\|
= \ip{\lambda}{x^*x}^{1/2} \|\mu\|
= \|\pi_0(x)\xi_0\| \|\mu\| \qquad (x\in A). \]
Thus there is $\eta_0\in K_0$ with $\|\eta_0\|=\|\mu\|$ and $(\pi_0(x)\xi_0|\eta_0)
= \ip{\mu}{x}$ for $x\in A$, as required.



\section{An informal discussion}

The proof is simple, and occurs in two steps:

\begin{itemize}
\item Show that $A^{**}$ with the Arens products forms a W$^*$-algebra.
\item Then argue that if $A\rightarrow M$ is an inclusion then $A^{**}\rightarrow
M$ will be onto, hence a metric surjection, hence allowing Hahn-Banach to do the
heavily lifting to show Kaplansky Density.
\end{itemize}

The only real difficulty is in checking carefully that it is possible to give
$A^{**}$ a W$^*$-algebra structure in a non-circular way.  For example, the development
of the universal enveloping algebra in \cite[Chapter~III, Section~2]{tak1} uses
Kaplansky density.

As far as I know from the literature, it is usual to use Kaplansky Density, at some point,
when considering Arens products and $C^*$-algebras.  Indeed, after we wrote this note
we found that Ozawa had suggested this sort of argument in a MathOverflow, \cite{oz},
but wondered if such an argument could be made non-circular (so, ``yes'').  Of course,
it remains quite possible such an argument has been observed before.




\begin{thebibliography}{aa}

\bibitem{oz} N. Ozawa (\url{https://mathoverflow.net/users/7591/narutaka-ozawa}), Ultraweak closure inside a closed ball, URL (version: 2012-07-17): \url{https://mathoverflow.net/q/102411}

\bibitem{tak1} M. Takesaki, {\it Theory of operator algebras. I}, reprint of the first (1979) edition, Encyclopaedia of Mathematical Sciences, 124, Springer-Verlag, Berlin, 2002. MR1873025

\end{thebibliography}


\end{document}
