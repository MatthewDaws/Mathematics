\documentclass[a4paper,11pt]{article}
\usepackage[utf8]{inputenc}
\usepackage[margin=2cm]{geometry}

\usepackage{xcolor}
\definecolor{myblue}{rgb}{0.1 0.1 0.6}
\usepackage{hyperref}
\hypersetup{
   colorlinks=true,
   linkcolor=myblue,
   citecolor=myblue,
   urlcolor=myblue
}
\usepackage{hyperref}

\usepackage{latexsym, amsmath, amsthm, amssymb}
\usepackage{url}
%\usepackage[all]{xy}
%\usepackage{dsfont}

\newcommand{\mc}[1]{{\mathcal{#1}}}
\newcommand{\ip}[2]{\langle{#1},{#2}\rangle}
\newcommand{\Rep}{\operatorname{Rep}}
\newcommand{\lin}{{\operatorname{lin}}}
\newcommand{\supp}{{\operatorname{supp}}}
\newcommand{\G}{{\mathbb{G}}}
\newcommand{\HH}{{\mathbb{H}}}
\newcommand{\hh}{\widehat}
\newcommand{\qaut}{\operatorname{QAut}}
\newcommand{\aut}{\operatorname{Aut}}
\newcommand{\mor}{\operatorname{Mor}}
\newcommand{\op}{{\operatorname{op}}}
\newcommand{\cop}{{\operatorname{cop}}}
\newcommand{\id}{{\operatorname{id}}}
\newcommand{\vnten}{\overline\otimes}
\newcommand{\WW}{\mathbb{W}}
%\newcommand{\Ww}{{W_{ur}}}
%\newcommand{\wW}{{W_{ru}}}
\newcommand{\Ww}{\mathds{W}}
\newcommand{\wW}{\text{\reflectbox{$\Ww$}}\:\!} % requires graphicx, only works in pdf
\newcommand{\pol}{\operatorname{Pol}}
\newcommand{\irr}{\operatorname{Irr}}

\newtheorem{lemma}{Lemma}[section]
\newtheorem{proposition}[lemma]{Proposition}
\newtheorem{theorem}[lemma]{Theorem}
\newtheorem{corollary}[lemma]{Corollary}

\theoremstyle{definition}
\newtheorem{definition}[lemma]{Definition}
\newtheorem{example}[lemma]{Example}
\newtheorem{examples}[lemma]{Examples}

\newtheorem{remarkx}[lemma]{Remark}
\newtheorem{remarksx}[lemma]{Remarks}
% Some hacks to get a symbol printed at the end of a remark, as it was very unclear (in my
% writing style) where a remark ended and the general flow of the paper (re)started.
\newenvironment{remark}
  {\pushQED{\qed}\renewcommand{\qedsymbol}{$\triangle$}\remarkx}
  {\popQED\endremarkx}
\newenvironment{remarks}
  {\pushQED{\qed}\renewcommand{\qedsymbol}{$\triangle$}\remarksx}
  {\popQED\endremarksx}


\begin{document}
\title{Cohen--Hewitt Factorisation: A brief history}
\author{Matthew Daws}
\maketitle

\begin{abstract}
We review some of the original literature around the Cohen--Hewitt factorisation theorem.
\end{abstract}

We fix some notation and terminology.  Let $A$ be a Banach algebra.  We will always assume that
the product is contractive: $\|ab\| \leq \|a\| \|b\|$ for $a,b \in A$.  A \emph{bounded left
approximate identity} (blai) for $A$ is a bounded net $(e_i)$ with $\lim_i \|e_ia - a\|=0$ for
each $a\in A$.  The \emph{bound} of $(e_i)$ is $\limsup_i \|e_i\|$.

For us, a \emph{left module} over $A$ is a Banach space $E$ which is algebraically a left
module, such that the action is contractive, meaning $\|a\cdot x\| \leq \|a\| \|x\|$ for
$a\in A, x\in E$.  We denote by $\overline{A\cdot E}$ the closed linear span of the set
$\{ a\cdot x : a\in A, x\in E \}$.  We say that $E$ is essential if $E = \overline{A\cdot E}$.


\section{Cohen's paper}

This is \cite{cohen}\footnote{A small rant: Why does Duke insist that one has a subscription to
access a paper from $\geq70$ years ago?} translated into slightly more modern language.

\begin{theorem}[{\cite[Theorem~1]{cohen}}]
Let $A$ be a Banach algebra with a bounded left approximate identity.  For each $a\in A$ and
$\epsilon>0$ there are $b,c\in A$ with:
\begin{enumerate}
\item $a=bc$;
\item $c$ belongs the the closed left ideal generated by $a$;
\item $\|a-c\| < \epsilon$.
\end{enumerate}
\end{theorem}



\section{Hewitt's paper}

The statement concerns, in modern language, left modules.  This paper works with a Banach
algebra $A$ and a left module $E$ such that $A$ possesses a blai $(e_i)$ which additionally
satisfies $\lim_i \|e_i\cdot x-x\| =0$ for each $x\in E$.  We see that this is equivalent to
$A$ having some blai, and $E$ being essential.

\begin{theorem}[{\cite[Theorem~2.5]{hewitt}}]\label{thm:ch}
Let $A$ be a Banach algebra with a blai of bound $C$, and let $E$ be an essential left module
over $A$. For each $x\in E$ and $\epsilon>0$ there is $a\in A, y\in E$ with:
\begin{enumerate}
\item $x = a\cdot y$;
\item $y$ belongs to the closure of $\{b\cdot x : b\in A\}$;
\item $\|x-y\| \leq \epsilon$;
\item $\|a\|\leq C$.
\end{enumerate}  
\end{theorem}


\section{Doran--Wichmann book}

This is \cite{dw} which undertakes a careful (and in our opinion extremely readable) survey
of the state of the art, circa 1979.  The factorisation result is considered in
\cite[Section~16]{dw} and the statement is identical to Theorem~\ref{thm:ch}.  No references
are given in the text, but extensive notes can be found starting in page~248.  From this,
it does appear to be reasonable to call Theorem~\ref{thm:ch} the Cohen--Hewitt Factorisation
Theorem.


\section{Further refinements}

In \cite[Theorem~2.9.24]{dales} a refinement of Theorem~\ref{thm:ch} is presented:

\begin{theorem}[{\cite[Theorem~2.9.24]{dales}}]
Let $A$ be a Banach algebra with a blai of bound $C$, and let $E$ be an essential left module
over $A$.  Let $(\alpha_n)$ be an increasing, unbounded sequence in $(1,\infty)$.
For each $x\in E$ and $\epsilon>0$ there is $a\in A$ and a sequence $(y_n)\in E$ with:
\begin{enumerate}
\item $x = a^n\cdot y_n$ for each $n$;
\item $\|x-y_n\| \leq \epsilon$;
\item $\|y_n\| \leq \alpha^n_n \|x\|$;
\item $\|a\| \leq C$.
\end{enumerate}  
\end{theorem}

Then \cite[Corollary~2.9.26]{dales} gives a statement equivalent to Theorem~\ref{thm:ch}.
The above is attributed to Cohen and Hewitt, and also \cite{as}.

Palmer's book \cite{palmer} considers factorisation results in Section~5.2, using the
language of representations instead of modules.  The results stated are equivalent to those
discussed above.


\section{Further reading}

A detailed history of the theorem can be found in \cite[pages~1033-34]{ross}.


\begin{thebibliography}{99}

\bibitem{as} G. R. Allan\ and\ A. M. Sinclair, Power factorization in Banach algebras with a bounded approximate identity, Studia Math. {\bf 56} (1976), no.~1, 31--38. MR0410380

\bibitem{cohen} P. J. Cohen, Factorization in group algebras, Duke Math. J. {\bf 26} (1959), 199--205. MR0104982

\bibitem{dales} Dales, H. G. Banach algebras and automatic continuity. London Mathematical Society Monographs. New Series, 24. Oxford Science Publications. The Clarendon Press, Oxford University Press, New York, 2000. xviii+907 pp. ISBN: 0-19-850013-0 MR1816726

\bibitem{dw} R. S. Doran\ and\ J. Wichmann, {\it Approximate identities and factorization in Banach modules}, Lecture Notes in Mathematics, 768, Springer-Verlag, Berlin, 1979. MR0555240

\bibitem{hewitt} E. Hewitt, The ranges of certain convolution operators, Math. Scand. {\bf 15} (1964), 147--155. MR0187016

\bibitem{palmer} T. W. Palmer, {\it Banach algebras and the general theory of $^*$-algebras. Vol. I}, Encyclopedia of Mathematics and its Applications, 49, Cambridge University Press, Cambridge, 1994. MR1270014

\bibitem{ross} K. A. Ross, A trip from classical to abstract Fourier analysis, Notices Amer. Math. Soc. {\bf 61} (2014), no.~9, 1032--1038. MR3241559

\end{thebibliography}

\end{document}

