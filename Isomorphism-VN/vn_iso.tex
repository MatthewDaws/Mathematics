\documentclass[a4paper,11pt]{article}
\usepackage[utf8]{inputenc}
\usepackage[margin=2cm]{geometry}

\usepackage{xcolor}
\definecolor{myblue}{rgb}{0.1 0.1 0.6}
% ``backref'' for drafting; see manual for further options
%\usepackage[backref]{hyperref}
%\usepackage{showkeys}
\usepackage{hyperref}
\hypersetup{
   colorlinks=true,
   linkcolor=myblue,
   citecolor=myblue,
   urlcolor=myblue
}

\usepackage[T1]{fontenc}
\usepackage{beton}
\usepackage[euler-digits, euler-hat-accent]{eulervm}
\DeclareFontSeriesDefault[rm]{bf}{sbc} 

\usepackage{amsmath,amssymb,amsthm}
%\usepackage[all]{xy}
\usepackage{url, enumitem}

\theoremstyle{plain}
\newtheorem{proposition}{Proposition}[section]
\newtheorem{theorem}[proposition]{Theorem}
\newtheorem{corollary}[proposition]{Corollary}
\newtheorem{lemma}[proposition]{Lemma}
\newtheorem{claim}[proposition]{Claim}
\newtheorem{definition}[proposition]{Definition}
\newtheorem{example}[proposition]{Example}
\newtheorem{question}[proposition]{Question}

\theoremstyle{remark}
\newtheorem{remarkx}[proposition]{Remark}
\newtheorem{remarksx}[proposition]{Remarks}
\newtheorem{workingx}[proposition]{Working}
% Some hacks to get a symbol printed at the end of a remark, as it was very unclear (in my
% writing style) where a remark ended and the general flow of the paper (re)started.
\newenvironment{remark}
  {\pushQED{\qed}\renewcommand{\qedsymbol}{$\triangle$}\remarkx}
  {\popQED\endremarkx}
\newenvironment{remarks}
  {\pushQED{\qed}\renewcommand{\qedsymbol}{$\triangle$}\remarksx}
  {\popQED\endremarksx}
\newenvironment{working}
  {\pushQED{\qed}\renewcommand{\qedsymbol}{$\triangle$}\workingx}
  {\popQED\endworkingx}


\newcommand{\mc}[1]{\mathcal{#1}}
\newcommand{\mf}[1]{\mathfrak{#1}}
\newcommand{\msf}[1]{\mathsf{#1}}

\newcommand{\ip}[2]{{\langle {#1} , {#2} \rangle}}
\newcommand{\lin}{\operatorname{lin}}
\newcommand{\id}{\operatorname{id}}

\newcommand{\vnten}{\bar\otimes}

\newcommand{\hh}{\widehat}
\newcommand{\G}{\mathbb{G}}
\renewcommand{\H}{\mathbb{H}}
\newcommand{\op}{{\operatorname{op}}}

\begin{document}

\title{Isomorphism of von Neumann algebras}
\author{Matt Daws}
\date{September 2024}
\maketitle

\begin{abstract}
We give a simple proof of the fact that isomorphic von Neumann algebras have unitarily equivalent amplifications.
\end{abstract}

\section{Introduction}

At a number of places in the literature, it is claimed that when $M\subseteq\mc B(H_M)$ and $N\subseteq\mc B(H_N)$ are von Neumann algebras which are $*$-isomorphic, then there is an amplification which is unitarily equivalent: there is some Hilbert space $L$ and a unitary $u\colon H_M\otimes L \to H_N\otimes L$ with $u(x\otimes 1) = (\theta(x)\otimes 1)u$ where $\theta \colon M\to N$ is the $*$-isomorphism.

This is stated without proof in \cite[Theorem~III.2.2.8]{Blackadar_OperatorAlgebrasBook}, for example, and I have seen references to e.g. \cite[Theorem~IV.5.5]{TakesakiI}, which doesn't even state this result explicitly.  A rather minimal proof is given in \cite[Theorem~2.4.26]{BR_OpAlgBook_1}.  The implication is that this result should be an obvious corollary from the ``structure theorem'' for $*$-homomorphism between von Neumann algebras, but my belief is that some extra ingredients are needed.  This note spells out these extra steps.

I had forgotten that I asked this on \texttt{math.stackexchange} \cite{4728018}, and provided my own answer, using the same technique as here.  The strategy is that given by Jones in his lecture notes \cite[Theorem~7.2.1]{JonesNotes}.

\section{The structure theorem}

What I refer to as the ``structure theorem'' for normal unital $*$-homomorphisms between von Neumann algebras is the following (see \cite[Theorem~III.2.2.8]{Blackadar_OperatorAlgebrasBook},  \cite[Theorem~IV.5.5]{TakesakiI} for example).

\begin{theorem}
Let $M\subseteq\mc B(H_M)$ be a von Neumann algebra, and let $\theta \colon M \to \mc B(K)$ be a normal unital $*$-homomorphism.  There is a Hilbert space $K'$ and an isometry $u\colon K \to H \otimes K'$ such that $u\theta(x) = (x\otimes 1)u$ for $x\in M$.  As such, $\theta(x) = u^*(x\otimes 1)u$ for $x\in M$, we have that $e = uu^*$ is a projection in $(M\otimes 1)' = M'\vnten\mc B(K')$, and so $u$ gives a spatial isomorphism between $\theta(M)$ and the induced von Neumann algebra $(M\otimes 1)_e$.
\end{theorem}
\begin{proof}
Given a unit vector $\xi\in K$, the functional $\omega_\xi\circ\theta$ is a normal state on $M$, and so there is a square-summable sequence $(\eta_n)$ in $H$ with $\omega_\xi\circ\theta = \sum_n \omega_{\eta_n}$.  Let $\eta = \sum_n \eta_n \otimes \delta_n \in H\otimes \ell^2$ with $\delta_n$ the standard orthonormal basis of $\ell^2$, so
\[ (\eta|(x\otimes 1)\eta) = \sum_n (\eta_n|x\eta_n) = (\xi|\theta(x)\xi) \qquad (x\in M). \]
By Zorn's Lemma, we may choose $(\xi_i)_{i\in I}$ a maximal family of unit vectors in $K$ such that $(\xi_i|\theta(x)\xi_j)=0$ for $i\not=j, x\in M$.  As $\theta$ is a $*$-homomorphism, this is equivalent to the subspaces $K_i = \{ \theta(x)\xi_i : x\in M \}\subseteq K$ being pairwise orthogonal.  For each $i$ choose $\eta_i\in H\otimes\ell^2$ associated to $\xi_i$.
%, and then the map $u_i \colon \theta(x)\xi_i \mapsto (x\otimes 1)\eta_i$ is an isometry (as $\|(x\otimes 1)\eta_i\|^2 = (\eta_i|(x^*x\otimes 1)\eta_i) = (\xi_i|\theta(x^*x)\xi_i) = \|\theta(x)\xi_i\|^2$.)  So $u_i$ extends to the closure $\overline{K_i}$, and by setting $u_i\equiv 0$ on $K_i^\perp$, we obtain a partial isometry.  As the initial subspaces of each $u_i$ are mutually orthogonal, we can define
We define
\[ u \colon \sum_i \theta(x_i)\xi_i \mapsto \sum_i (x_i\otimes 1)\eta_i\otimes\delta_i, \]
for any finitely supported family $(x_i)$ in $M$, where $(\delta_i)$ is the canonical basis of $\ell^2(I)$, and then claim that $u$ is an isometry.  Indeed,
\begin{align*}
\Big\| \sum_i (x_i\otimes 1)\eta_i\otimes\delta_i \Big\|^2
&= \sum_i \| (x_i\otimes 1)\eta_i \|^2
= \sum_i (\eta_i|(x_i^*x_i\otimes 1)\eta_i)
= \sum_i (\xi_i|\theta(x_i^*x_i)\xi_i) \\
&= \Big( \sum_i \theta(x_i)\xi_i \Big| \sum_j \theta(x_j)\xi_j \Big)
= \Big\| \sum_i \theta(x_i)\xi_i \Big\|^2,
\end{align*}
in the last step using that $K_i \perp K_j$ for $i\not=j$.  So $u$ extends to an isometry on the closure of its domain.  If this is not all of $K$, then there is a unit vector $\xi_0$ orthogonal to each $K_i$, which contradicts the maximality of our family $(\xi_i)$.

Set $K' = \ell^2 \otimes \ell^2(I)$ so $u\colon K \to H\otimes K'$.  Almost by definition, $u\theta(x) = (x\otimes 1)u$ for $x\in M$, so as $u$ is an isometry, $\theta(x) = u^*(x\otimes 1)u$.  Then also $\theta(x)u^* = u^*(x\otimes 1)$ for each $x\in M$, and hence $uu^* (x\otimes 1) = (x\otimes 1) uu^*$ so $e = uu^* \in (M\otimes 1)' = M' \vnten \mc B(K')$.  Then $u$ is a unitary between $K$ and the image of $e$, and so $u$ gives a spatial isomorphism between $(M\otimes 1)_e$ and $\theta(M)$.
\end{proof}

Notice that when $K$ is separable, we can choose the family $(\xi_i)$ by using a version of the Gram--Schmidt procedure, starting from some dense subset of $K$ say.  Thus in this case $I$ is countable, and we can choose $K'$ to be separable.


\section{The case of isomorphisms}

When $\theta$ is an isomorphism (equivalently, $\theta$ is injective and $N = \theta(M)$) we hence obtain $u\colon H_N \to H_M \otimes K$ and $v\colon H_M \to H_N\otimes K'$ both isometries, with $u\theta(x)=(x\otimes 1)u$ and $vx=(\theta(x)\otimes 1)v$ for $x\in M$.  However, there seems no particular reason for a direct relationship between $u$ and $v$ and $K$ and $K'$.

We proceed by using the comparison theory of projections, \cite[Section~V.1]{TakesakiI}, \cite[Chapter~6]{KadisonRingroseII}.  Recall that projections $e,f\in M$ are equivalent, written $e\sim f$, when there is a partial isometry $u\in M$ with $u^*u=e, uu^*=f$.  We further write $e \preceq f$ when $e$ is equivalent to a subprojection of $f$, that is, there is $u\in M$ with $u^*u=e$ and $uu^* \leq f$, that is, $f uu^* = uu^*$.  A variant of the Cantor--Bernstein theorem shows that if $e \preceq f$ and $f \preceq e$ then $e\sim f$, see \cite[Proposition~V.1.3]{TakesakiI} for example.  We shall make use of the result in an essential way.

Define
\[ M_1 = \Big\{ \begin{pmatrix} x & 0 \\ 0 & \theta(x) \end{pmatrix}: x\in M \Big\} \subseteq \mc B(H_M \oplus H_N), \]
a von Neumann algebra.  We identify
\begin{equation} \label{eq:1}
\mc B(H_M \oplus H_N) \cong
\Big\{ \begin{pmatrix} a & b \\ c & d \end{pmatrix} : a\in\mc B(H_M), d\in\mc B(H_N), b,c^*\in\mc B(H_N, H_M) \Big\}.
\end{equation}
A simple calculation shows that
\[ M_1' = \Big\{ \begin{pmatrix} a & b \\ c & d \end{pmatrix} : a\in M', d\in\theta(M)', xb=b\theta(x), cx=\theta(x)c \ (x\in M) \Big\}. \]
Given any Hilbert space $L$, we identify $(H_M \oplus H_N) \otimes L$ with $(H_M\otimes L) \oplus (H_N\otimes L)$ and so have a similar identification as \eqref{eq:1}.  Let $w\in M_1'\vnten\mc B(L)$ be a partial isometry with
\[ w^*w = \begin{pmatrix} 1 & 0 \\ 0 & 0 \end{pmatrix}, \quad
ww^* = \begin{pmatrix} 0 & 0 \\ 0 & 1 \end{pmatrix}
\quad\implies\quad
w = w \begin{pmatrix} 1 & 0 \\ 0 & 0 \end{pmatrix} =
\begin{pmatrix} 0 & 0 \\ 0 & 1 \end{pmatrix} w
\quad\implies\quad
w = \begin{pmatrix} 0 & 0 \\ u & 0 \end{pmatrix}, \]
where $u \colon H_M\otimes L \to H_N\otimes L$ satisfies $u^*u = 1_{H_M}\otimes 1_L$ and $uu^* = 1_{H_N}\otimes 1_L$, that is, $u$ is unitary.  As $w\in M_1'\vnten\mc B(L)$, we have that $u(x\otimes 1) = (\theta(x)\otimes 1)u$.  Conversely, any such $u$ gives rise to such a $w$.

So we exactly wish to show that such a $w$ exists.  By the theorem quoted above, it is enough to show that inside $M_1'\vnten\mc B(L)$ we have
\[ \begin{pmatrix} 1 & 0 \\ 0 & 0 \end{pmatrix} \preceq \begin{pmatrix} 0 & 0 \\ 0 & 1 \end{pmatrix},
\quad
\begin{pmatrix} 0 & 0 \\ 0 & 1 \end{pmatrix} \preceq  
\begin{pmatrix} 1 & 0 \\ 0 & 0 \end{pmatrix}. \]

Applying the structure theorem to $\theta$, as above, we have $u\colon H_N \to H_M \otimes K$ with $u^*u=1$ and $u\theta(x) = (x\otimes 1)u$.  For a set $I$ with sufficiently large cardinality, we have $K \otimes \ell^2(I) \cong \ell^2(I)$ (for example, if $K$ is separable, $I=\mathbb N$ suffices).  Then $u\otimes 1 \colon H_N \otimes \ell^2(I) \to H_M \otimes K \otimes \ell^2(I) \cong H_M \otimes \ell^2(I)$ and $(u\otimes 1)(\theta(x)\otimes 1) = (x\otimes 1)u\otimes 1 \cong (x\otimes 1)(u\otimes 1)$.  So $u_1 = u\otimes 1$ satisfies $u_1^*u_1=1$ and $u_1u_1^* \leq 1$, showing that
\[ \begin{pmatrix} 1 & 0 \\ 0 & 0 \end{pmatrix} \preceq \begin{pmatrix} 0 & 0 \\ 0 & 1 \end{pmatrix} \]
inside $M_1' \vnten \mc B(\ell^2(I))$.  Applying the same argument to $\theta^{-1}$ shows the other partial order, but possibly for a different index set $I$.  Of course, we can then simply choose the $I$ with the larger cardinality, and we have verified both partial orders, as required.


\section{A worked example}

We give an example to show that things can be a little complicated.  It is a fun exercise to show that if $H,K$ are Hilbert spaces, then a normal unital $*$-homomorphism $\theta \colon \mc B(H)\to \mc B(K)$ must take the following form: $K\cong H\otimes K'$ for some $K'$, and under this isomorphism, $\theta(x)=x\otimes 1$.  (For example, use the structure theorem, and note that $(\mc B(H)\otimes 1)' = 1\otimes\mc B(K')$.)

Now let $M$ be the von Neumann algebraic direct sum of finite-dimensional matrix algebras $\mathbb M_{n_\alpha}$.  Let $\theta \colon M \to \mc B(K)$ be an injective unital normal $*$-homomorphism.  Let $1_\alpha \in \mathbb M_{n_\alpha}$ be the units of the matrix blocks, so $(1_\alpha)$ are minimal central projections in $M$, each mutually orthogonal.  The same is hence true of $(\theta(1_\alpha))$ in $\mc B(K)$, and so $K$ decomposes as the direct sum of orthogonal subspaces $K_\alpha = \theta(1_\alpha)H$.  The restriction of $\theta$ to the matrix block $\alpha$ gives a non-zero $*$-homomorphism $\mathbb M_{n_\alpha} \to \mc B(K_\alpha)$ and so $K_\alpha \cong \mathbb C^{n_\alpha} \otimes H_\alpha$ say, with $\mathbb M_{n_\alpha}$ acting as $x\otimes 1$.  Any family $(H_\alpha)$ could occur in this way.

So, given two embeddings $M\subseteq \mc B(H_1)$ and $M\subseteq\mc B(H_2)$, we have
\[ H_k = \bigoplus_\alpha \mathbb C^{n_\alpha} \otimes H_\alpha^{(k)}, \]
for $k=1,2$.  Tensoring $H_k$ with $L$ is equivalent to tensoring $H_\alpha^{(k)}$ with $L$, for each $\alpha$.  Any unitary equivalence between these representations must restrict to unitaries $H^{(1)}_\alpha \to H^{(2)}_\alpha$ for each $\alpha$.

It is now clear that $H_1\otimes L$ and $H_2\otimes L$ will admit a unitary intertwiner when $L$ is sufficiently large, but that not just any choice of $L$ will work.

We now consider when there is an isometry $u\colon H_2 \to H_1\otimes K'$ intertwining the representations.  Again, $u$ must restrict to the components
\[ u_\alpha \colon \mathbb C^{n_\alpha} \otimes H_\alpha^{(1)} \to
\mathbb C^{n_\alpha} \otimes (H_\alpha^{(2)} \otimes K'), \quad
u_\alpha(x\otimes 1) = (x\otimes 1\otimes 1)u_\alpha \qquad (x\in\mathbb M_{n_\alpha}), \]
for each $\alpha$.  It follows that $u_\alpha = 1\otimes v_\alpha$ for some isometry $u'_\alpha \colon H_\alpha^{(1)} \to H_\alpha^{(2)} \otimes K'$, which may be arbitrary.  Again, this puts constraints on the size of $K'$.  It also shows that if $v\colon H_1 \to H_2\otimes K''$ is an intertwiner going the other way, there need indeed be little relation between $u$ and $v$, as the associated isometries $v'_\alpha \colon H^{(2)}_\alpha \to H^{(1)}_\alpha \otimes K''$ are again arbitrary.

\bibliographystyle{plain}
\bibliography{thebib.bib}

\end{document}
